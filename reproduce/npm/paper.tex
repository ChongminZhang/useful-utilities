\published{Geophysical Prospecting, 66, 85-97 (2018)}
  
\title{Data-driven time-frequency analysis of seismic data using 
non-stationary Prony method} 
\renewcommand{\thefootnote}{\fnsymbol{footnote}}
\address{
\footnotemark[1] The College of Science\\
China University of Petroleum\\
Beijing, China
\footnotemark[2] Bureau of Economic Geology,\\
John A. and Katherine G. Jackson School of Geosciences\\
The University of Texas at Austin\\
University Station, Box X\\
Austin, TX, USA, 78731-8924\\
\footnotemark[3] Previously:  The University of Texas at Austin; Currently: Oak Ridge National Laboratory
}

\author{Guoning Wu\footnotemark[1], Sergey Fomel\footnotemark[2], Yangkang Chen\footnotemark[3]}

%\footer{GEO-2016}
%\lefthead{Wu etc.}
\righthead{Time-frequency analysis}
\maketitle

\begin{abstract}
The empirical mode decomposition aims to decompose the input signal 
into a small number of components named intrinsic mode functions 
with slowly varying amplitudes and frequencies. In spite of its simplicity
and usefulness, however, the empirical mode decomposition lack 
solid mathematical foundation. In this paper, we describe a method to 
extract the intrinsic mode functions of the input signal using 
non-stationary Prony method. The proposed method captures the 
philosophy of the empirical mode decomposition, but use a different 
method to compute the intrinsic mode functions. 
Having the intrinsic mode functions obtained, we then compute 
the spectrum of the input signal using Hilbert transform. Synthetic and field data 
validate the proposed method can correctly compute the 
spectrum of the input signal, and could be used in seismic data analysis to facilitate 
interpretation.
\end{abstract}

\section{INTRODUCTION}

Time-frequency analysis maps an 1D time signal into 2D time and frequency domains, 
which can capture the non-stationary character of seismic data. Time-frequency 
analysis is a fundamental tool for seismic data analysis and geological 
interpretation \cite[]{Castagna2003,Reine2009,yangkang2014sswt,liuwei2016}. 
Conventional time-frequency methods, such as short time Fourier transform \cite[]{Cohen1989}, 
wavelet transform \cite[]{Mallat1989} and S-transform \cite[]{Stockwell1996} 
are under the control of Heisenberg/Gabor uncertainty principle, which states 
that we cannot have the energy arbitrarily located in both time and frequency 
domains \cite[]{mallat}. Moreover, short time Fourier transform, wavelet transform 
and S-transform are using a windowing process, which often brings smearing and 
leakage \cite[]{Tary2014}. Therefore spurious frequencies are often generated, 
which will "color" the real time-frequency map and affects the interpretation. 
In recent years, many new methods were proposed such as matching pursuit \cite[]{Mallat1993}, 
basis pursuit \cite[]{basispursuit}, empirical mode decomposition \cite[]{emd,emdseis}, 
the synchrosqueezing wavelet transform \cite[]{daubechies} and its variants such as the 
synchrosqueezing short time Fourier transform \cite[]{Oberlin2014}, 
the synchrosqueezing S-transform \cite[]{Synchrostran}. The matching pursuit and basis pursuit methods 
represent the energies of the input signal by the energies of atoms found using 
different methods, which prevents smearing and leakage, creating highly 
localized time-frequency decompositions. The efficiency of these two methods depend
on the predefined dictionary \cite[]{Tary2014}. The empirical mode decomposition 
method decomposes a signal into symmetric, narrow-band waveforms named intrinsic 
mode functions to compress artificial spectra caused by sudden changes and to 
therefore improve the time-frequency resolution \cite[]{han}. However, the empirical mode
decomposition also suffers from mode mixing and splitting problems. 
In order to solve the above problems, alternative methods were developed 
based on empirical mode decomposition: ensemble empirical mode decomposition\cite[]{eemd}, 
complete ensemble empirical mode decomposition \cite[]{ceemd}. However, these two methods, 
like the empirical mode decomposition, are still "empirical" because 
their sketchy mathematical justifications. The synchrosqueezing wavelet 
transform \cite[]{daubechies} and its variants capture the philosophy of 
empirical mode decomposition, but use a different method to compute the 
intrinsic mode functions providing a rigorous mathematical framework.

Similar to the Fourier transform, the Prony method \cite[]{prony} decomposes 
a signal into a series of damped exponentials or sinusoids in a data-driven manner, 
which allows for the estimation of frequencies, amplitudes, phases and damping 
components of a signal. \citet[]{fomel2013} proposed the non-stationary Prony method (NPM)
based on regularized non-stationary auto-regression. The NPM decomposes a signal into 
intrinsic mode functions with controlled smoothness of amplitudes and frequencies 
like the empirical mode decomposition does, but uses NPM instead.
Unlike Fourier transform, the coefficients of the extracted intrinsic mode functions 
for the Prony method do not clearly define the energy distribution for the input 
signal in the time-frequency domain. Therefore, the NPM used by Fomel does not clearly
define a "real" time-frequency map but a "time-component" map. In this paper, we couple the NPM 
\cite[]{fomel2013} and the Hilbert transform to give a time-frequency decomposition. 
The proposed method has a rigorous mathematical framework. Furthermore, 
synthetic and real data tests confirm that the intrinsic mode functions derived by the proposed 
method are more smooth with respect to the amplitudes and frequencies
than the intrinsic mode functions of ensemble empirical mode decomposition \cite[]{eemd}. 
Synthetic and real data tests also confirm that the proposed method
has a higher time-frequency resolution than the ensemble empirical mode decomposition.
The proposed method can be used to facilitate seismic interpretation.

\section{THEORY}
We give a short description of the theories for empirical mode decomposition, Prony, and 
NPM. For details of the Prony and NPM see Appendix.
\subsection{Empirical Mode Decomposition}
Empirical mode decomposition is a data-driven method, which is a powerful 
tool for non-stationary signal analysis \cite[]{emd}. This method decomposes 
a signal into slowly varying time dependent amplitudes and phases components named intrinsic 
mode functions. The time-frequency decomposition for the input signal is attributed to 
the Hilbert transform of the intrinsic mode functions extracted by the sifting process\cite[]{han}. 
If $s(t)$ is the input signal, the empirical mode decomposition can be written as:
\begin{equation}
    \label{eq:eq1}
    s(t) =\sum_{k=1}^{K} s_k (t)= \sum_{k=1}^{K} A_k (t)\cos (\phi_k(t)),
\end{equation}
where $A_k(t)$ measures amplitude modulation, and $\phi_k(t)$ measures 
phase oscillation. Each $s_k(t)$ has a narrow-band 
waveform and an instantaneous frequency that is smooth and positive. 
The empirical mode decomposition is powerful, but its mathematical theory is sketchy.

\subsection{Prony Method}
Prony method extracts damped complex exponential functions (or sinusoids) 
from a given signal by solving a set of linear equations 
\cite[]{prony,prony2003,ThomasPeter2013,Mitrofanov2015}. 
The Prony method allows for estimation of frequencies, amplitudes and phases 
of a signal (For details see Appendix). Assume we want to solve the problem:
\begin{equation}
    \label{eq:eq2}
     x[n] = \sum_{k=1}^{M}A_k e^{(\alpha_k + j\omega_k)(n-1)\Delta t + j\phi_k},
\end{equation}
if let $h_k = A_k e^{j\phi_k}, z_k = e^{(\alpha_k + j\omega_k)\Delta t}$, we derive the concise form
\begin{equation}
    \label{eq:eq3}
     x[n] = \sum_{k=1}^{M}h_k z_k^{n-1}.
\end{equation}
The above M equations can be written in a matrix form:
\begin{equation}
    \label{eq:eq4}
    \left[ \begin{array}{cccc}
        z_1^0 & z_2^0 & \cdots & z_M^0\\
        z_1^1 & z_2^1 & \cdots & z_M^1\\
        \vdots & \vdots & & \vdots\\
        z_1^{M-1} & z_2^{M-1} & \cdots & z_M^{M-1}
    \end{array} \right]
    \left[ \begin{array}{c} h_1\\ h_2\\ \vdots \\ h_M
    \end{array} \right] =
    \left[ \begin{array}{c} x[1]\\ x[2]\\ \vdots \\ x[M]
    \end{array} \right].
\end{equation}
The above $z_k, k=1,2, \cdots , M$ of equation~\ref{eq:eq4} 
can be computed by solving a polynomial of the form:
\begin{equation}
    \label{eq:eq5}
    \mathbf{P}(z)  = \prod_{k=1}^{M}(z-z_k),
\end{equation}
equation~\ref{eq:eq5} can also be written in a form:
\begin{equation}
    \label{eq:eq6}
    \mathbf{P}(z) = a_0 z^M + a_1 z^{M-1} + \cdots + a_{M-1}z + a_M.
\end{equation}
The coefficients $a_k$ of the polynomial can be computed by solving the following equation:
\begin{equation}
    \label{eq:eq7}
    \sum_{m=0}^{M}a_mx[n-m] = 0.
\end{equation}
We use the method proposed by \citet[]{roots} to compute the roots $z_k$ of equation~\ref{eq:eq6}. 
If the roots are solved, 
the $h_k$ can be computed using equation~\ref{eq:eq3}.
Finally, the components are 
computed based on the equation below(For details see Appendix):
\begin{equation}
    c_k[n] = h_kz_k^{n-1} = h_kz_k^{n-1}, k=1,2,\cdots,M.
\end{equation}

\subsection{Non-stationary Prony method}
Equation~\ref{eq:eq7} can be written as:
\begin{equation}
    \label{eq:eq9}
    \sum_{m=1}^{M}\hat{a}_m x[n-m] = x[n].
\end{equation}
If the $\hat{a}_m$ in equation~\ref{eq:eq9} are time dependent, then we have:
\begin{equation}
    \label{eq:eq10}
    \sum_{m=1}^{M}\hat{a}_m[n] x[n-m] \approx x[n],
\end{equation}
which is an under-determined linear system. There are many methods for solving under-determined 
linear system, such as Tikhonov method \cite[]{tikhonov}.
In this paper, we apply shaping regularization \cite[]{shape,lpf} to regularize
the under-determined linear system, and obtain (for details see Appendix):
\begin{equation}
    \label{eq:eq11}
    \mathbf{\hat{a}} = \mathbf{F}^{-1}\mathbf{\eta},
\end{equation}
where $\mathbf{\hat{a}}$ is a vector composed of $\hat{a}_m [n]$, 
the elements of vector $\mathbf{\eta}$ are 
$\eta_i[n] = \mathbf{S}[x_i^* [n] x[n]] $, where $x_i[n] = x[n-i]$, $x_i^*[n]$ stands for the complex conjugate of $x_i[n]$
and $\mathbf{S}$ is the shaping operator. 
The elements of matrix $\mathbf{F}$ are:
\begin{equation}
    \label{eq:eq12}
    {F}_{ij}[n] = \sigma^2 \delta_{ij} + \mathbf{S}[x_i^*[n]x_j[n] - \sigma^2 \delta_{ij}],
\end{equation}
where $\sigma$ is the regularization parameter.
Solving equation~\ref{eq:eq11}, we obtain the 
coefficients vector $\hat{a}_m[n]$ and form a polynomial below:
\begin{equation}
    \mathbf{P}(z) = z^M + \hat{a}_1[n] z^{M-1} + \cdots + \hat{a}_M[n].
\end{equation}
For the roots computation $\hat{z}_m[n], m=1,2,\cdot,M$ of the above polynomial, we 
use the method proposed by \citet[]{roots}. 
The instantaneous frequency of each different component is derived from the following equation:
\begin{equation}
    f_m[n] = \Re \left[ \arg \left( \frac{\hat{z}_m[n]}{2\pi \Delta t} \right) \right].
\end{equation}
From the instantaneous frequency, we compute the local phase according to the following equation:
\begin{equation}
    \Phi_m[n] = 2\pi\sum_{k=0}^{n}f_m[k]\Delta t.
\end{equation}
Solving the following equation using 
regularized non-stationary regression method \cite[]{fomel2013}:
\begin{equation}
    \label{eq:eq13}
    x[n] = \sum_{m=1}^{M} \hat{A}_m[n]e^{j \Phi_m[n]} =
     \sum_{m=1}^{M} c_m[n].
\end{equation} 
Finally the narrow-band intrinsic mode functions $c_m[n]$ are computed based on equation~\ref{eq:eq13}

\section{EXAMPLES}
We use synthetic signals and real field data to test the proposed method.
\subsection{Benchmark examples}
We use a simple synthetic signal to test the proposed method. 
Figure~\ref{fig:hsig} is a synthetic signal from \citet[]{hou}. 
The three components of the signal are shown in Figure~\ref{fig:sigs}. 
Figure~\ref{fig:emd} and Figure~\ref{fig:nar} show the intrinsic mode functions 
extracted respectively by ensemble empirical mode decomposition and NPM methods. 
From the figures, we see that the NPM method 
accurately identifies the three components that the signal has. 
The intrinsic mode functions derived by the NPM are 
more smooth with respect to amplitudes and frequencies 
compared with the intrinsic mode functions obtained by ensemble empirical mode decomposition. 
For ensemble empirical mode decomposition, we repeat the empirical mode decomposition 25 times with different level of noises 
to generate the ensemble empirical mode decomposition results.
The time-frequency distributions of the input signal are the 
Hilbert transform of the intrinsic mode functions. 
Figure~\ref{fig:htf},~\ref{fig:htfemd} and~\ref{fig:htfnar} 
are respectively the time-frequency distributions using local attribute \cite[]{guochang2011}, 
ensemble empirical mode decomposition \cite[]{eemd} and the proposed method for the synthetic 
signal of Figure~\ref{fig:hsig}. 
Figure~\ref{fig:msig} is an another synthetic signal. 
Figure~\ref{fig:mtf},~\ref{fig:mtfemd} and~\ref{fig:mtfnar}
are respectively the time-frequency maps using local attribute 
\cite[]{guochang2011}, ensemble empirical mode decomposition \cite[]{eemd} and the proposed method. 
From the figures, we see that the energies compactly spread over the instantaneous 
frequencies for the ensemble empirical mode decomposition. 
However, the energies are not steadily distributed for the ensemble empirical mode decomposition. 
The proposed method provides a steady and compact energies distribution, which 
sharpen the time-frequency distribution.
\inputdir{hou}
\plot{hsig}{width=0.7\columnwidth}{Synthetic signal.}

\plot{sigs}{width=0.7\columnwidth} {Components of the synthetic 
signal of Figure~\ref{fig:hsig} .}
\plot{emd}{width=0.7\columnwidth} {Components of the synthetic 
signal of Figure~\ref{fig:hsig} using ensemble empirical mode decomposition.}
\plot{nar}{width=0.7\columnwidth} {Components of the synthetic 
signal of Figure~\ref{fig:hsig} using NPM.}
\multiplot{3}{htf,htfemd,htfnar}{width=0.7\columnwidth}
  {(a) Time-frequency map for synthetic signal of Figure~\ref{fig:hsig} using local attribute. 
  (b) Time-frequency map for synthetic signal of Figure~\ref{fig:hsig} using ensemble empirical mode decomposition. 
  (c) Time-frequency map  for synthetic signal of Figure~\ref{fig:hsig} using the proposed method.}

\inputdir{mirko}
\plot{msig}{width=0.7\columnwidth}{Synthetic signal.}
\multiplot{3}{mtf,mtfemd,mtfnar}{width=0.7\columnwidth}
  {(a) Time-frequency map for synthetic signal of Figure~\ref{fig:msig} using local attribute. 
  (b) Time-frequency map for synthetic signal of Figure~\ref{fig:msig} using ensemble empirical mode decomposition. 
  (c) Time-frequency map  for synthetic signal of Figure~\ref{fig:msig} using the proposed method.}

\subsection{Field examples}
Figure~\ref{fig:trace} is a seismic trace from marine survey. 
Figure~\ref{fig:stf},~\ref{fig:tfemd} and~\ref{fig:tfnar} are the time-frequency 
distributions of the trace using local attribute \cite[]{guochang2011}, 
ensemble empirical mode decomposition and the proposed method.
We can see that the energies distributions for ensemble empirical mode decomposition and the proposed method 
are much like each other. Both the ensemble empirical mode decomposition and the proposed method using 
the Hilbert transform of the intrinsic mode functions to represent the time-frequency distributions 
for the input signal. The results confirm that they both reveal the time-frequency
character of the input signal.

\inputdir{trace}
\plot{trace}{width=0.7\columnwidth}{Seismic trace from marine survey.}
\multiplot{3}{stf,tfemd,tfnar}{width=0.7\columnwidth}
  {(a) Time-frequency map for synthetic signal of Figure~\ref{fig:trace} using local attribute. 
  (b) Time-frequency map for synthetic signal of Figure~\ref{fig:trace} using ensemble empirical mode decomposition. 
  (c) Time-frequency map  for synthetic signal of Figure~\ref{fig:trace} using the proposed method.}
Low-frequency anomalies are often attributed to abnormally high attenuation
in gas-filled reservoirs and can be used as a hydrocarbon indicator \cite[]{Castagna2003}.
The mechanisms of low-frequency anomalies associated with hydrocarbon reservoirs 
are not clearly understood \cite[]{Ebrom2004, Kazemeini2009}.
Figure~\ref{fig:data} is a 2D field seismic data. 
Figure~\ref{fig:LC_TimeFreqSlice3} and~\ref{fig:LC_TimeFreqSlice6}, 
~\ref{fig:GasSmoothSliceEmd3} and~\ref{fig:GasSmoothSliceEmd6},
~\ref{fig:GasSmoothSliceNar3} and~\ref{fig:GasSmoothSliceNar6} are
the 30Hz and 60Hz constant frequency slices using local attribute, ensemble empirical mode decomposition and 
the proposed method.
From the above figures, we see that there is a low frequency anomaly 
in the upper left part of the data section indicated by the text boxes
"Gas?" for the ensemble empirical mode decomposition and the proposed methods, 
which may correspond to gas presentation.
\inputdir{vecta-stlc}
\plot{data}{width=0.7\columnwidth}
{2D seismic data section.}

\inputdir{vecta-nar}
\multiplot{6}{LC_TimeFreqSlice3,LC_TimeFreqSlice6,GasSmoothSliceEmd3,GasSmoothSliceEmd6,GasSmoothSliceNar3,GasSmoothSliceNar6}{width=0.45\columnwidth}
{
   (a) 30Hz slice time-frequency map using local attribute method.
   (b) 60Hz slice time-frequency map using local attribute method.
   (c) 30Hz slice time-frequency map using ensemble empirical mode decomposition method.
   (d) 60Hz slice time-frequency map using ensemble empirical mode decomposition method.
   (e) 30Hz slice time-frequency map of the proposed method. 
   (f) 60Hz slice time-frequency map of the proposed method. 
}

Figure~\ref{fig:TimeFreqCube_LC},~\ref{fig:TimeFreqCubeEmd} and~\ref{fig:TimeFreqCubeNar} are the 
full time-frequency cubes computed respectively using local attribute, ensemble empirical mode decomposition and the proposed 
methods. The main panels show constant frequency slices. 
The right hand side panels show the time-frequency maps of the 150th trace. 
The top panels show the time-frequency maps of 0.6s time-depth signal. 
From the right and top side panels we see that there are a lot of noise in the high 
frequency domain for the ensemble empirical mode decomposition and local attribute methods compared with the proposed method. 
\inputdir{vecta-nar}
\multiplot{4}{TimeFreqCube_LC,TimeFreqCubeEmd,TimeFreqCubeNar}{width=0.4\columnwidth}
{
   (a) Time-frequency cube using local attribute method. 
   (b) Time-frequency cube using ensemble empirical mode decomposition method. 
   (c) Time-frequency cube using NPM method. 
}

\section{CONCLUSION}
We proposed to compute the time-frequency map of an input signal based on 
NPM coupled with Hilbert spectral analysis. 
The proposed method is an empirical mode decomposition-like method, but using NPM 
to compute its intrinsic mode functions. Compared with the Fourier 
transform, the proposed method is data-driven and needs much less base functions 
to approximate the original signal. Since the NPM 
results an under-determined linear system, we use shaping regularization to 
regularize it. The regularization makes the intrinsic mode functions more 
smooth with respect to the amplitudes and frequencies 
compared with the intrinsic mode functions of the empirical mode decomposition. 
There are many time-frequency methods, which one is the best? This is a difficult 
question to answer. Methods are good for some type signals, maybe not good for 
other type signals.  

\citet[]{eemd_comp} pointed out that the complexity of empirical mode decomposition/ensemble
empirical mode decomposition is $41*\mathbf{N}_E*\mathbf{N}_S*n(\log_2n)=O(n\log n)$
, where $n$ is the data length and the parameters $\mathbf{N}_E$ and $\mathbf{N}_S$ are the
ensemble and sifting numbers respectively. For the non-stationary Prony method, the computation
complexity is mainly attributed to the polynomial zero-finding. We used the pseudo-zeros 
method to compute the pseudo-spectra of the associated balanced 
companion matrix \cite[]{roots}, which requires approximate $\mathbf{N}^3$ works, where $\mathbf{N}$
is the polynomial degree number. Therefore, the total computation complexity is 
$\mathbf{N}^3*\frac{n}{\mathbf{N}}=n*\mathbf{N}^2$, where $n$ is the data length.
In this paper, we choose $\mathbf{N}=5$, and therefore the total computation complexity is approximate 
$n*5^2 = O(n)$.
\section{ACKNOWLEDGMENTS}
We would like to thank Zhiguang Xue, and Junzhe Sun for their constructive 
suggestions. The first author thanks China University of Petroleum-Beijing 
for supporting his visiting to the Bureau of Economic Geology at UT Austin. 
The work is partially supported by Youth Science Foundation of China 
University of Petroleum at Beijing (Grant NO. 01jb0508) and Science Foundation 
of China University of Petroleum-Beijing (Grant NO.2462015YQ0604).

\newpage
\appendix
\section{Appendix A}
\section{Prony method}
Prony method can extract damped complex exponential signals 
from a given data by solving a set of linear equations 
\cite[]{prony,prony2003,ThomasPeter2013,Mitrofanov2015}. 
Assume the N complex data samples $x[1],x[2],\cdots x[N]$,
 we approximate the data by $M$ exponential functions:
\begin{equation}
    \label{eqa:eq0}
     x[n] \approx \sum_{k=1}^{M}A_k e^{(\alpha_k + j\omega_k)(n-1)\Delta t + j\phi_k},
\end{equation}
where $A_k$ is the amplitude, $\Delta t$ is the sampling period, 
$\alpha_k$ is the damping factor, $\omega_k$ is the angular frequency, $\phi_k$ is 
the initial phase.
If we let $h_k = A_k e^{j\phi_k}, z_k = e^{(\alpha_k + j\omega_k)\Delta t}$, 
we then derive the concise form below:
\begin{equation}
    \label{eqa:eq1}
     x[n] \approx \sum_{k=1}^{M}h_k z_k^{n-1}.
\end{equation}
The approximation problem above can be solved based on the error minimization:
\begin{equation}
    \label{eqa:eq2}
    \mathbf{min} \sum_{n=1}^{N} \left|\epsilon[n]\right|^2 = 
    \mathbf{min}\sum_{n=1}^{N}\left | x[n] - \sum_{k=1}^{M}h_k z_k^{n-1}\right|^2.
\end{equation}
This turns to be a nonlinear problem. It can be solved using 
Prony method that utilizes linear equation solutions.
If there are as many data samples as parameters of the approximation problem,
the above M equations\textbf{~\ref{eqa:eq1}} can be expressed:
\begin{equation}
    \label{eqa:eq3}
     x[n] = \sum_{k=1}^{M}h_k z_k^{n-1}.
\end{equation}
\textbf{~\ref{eqa:eq3}} can be written in a matrix form as below:
\begin{equation}
    \label{eqa:eq4}
    \left[ \begin{array}{cccc}
        z_1^0 & z_2^0 & \cdots & z_M^0\\
        z_1^1 & z_2^1 & \cdots & z_M^1\\
        \vdots & \vdots & & \vdots\\
        z_1^{M-1} & z_2^{M-1} & \cdots & z_M^{M-1}
    \end{array} \right]
    \left[ \begin{array}{c} h_1\\ h_2\\ \vdots \\ h_M
    \end{array} \right] =
    \left[ \begin{array}{c} x[1]\\ x[2]\\ \vdots \\ x[M]
    \end{array} \right].
\end{equation}
Prony proposed to define the polynomial that has the 
above $z_k, k=1,2, \cdots , M$ as its roots \cite[]{prony}:
\begin{equation}
    \label{eqa:eq5}
    \mathbf{P}(z)  = \prod_{k=1}^{M}(z-z_k).
\end{equation}
Equation \textbf{~\ref{eqa:eq5}} can be rewritten in the form below:
\begin{equation}
    \label{eqa:eq6}
    \mathbf{P}(z) = a_0 z^M + a_1 z^{M-1} + \cdots + a_{M-1}z + a_M.
\end{equation}
Shifting the index on equation\textbf{~\ref{eqa:eq3}} from $n$ to $n-m$
, and multiplying by parameter $a[m]$, then we derive:
\begin{equation}
    \label{eqa:eq7}
    \sum_{m=0}^{M}a_mx[n-m] = \sum_{k=1}^{M}h_kz_k^{n-M-1}
    \sum_{m=0}^M a[m]z_{k}^{M-m}.
\end{equation}
Notice $z_k, k=1,2,\cdots, M$ are roots of equation\textbf{~\ref{eqa:eq6}},
then equation\textbf{~\ref{eqa:eq7}} be written as:
\begin{equation}
    \label{eqa:eq8}
    \sum_{m=0}^{M}a_mx[n-m] = 0.
\end{equation}
Solving equation\textbf{~\ref{eqa:eq8}} for the polynomial coefficients.
In subsequent steps we compute the frequencies, damping factors 
and the phases according to Algorithm \ref{pro:pronyalgorithm}. 
After all the parameters are computed, we then
compute the components of the input signal. For details see 
Algorithm \ref{pro:pronyalgorithm} as follows:
\begin{algorithm}
    \caption{Prony method} 
    \label{pro:pronyalgorithm}
  \begin{algorithmic}[1]
      \State \textit{Find coefficients:} $\displaystyle a_k,k=1,2,\cdots,M \gets \sum_{m=0}^{M}a_mx[n-m] = 0.$
      \State \textit{Find roots:} $\displaystyle z_k,k=1,2,\cdots,M \gets \sum_{m=0}^{M}a_mz^{M-m}= 0.$
      \State \textit{Compute frequencies:} $\displaystyle
      \omega_k, k=1,2,\cdots,M \gets \Re\left\{\mathbf{arg}\left(\frac{z_k}{(k-1)\Delta t}\right)\right\}, k=1,2,\cdots,M.$
      \State \textit{Compute:} $\displaystyle
      A_k e^{\alpha_k (n-1)\Delta t + j\phi_k}, k=1,2,\cdots,M \gets 
      x[n] = \sum_{k=1}^{M}A_k e^{(\alpha_k + j\omega_k)(n-1)\Delta t + j\phi_k}.$
      \State \textit{Compute components:} $\displaystyle
      A_k e^{(\alpha_k + j\omega_k)(n-1)\Delta t + j\phi_k}.$
  \end{algorithmic}
\end{algorithm}

\section{non-stationary Prony method and shaping regularization}
The equation\textbf{~\ref{eqa:eq8}} can be rewritten as
\begin{equation}
    \label{eqa:eq9}
\sum_{m=1}^{M}\hat{a}_m x[n-m] = x[n].
\end{equation}
If the $\hat{a}_m, m=1,2,\cdots,M$ in equation\textbf{~\ref{eqa:eq9}} are time dependent, then we have
\begin{equation}
    \label{eqa:eq10}
    \sum_{m=1}^{M}\hat{a}_m[n] x[n-m] \approx x[n],
\end{equation}
which is under-determined. There many methods for the solving 
under-determined linear system. For example, \citet[]{tikhonov} 
used the regularization method for making the under-determined 
problem well-posed by adding constraints on the estimated model.
\subsection{Shaping regularization}
\citet[]{shape,lpf} introduces shaping regularization in inversion problem, 
which regularizes the under-determined linear system by mapping the model
to the space of acceptable models.
Consider a linear system given as $\mathbf{Fx = \mathbf{b}}$,
where $\mathbf{F}$ is the forward-modeling map, $\mathbf{x}$
is the model vector, and $\mathbf{b}$ is the data vector.
Tikhonov regularization method amounts to minimize the least square problem bellow 
\cite[]{tikhonov}:
\begin{equation}
    \label{eqa:eq11}
    \mathbf{min} \|\mathbf{Fx}-\mathbf{b}\|^2+\epsilon^2\|\mathbf{Dx}\|^2,
\end{equation}
where $\mathbf{D}$ is the regularization operator, and $\epsilon$ is a scalar parameter.
The solution for equation\textbf{~\ref{eqa:eq11}} is:
\begin{equation}
    \label{eqa:eq12}
    \mathbf{\hat{x}} = \left(\mathbf{F}^T\mathbf{F}+\epsilon^2 \mathbf{D}^T\mathbf{D}\right)^{-1}\mathbf{F}^T\mathbf{b},
\end{equation}
Where $\mathbf{\hat{x}}$ is the least square approximated of $\mathbf{x}$, $\mathbf{F}^T$
is the adjoint operator.
If the forward operator $\mathbf{F}$ is simply the identity operator, the solution of 
equation\textbf{~\ref{eqa:eq12}} is the form below:
\begin{equation}
    \label{eqa:eq13}
    \mathbf{\hat{x}} = \left(\mathbf{I}+\epsilon^2 \mathbf{D}^T\mathbf{D}\right)^{-1}\mathbf{b},
\end{equation}
which can be viewed as a smoothing process. If we let:
\begin{equation}
    \label{eqa:eq14}
    \mathbf{S} = \left(\mathbf{I}+\epsilon^2 \mathbf{D}^T\mathbf{D}\right)^{-1},
\end{equation}
or 
\begin{equation}
    \label{eqa:eq15}
    \epsilon^2 \mathbf{D}^T\mathbf{D} = \mathbf{S}^{-1} - \mathbf{I}.
\end{equation}
Substituting equation \textbf{~\ref{eqa:eq15}} into equation \textbf{~\ref{eqa:eq12}}
yields a solution by shaping regularization:
\begin{equation}
    \label{eqa:eq16}
    \mathbf{\hat{x}} = \left(\mathbf{F}^T\mathbf{F}+\mathbf{S}^{-1}-\mathbf{I}\right)^{-1}\mathbf{F}^T\mathbf{b}
     = \left[\mathbf{I} + \mathbf{S}\left(\mathbf{F}^T\mathbf{F}-\mathbf{I}\right)\right]^{-1}\mathbf{SF}^T\mathbf{b}.
\end{equation}
The forward operator $\mathbf{F}$ may has physical units that require scaling. Introducing 
scaling $\lambda$ into $\mathbf{F}$, equation \textbf{\ref{eqa:eq16}} be written as:
\begin{equation}
    \label{eqa:eq17}
    \mathbf{\hat{x}} = \left[\lambda^2\mathbf{I} + \mathbf{S}\left(\mathbf{F}^T\mathbf{F}-\lambda^2\mathbf{I}\right)\right]^{-1}\mathbf{SF}^T\mathbf{b}.
\end{equation}
If $\mathbf{S}=\mathbf{HH}^T$ with square and invertible $\mathbf{H}$. Equation \textbf{~\ref{eqa:eq17}} can be written as:
\begin{equation}
    \label{eqa:eq18}
    \mathbf{\hat{x}} = \mathbf{H}\left[\lambda^2\mathbf{I} + \mathbf{H}^T\left(\mathbf{F}^T\mathbf{F}-\lambda^2\mathbf{I}\right)\mathbf{H}\right]^{-1}\mathbf{H}^T\mathbf{F}^T\mathbf{b}.
\end{equation}
The conjugate gradient algorithm can be used for the solution of the equation\textbf{~\ref{eqa:eq18}}.

\subsection{Non-stationary Prony method}
Equation \textbf{\ref{eqa:eq10}} can be written as a matrix form:
\begin{equation}
    \label{eqa:eq19}
    \sum_{m=1}^{M}\mathbf{\hat{a}}_m(t)\mathbf{x}_m(t) \approx \mathbf{d}(t),
\end{equation}
where $\mathbf{d}(t) = \mathbf{x}(t)$, $\mathbf{x}_m(t) = \mathbf{x}(t-m\Delta t)$ is the time shift
of the input signal $\mathbf{x}(t)$ and $\mathbf{\hat{a}}_m(t)$ is the time-dependant coefficients.
We solve the under-determined linear system by using the shaping regularization method.
The solution is the form below:
\begin{equation}
    \label{eqa:eq20}
    \mathbf{a = F^{-1}\eta},
\end{equation}
where $\mathbf{a}$ is a vector of $\hat{a}(t)$, the elements of vector $\mathbf{\eta}$ is:
\begin{equation}
    \label{eqa:eq21}
    \mathbf{\eta}_i(t) = \mathbf{S}\left[\mathbf{x}_i^*(t)\mathbf{d}(t)\right],
\end{equation}
the elements of the matrix $\mathbf{F}$ is:
\begin{equation}
    \label{eqa:eq22}
    \mathbf{F}_{ij}= \sigma^2 \mathbf{\delta}_{ij} + S[\mathbf{x}_i^*(t)\mathbf{x}_j(t) - \sigma^2 \mathbf{\delta}_{ij}]
\end{equation}
where $\sigma$ is the regularization parameter, $\mathbf{S}$ is a shaping operator,
and $\mathbf{x}_i^*(t)$ stands for the complex conjugate of $\mathbf{x}_i(t)$.
We can use the conjugate gradient method to find the solution of the linear system.
The NPM \cite[]{fomel2013} can be summarized as follows:
\begin{algorithm}
    \caption{non-stationary Prony method} 
    \label{pro:nopronyalgorithm}
  \begin{algorithmic}[1]
      \State \textit{Find time dependent coefficients using auto-regression method:} 
        $\displaystyle \hat{a}_k[n],k=1,\cdots,M \gets \sum_{m=0}^{M}\hat{a}_m[n]x[n-m] = 0.$
      \State \textit{Find time dependent roots:} 
         $\displaystyle \hat{z}_k[n],k=1,2,\cdots,M \gets \sum_{m=0}^{M}\hat{a}_m[n]z^{M-m}= 0.$
      \State \textit{Compute time dependent frequencies:}
         $\displaystyle \hat{\omega}_k[n], k=1,\cdots,M\gets\Re\left\{\mathbf{arg}\left(\frac{\hat{z}_k[n]}{\Delta t}\right)\right\}, k=1,\cdots,M.$
      \State \textit{Compute the time dependent phase:} 
       $\displaystyle \hat{\phi}_k[n] = \sum_{k=0}^{n}\hat{\omega}_k[n]\Delta t.$ 
      \State \textit{Compute components using auto-regression method:} 
        $\displaystyle \hat{c}_m[n], m=1,\cdots M \gets \mathbf{x}[n] \approx \sum_{m=1}^M \hat{A}_m[n]e^{j\hat{\phi}_m[n]}=\sum_{m=1}^M\hat{c}_m[n]$
  \end{algorithmic}
\end{algorithm}
After we decompose the input signal into narrow-band components, 
we compute the time-frequency distribution of the input signal 
using the Hilbert transform of the intrinsic mode functions.

\newpage
\bibliographystyle{seg}
\bibliography{npm}

