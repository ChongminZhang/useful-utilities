\chapter{Non-Dimensionalization Conventions}\label{cha:Non-Dimensionalization-Conventions}

All physical parameters used are non-dimensionalized internally in the code
in order to work in a reference Earth of radius 1. In Table~{\small \ref{table:conventions} in the right column are the values by which
we divide the parameters of the left column internally to perform the calculations.
The values output and saved by the code (seismograms, sensitivity kernels...) are then
scaled back to the right physical dimensions before being saved.
}%
\begin{table}[ht]
\noindent
\begin{centering}
{\small }\begin{tabular}{|c|c|}
\hline
quantity (units)  & divided internally by \\
\hline
distance (m)  & \texttt{R\_EARTH} \\
time (s)  & $1/\sqrt{\texttt{PI}\times\texttt{GRAV}\times\texttt{RHOAV}}$ \\
density (kg/m$^{3}$)  & \texttt{RHOAV} \\
\hline
\end{tabular}
\par
\end{centering}{\small \par}

\caption{Non-dimensionalization convention employed internally by the code. The constants \texttt{R\_EARTH}
(the radius of the Earth), \texttt{PI} (the number $\pi$), \texttt{GRAV}
(the universal gravitational constant), and \texttt{RHOAV} (the Earth's
average density) are defined in the \texttt{constants.h} file. }

{\small \label{table:conventions} }
\end{table}
{\small \par}


