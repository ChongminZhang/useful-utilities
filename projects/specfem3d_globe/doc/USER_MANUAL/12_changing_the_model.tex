\chapter{Changing the Model}\label{cha:-Changing-the}

In this section we explain how to change the crustal, mantle, or inner
core models. These changes involve contributing specific subroutines
that replace existing subroutines in the \texttt{SPECFEM3D\_GLOBE}
package.


\section{Changing the Crustal Model}\label{sec:Changing-the-Crustal}

The 3D crustal model Crust2.0 \citep{BaLaMa00} is superimposed onto
the mesh by the subroutine \texttt{model\_crust}~\\
\texttt{.f90}. To accomplish this, the flag \texttt{CRUSTAL}, set
in the subroutine \texttt{get\_model\_parameters.f90}, is used
to indicate a 3D crustal model. When this flag is set to \texttt{.true.},
the crust on top of the 1D reference model (PREM, IASP91, or AK135F\_NO\_MUD)
is removed and replaced by extending the mantle. The 3D crustal model
is subsequently overprinted onto the crust-less 1D reference model.
The call to the 3D crustal routine is of the form
\begin{verbatim}
call model_crust(lat,lon,r,vp,vs,rho,moho,foundcrust,CM_V,elem_in_crust)
\end{verbatim}

\noindent
Input to this routine consists of:
\begin{description}
\item [{\texttt{lat}}] Latitude in degrees.
\item [{\texttt{lon}}] Longitude in degrees.
\item [{\texttt{r}}] Non-dimensionalized radius ($0<\texttt{r}<1$).
\end{description}
%
Output from the routine consists of:
\begin{description}
\item [{\texttt{vp}}] Non-dimensionalized compressional wave speed at location
(\texttt{lat},\texttt{lon},\texttt{r}).
\item [{\texttt{vs}}] Non-dimensionalized shear wave speed.
\item [{\texttt{rho}}] Non-dimensionalized density.
\item [{\texttt{moho}}] Non-dimensionalized Moho depth.
\item [{\texttt{found\_crust}}] Logical that is set to \texttt{.true.}
only if crust exists at location (\texttt{lat},\texttt{lon},\texttt{r}),
i.e., \texttt{.false.} for radii \texttt{r} in the mantle. This flags
determines whether or not a particular location is in the crust and,
if so, what parameters to assign to the mesh at this location.
\item [{\texttt{CM\_V}}] Fortran structure that contains the parameters,
variables and arrays that describe the model.
\item [{\texttt{elem\_in\_crust}}] Logical that is used to force the
routine to return crustal values, even if the location would
be below the crust.
\end{description}
All output needs to be non-dimensionalized according to the convention
summarized in Appendix~\ref{cha:Non-Dimensionalization-Conventions}.
You can replace this subroutine by your own routine \textit{provided
you do not change the call structure of the routine}, i.e., the new
routine should take exactly the same input and produce the required,
properly non-dimensionalized output.

Part of the file \texttt{model\_crust.f90} is the subroutine \texttt{model\_crust\_broadcast}.
The call to this routine takes argument \texttt{CM\_V} and is used
to once-and-for-all read in the databases related to Crust2.0 and
broadcast the model to all parallel processes. If
you replace the file \texttt{model\_crust.f90} with your own implementation,
you \textit{must} provide a \texttt{model\_crust\_broadcast} routine,
even if it does nothing. Model constants and variables read by the
routine \texttt{model\_crust\_broadcast} are passed to the subroutine
\texttt{read\_crust\_model} through the structure \texttt{CM\_V}.
An alternative crustal model could use the same construct. Please
feel free to contribute subroutines for new models and send them to
us so that they can be included in future releases of the software.

\begin{quote}
\textbf{NOTE:} If you decide to create your own version of file \texttt{model\_crust.f90},
you must add calls to \texttt{MPI\_BCAST} in the subroutine \texttt{model\_crust\_broadcast}
after the call to the \texttt{read\_crust\_model} subroutine that
reads the isotropic mantle model once and for all in the mesher. This
is done in order to read the (potentially large) model data files
on the master node (which is the processor of rank 0 in our code)
only and then send a copy to all the other nodes using an MPI broadcast,
rather than using an implementation in which all the nodes would read
the same model data files from a remotely-mounted home file system,
which could create a bottleneck on the network in the case of a large
number of nodes. For example, in the current call to that routine
from \texttt{model\_crust.f90,} we write:

{\footnotesize
\begin{verbatim}
! the variables read are declared and stored in structure CM_V
  if(myrank == 0) call read_crust_model(CM_V)

! broadcast the information read on the master to the nodes
  call MPI_BCAST(CM_V%thlr,NKEYS_CRUST*NLAYERS_CRUST,MPI_DOUBLE_PRECISION,0,MPI_COMM_WORLD,ier)
  call MPI_BCAST(CM_V%velocp,NKEYS_CRUST*NLAYERS_CRUST,MPI_DOUBLE_PRECISION,0,MPI_COMM_WORLD,ier)
  call MPI_BCAST(CM_V%velocs,NKEYS_CRUST*NLAYERS_CRUST,MPI_DOUBLE_PRECISION,0,MPI_COMM_WORLD,ier)
  call MPI_BCAST(CM_V%dens,NKEYS_CRUST*NLAYERS_CRUST,MPI_DOUBLE_PRECISION,0,MPI_COMM_WORLD,ier)
  call MPI_BCAST(CM_V%abbreviation,NCAP_CRUST*NCAP_CRUST,MPI_CHARACTER,0,MPI_COMM_WORLD,ier)
  call MPI_BCAST(CM_V%code,2*NKEYS_CRUST,MPI_CHARACTER,0,MPI_COMM_WORLD,ier)
\end{verbatim}
}
\end{quote}

\section{Changing the Mantle Model}\label{sec:Changing-the-Mantle}

This section discusses how to change isotropic and anisotropic 3D
mantle models. Usually such changes go hand-in-hand with changing
the 3D crustal model.


\subsection{Isotropic Models}\label{sub:Isotropic-Models}

The 3D mantle model S20RTS \citep{RiVaWo99} is superimposed onto
the mantle mesh by the subroutine \texttt{model\_s20rts.f90}. The
call to this subroutine is of the form
\begin{verbatim}
call model_s20rts(radius,theta,phi,dvs,dvp,drho,D3MM_V)
\end{verbatim}

\noindent
Input to this routine consists of:
\begin{description}
\item [{\texttt{radius}}] Non-dimensionalized radius ($\texttt{RCMB/R\_ EARTH}<\texttt{r}<\texttt{RMOHO/R\_ EARTH}$;
for a given 1D reference model, the constants \texttt{RCMB} and \texttt{RMOHO}
are set in the \texttt{\small get\_model\_parameters}\texttt{.f90}
file). The code expects the isotropic mantle model to be defined between
the Moho (with radius \texttt{RMOHO} in m) and the core-mantle boundary
(CMB; radius \texttt{RCMB} in m) of a 1D reference model. When a 3D
crustal model is superimposed, as will usually be the case, the 3D
mantle model is stretched to fill any potential gap between the radius
of the Moho in the 1D reference model and the Moho in the 3D crustal
model. Thus, when the Moho in the 3D crustal model is shallower than
the Moho in the reference model, e.g., typically below the oceans,
the mantle model is extended to fill this gap.
\item [{\texttt{theta}}] Colatitude in radians.
\item [{\texttt{phi}}] Longitude in radians.
\end{description}
%
Output from the routine are the following non-dimensional perturbations:
\begin{description}
\item [{\texttt{dvs}}] Relative shear-wave speed perturbations $\delta\beta/\beta$
at location (\texttt{radius},\texttt{theta},\texttt{phi}).
\item [{\texttt{dvp}}] Relative compressional-wave speed perturbations
$\delta\alpha/\alpha$.
\item [{\texttt{drho}}] Relative density perturbations $\delta\rho/\rho$.
\item [{\texttt{D3MM\_V}}] Fortran structure that contains the parameters,
variables and arrays that describe the model.
\end{description}
You can replace the \texttt{model\_s20rts.f90} file with your own
version \textit{provided you do not change the call structure of the
routine}, i.e., the new routine should take exactly the same input
and produce the required relative output.

Part of the file \texttt{model\_s20rts.f90} is the subroutine
\texttt{model\_s20rts\_broadcast}.
The call to this routine takes argument\texttt{ D3MM\_V} and is used
to once-and-for-all read in the databases related to S20RTS. If you
replace the file \texttt{model\_s20rts.f90} with your own implementation,
you \textit{must} provide a \texttt{model\_s20rts\_broadcast} routine,
even if it does nothing. Model constants and variables read by the
routine \texttt{model\_s20rts\_broadcast} are passed to the subroutine
\texttt{read\_model\_s20rts} through the structure \texttt{D3MM\_V.}
An alternative mantle model should use the same construct.

\begin{quote}
\textbf{NOTE:} If you decide to create your own version of file \texttt{model\_s20rts.f90},
you must add calls to \texttt{MPI\_BCAST} in the subroutine \texttt{model\_s20rts\_broadcast}
after the call to the \texttt{read\_model\_s20rts} subroutine that
reads the isotropic mantle model once and for all in the mesher. This
is done in order to read the (potentially large) model data files
on the master node (which is the processor of rank 0 in our code)
only and then send a copy to all the other nodes using an MPI broadcast,
rather than using an implementation in which all the nodes would read
the same model data files from a remotely-mounted home file system,
which could create a bottleneck on the network in the case of a large
number of nodes. For example, in the current call to that routine
from \texttt{model\_s20rts.f90,} we write:

{\footnotesize
\begin{verbatim}
! the variables read are declared and stored in structure D3MM_V
  if(myrank == 0) call read_model_s20rts(D3MM_V)

! broadcast the information read on the master to the nodes
  call MPI_BCAST(D3MM_V%dvs_a,(NK+1)*(NS+1)*(NS+1),MPI_DOUBLE_PRECISION,0,MPI_COMM_WORLD,ier)
  call MPI_BCAST(D3MM_V%dvs_b,(NK+1)*(NS+1)*(NS+1),MPI_DOUBLE_PRECISION,0,MPI_COMM_WORLD,ier)
  call MPI_BCAST(D3MM_V%dvp_a,(NK+1)*(NS+1)*(NS+1),MPI_DOUBLE_PRECISION,0,MPI_COMM_WORLD,ier)
  call MPI_BCAST(D3MM_V%dvp_b,(NK+1)*(NS+1)*(NS+1),MPI_DOUBLE_PRECISION,0,MPI_COMM_WORLD,ier)
  call MPI_BCAST(D3MM_V%spknt,NK+1,MPI_DOUBLE_PRECISION,0,MPI_COMM_WORLD,ier)
  call MPI_BCAST(D3MM_V%qq0,(NK+1)*(NK+1),MPI_DOUBLE_PRECISION,0,MPI_COMM_WORLD,ier)
  call MPI_BCAST(D3MM_V%qq,3*(NK+1)*(NK+1),MPI_DOUBLE_PRECISION,0,MPI_COMM_WORLD,ier)
\end{verbatim}
}
\end{quote}


\subsection{Anisotropic Models}\label{sub:Anisotropic-Models}

Three-dimensional anisotropic mantle models may be superimposed on
the mesh based upon the subroutine
\begin{verbatim}
model_aniso_mantle.f90
\end{verbatim}
The call to this subroutine is of the form
\begin{verbatim}
call model_aniso_mantle(r,theta,phi,rho, &
         c11,c12,c13,c14,c15,c16,c22,c23,c24,c25,c26, &
         c33,c34,c35,c36,c44,c45,c46,c55,c56,c66,AMM_V)
\end{verbatim}

\noindent
Input to this routine consists of:
\begin{description}
\item [{\texttt{r}}] Non-dimensionalized radius ($\texttt{RCMB/R\_ EARTH}<\texttt{r}<\texttt{RMOHO/R\_ EARTH}$;
for a given 1D reference model, the constants \texttt{RCMB} and \texttt{RMOHO}
are set in the \texttt{\small get\_model\_parameters}\texttt{.f90}
file). The code expects the anisotropic mantle model to be defined
between the Moho and the core-mantle boundary (CMB). When a 3D crustal
model is superimposed, as will usually be the case, the 3D mantle
model is stretched to fill any potential gap between the radius of
the Moho in the 1D reference model and the Moho in the 3D crustal
model. Thus, when the Moho in the 3D crustal model is shallower than
the Moho in the reference model, e.g., typically below the oceans,
the mantle model is extended to fill this gap.
\item [{\texttt{theta}}] Colatitude in radians.
\item [{\texttt{phi}}] Longitude in radians.
\end{description}
%
Output from the routine consists of the following non-dimensional
model parameters:
\begin{description}
\item [{\texttt{rho}}] Non-dimensionalized density $\rho$.
\item [{\texttt{c11},}] \textbf{$\cdots$,} \texttt{\textbf{c66}} 21 non-dimensionalized
anisotropic elastic parameters.
\item [{\texttt{AMM\_V}}] Fortran structure that contains the parameters,
variables and arrays that describe the model.
\end{description}
You can replace the \texttt{model\_aniso\_mantle.f90} file by
your own version \textit{provided you do not change the call structure
of the routine}, i.e., the new routine should take exactly the same
input and produce the required relative output. Part of the file \texttt{model\_aniso\_mantle.f90}
is the subroutine \texttt{model\_aniso\_mantle\_broadcast}. The call to
this routine takes argument \texttt{AMM\_V} and is used to once-and-for-all
read in the static databases related to the anisotropic model. When
you choose to replace the file \texttt{model\_aniso\_mantle.f90}
with your own implementation you \textit{must} provide a \texttt{model\_aniso\_mantle\_broadcast}
routine, even if it does nothing. Model constants and variables read
by the routine \texttt{model\_aniso\_mantle\_broadcast} are passed through the
structure \texttt{AMM\_V}. An alternative anisotropic mantle model
should use the same construct.

\begin{quote}
\textbf{NOTE:} If you decide to create your own version of file \texttt{model\_aniso\_mantle.f90},
you must add calls to \texttt{MPI\_BCAST} in file \texttt{model\_aniso\_mantle.f90}
after the call to the \texttt{read\_aniso\_mantle\_model} subroutine
that reads the anisotropic mantle model once and for all in the mesher.
This is done in order to read the (potentially large) model data files
on the master node (which is the processor of rank 0 in our code)
only and then send a copy to all the other nodes using an MPI broadcast,
rather than using an implementation in which all the nodes would read
the same model data files from a remotely-mounted home file system,
which could create a bottleneck on the network in the case of a large
number of nodes. For example, in the current call to that routine
from \texttt{model\_aniso\_mantle.f90,} we write:

{\footnotesize
\begin{verbatim}
! the variables read are declared and stored in structure AMM_V
  if(myrank == 0) call read_aniso_mantle_model(AMM_V)

! broadcast the information read on the master to the nodes
  call MPI_BCAST(AMM_V%npar1,1,MPI_INTEGER,0,MPI_COMM_WORLD,ier)
  call MPI_BCAST(AMM_V%beta,14*34*37*73,MPI_DOUBLE_PRECISION,0,MPI_COMM_WORLD,ier)
  call MPI_BCAST(AMM_V%pro,47,MPI_DOUBLE_PRECISION,0,MPI_COMM_WORLD,ier)
\end{verbatim}
}
\end{quote}

Rotation of the anisotropic tensor elements from one coordinate system
to another coordinate system may be accomplished based upon the subroutine
\texttt{rotate\_aniso\_tensor}. Use of this routine requires understanding
the coordinate system used in \texttt{SPECFEM3D\_GLOBE}, as discussed
in Appendix~\ref{cha:Reference-Frame-Convention}.


\subsection{Point-Profile Models}\label{sub:Point-Profile-Models}

In order to facilitate the use of your own specific mantle model, you can choose \texttt{PPM} as model in the \texttt{DATA/Par\_file} file
and supply your own model as an ASCII-table file. These generic models are given as depth profiles at a specified lon/lat location
and a perturbation (in percentage) with respect to the shear-wave speed values from PREM. The ASCII-file should have a format like:

{\footnotesize
\begin{verbatim}
#lon(deg), lat(deg), depth(km), Vs-perturbation wrt PREM(%), Vs-PREM (km/s)
 -10.00000       31.00000       40.00000      -1.775005       4.400000
 -10.00000       32.00000       40.00000      -1.056823       4.400000
 ...
\end{verbatim}
}
\noindent
where the first line is a comment line and all following ones are specifying the Vs-perturbation at a lon/lat location and a given depth.
The last entry on each line is specifying the absolute value of Vs (however this value is only given as a supplementary information
and not used any further). The background model is PREM with a transverse isotropic layer between Moho and 220~km depth.
The specified Vs-perturbations are added as isotropic perturbations. Please see the file \texttt{DATA/PPM/README}
for more informations how to setup the directory \texttt{DATA/PPM} to use your own ASCII-file.



To change the code behavior of these PPM-routines, please have a look at the implementation in the source code
file \texttt{model\_ppm.f90} and set the flags and scaling factors as needed for your purposes.
Perturbations in density and Vp may be scaled to the given Vs-perturbations with constant scaling factors by setting the appropriate
values in this source code file. In case you want to change the format of the input ASCII-file, see more details in the Appendix \ref{cha:Troubleshooting}.




\section{Anelastic Models}\label{sec:Anelastic-Models}

Three-dimensional anelastic (attenuation) models may be superimposed
onto the mesh based upon your subroutine .
\texttt{model\_atten3D.f90}.
The call to this routine would be as follows
\begin{verbatim}
call model_atten3D(radius, colatitude, longitude, Qmu, QRFSI12_Q, idoubling)
\end{verbatim}

\noindent
Input to this routine consists of:
\begin{description}
\item [{\texttt{radius}}] scaled radius of the earth: $0\,(\mathrm{center})<=r\,<=1$(surface)
\item [{\texttt{latitude}}] Colatitude in degrees: $0^{\circ}<=\theta<=180^{\circ}$
\item [{\texttt{longitude}}] Longitude in degrees: $-180^{\circ}<=\phi<=180^{\circ}$
\item [{\texttt{QRFSI12\_Q}}] Fortran structure that contains the parameters,
variables and arrays that describe the model
\item [{\texttt{idoubling}}] value of the doubling index flag in each radial region of the mesh
\end{description}
%
Output to this routine consists of:
\begin{description}
\item [{\texttt{Qmu}}] Shear wave quality factor: $0<Q_{\mu}<5000$
\end{description}
A 3-D attenuation model QRFSI12 \citep{DaEkDz08}
is provided, as well as 1-D models with a PREM and a 1DREF
attenuation structure. By default the PREM attenuation model is taken,
using the routine \texttt{model\_attenuation\_1D\_PREM},
found in \texttt{model\_attenuation.f90}.

To create your own three-dimensional attenuation model, you add your model
using a routine like the ~\\
\texttt{model\_atten3D\_QRFSI12} subroutine
and the example routine above as a guide and replace the call in file
\texttt{meshfem3D\_models.f90} to your own subroutine.

Note that the resolution and maximum value of anelastic models are
truncated. This speeds the construction of the standard linear solids
during the meshing stage. To change the resolution, currently at one
significant figure following the decimal, or the maximum value (5000),
consult \texttt{constants.h}. In order to prevent unexpected results,
quality factors $Q_{\mu}$ should never be equal to 0 outside of the
inner core.



