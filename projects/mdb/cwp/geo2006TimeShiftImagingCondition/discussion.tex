\section{Discussion}
As discussed in one of the preceding sections, time-shift 
gathers consist of linear events with slopes corresponding
to the local migration velocity.
In contrast, space-shift gathers consist of events focused 
at $\hh=0$.
Those events can be mapped to the angle-domain using 
transformations \ren{angX} and \ren{angT}, respectively.

In order to understand the angle-domain mapping, 
we consider a simple synthetic in which we model
common-image gathers corresponding to incidence at
a particular angle.
The experiment is depicted in Figure~\ref{fig:ttest}
for time-shift imaging, and in Figure~\ref{fig:htest}
for space-shift imaging.
For this experiment, the sampling parameters are the following:
$\Delta z=0.01$~km,
$\Delta h=0.02$~km, and
$\Delta \tau=0.01$~s.

A reflection event at a single angle of incidence 
maps in common-image gathers as a line of a given slope.
The left panels in Figures~\ref{fig:ttest} and \ref{fig:htest}
show $3$ cases, corresponding to angles of
$0^\circ$, $20^\circ$ and $40^\circ$.
Since we want to analyze how such events map to angle, 
we subsample each line to $5$ selected samples lining-up
at the correct slope.

The middle panels in Figures~\ref{fig:ttest} and \ref{fig:htest}
show the data in the left panels after slant-stacking
in $z-\tt$ or $z-h$ panels, respectively.
Each individual sample from the common-image gathers maps
in a line of a different slope intersecting in a point.
For example, normal incidence in a time-shift gather
maps at the migration velocity $\nu=2$~km/s 
(Figure~\ref{fig:ttest} top row, middle panel), and
normal incidence in a space-shift gather maps at
slant-stack parameter $\tan \t=0$.

The right panels in Figures~\ref{fig:ttest} and \ref{fig:htest}
show the data from the middle panels after mapping to angle
using \reqs{angT} and \ren{angX}, respectively.
All lines from the slant-stack panels map into curves
that intersect at the angle of incidence.

We note that all curves for the time-shift angle-gathers
have zero curvature at normal incidence.
Therefore, the resolution of the time-shift mapping around 
normal incidence is lower than the corresponding
space-shift resolution. 
However, the storage and computational cost of time-shift 
imaging is smaller than the cost of equivalent space-shift 
imaging. 
The choice of the appropriate imaging condition depends on 
the imaging objective and on the trade-off 
between the cost and the desired resolution.

% ------------------------------------------------------------
\inputdir{icomp}
% ------------------------------------------------------------

% ------------------------------------------------------------
\plot{ttest}{width=6.0in}
{Image-gather formation using time-shift imaging.
Each row depicts an event at 
$0 ^\circ$ (top),
$20^\circ$ (middle), and
$40^\circ$ (bottom).
Three columns correspond to
subsampled time-shift gathers (left),
slant-stacked gathers (middle), and
angle-gathers (right).}
% ------------------------------------------------------------

% ------------------------------------------------------------
\plot{htest}{width=6.0in}
{Image-gather formation using space-shift imaging.
Each row depicts an event at 
$0 ^\circ$ (top),
$20^\circ$ (middle), and
$40^\circ$ (bottom).
Three columns correspond to
subsampled space-shift gathers (left),
slant-stacked gathers (middle), and
angle-gathers (right).}
% ------------------------------------------------------------
