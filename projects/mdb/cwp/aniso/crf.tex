\section{CURVILINEAR REFLECTOR}
%%%%%%%%%%%%%%%%%%%%%%%%%%%%%

Reflector curvature can also cause nonhyperbolic reflection moveout. 
In isotropic media, local dip of the reflector influences the normal-moveout velocity, while reflector 
curvature introduces nonhyperbolic moveout. When overlaying layer is also anisotropic, both hyperbolic and nonhyperbolic moveouts for reflections from curved reflectors also become 
functions of the anisotropic parameters.

\subsection{Curved reflector beneath isotropic medium}
%%%%%%%%%%%%%%%%%%%%%%
If the reflector has the shape of a dipping plane beneath a homogeneous
isotropic medium, the reflection moveout in the dip direction is a hyperbola 
\cite[]{GEO36-03-05100516}
\begin{equation}
t^2(l) = t_0^2 + {l^2 \over V_n^2}\;.
\label{eqn:ttplane}
\end{equation}
Here 
\begin{eqnarray}
t_0 & = & {{2\,L} \over V_z}\;,
\label{eqn:t0plane} \\
V_n & = & {{V_z} \over {\cos{\alpha}}}\;, 
\label{eqn:vnplane}
\end{eqnarray}
$L$ is the length of the zero-offset ray, and $\alpha$ is the
reflector dip. Formula (\ref{eqn:ttplane}) is inaccurate if the
reflector is both dipping and curved. The Taylor series expansion for
moveout in this case has the form of equation
(\ref{eqn:taylor}), with coefficients \cite[]{fomel}
\begin{eqnarray}
a_2 & = & {{\cos^2{\alpha}\,\sin^2{\alpha}\,G} 
\over {4\,V_z^2\,L^2}}\;,
\label{eqn:a2plane} \\
a_3 & = & - {{\cos^2{\alpha}\,\sin^2{\alpha}\,G^2} 
\over {16\,V_z^2\,L^4}}\,\left(\cos{2 \alpha} + \sin{2 \alpha}\,
{{G\,K_3} \over {K_2^2\,L}}\right)\;,
\label{eqn:a3plane}
\end{eqnarray}
where 
\begin{equation}
G={{K_2\,L} \over {1 + K_2\,L}}\;,
\label{eqn:g}
\end{equation}
$K_2$ is the reflector curvature [defined by equation~(\ref{eqn:k2})]
at the reflection point of the zero-offset ray, and $K_3$ is the
third-order curvature [equation~(\ref{eqn:k3})]. If the reflector has
an explicit representation $z=z(x)$, then the parameters in equations
(\ref{eqn:a2plane}) and (\ref{eqn:a3plane}) are
\begin{eqnarray}
\tan{\alpha} & = & {{d z} \over {d x}}\;,
\label{eqn:dip} \\
L & = & {z \over {\cos{\alpha}}}\;,
\label{eqn:l} \\
K_2 & = & {{d^2 z} \over {d x^2}}\,\cos^3{\alpha}\;,
\label{eqn:k2} \\
K_3 & = & {{d^3 z} \over {d x^3}}\,\cos^4{\alpha} - 
3\,K_2^2\,\tan{\alpha}\;.
\label{eqn:k3}
\end{eqnarray}
\par
Keeping only three terms in the Taylor series leads to the approximation
\begin{equation}
t^2(l) = t_0^2 + {l^2 \over V_n^2} + {{G\,l^4 \tan^2{\alpha}} \over 
{V_n^2\,\left(V_n^2 t_0^2 + G\,l^2\right)}}\;,
\label{eqn:CRapprox}
\end{equation}
where we included the denominator in the third term to ensure that the
traveltime behavior at large offsets satisfies the obvious limit
\begin{equation}
\lim_{l \rightarrow \infty} t^2(l) = {l^2 \over V_z^2}\;.
\label{eqn:CRlimit}
\end{equation}
As indicated by equation (\ref{eqn:k2}), the sign of the curvature $K_2$ is
positive if the reflector is locally convex (i.e., an anticline-type). 
The sign of $K_2$ is negative for concave, syncline-type reflectors. 
%??? Am I correct saying that ? --
Therefore, the coefficient $G$
expressed by equation (\ref{eqn:g}) and, likewise, the nonhyperbolic term in
(\ref{eqn:CRapprox}) can take both positive and negative values. This means
that only for concave reflectors in homogeneous media do nonhyperbolic
moveouts resemble those in VTI and vertically heterogeneous media.
Convex surfaces produce nonhyperbolic moveout with the opposite sign.
Clearly, equation (\ref{eqn:CRapprox}) is not accurate for strong
negative curvatures $K_2 \approx -1 / L$, 
%??? Is this correct? --
which cause focusing of the
reflected rays and triplications of the reflection traveltimes.
\par
In order to evaluate the accuracy of approximation (\ref{eqn:CRapprox}), we
can compare it with the exact expression for a point diffractor, which 
is formally a convex reflector with an
infinite curvature. The exact expression for normal moveout 
in the present notation is
\begin{equation}
t(l) = {{\sqrt{z^2 + {(z\,\tan{\alpha} - l/2)^2}} +
\sqrt{z^2 + {(z\,\tan{\alpha} + l/2)^2}}} \over {V_z}}\;,
\label{eqn:ttdiff}
\end{equation}
where $z$ is the depth of the diffractor, and $\alpha$ is the angle from 
vertical of the
zero-offset ray. Figure \ref{fig:nmoerr} shows the relative error of
approximation (\ref{eqn:CRapprox}) as a function of the ray angle for
offset $l$ twice the diffractor depth $z$. The
maximum error of about 1\% occurs at $\alpha \approx 50^{\circ}$.
We can expect equation (\ref{eqn:CRapprox}) to be even more accurate
for reflectors with smaller curvatures.

\plot{nmoerr}{height=2in}{Relative error $e$ of the
nonhyperbolic moveout approximation~\protect(\ref{eqn:CRapprox}) for a 
point diffractor. The error corresponds to offset $l$ twice the diffractor depth $z$ and is plotted against the angle from vertical $\alpha$ of
the zero-offset ray.}
%??? -- Am I right?
%??? -- Remove the words ``Relative error'' from the plot.
%??? -- Replace ``angle'' with ``$\alpha$''.
%}

\inputdir{XFig}

\subsection{Curved reflector beneath homogeneous VTI medium} 
%%%%%%%%%%%%%%%%%%%%%%%%%%%%%%%%%%%%%%%%%%%%%%%%%%%%
For a dipping curved reflector in a homogeneous VTI medium,
the ray trajectories of the incident and reflected waves are straight,
but the location of the reflection point is no longer controlled by
the isotropic laws. To obtain analytic expressions in this
model, we use the theoremthat connects the derivatives of the
common-midpoint traveltime with the derivatives of the one-way
traveltimes for an imaginary wave originating at the reflection point
of the zero-offset ray. This theorem, introduced for the
second-order derivatives by \cite{chergr},
is usually called the normal incidence point (NIP) theorem
\cite[]{hubkrey,GEO48-08-10511062}. Although the original proof did not
address anisotropy, it is applicable to anisotropic media because it is based
on the fundamental Fermat's principle. The ``normal incidence'' point
in anisotropic media is the point of incidence for the zero-offset ray
(which is, in general, not normal to the reflector). In Appendix~A, we
review the NIP theorem, as well as its extension to the high-order
traveltime derivatives \cite[]{fomel}.
\par
Two important equations derived in Appendix~A are:
\begin{eqnarray}
\left.{{\partial^2 t} \over {\partial l^2}}\right|_{l=0} & = &
\frac{1}{2}\,{{\partial^2 T} \over {\partial y^2}}\;,
\label{eqn:NIP2} \\
\left.{{\partial^4 t} \over {\partial l^4}}\right|_{l=0} & = &
\frac{1}{8}\,{{\partial^4 T} \over {\partial y^4}}-
\frac{3}{8}\,\left({{\partial^2 T} \over {\partial x^2}}\right)^{-1}\,
\left({{\partial^3 T} \over {\partial y^2\,\partial x}}\right)^2\;,
\label{eqn:NIP4} 
\end{eqnarray}
where $T(x,y)$ is the one-way traveltime of the direct wave propagating from
the reflection point $x$ to the point $y$ at the surface $z=0$. All 
derivatives in equations (\ref{eqn:NIP2}) and (\ref{eqn:NIP4}) are evaluated 
at the zero-offset ray. Both equations are based
solely on Fermat's principle and, therefore, remain valid in any type of
media for reflectors of an arbitrary shape, assuming that the traveltimes
possess the required order of smoothness. It is especially convenient
to use equations (\ref{eqn:NIP2}) and (\ref{eqn:NIP4}) in 
homogeneous media, where the direct traveltime $T$ can be expressed explicitly.
\par
To apply equations (\ref{eqn:NIP2}) and (\ref{eqn:NIP4}) in VTI
media, we need to start with tracing the zero-offset ray. According to 
Fermat's principle, the ray trajectory must correspond to an extremum of
the traveltime. For the zero-offset ray, this simply means that the
one-way traveltime $T$ satisfies the equation
\begin{equation}
{{\partial T} \over {\partial x}} = 0\;,
\label{eqn:fermat0}
\end{equation}
where
\begin{equation}
T(x,y) = {\sqrt{z^2(x) + (x-y)^2} \over {V_g(\psi(x,y))}}\;.
\label{eqn:defT}
\end{equation}
Here, the function $z(x)$ describes the reflector shape, and $\psi$ is the ray angle given by the trigonometric relationship 
(Figure~\ref{fig:nmoray})
\begin{equation}
\cos{\psi(x,y)} = {z(x) \over \sqrt{z^2(x) + (x-y)^2}}\;.
\label{eqn:defpsi}
\end{equation}
Substituting approximate equation (\ref{eqn:vgeta}) for the group velocity
$V_g$ into equation (\ref{eqn:defT}) and linearizing it with respect to the
anisotropic parameters $\delta$ and $\eta$, we can solve equation
(\ref{eqn:fermat0}) for $y$, obtaining
\begin{equation}
y = x + z\,\tan{\alpha}\,(1 + 2\,\delta + 4\,\eta\,\sin^2{\alpha})
\label{eqn:defy}
\end{equation}
or, in terms of $\psi$,
\begin{equation}
\tan{\psi} = \tan{\alpha}\,(1 + 2\,\delta + 4\,\eta\,\sin^2{\alpha})\;,
\label{eqn:a2p}
\end{equation}
where $\alpha$ is the local dip of the reflector at the reflection
point $x$. Equation (\ref{eqn:a2p}) shows that, in VTI media, the
angle $\psi$ of the zero-offset ray differs from the reflector dip
$\alpha$ (Figure~\ref{fig:nmoray}). As one might expect, the relative
difference is approximately linear in Thomsen anisotropic parameters.

\plot{nmoray}{height=4in}{Zero-offset reflection from a
curved reflector beneath a VTI medium (a scheme). Note that the ray angle
$\psi$ is not equal to the local reflector dip $\alpha$.}
%??? -- Make the arcs for $\psi$ and $\alpha$ touch the straight lines.
%??? -- Put the reflection point ON the reflector, not above it.}
\par
Now we can apply equation (\ref{eqn:NIP2}) to evaluate the second term of the
Taylor series expansion (\ref{eqn:taylor}) for a curved
reflector. The linearization in anisotropic parameters leads to the expression
\begin{equation}
a_1 = {1 \over V_n^2} = 
{\cos^2{\alpha} \over {V_z^2\,
\left(1 + 2\,\delta\,(1 + \sin^2{\alpha}) + 
6\,\eta\,\sin^2{\alpha}\,(1+\cos^2{\alpha})\right)}}\;,
\label{eqn:tsvan}
\end{equation}
which is equivalent to that derived by \cite{tsvan1995}. As in
isotropic media, the normal-moveout velocity does not depend on the
reflector curvature. Its dip dependence, however, is an important
indicator of anisotropy, especially in areas of conflicting dips
\cite[]{aktsvan}.  \par Finally, using equation (\ref{eqn:NIP4}), we
determine the third coefficient of the Taylor series. After
linearization in anisotropic parameters and lengthy algebra, the
result takes the form
\begin{equation}
a_2 = {A \over {V_n^4\,t_0^2}}\;,
\label{eqn:a2new}
\end{equation}
where
\begin{eqnarray}
A = G\,\tan^2{\alpha} + 
2\,\delta\,G\,\sin^2{\alpha}\,(2 + \tan^2{\alpha} - G) -
2\,\eta\,(1 - 4\,\sin^2{\alpha}) +
\nonumber \\
+ 4\,\eta\,G\,\sin^2{\alpha}\,\left(
6\,\cos^2{\alpha} + \sin^2{\alpha}\,(\tan^2{\alpha}-3\,G)\right)\;,
\label{eqn:Anew}
\end{eqnarray}
and the coefficient $G$ is defined by equation (\ref{eqn:g}). For zero curvature (a plane reflector) $G = 0$, and
the only term remaining in equation (\ref{eqn:Anew}) is
\begin{equation}
A = - 2\,\eta\,(1 - 4\,\sin^2{\alpha})\;.
\label{eqn:zerog}
\end{equation}
If the reflector is curved, we can rewrite the
isotropic equation (\ref{eqn:CRapprox}) in the form
\begin{equation}
t^2(l) = t_0^2 + {l^2 \over V_n^2} + {{A\,l^4} \over 
{V_n^2\,\left(V_n^2 t_0^2 + G\,l^2\right)}}\;,
\label{eqn:TICRapprox}
\end{equation}
where the normal-moveout velocity $V_n$ and the quantity $A$ are given by 
equations~(\ref{eqn:tsvan}) and~(\ref{eqn:Anew}), respectively. 
Equation~(\ref{eqn:TICRapprox}) approximates the nonhyperbolic moveout 
in homogeneous VTI media above a curved reflector. For small
curvature, the accuracy of this equation at finite offsets
can be increased by modifying the denominator in the quartic term  similarly
to that done by \cite{grektsvan} for VTI media.
