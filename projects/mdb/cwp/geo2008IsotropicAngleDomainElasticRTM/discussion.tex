\section{Discussion}

\geouline{Our presentation of the angle-domain reverse-time migration
method outlined in the preceding sections deliberately ignores several
practical challenges in order to maintain the focus of this paper to
the actual elastic imaging condition. However, for completeness, we
would like to briefly mention several complementary issues that need
to be addressed in conjunction with the imaging condition in order to
design a practical method for elastic reverse-time migration.}

%% 
 % discuss mode conversion at the injection point
 %%

\geouline{First, reconstruction of the receiver wavefield requires that
  the multicomponent recorded data be injected into the model in
  reverse-time. In other words, the recorded data act as a
  displacement sources at receiver positions. In elastic materials,
  displacement sources trigger both compressional and transverse wave
  modes, no matter what portion of the recorded elastic wavefield is
  used as a source. For example, injecting a recorded compressional
  mode triggers both a compressional (physical) mode and a transverse
  (non-physical) mode in the subsurface . Both modes propagate in the
  subsurface and might correlate with wave modes from the
  source side. There are several ways to address this
  problem, such as by imaging in the angle-domain where the
  non-physical modes appear as events with non-flat moveout. We can
  make an analogy between those non-physical waves and multiples
  that also lead to non-flat events in the angle-domain. Thus, the
  source injection artifacts might be eliminated by filtering the
  migrated images in the angle domain, similar to the technique
  employed by} \cite{SavaGuitton.geo.mat} \geouline{for suppressing
  multiples after imaging.}

%% 
 % discuss the need for additional mode separation on the surface
 % discuss surface waves
 %%

\geouline{Second, the data recorded at a free surface contain both
  up-going and down-going waves. Ideally, we should use only the
  up-going waves as a source for reconstructing the elastic wavefields
  by time-reversal. In our examples, we assume an absorbing surface in
  order to avoid this additional complication and concentrate on the
  imaging condition. However, practical implementations require
  directional separation of waves at the surface}
\cite[]{WapenaarHaime1990,WapenaarEtAl1990,
  AmundsenReitan1995,AmundsenEtAl2001,HouMarfurt2002}.
\geouline{Furthermore, a free surface allows other wave modes to be
  generated in the process of wavefield reconstruction using the
  elastic wave-equation, e.g. Rayleigh and Love waves. Although those
  waves do not propagate deep into the model, they might interfere
  with the directional wavefield separation at the surface.}

%% 
 % discuss amplitude treatment and AVA
 %%

\geouline{ Third, we suggest in this paper that angle-dependent
  reflectivity constructed using extended imaging conditions might
  allow for elastic AVA analysis. This theoretical possibility
  requires that the wavefields are correctly reconstructed in the
  subsurface to account for accurate amplitude variation. For example,
  boundaries between regions with different material properties need
  to be reasonably located in the subsurface to generate correct mode
  conversions, and the radiation pattern of the source also needs to
  be known. Neither one of these aspects is part of our analyses, but
  they represent important considerations for practical elastic
  wavefield imaging.}


%% 
 % discuss wave-mode separation in anisotropic media
 %%

\geouline{Fourth, the wave-mode separation using divergence and curl
  operators, as required by Helmholtz decomposition, does not work
  well in elastic anisotropic media. Anisotropy requires that the
  separation operators take into account the local
  anisotropic parameters that may vary spatially}
\cite[]{YanSava.geo.separate}. \geouline{However, we do not discuss
  anisotropic wave-mode decomposition in this paper and restrict our
  attention to angle-domain imaging in isotropic models.}
