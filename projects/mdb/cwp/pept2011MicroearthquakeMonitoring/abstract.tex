% ------------------------------------------------------------
\begin{abstract}

%% 
 % the motivation/objective
 %%
Micro-seismicity can be used to monitor the migration of fluids during
reservoir production and hydro-fracturing operations in brittle
formations or for studies of naturally occurring earthquakes in fault
zones. Micro-earthquake locations can be inferred using wave-equation
imaging under the exploding reflector model, assuming densely sampled
data and known velocity.
%% 
 % sparse data - what is the problem?
 %%
Seismicity is usually monitored with sparse networks of seismic
sensors, for example located in boreholes. The sparsity of the sensor
network itself degrades the accuracy of the estimated locations, even
when the velocity model is accurately known. This constraint limits
the resolution at which fluid pathways can be inferred.
%% 
 % sources of wavefield randomness
 %%
Wavefields reconstructed in known velocity using data recorded with
sparse arrays can be described as having a random character due to the
incomplete interference of wave components. Similarly, wavefields
reconstructed in unknown velocity using data recorded with dense
arrays can be described as having a random character due to the
inconsistent interference of wave components. In both cases, the
\geosout{reconstructed wavefields are characterized by} random
fluctuations \geosout{which} obstruct focusing that occurs at source
locations.
%% 
 % what can we do about it?
 %%
This situation can be improved using interferometry in the imaging
process. Reverse-time imaging with an interferometric imaging
condition attenuates random fluctuations, thus producing crisper
images which support the process of robust automatic micro-earthquake
location.
%% 
 % 2D vs. 3D
 %%
The similarity of random wavefield fluctuations due to model
fluctuations and sparse acquisition \geosout{, as well as the
  applicability of conventional and interferometric imaging
  techniques,} are illustrated in this paper with a \geosout{complex}
\geouline{realistic} synthetic example.

\end{abstract}