\section{Introduction}

%% 
 % the need for velocity estimation
 %%

Accurate wave-equation depth imaging requires accurate knowledge of a
velocity model. The velocity model is used in the process of wavefield
reconstruction, irrespective of the method used to solve the acoustic
wave-equation, e.g. by integral methods (Kirchhoff migration), or by
differential/spectral methods (migration by wavefield extrapolation
and reverse-time migration).

%% 
 % velocity estimation strategies (tomography vs. MVA)
 % what is the difference between tomography and MVA?
 %%

Generally-speaking, there are two possible strategies for velocity
estimation from surface seismic data in the context of wavefield depth
migration. The two strategies differ by the domain in which the
information used to update the velocity model is collected. The first
strategy is formulated in the \textit{data space}, prior to migration,
and it involves matching of recorded and simulated data using an
approximate (background) velocity model. Techniques in this category
are known by the name of tomography (or inversion).  
%\onecolumn 
The
second strategy is formulated in the \textit{image space}, after
migration, and it involves measuring and correcting image features
that indicate model inaccuracies. Techniques in this category are
known as migration velocity analysis (MVA), since they involve
migrated images and not the recorded data directly.

%% 
 % implementation strategies for MVA and tomography?
 % (rays vs. waves)
 %%

Tomography and migration velocity analysis can be implemented in
various ways that can be classified based on the carrier of
information from the data or image to the velocity model. Thus, we can
distinguish between \textit{ray-based} methods and \textit{wave-based}
methods. This terminology is applicable to both tomography and
migration velocity analysis.
% rays
For ray-based methods, the carrier of information are wide-band rays
traced using a background velocity model from picked events in the
data (or image). Methods in this category are known as
\textit{traveltime tomography} \cite[]{GEO50-06-09030923} and
\textit{traveltime MVA}, sometimes described as image-space traveltime
tomography
\cite[]{GEO57-05-06800692,AlYahya.thesis,Fowler.thesis,Etgen.thesis,GEO60-02-04760490,GEO66-03-08450860,GEO67-04-12021212,GEO67-04-12131224,GEO68-03-10081021,GPR52-06-06710681,GEO69-02-05330546}.
% waves
For wave-based methods, the carrier of information are band-limited
wavefields constructed using a background velocity model. Methods in
this category are known as \textit{wave-equation tomography}
\cite[]{GEO51-07-13871403,tarantola.elsevier,GEO54-12-15751586,GEO57-01-00150026,GEO64-03-08880901,2004_Pratt_BPsalt},
and \textit{wave-equation MVA}
\cite[]{BiondiSava.segab.1999,SavaBiondi.gp.wemva1,SavaBiondi.gp.wemva2,ShenSymes.segab.2005,Albertin.eageab.2006,MaharramovAlbertin.segab.2007}.
This paper concentrates on methods from the latter category.

%% 
 % why are waves better than rays?
 %%

The volume of information used for model updates using wave-based
methods is at least one order of magnitude larger than the volume of
information used for ray-based methods. Thus, a fundamental
question we should ask is what is gained by using wave-based
methods over ray-based methods. First, modern imaging
applications using wave-based methods (downward continuation or
reverse-time extrapolation) require consistent velocity estimation
methods which interact with model in the same frequency band as the
migration methods. Second, wave-based methods are robust
(i.e. stable) for models with large and sharp velocity variations
(e.g. salt). Third, wave-based methods describe naturally all
propagation paths, thus they can easily handle multi-pathing
characterizing wave propagation in media with large velocity
variations.

%% 
 % WE tomography vs. WE MVA
 % match data or match images
 %%

Wave-equation tomography and wave-equation MVA have both similarities
and differences.
%
Wave-equation tomography uses the advantage that the residual used for
velocity updating is obtained by a direct comparison between recorded
and measured data. In contrast, wave-equation MVA uses the property
that the residual used for velocity updating is obtained by a
comparison between a reference image and an improved version of it.
%
On the other hand, wave-equation tomography has the disadvantage that
the kinematics of events in the data domain are more complex than in
the image domain. In addition, the dimensionality of the space in
which to evaluate the misfit between recorded and simulated data is
higher too, potentially making a comparison more complex.
%
Also, wave-equation MVA optimizes directly the desired end product,
i.e. the migrated image, which potentially makes this technique more
``interpretive'' and less of a computational ``black-box''.

%% 
 % where does MVA information come from?
 % (moveout vs. focusing)
 % (offset-gathers vs. angle-gathers)
 %%

One important component of MVA methods is what type of measurement on
migrated images is used to evaluate its deficiencies. Although
strictly related to one-another, we can describe two kinds of
information available to quantify image quality. First is
\textit{focusing analysis}, which evaluates whether point-like events,
e.g. fault truncations, are focused in migrated images at their
correct position. Image enhancement for wave-equation MVA can be
formulated purely based on this type of information, which makes the
techniques fast since it only operates with zero-offset data
\cite[]{Sava.geo.fva}. Second is \textit{moveout analysis}, which is
the case for all conventional MVA techniques, whether using rays or
waves. In this case, we can formulate wave-equation MVA based on
analysis of moveout of common-image gathers using velocity scans
\cite[]{BiondiSava.segab.1999,SavaBiondi.gp.wemva1,SavaBiondi.gp.wemva2,SoubarasGratacos.2007}
or based on analysis of differential semblance of nearby traces in
similar common-image gathers \cite[]{ShenSymes.segab.2005}.

%% 
 % can also use amplitude information! (4D - Albertin)
 %%

Moveout analysis requires construction of common-image gathers (CIGs)
characterizing the dependence of reflectivity function of various
parameters used to parametrize multiple experiments used for
imaging. There are two main alternatives for common-image gather
construction. First, we can construct \textit{offset-domain} CIGs
\cite[]{GEO49-10-16641674}, when reflectivity depends on
source-receiver offset on the acquisition surface, which is a data
parameter. Second, we can construct \textit{angle-domain} CIGs
\cite[]{GEO55-09-12231234, SEG-1999-08240827, SEG-2000-08300833,
GEO67-03-08830889, XieWu.adcig, GEO68-03-10651074, SEG-2003-08890892,
Fomel.seg.3dadcig, GEO69-05-12831298}, when reflectivity depends on
the angles of incidence at the reflection point, which is an image
parameter. For wave-equation migration, offset-domain CIGs are not a
practical solution, since the information from multiple offsets (or
multiple seismic experiments) mixes in the process of
migration. Furthermore, angle-domain CIGs suffer from fewer artifacts
than offset-domain CIGs, due to the fact that reflectivity
parametrization for angle-gathers occurs after wavefield
reconstruction in the subsurface, as oppose to offset-gathers when
reflectivity parametrization is related to data parameters
\cite[]{GEO69-02-05620575}.

%% 
 % what is the problem with wave-based methods?
 %%

Both wave-equation tomography and wave-equation MVA methods are based
on a fundamental ``small perturbation'' assumption, which requires a
reasonably-good starting model. This requirement represents a drawback
which is responsible for the main difficulty of methods in both
categories. For wave-equation tomography or inversion, we can update
models based on differences between recorded and simulated data. If
the starting background model is not accurate enough, we run the risk
of subtracting wavefields corresponding to different events. Likewise,
for wave-equation MVA, we update the model based on differences
between two images, one simulated in the background model and an
enhanced version of this image. If the enhanced version of the image
goes too far from the reference image, we run the risk of subtracting
events corresponding to different reflectors. This phenomenon is
usually referred-to in the literature as ``cycle skipping'' and
various strategies have been designed to ameliorate this problem,
e.g. by optimal selection of frequencies used for velocity analysis
\cite[]{GEO69-01-02310248,AlbertinFreq}. However, alternative methods
used to evaluate image accuracy, e.g. differential semblance
\cite[]{GEO56-05-06540663,SEG-2003-21322135}, have the best potential
to ameliorate this situation. In this case wave-equation MVA analyzes
difference between image traces within common-image gathers which are
likely to be similar enough from one-another such as to avoid the
cycle-skipping problem. Even in this case, the assumption made is that
the nearby traces in a gather are sampled well-enough, i.e. the
seismic events differ by only a fraction of the wavelet, which is a
function of image sampling and frequency content. In practice, there
is no absolute guarantee that nearby events are closely related to one
another, although this is more likely to be true for DSO than it is
for direct image differencing.

%% 
 % what are we discussing about in this paper?
 %%

In this paper, we concentrate on the implementation of the
wave-equation migration velocity analysis operators for various
wave-equation migration configurations. The main objective of the
paper is to derive the linearized operators linking perturbations of
the slowness model to the corresponding perturbations of the seismic
wavefield and image. All our theoretical development is formulated
under the single scattering (Born) approximation applied to acoustic
waves. We begin by describing the MVA operators corresponding to
zero-offset, survey-sinking and shot-record wave-equation migration
frameworks.  We describe the theoretical background for each operator
and emphasize the similarities and differences between the different
operators.  Throughout the paper, we use pseudo-code to illustrate
implementation strategies and data flow for the various wave-equation
MVA operators. Finally, we illustrate the wave-equation MVA operators
with impulse responses corresponding to simple and complex models. We
leave outside the scope of this paper the procedure in which the
discussed operators are used for migration velocity analysis.
