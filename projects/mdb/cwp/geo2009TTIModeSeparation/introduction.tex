
% ------------------------------------------------------------
\section{Introduction}

%ADD  Elastic imaging context, compare to acoustic imaging
%Review
Although acoustic migration is currently the most common seismic
imaging procedure, elastic imaging, with the addition of
converted-waves, has been recognized to have potential advantages in
seeing through gas-charged sediments and in structural and near
surface imaging~\cite[]{GEO68-01-00400057}. Two options are available
for elastic imaging: 1) one can separate wave-modes at the surface and
image with the separated PP and PS data using acoustic migration
tools~\cite[]{GEO69-01-02860297}, or 2) one can extrapolate the
recorded multicomponent data and image with the reconstructed elastic
wavefields by applying an imaging condition to wave-modes separated in the
vicinity of the image points~\cite[]{yan:S229}.  The first approach
benefits from the simplicity of imaging with scalar waves, but it is
based on the assumption that P and S modes can be successfully
separated on the recording surface, which is difficult for complicated
datasets. The second approach reconstructs elastic wavefields in the
subsurface, thus capturing all possible wave-mode transmissions and
reflections, although it increases the computational cost in elastic
wavefields modeling. In addition, the elastic migration technique
requires wave-mode separation before the application of an imaging
condition to avoid crosstalk between different wave-modes.

% isotropic and VTI, how does separation work
%Claim
%MAKE ISOTROPIC AND ANISOTROPIC SYMMETRIC
In isotropic media, the P- and S-modes (shear waves do not split in
isotropic media) can easily be separated by taking the divergence and
curl of the elastic wavefield~\cite[]{akirichards.2002}, and the
procedure is effective in homogeneous as well as heterogeneous
media. This is because, in the far field, P- and S-waves are polarized
parallel and perpendicular to the wave vectors, respectively. The
polarization directions of the P- and S-waves only depend on the wave
propagation direction and are not altered by the medium.  Therefore,
the wave-mode separators are invariant with space, and divergence and
curl can always be used to separate compressional (scalar) and shear
(vector) wave-modes.

However, divergence and curl do not fully separate wave-modes in
anisotropic media, because P- and S-waves are not polarized parallel
and perpendicular to the wave vectors.
\cite{GEO55-07-09140919} separate wave-modes in
homogeneous VTI (vertically transversely isotropic) media by
projecting the vector wavefields onto the polarization vectors of each
mode.  In VTI media, the polarization vectors of P- and SV-waves
depend on the anisotropy parameters $\epsilon$ and
$\delta$~\cite[]{thomsen:1954} and are spatially-varying when the
medium is inhomogeneous. Therefore, \cite{yan:WB19} separate
wave-modes in heterogeneous VTI media by filtering the wavefields with
spatially varying separators in the space domain and show that
separation is effective even for complex geology with high
heterogeneity.


% TTI, importance  % TTI, what needs to be changed
However, VTI models are suitable only for limited geological settings
with horizontal layering.  Many case studies have shown that TTI
(tilted transversely isotropic) models better represent complex
geologies like thrusts and fold belts, e.g., the Canadian
Foothills~\cite[]{SEG-1991-0207}. Using the VTI assumption to image
structures characterized by TTI anisotropy introduces both kinematic
and dynamical errors in migrated images. For
example, \cite{vestrum:1239} and \cite{isaac:1230} show that seismic
structures can be mispositioned if isotropy, or even VTI anisotropy,
is assumed when the medium above the imaging targets is TTI. To carry
out elastic wave-equation migration for TTI models and apply the
imaging condition that crosscorrelates the separated wave-modes, the
wave-mode separation algorithm needs to be adapted to TTI media.  For
sedimentary layers bent under geological forces, TTI migration models
usually incorporate locally varying tilts, and the local symmetry axes
are assumed to be orthogonal to the reflectors throughout the
model~\cite[]{charles:VE261,AlkhalifahSava.geo.2010}. Therefore, in
complex TI models, both the local anisotropy parameters $\epsilon$ and
$\delta$, and the local symmetry axes with tilt $\nu$ and azimuth
$\alpha$ can be space-dependent.
%3D VTI

This technique of separation by projecting the vector wavefields onto
polarization vectors has been applied only to 2D VTI
models~\cite[]{dellinger.thesis,yan:WB19} and for P-mode separation
for 3D VTI models~\cite[]{dellinger.thesis}. For 3D models, the main
challenge resides in the fact that fast and slow shear modes have
non-linear polarizations along symmetry-axis propagation
directions. It is possible to apply the 2D separation method to 3D TTI
models using the following procedure. First, project the elastic
wavefields onto symmetry planes (which contains P- and SV-modes) and
their orthogonal directions (which contain the SH-mode only); then
separate P- and SV-modes in the symmetry planes using divergence and
curl operators for isotropic media or polarization vector projection
for TI media. However, this approach is difficult as wavefields are
usually constructed in Cartesian coordinates and symmetry planes of
the models do not align with the Cartesian coordinates. Furthermore,
for heterogeneous models, the symmetry planes change spatially, which
makes projection of wavefields onto symmetry planes impossible. To
avoid these problems, I propose a simpler and more straightforward
solution to separate wave-modes with 3D operators, which eliminates
the need for projecting the wavefields onto symmetry planes. The new
approach constructs shear-wave filters by exploiting the mutual
orthogonality of shear modes with the P mode, whose polarization
vectors are computed by solving 3D Christoffel equations.

% Agenda
In this chapter, I briefly review wave-mode separation for 2D VTI media
and then I extend the algorithm to symmetry planes of TTI media.
Then, I generalize the wave-mode separation to 3D TI media. Finally,
I demonstrate wave-mode separation in 2D with homogeneous and
heterogeneous examples and separation in 3D with a homogeneous TTI
example.

