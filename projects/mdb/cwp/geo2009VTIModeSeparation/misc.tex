%\def\cwplogo{
%\pgfputat{\pgfxy(10,-5)}{\pgfbox[left,bottom]{\includegraphics[width=.1\textwidth]{XFig/Fig/cwp}}}
%
%}



\def\SSMIG{
\begin{figure}[h]
\centering
\setlength{\unitlength}{1in}
%
\begin{picture}(4,2.0)
\thicklines
\large
%
%\put(-0.50,2.00){\makebox(4,0.5){\underline{ \textbf{source/receiver} }}}
%
\put(-0.50,1.60){\makebox(4,0.5){\;\;$\us\lp \xx,\zz,\tt \rp$\;\;$\ur\lp \xx,\zz,\tt \rp$ }}
\put( 1.50,1.60){\vector(0,-1){0.3}}
\put(-0.50,0.80){\makebox(4,0.5){$\II\lp \xx,\zz,\hx \rp$}}
\put( 1.50,0.80){\vector(0,-1){0.3}}
\put(-0.50,0.00){\makebox(4,0.5){$\II\lp \xx,\zz,\mathbf{\theta} \rp$}}
%
\put( 2.00,1.20){\makebox(3,0.5)[c]{\textbf{extended I.C.}}}
\put( 2.00,0.40){\makebox(3,0.5)[c]{\textbf{angle decomposition}}}
%\put( 2.00,0.00){\makebox(3,0.5)[c]{\textbf{image}}}
\end{picture}
\end{figure}
}



\def\CURL#1{\nabla \times {#1}}
\def\GRAD#1{\nabla {#1}}
\def\DIV#1{\nabla \cdot {#1}}
\def\LAPL#1{\nabla^2 {#1}}

\def\us{\mathbf {u_s}}
\def\ur{\mathbf {u_r}}
\def\xx{\mathbf x}
\def\zz{\mathbf z}
\def\tt{\mathbf t}
\def\hx{\mathbf {h_x}}
\def\uu{\mathbf u}
\def\II{\mathbf I}

%\def\xx{ x,z}
%\def\ofx { \lp \xx   \rp}
%\def\ofxt{ \lp \xx,t \rp}

\def\IM#1{  {I}_{#1}}  
\def\US#1{{u_s}_{#1}}
\def\UR#1{{u_r}_{#1}}

\def\P#1{ \Theta_{#1} }
\def\S#1{   \Psi_{#1} }

% ------------------------------------------------------------
\def\zis{
\put(-4,16){\rotatebox{90}{\normalsize {\bf depth}   }}
\put(40,-3){               \normalsize {\bf position} }
}

\def\zi#1#2{
\put(5,22){\rotatebox{90}{\normalsize {\bf depth}   }}
\put(40,-3){               \normalsize {\bf position} }
\put(95,0){               \normalsize {\bf #1} }
\put(95,55){               \normalsize {\bf #2} }
}
%for add a scale bar

\def\zj{
\put(-8,12){\rotatebox{90}{\normalsize {\bf depth}   }}
\put(35,-6){               \normalsize {\bf position} }
}


\def\zd{
\put(-4,45){\rotatebox{90}{\normalsize {\bf time}   }}
\put(10,-4){               \normalsize {\bf position} }
}

\def\za{
\put(-5,45){\rotatebox{90}{\normalsize {\bf }   }}
\put(15,-4){               \normalsize {\bf angle} }
\put(-6,100){               \normalsize {\bf -60} }
\put(20.75,100){               \normalsize {\bf 0} }
\put(45,100){               \normalsize {\bf 60} }
}

\def\zc{
\put(-5,45){\rotatebox{90}{\normalsize {\bf      }   } }
\put(10,-4){              {\normalsize {\bf space lag}   } }

}


\def\zaline{
\za
\lvertline
}

\def\model{
\put(-5,45){\rotatebox{90}{\normalsize {\bf depth}   }}
\put(4,-4){               \normalsize {\bf position} }
}

\def\Ilum
{\zis
\put(59,43){\normalsize {\bf S} }
\put(-2,43){\normalsize {\bf R}   }
}
%-------------------------------------------------------------------------
\def\cig{
\put(-4,16){\rotatebox{90}{\normalsize {\bf depth}   }}
\put(40,-3){               \normalsize {\bf position} }
\put(45,45){\bf CIG }
\svertline
}
\def\svertline{
\linethickness{1pt}
\put(50,0.7) {\line(0,1){41}}
}
\def\lvertline{
\linethickness{.5pt}
\put(25.5,1) {\line(0,1){98}}
}

% ------------------------------------------------------------

%% 
 % 45 degree red arrow
 %%
\def\arrow45#1#2{
\thicklines
\put(#1,#2){\red {\vector(-1,-1){10}} }
}

% ------------------------------------------------------------

% \put(xcoordinate, ycoordinate){\line(xslope,yslope){length}

%The slope is determined by (x-slope,y-slope) which are signed integers of magnitude less than or equal to 6 and which have no common divisor except for plus or minus one; For example (1,0) is a horizontal line; (1,1) gives a slope of 45 degress; (0,1) is a vertical line; (-1,1) is 135 degrees.

%The horizontal extent of the line is given by the length parameter, except in the case of a vertical line in which case length specifies the vertical height.


%% 
 % define \matrix
 %%
\def\mtrx#1 
{
\begin{matrix}#1\end{matrix}
}
\def\pmtrx#1 
{
\begin{pmatrix}#1\end{pmatrix}
}
\def\bmtrx#1 
{
\begin{bmatrix}#1\end{bmatrix}
}
\def\Bmtrx#1 
{
\begin{Bmatrix}#1\end{Bmatrix}
}
\def\vmtrx#1 
{
\begin{vmatrix}#1\end{vmatrix}
}
\def\Vmtrx#1 
{
\begin{Vmatrix}#1\end{Vmatrix}
}
