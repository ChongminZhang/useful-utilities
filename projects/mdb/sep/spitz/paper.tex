
\lefthead{Claerbout \& Fomel}
\righthead{Estimating the signal PEF}
\footer{SEP--103}

\published{SEP report, 103, 211-219 (2000)}
\title{Spitz makes a better assumption for the signal
  PEF}

\email{claerbout@stanford.edu, sergey@sep.stanford.edu}
%  This paper should have been in SEP-102 but tragedy
%    intervened.}
\author{Jon Claerbout and Sergey Fomel}
\def\eq{\quad =\quad}
\maketitle

\begin{abstract}
In real-world extraction of signal from data we are not given the
needed signal prediction-error filter (PEF).  Claerbout has taken $S$,
the PEF of the signal, to be that of the data, $S\approx D$.  Spitz
takes it to be $S\approx D/N$.  Where noises are highly predictable in
time or space, Spitz gets significantly better results.
Theoretically, a reason is that the essential character of a PEF is
contained {\it where it is small.}
\end{abstract}

\inputdir{sign}

\section{INTRODUCTION}

Knowledge of signal spectrum and noise spectrum
allows us to find filters for optimally
separating data $\mathbf d$ into two components, signal $\mathbf s$
and noise $\mathbf n$ \cite[]{gee}.
Actually, it is the inverses of these spectra
which are required.
In Claerbout's textbook example \cite[]{gee}
he estimates these inverse spectra by estimating prediction-error filters
(PEFs) from the data.
He estimates both a signal PEF and a noise PEF from the same data $\mathbf d$.
A PEF based on data $\mathbf d$ might be expected to be named the data PEF $D$,
but Claerbout estimates two different PEFS from $\mathbf d$
and calls them the signal PEF $S$ and the noise PEF $N$.
They differ by being estimated with different number
of adjustable coefficients, one matching a signal model
(two plane waves) having three positions on the space axis,
the other matching a noise model
having one position on the space axis.
\par
Meanwhile, using a different approach,
\cite{TLE18-01-00550058} concludes
that the signal, noise, and data inverse spectra
should be related by $D=SN$.
The conclusion we reach in this paper
is that Claerbout's estimate of $S$ is more
appropriately an estimate of the data PEF $D$.
To find the most appropriate $S$ and $N$ we
should use both the ``variable templates'' idea of Claerbout
and the $D\approx SN$ idea of Spitz.
Here we first
provide a straightforward derivation of the Spitz insight
and then we show some experimental results.

\section{BASIC THEORY}
Signal spectrum
plus the noise spectrum gives the data spectrum.
Since a prediction-error filter tends to the inverse of a spectrum
we have
\begin{eqnarray}
        {1\over \overline{D}D}
        &=&
        {1\over \overline{S}S}
        \ + \ 
        {1\over \overline{N}N}
\label{eqn:powersadd}
\\
        {1\over \overline{D}D}
        &=&
        { \overline{S}S} \ + \ { \overline{N}N}
         \over
         { \overline{SN} SN}
\end{eqnarray}
or
\begin{equation}
\overline{D}D
\eq
{
        { \overline{SN} SN}
 \over
        { \overline{S}S} \ + \ { \overline{N}N}
}
\label{eqn:truth}
\end{equation}

Now we are ready for the Spitz approximation.
Spitz builds his applications upon
the assumption that we can estimate $D$
and $N$ from suitable chunks of raw data.
His result may be obtained  from
(\ref{eqn:truth}) by ignoring its denominator getting
$D\approx SN$ or
\begin{equation}
S \quad\approx\quad D/N
\end{equation}
Ignoring
the denominator
in equation (\ref{eqn:truth}),
is not so terrible an approximation
as it might seem.
Remember that PEFs are important {\it where they are small}
because they are used as weighting functions.
Where weighting functions are small,
solutions are expected to be large.
%As terrible as it may seem
%to ignore the denominator
%it does seem to be a better approximation than
%Claerbout's assumption that
%\begin{equation}
%S \quad\approx\quad D
%\end{equation}
%Claerbout's assumption seems to be that he can find a chunk
%of data where signal is relatively unpolluted by noise.
%\par
%Why does Spitz's choice seem to work better than Claerbout's?
%The important part of a PEF is where it is small.
%PEFs are used as weighting functions.
%Where they are small is where solutions become large.
%Also, it is the inverse PEF that is something physical.
Although Claerbout's assumption
$ S \approx D$ might be somewhat valid for signal and data
{\it spectra},
it is much less valid for their
{\it PEFs.}
In practice,
signal unpolluted with noise is usually not available.
Even a very good chunk of data
tends to yield a poor estimate of the signal PEF $S$
because the holes in the signal spectrum are
easily intruded with noise.

\par
Obviously the major difference between
$ S \approx D$ and $S \approx D/N$ is where the noise is large.
Thus it is for ``organized and predictable'' noises (small $N$)
where we expect to see the main difference.

\par
Theoretically, we need not make the Spitz approximation.
We could solve (\ref{eqn:powersadd}) for $S$ by spectral factorization.
Although the $S$ obtained would be more theoretically satisfying,
there would be some practical disadvantages.
Getting the signal spectrum
by subtracting that of the noise from that of the data
leaves the danger of a negative result
(which explodes the factorization).
Thus,
maintaining spectral positivity would require extra care.
All these extra burdens are avoided by making the Spitz approximation.
All the more so in applications with continuously varying estimates.

%\par
%The reality of the Claerbout book approach is a little different
%from the oversimplification above.
%Our estimates of D and N depend
%not only on their training regions
%but on their allowed PEF templates.
%In my book GEE, $D$ and $N$ are both estimated on the
%{\it same} chunk of data (same training region)
%but they are estimated with different PEF templates.
%The signal template exists on both time and space axes
%whereas the noise template is limited to the time axis.
%Thus the assumption (valid for the book test case) is that
%the noise is spatially white.


%\section{ACKNOWLEDGEMENT}
%I'm not sure where 
%the important earlier work of others such as
%(1) Abma,
%(2) Gulanay,
%(3) Subaras, and
%(4) Ozdemir, Ozbeck, Ferber, and Zerouk
%fit into this picture.



\section{Signal and noise separation}

We assume that the data vector $\mathbf{d}$ is composed of the signal
and noise components $\mathbf{s}$ and $\mathbf{n}$:
\begin{equation}
  \label{spn}
  \mathbf{d = s + n}\;.
\end{equation}
If both the signal and noise prediction-error filters $S$ and
$N$ are known, then the signal can be extracted from the data
by solving the following system by the least squares method:
\begin{eqnarray}
\label{eqn:noisereg}
0 & \approx &          \mathbf N \mathbf n = \mathbf N ( \mathbf d - \mathbf s)\;;  \\
\label{eqn:signalreg}
0 & \approx & \epsilon \mathbf S \mathbf s\;,
\end{eqnarray} 
where $\epsilon$ is a scalar scaling coefficient, reflecting the
presumed signal-to-noise ration \cite[]{gee}. 
\par
The formal solution of system~(\ref{eqn:noisereg}-\ref{eqn:signalreg})
has the form of a \emph{projection filter}:
\begin{equation}
  \label{eqn:sfilter}
  \mathbf s =
  \left(
    \mathbf N' \mathbf N
    \over
    \mathbf N' \mathbf N \ + \ \epsilon^2 \mathbf S'\mathbf S 
  \right) \ \mathbf d\;.
\end{equation}
Analogously, the signal vector is expressed as
\begin{equation}
  \label{eqn:nfilter}
  \mathbf n = \mathbf d - \mathbf s =
  \left(
    \epsilon^2 \mathbf S' \mathbf S
    \over
    \mathbf N' \mathbf N \ + \ \epsilon^2 \mathbf S'\mathbf S 
  \right) \ \mathbf d\;.
\end{equation}
In 1-D or $F$-$X$ setting, one can accomplish the division in
formulas~(\ref{eqn:sfilter}) and~(\ref{eqn:nfilter}) directly by
spectral factorization and inverse recursive filtering
\cite[]{SEG-1995-0711,SEG-1994-1576}. A similar approach can be applied
in the case of $T$-$X$ or $F$-$XY$ filtering with the help of the
helix transform \cite[]{GEO63-05-15321541,SEG-1999-12311234} or by
solving system ~(\ref{eqn:noisereg}-\ref{eqn:signalreg}) directly with
an iterative method \cite[]{Abma.sepphd.88}.
\par
%A problem with the outlined signal-noise separation technique is that
%the signal prediction-error filter may not be know a priori. We can
%easily estimate the data PEF $\mathbf D$ by the usual technique
%(minimizing the power of $\mathbf{D d}$. 
%We can also assume that the
%noise PEF $\mathbf{N}$ is available from a prescribed noise model.
Claerbout's approach, implemented in the examples of \emph{GEE}
\cite[]{gee}, is to estimate the signal and noise PEFs $S$ and $N$ from
the data $\mathbf{d}$ by specifying different shape templates for these
two filters. The filter estimates can be iteratively refined after the
initial signal and noise separation.  In some examples, such as those
shown in this paper, the signal and noise templates are not easily
separated. When the signal template behaves as an extension of the
noise template so that the shape of $S$ completely embeds the shape of
$N$, our estimate of $S$ serves as a predictor of both signal and
noise. We might as well consider it as $D$, the prediction-error
filter for the data.

%start with assuming that the signal and data PEFs are
%equivalent. This assumption leads to the following algorithm:
% \begin{enumerate}
% \item Estimate $\mathbf D$ and $\mathbf N$.
% \item Assume $\mathbf S \approx \mathbf D$.
% \item Solve the least-square system~(\ref{eqn:noisereg}-\ref{eqn:signalreg}).
% \item Iterate if necessary.
% \end{enumerate}
 \par
 \cite{TLE18-01-00550058} argues that the data PEF $D$ can
 be regarded as the convolution of the signal and noise PEFs
 $S$ and $N$. 

This assertion suggests the following
 algorithm:
 \begin{enumerate}
 \item Estimate $D$ and $N$.
 \item Estimate $S$ by deconvolving (polynomial division)
   $D$ by $N$.
 \item Solve the least-square system~(\ref{eqn:noisereg}-\ref{eqn:signalreg}).
 \end{enumerate}
 To avoid the division step, we suggest a simple modification of
 Spitz's algorithm, which results from multiplying both equations in
 system~(\ref{eqn:noisereg}-\ref{eqn:signalreg}) by the noise
 filtering operator~$\mathbf{N}$. The resulting system has the form
\begin{eqnarray}
  \label{eqn:noisereg2}
  0 & \approx &          \mathbf N^2 \mathbf n = \mathbf N^2 ( \mathbf d - \mathbf s)\;;
  \\
  \label{eqn:signalreg2}
  0 & \approx & \epsilon \mathbf N \mathbf S \mathbf s = \epsilon \mathbf D \mathbf s\;.
\end{eqnarray}
The modified algorithm is
\begin{enumerate}
\item Estimate $D$ and $N$.
\item Convolve $N$ with itself.
\item Solve the least-square system~(\ref{eqn:noisereg2}-\ref{eqn:signalreg2}).
\end{enumerate}
The formal least-squares solution of
system~(\ref{eqn:noisereg2}-\ref{eqn:signalreg2}) is
\begin{equation}
  \label{eqn:sfilter2}
  \mathbf s =
  \left(
    \mathbf {N' N' N N}
    \over
    \mathbf {N' N' N N} \ + \ \epsilon^2 \mathbf D'\mathbf D 
  \right) \ \mathbf d \  =
  \left(
    \mathbf {N' N' N N}
    \over
    \mathbf {N' N' N N} \ + \ \epsilon^2 \mathbf {N' S' S N} 
  \right) \ \mathbf d\;.
\end{equation}
Comparing~(\ref{eqn:sfilter2}) with~(\ref{eqn:sfilter}), we can see
that both the numerator and the denominator in the two expressions
differ by the same multiplier $\mathbf{N' N}$. This multiplication
should not effect the result of projection filtering.
\par
Figure~\ref{fig:signoi0} shows a simple example of signal and noise
separation taken from \emph{GEE} \cite[]{gee}. The signal consists of
two crossing plane waves with random amplitudes, and the noise is
spatially random. The data and noise $T$-$X$ prediction-error filters
were estimated from the same data by applying different filter
templates. The template for $D$ is
\begin{verbatim}
  a a
  a a
  a a
1 a a
a a a
a a a
a a a
\end{verbatim}
where the \texttt{a} symbol represents adjustable coefficients.  The
data filter shape has three columns, which allows it to predict two
plane waves with different slopes.  The noise filter $N$ has
only one column. Its template is
\begin{verbatim}
1
a
a
a
\end{verbatim}
The noise PEF can estimate the temporal spectrum but would fail to
capture the signal predictability in the space direction.
Figure~\ref{fig:signoi} shows the result of applying the modified
Spitz method according to
equations~~(\ref{eqn:noisereg2}-\ref{eqn:signalreg2}).
Comparing
figures~\ref{fig:signoi0} and~\ref{fig:signoi},
we can see that using
a modified system of equations brings
a slightly modified result with more noise in the signal
but more signal in the noise.
It is as if $\epsilon$ has changed,
and indeed this could be the principal effect
of neglecting the denominator in equation (\ref{eqn:truth}).

%only a slight improvement to the original method.
%approaches perform fairly well with Spitz's method producing slightly
%better separation.

\plot{signoi0}{width=6in,height=3in}{Signal and noise separation with the
  original GEE method.  The input signal is on the left.  Next is that
  signal with random noise added.  Next are the estimated signal and
  the estimated noise.}

\plot{signoi}{width=6in,height=3in}{Signal and noise separation
  with the modified Spitz method.  The input signal is on the left.
  Next is that signal with random noise added.  Next are the estimated signal
  and the estimated noise.}

\par
To illustrate a significantly different result
using the Spitz insight we examine the new situation shown in
Figures~\ref{fig:planes90} and~\ref{fig:planes}.
The wave with the positive slope is considered to be
regular noise;
the other wave is signal.
The noise PEF $N$ was
estimated from the data by restricting the filter shape so that it
could predict only positive slopes. The corresponding template is
\begin{verbatim}
  a 
1 a 
\end{verbatim}
The data PEF template is
\begin{verbatim}
  a a
  a a
1 a a
a a a
a a a
\end{verbatim}
Using the data PEF as a substitute for the signal PEF produces a poor
result, shown in Figure~\ref{fig:planes90}.  We see a part of the
signal sneaking into the noise estimate. Using the modified Spitz
method, we obtain a clean separation of the plane waves
(Figure~\ref{fig:planes}).

\plot{planes90}{width=6in,height=3in}{Plane wave separation with the 
  GEE method.  The input signal is on the left.  Next is that signal
  with noise added.  Next are the estimated signal and the estimated
  noise.}

\plot{planes}{width=6in,height=3in}{Plane wave separation with the
  modified Spitz method.  The input signal is on the left.  Next is
  that signal with noise added.  Next are the estimated signal and the
  estimated noise.}
 
\cite{Clapp.sep.102.bob2,Clapp.sep.103.bob2} and
\cite{Brown.sep.102.morgan1} show applications of the
least-squares signal-noise separation to multiple and ground-roll
elimination.

\section{Acknowledgments}

Conversations with our colleagues Bob Clapp and Morgan Brown led us to
a better understanding of the Spitz approach.

\bibliographystyle{seg}
\bibliography{SEG,SEP2,spitz}


