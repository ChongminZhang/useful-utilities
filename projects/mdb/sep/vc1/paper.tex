\lefthead{Fomel}
\righthead{Velocity continuation}
\footer{SEP--92}
\published{Geophysics, 68, 1650-1661 (2003)}
\title{Velocity continuation and the anatomy of \\
residual prestack time migration}

\email{sergey@sep.stanford.edu}
\author{Sergey Fomel}

\maketitle

\begin{abstract} 
  Velocity continuation is an imaginary continuous process of
  seismic image transformation in the post-migration domain. It generalizes
  the concepts of residual and cascaded migrations. Understanding the laws
  of velocity continuation is crucially important for a successful application
  of time migration velocity analysis.  These laws predict the changes
  in the geometry and intensity of reflection events on migrated images with
  the change of the migration velocity.  In this paper, I derive kinematic
  and dynamic laws for the case of prestack residual migration from simple
  geometric principles. The main theoretical result is a decomposition
  of prestack velocity continuation into three different components
  corresponding to residual normal moveout, residual dip moveout, and residual
  zero-offset migration. I analyze the contribution and properties of each of
  the three components separately. This theory forms the basis for
  constructing efficient finite-difference and spectral algorithms for
  time migration velocity analysis.
\end{abstract}

\section{Introduction}
%%%%%%%%%%%%%%%%%%

\begin{comment}
Migration velocity analysis is a routine part of prestack time
migration applications. It serves both as a tool for velocity
estimation \cite[]{FBR08-06-02240234} and as a tool for optimal stacking
of migrated seismic sections and modeling zero-offset data for depth
migration \cite[]{GEO62-02-05680576}. In the most common form, migration
velocity analysis amounts to residual moveout correction on CRP
(common reflection point) gathers. However, in the case of dipping
reflectors, this correction does not provide optimal focusing of
reflection energy, since it does not account for lateral movement of
reflectors caused by the change in migration velocity. In other words,
different points on a stacking hyperbola in a CRP gather can
correspond to different reflection points at the actual reflector. The
situation is similar to that of the conventional NMO velocity
analysis, where the reflection point dispersal problem is usually
overcome with the help of DMO \cite[]{FBR04-07-00070024,dmo}. An
analogous correction is required for optimal focusing in the
post-migration domain. In this paper, I propose and test velocity
continuation as a method of migration velocity analysis. The method
enhances the conventional residual moveout correction by taking into
account lateral movements of migrated reflection events.
\end{comment}

The conventional approach to seismic migration theory
\cite[]{Claerbout.blackwell.85,Berkhout.mig.14A.1985} employs the
downward continuation concept. According to this concept, migration
extrapolates upgoing reflected waves, recorded on the surface, to the
place of their reflection to form an image of subsurface
structures.
%In order to understand the concept of velocity continuation, we need
%to look at the fundamentals of seismic time migration. 
Post-stack time migration possesses peculiar properties, which can
lead to a different viewpoint on migration.  One of the most
interesting properties is an ability to decompose the time migration
procedure into a cascade of two or more migrations with smaller
migration velocities. This remarkable property is described by
\cite{GEO50-01-01100126} as {\em residual migration}.
\cite{GEO52-05-06180643} generalized the method of residual
migration to one of {\em cascaded migration.} Cascading
finite-difference migrations overcomes the dip limitations of
conventional finite-difference algorithms \cite[]{GEO52-05-06180643};
cascading Stolt-type {\em f-k} migrations expands their range of
validity to the case of a vertically varying velocity
\cite[]{GEO53-07-08810893}. Further theoretical generalization sets
the number of migrations in a cascade to infinity, making each step in
the velocity space infinitesimally small. This leads to a partial
differential equation in the time-midpoint-velocity space, discovered
by \cite{Claerbout.sep.48.79}. Claerbout's equation describes the
process of {\em velocity continuation,} which fills the velocity space
in the same manner as a set of constant-velocity migrations. Slicing
in the migration velocity space can serve as a method of velocity
analysis for migration with nonconstant velocity
\cite[]{shurtleff,SEG-1984-S1.8,Fowler.sepphd.58, GEO57-01-00510059}.

\new{The concept of velocity continuation was introduced in the earlier
publications} \cite[]{me,SEG-1997-1762}.  \cite{hubral} and
\cite{GEO62-02-05890597} \new{
use the term \emph{image waves} to describe a
similar idea.} \cite{adler} \new{generalizes it to the case of variable
background velocities under the name \emph{Kirchhoff image propagation}.  The
importance of this concept lies in its ability to predict changes in the
geometry and intensity of reflection events on seismic images with the change
of migration velocity. While conventional approaches to migration velocity
analysis methods take into account only vertical movement of reflectors}
\cite[]{FBR08-06-02240234,GEO60-01-01420153}, 
\new{velocity continuation attempts to describe both
vertical and lateral movements, thus providing for optimal focusing in
velocity analysis applications} \cite[]{SEG-2001-11071110,second}. 

In this paper, I describe the velocity continuation theory for the case of
prestack time migration, connecting it with the theory of prestack residual
migration \cite[]{Al-Yahya.sep.50.219,Etgen.sepphd.68,GEO61-02-06050607}. By
exploiting the mathematical theory of characteristics, a simplified kinematic
derivation of the velocity continuation equation leads to a differential
equation with correct dynamic properties. 
%In the post-stack case, the
%solution of the boundary-value problem, associated with this equation,
%coincides precisely with the operators of Kirchoff migration, traditionally
%derived from a completely different prospective.  
In practice, one can
accomplish dynamic velocity continuation by integral, finite-difference, or
spectral methods.
\begin{comment}
For practical
applications, I chose the Fourier spectral method. The method has its
limitations \cite[]{Fomel.sep.97.sergey2}, but looks optimal in terms of
the accuracy versus efficiency trade-off.
 
Applying velocity continuation to migration velocity analysis involves
the following steps: 
\begin{enumerate}
\item prestack common-offset (and common-azimuth) migration - to
  generate the initial data for continuation,
\item velocity continuation with stacking across different offsets -
  to transform the offset data dimension into the velocity dimension,
\item picking the optimal velocity and slicing through the migrated
  data volume - to generate an optimally focused image.
\end{enumerate}
The final part of this paper includes a demonstration of all three
steps on a simple two-dimensional dataset.
\end{comment}
The accompanying paper \cite[]{second} introduces one of the possible
numerical implementations and demonstrates its application on a \new{field}
data example.

\new{
The paper is organized into two main sections. First, I derive the kinematics
of velocity continuation from the first geometric principles. I identify three
distinctive terms, corresponding to zero-offset residual migration, residual
normal moveout, and residual dip moveout. Each term is analyzed separately to
derive an analytical prediction for the changes in the geometry of 
traveltime curves (reflection events on migrated images) with the change of
migration velocity. Second, the dynamic behavior of seismic images is
described with the help of partial differential equations and their
solutions. Reconstruction of the dynamical counterparts for kinematic
equations is not unique. However, I show that, with an appropriate selection of
additional terms, the image waves corresponding to the velocity continuation
process have the correct dynamic behavior. In particular, a special 
boundary value problem with the zero-offset velocity continuation equation
produces the solution identical to the conventional Kirchoff time migration.}

\begin{comment}
It is important to note that although the velocity continuation result
could be achieved in principle by using prestack residual migration in
Kirchhoff \cite[]{Etgen.sepphd.68} or Stolt \cite[]{GEO61-02-06050607}
formulation, the first is evidently inferior in efficiency, and the
second is not convenient for velocity analysis across different
offsets, because it mixes them in the Fourier domain.
\end{comment}

\section{KINEMATICS OF VELOCITY CONTINUATION}

From the kinematic point of view, it is convenient to describe the
reflector as a locally smooth surface $z = z(x)$, where $z$ is the
depth, and $x$ is the point on the surface ($x$ is a two-dimensional
vector in the 3-D problem). The image of the reflector obtained after
a common-offset prestack migration with a half-offset $h$ and a
constant velocity $v$ is the surface $z = z(x;h,v)$. Appendix A
provides the derivations of the partial differential equation
describing the image surface in the depth-midpoint-offset-velocity
space. The purpose of this section is to discuss the laws of kinematic
transformations implied by the velocity continuation equation. Later
in this paper, I obtain dynamic analogs of the kinematic
relationships in order to describe the continuation of migrated
sections in the velocity space.

The kinematic equation for prestack velocity continuation, derived in
Appendix A, takes the following form:
\begin{equation}
{{\partial \tau} \over {\partial v}} = 
v\,\tau\,\left({{\partial \tau} \over {\partial x}}\right)^2 +
{{h^2} \over {v^3\,\tau}}\,- 
\frac{h^2 v}{\tau}\,
\left({{\partial \tau} \over {\partial x}}\right)^2\,
\left({{\partial \tau} \over {\partial h}}\right)^2\;.
\label{eq:eikonal} 
\end{equation}
Here $\tau$ denotes the one-way vertical traveltime $\left(\tau = {z
\over v}\right)$. The right-hand side of equation (\ref{eq:eikonal})
consists of three distinctive terms. Each has its own geophysical meaning. The first term is the only one remaining
when the half-offset $h$ equals zero. This term corresponds to the procedure of
{\em zero-offset residual migration}.  Setting the traveltime dip to
zero eliminates the first and third terms, leaving the second,
dip-independent one. One can associate the second term with the process of
{\em residual normal moveout}. The third term is both dip- and offset-
dependent. The process that it describes is {\em residual dip
moveout}. It is convenient to analyze each of the three processes
separately, evaluating their contributions to the cumulative process
of prestack velocity continuation.

\subsection{Kinematics of Zero-Offset Velocity Continuation}
%%%%%%%%%%%%%%%%%%%%%%%%%%%%%%%%%%%%%%%%%%%%%%%%%%%%%%
The kinematic equation for zero-offset velocity continuation is
\begin{equation}
{{\partial \tau} \over {\partial v}} = 
v\,\tau\,\left({{\partial \tau} \over {\partial x}}\right)^2\;.
\label{eq:POMeikonal} 
\end{equation}
The typical boundary-value problem associated with it is to find the
traveltime surface $\tau_2(x_2)$ for a constant velocity $v_2$, given the
traveltime surface $\tau_1(x_1)$ at some other velocity $v_1$. Both surfaces
correspond to the reflector images obtained by time migrations with the
specified velocities.  When the migration velocity approaches zero, post-stack
time migration approaches the identity operator. Therefore, the case of $v_1 =
0$ corresponds kinematically to the zero-offset (post-stack) migration, and
the case of $v_2 = 0$ corresponds to the zero-offset modeling (demigration).
\new{The variable $x$ in equation~(\ref{eq:POMeikonal}) describes both the surface
midpoint coordinate and the subsurface image coordinate. One of them is
continuously transformed into the other in the velocity continuation process.}

The appropriate mathematical method of solving the kinematic
problem posed above is the method of characteristics \cite[]{kurant2}. The
characteristics of equation (\ref{eq:POMeikonal}) are the trajectories
followed by individual points of the reflector image in the velocity
continuation process. These trajectories are called {\em velocity rays} 
\cite[]{me,them,adler}. Velocity rays are defined by the system of ordinary
differential equations derived from (\ref{eq:POMeikonal}) according to the
\new{Hamilton-Jacobi theory}:
\begin{eqnarray}
{{{dx} \over {dv}} = - 2\,v\,\tau\,\tau_x} & , &
{{{d\tau} \over {dv}} = - \tau_v}\;,
\label{eq:velray1} \\
{{{d\tau_x} \over {dv}} = v\,\tau_x^3} & , &
{{{d\tau_v} \over {dv}} = \left(\tau + v\,\tau_v\right)\,\tau_x^2}\;,
\label{eq:velray2} 
\end{eqnarray}
\new{where $\tau_x$ and $\tau_v$ are the phase-space parameters}.
An additional constraint for $\tau_x$ and $\tau_v$
follows from equation (\ref{eq:POMeikonal}), rewritten in the form
\begin{equation}
\tau_v = v\,\tau\,\tau_x^2\;. 
\label{eq:equiveikonal} 
\end{equation}
%One can easily solve the system of equations (\ref{eq:velray1}) and
%(\ref{eq:velray2}) by the classic mathematical methods for ordinary
%differential equations. 
The general solution of the system of
equations~(\ref{eq:velray1}-\ref{eq:velray2}) takes the
parametric form
\begin{eqnarray}
x(v) & = & A - C v^2\;,\quad
\tau^2(v) = B - C^2\,v^2\;,
\label{eq:velrayg1} \\ 
\tau_x(v) & = & {C \over {\tau(v)}}\;,\quad
\tau_v(v) = {{C^2\,v} \over {\tau(v)}}\;,
\label{eq:velrayg2} 
\end{eqnarray}
where $A$, $B$, and $C$ are constant along each individual velocity
ray. These three constants are determined from the boundary conditions
as
\begin{equation}
A = x_1 + v_1^2\,\tau_1\,{{\partial \tau_1} \over {\partial x_1}} = x_0\;,
\label{eq:a} 
\end{equation}
\begin{equation}
B = \tau_1^2\,\left(1 + v_1^2\,
\left({{\partial \tau_1} \over {\partial x_1}}\right)^2\right) = \tau_0^2\;,
\label{eq:b} 
\end{equation}
\begin{equation}
C = \tau_1\,{{\partial \tau_1} \over {\partial x_1}} = 
\tau_0\,{{\partial \tau_0} \over {\partial x_0}}\;,
\label{eq:c} 
\end{equation}
where $\tau_0$ and $x_0$ correspond to the zero velocity (unmigrated
section), while $\tau_1$ and $x_1$ correspond to the velocity $v_1$.
% Equations (\ref{eq:a}), (\ref{eq:b}), and (\ref{eq:c}) have a clear
%geometric meaning illustrated in Figure \ref{fig:vlczor}. 
The
simple relationship between the midpoint derivative of the vertical
traveltime and the local dip angle $\alpha$ (appendix A),
\begin{equation}
{{\partial \tau} \over {\partial x}} = 
{{\tan{\alpha}} \over v}\;,
\label{eq:dtaudx} 
\end{equation}
shows that equations (\ref{eq:a}) and (\ref{eq:b}) are precisely equivalent
to the evident geometric relationships (Figure~\ref{fig:vlczor})
\begin{equation}
x_1 + v_1\,\tau_1\,\tan{\alpha} = x_0\;,
\;{\tau_1 \over {\cos{\alpha}}} = \tau_0\;.
\label{eq:evident}
\end{equation}
Equation (\ref{eq:c}) states that the points on a velocity ray correspond
to a single reflection point, constrained by the values of $\tau_1$,
$v_1$, and $\alpha$.  As follows from equations (\ref{eq:velrayg1}), the
projection of a velocity ray to the time-midpoint plane has the
parabolic shape $x(\tau) = A + (\tau^2 - B) / C$, which has been
noticed by \cite{GEO46-05-07170733}. On the depth-midpoint plane, the
velocity rays have the circular shape $z^2(x) = (A - x)\,B / C - (A -
x)^2$, described by \cite{them} as ``Thales circles.''

\inputdir{XFig}

\sideplot{vlczor}{width=0.9\textwidth}{Zero-offset reflection in a
  constant velocity medium (a scheme).}

For an example of kinematic continuation by velocity rays, let us
consider the case of a point diffractor. If the diffractor location in
the subsurface is the point ${x_d,z_d}$, then the reflection traveltime at
zero offset is defined from Pythagoras's theorem as the hyperbolic
curve
\begin{equation}
\tau_0(x_0) = {{\sqrt{z_d^2 + (x_0 - x_d)^2}} \over v_d}\;,
\label{eq:dift0}
\end{equation}
where $v_d$ is half of the actual velocity. Applying equations
(\ref{eq:velrayg1}) produces the following mathematical expressions
for the velocity rays:
\begin{eqnarray}
x(v) & = & x_d\,{v^2 \over v_d^2} + 
x_0\,\left(1 -  {v^2 \over v_d^2}\right)\;,
\label{eq:difrayg1} \\ 
\tau^2(v) & = & \tau_d^2 + {{(x_0 - x_d)^2} \over v_d^2}\,
\left(1 -  {v^2 \over v_d^2}\right)\;,
\label{eq:difrayg2} 
\end{eqnarray}
where $\tau_d = {z_d \over v_d}$.
Eliminating $x_0$ from the system of equations (\ref{eq:difrayg1}) and
(\ref{eq:difrayg2}) leads to the expression for the velocity continuation
``wavefront'': 
\begin{equation}
\tau(x)=\sqrt{\tau_d^2 + {{(x - x_d)^2} \over {v_d^2 - v^2}}}\;.
\label{eq:diffront}
\end{equation}
For the case of a point diffractor, the wavefront corresponds precisely
to the summation path of the residual migration operator
\cite[]{GEO50-01-01100126}. It has a hyperbolic shape when $v_d > v$
(undermigration) and an elliptic shape when $v_d < v$
(overmigration). The wavefront collapses to a point when the velocity
$v$ approaches the actual effective velocity $v_d$. At zero
velocity, $v=0$, the wavefront takes the familiar form of the post-stack migration
hyperbolic summation path. The form of the velocity rays and wavefronts
is illustrated in the left plot of Figure \ref{fig:vlcvrs}.

\inputdir{Sage}

\plot{vlcvrs}{width=6in}{Kinematic
velocity continuation in the post-stack migration domain. Solid lines
denote wavefronts: reflector images for different migration
velocities; dashed lines denote velocity rays. a: the case of a
point diffractor. b: the case of a dipping plane reflector.}

Another important example is the case of a dipping plane reflector. For
simplicity, let us put the origin of the midpoint coordinate $x$ at the point
of the plane intersection with the surface of observations. In this case, the
depth of the plane reflector corresponding to the surface point $x$ has the
simple expression
\begin{equation}
z_p(x) = x\,\tan{\alpha}\;,
\label{eq:plane}
\end{equation}
where $\alpha$ is the dip angle. The zero-offset reflection traveltime
$\tau_0(x_0)$ 
is the plane with a changed angle. It can be expressed as
\begin{equation}
\tau_0(x_0) = p\,x_0\;,
\label{eq:plt0}
\end{equation}
where $p = {{\sin{\alpha}}\over v_p}$, and $v_p$ is half of the actual
velocity. Applying formulas (\ref{eq:velrayg1}) leads to the following
parametric expression for the velocity rays:
\begin{eqnarray}
x(v) & = & x_0\,(1 -  p^2\,v^2)\;,
\label{eq:plrayg1} \\ 
\tau(v) & = & p\,x_0\,\sqrt{1 -  p^2\,v^2}\;.
\label{eq:plrayg2} 
\end{eqnarray}
Eliminating $x_0$ from the system of equations (\ref{eq:plrayg1}) and
(\ref{eq:plrayg2}) shows that the velocity continuation wavefronts are
planes with a modified angle:
\begin{equation}
\tau(x)={{p\,x} \over {\sqrt{1 -  p^2\,v^2}}}\;.
\label{eq:plfront}
\end{equation}
The right plot of Figure \ref{fig:vlcvrs} shows the geometry of the
kinematic velocity continuation for the case of a plane reflector.

\subsection{Kinematics of Residual NMO}
%%%%%%%%%%%%%%%%%%%%%%%%%%%%%%%%%%
The residual NMO differential equation is the second term in
equation~(\ref{eq:eikonal}):
\begin{equation}
{{\partial \tau} \over {\partial v}} = 
{{h^2} \over {v^3\,\tau}}\;.
\label{eq:ResNMOeikonal} 
\end{equation}
Equation (\ref{eq:ResNMOeikonal}) does not depend on the midpoint
$x$. This fact indicates the one-dimensional nature of normal
moveout. The general solution of equation (\ref{eq:ResNMOeikonal}) is
obtained by simple integration. It takes the form
\begin{equation}
\tau^2(v) = C - {h^2 \over v^2} = \tau_1^2 + 
h^2\,\left({1 \over v_1^2} - {1 \over v^2}\right)\;,
\label{eq:ResNMO} 
\end{equation}
where $C$ is an arbitrary velocity-independent constant, and I have chosen the
constants $\tau_1$ and $v_1$ so that $\tau(v_1) = \tau_1$.
\new{Equation~(\ref{eq:ResNMO}) is applicable only for $v$ different from zero.}

For the case of a point diffractor, equation (\ref{eq:ResNMO}) easily
combines with the zero-offset solution (\ref{eq:diffront}). The result
is a simplified approximate version of the prestack residual migration
summation path:
\begin{equation}
\tau(x)=\sqrt{\tau_d^2 + 
{{(x - x_d)^2} \over {v_d^2 - v^2}} +
h^2\,\left({1 \over v_d^2} - {1 \over v^2}\right)}\;.
\label{eq:PSRMfront}
\end{equation}
Summation paths of the form (\ref{eq:PSRMfront}) for a set of diffractors
with different depths are plotted in Figures \ref{fig:vlcve1} and
\ref{fig:vlcve2}. The parameters chosen in these plots allow a direct
comparison with Etgen's Figures 2.4 and 2.5 \cite[]{Etgen.sepphd.68},
based on the exact solution and reproduced in Figures \ref{fig:vlcve3} and
\ref{fig:vlcve4}. The comparison shows that the approximate
solution (\ref{eq:PSRMfront}) captures the main features of the prestack
residual migration operator, except for the residual DMO cusps
appearing in the exact solution when the diffractor depth is smaller
than the offset.

\sideplot{vlcve1}{width=\textwidth}{Summation paths of the simplified
  prestack residual migration for a series of depth diffractors.
  Residual slowness $v/v_d$ is 1.2; half-offset $h$ is 1 km. This
  figure is to be compared with Etgen's Figure 2.4, reproduced in
  Figure~\ref{fig:vlcve3}.}

\sideplot{vlcve2}{width=\textwidth}{Summation paths of the simplified
  prestack residual migration for a series of depth diffractors.
  Residual slowness $v/v_d$ is 0.8; offset $h$ is 1 km. This figure is
  to be compared with Etgen's Figure 2.5, reproduced in
  Figure~\ref{fig:vlcve4}.}

Neglecting the residual DMO term in residual migration is
approximately equivalent in accuracy to neglecting the DMO step in
conventional processing. Indeed, as follows from the geometric analog
of equation (\ref{eq:eikonal}) derived in Appendix A
[equation~(\ref{eq:zeikonal})], dropping the residual
DMO term corresponds to the condition
 \begin{equation}
\tan^2{\alpha}\,\tan^2{\theta} \ll 1\;,
\label{eq:tatg}
\end{equation}
where $\alpha$ is the dip angle, and $\theta$ is the reflection angle.
As shown by \cite{GEO45-12-17531779}, the conventional processing
sequence without the DMO step corresponds to the separable
approximation of the double-square-root equation (\ref{eq:DSR}):
\begin{equation}
\sqrt{1 -  v^2\,\left({{\partial t} \over {\partial s}}\right)^2} +
\sqrt{1 -  v^2\,\left({{\partial t} \over {\partial r}}\right)^2}
\approx
2\,\sqrt{1 -  v^2\,\left({{\partial t} \over {\partial x}}\right)^2} +
2\,\sqrt{1 -  v^2\,\left({{\partial t} \over {\partial h}}\right)^2} -
2\;,
\label{eq:Sep}
\end{equation}
where $t$ is the reflection traveltime, and $s$ and $r$ are the source and
receiver coordinates: $s=x-h$, $r=x+h$.
In geometric terms, approximation (\ref{eq:Sep}) transforms to
\begin{equation}
\cos{\alpha}\,\cos{\theta}
\approx
\sqrt{1 -  \sin^2{\alpha}\,\cos^2{\theta}} +
\sqrt{1 -  \sin^2{\theta}\,\cos^2{\alpha}} - 1\;.
\label{eq:Sepgeom}
\end{equation}
Taking the difference of the two sides of
equation~(\ref{eq:Sepgeom}), one can estimate its accuracy by the
first term of the Taylor series for small $\alpha$ and $\theta$. The
estimate is ${3 \over 4}\,\tan^2{\alpha}\,\tan^2{\theta}$
\cite[]{GEO45-12-17531779}, which agrees qualitatively with
(\ref{eq:tatg}). Although approximation (\ref{eq:PSRMfront}) fails in situations
where the dip moveout correction is necessary, it is significantly
more accurate than the 15-degree approximation of the
double-square-root equation, implied in the migration velocity
analysis method of \cite{GEO49-10-16641674} and \cite{abma}.  The
15-degree approximation
\begin{equation}
\sqrt{1 -  v^2\,\left({{\partial t} \over {\partial s}}\right)^2} +
\sqrt{1 -  v^2\,\left({{\partial t} \over {\partial r}}\right)^2}
\approx
2 - {v^2 \over 2}\,
\left(  \left({{\partial t} \over {\partial s}}\right)^2 +
        \left({{\partial t} \over {\partial r}}\right)^2\right)
\label{eq:FDeg}
\end{equation}
corresponds geometrically to the equation 
\begin{equation}
2\,\cos{\alpha}\,\cos{\theta}
\approx
{{3 + \cos{2\alpha}\,\cos{2\theta}} \over 2}\;.
\label{eq:FDgeom}
\end{equation} 
Its estimated accuracy (from the first term of the Taylor series)
is ${1 \over 8}\,\tan^2{\alpha} + {1 \over
8}\,\tan^2{\theta}$.  Unlike the separable approximation, which is
accurate separately for zero offset and zero dip, the 15-degree
approximation fails at zero offset in the case of a steep dip and at zero
dip in the case of a large offset.

\subsection{Kinematics of Residual DMO}   
%%%%%%%%%%%%%%%%%%%%%%%%%%%%%%%%%%
The partial differential equation for kinematic residual DMO is the
third term in equation~(\ref{eq:eikonal}):
\begin{equation}
{{\partial \tau} \over {\partial v}} = 
- {{h^2 v} \over {\tau}}\,
\left({{\partial \tau} \over {\partial x}}\right)^2\,
\left({{\partial \tau} \over {\partial h}}\right)^2\;.
\label{eq:ResDMOeikonal} 
\end{equation}
It is more convenient to consider the residual dip-moveout process
coupled with residual normal moveout. \cite{Etgen.sepphd.68} describes
this procedure as the cascade of inverse DMO with the initial velocity
$v_0$, residual NMO, and DMO with the updated velocity $v_1$. The
kinematic equation for residual NMO+DMO is the sum of the two terms in
(\ref{eq:eikonal}):
\begin{equation}
{{\partial \tau} \over {\partial v}} = 
{{h^2} \over {v^3\,\tau}}\,\left(1-v^4\,
\left({{\partial \tau} \over {\partial x}}\right)^2\,
\left({{\partial \tau} \over {\partial h}}\right)^2\right)\;.
\label{eq:DMONMOeikonal} 
\end{equation}
\begin{comment}
If the boundary data for equation (\ref{eq:DMONMOeikonal}) are on a
common-offset gather, it is appropriate to rewrite this equation
purely in terms of the midpoint derivative ${{\partial \tau} \over
{\partial x}}$, eliminating the offset-derivative term ${{\partial
\tau} \over {\partial h}}$. The resultant expression, derived in
Appendix A, has the form
\begin{equation}
v^3\,{{\partial \tau} \over {\partial v}} = 
{{2\,h^2} \over
{\sqrt{\tau^2 + 4\,h^2\,
Q\left(v,{{\partial \tau} \over {\partial x}}\right)} + \tau}}\;,
\label{eq:noth} 
\end{equation}
where 
\begin{equation}
Q(v,\tau_x) = {{\tau_x^2} \over 
{\left(1 + v^2\,\tau_x^2\right)^2}}\;.   
\label{eq:qtx} 
\end{equation}
\end{comment}

\new{The derivation of the residual DMO+NMO kinematics is detailed in
  Appendix B.}  Figure \ref{fig:vlcvcp} illustrates it with the
theoretical impulse response curves. Figure \ref{fig:vlccps} compares the
theoretical curves with the result of an actual cascade of the inverse
DMO, residual NMO, and DMO operators.

\plot{vlcvcp}{width=6in,height=3.5in}{Theoretical
kinematics of the residual NMO+DMO impulse responses for three
impulses. Left plot: the velocity ratio $v_1/v_0$ is $1.333$. Right
plot: the velocity ratio $v_1/v_0$ is $0.833$. In both cases the
half-offset $h$ is 1 km.}

\inputdir{resdmo}

\plot{vlccps}{width=6in,height=3.5in}{The result of
residual NMO+DMO (cascading inverse DMO, residual NMO, and DMO) for
three impulses. Left plot: the velocity ratio $v_1/v_0$ is
$1.333$. Right plot: the velocity ratio $v_1/v_0$ is $0.833$. In both
cases the half-offset $h$ is 1 km.}

\inputdir{Sage}

Figure \ref{fig:vlcvrd} illustrates the residual NMO+DMO velocity
continuation for two particularly interesting cases. The left plot
shows the continuation for a point diffractor. One can see that when
the velocity error is large, focusing of the velocity rays forms a
distinctive loop on the zero-offset hyperbola. The right plot illustrates
the case of a plane dipping reflector. The image of the reflector
shifts both vertically and laterally with the change in NMO
velocity.

\plot{vlcvrd}{width=6in}{Kinematic
velocity continuation for residual NMO+DMO. Solid lines denote
wavefronts: zero-offset traveltime curves; dashed lines denote
velocity rays. a: the case of a point diffractor; the velocity
ratio $v_1/v_0$ changes from $0.9$ to $1.1$. b:
the case of a dipping plane reflector; the velocity
ratio $v_1/v_0$ changes from $0.8$ to $1.2$. In both cases, the
half-offset $h$ is 2 km.}

The full residual migration operator is the chain of
residual zero-offset migration and residual NMO+DMO. I illustrate the
kinematics of this operator in Figures \ref{fig:vlcve3} and \ref{fig:vlcve4},
which are designed to match Etgen's Figures 2.4 and 2.5
\cite[]{Etgen.sepphd.68}. A comparison with Figures \ref{fig:vlcve1} and
\ref{fig:vlcve2} shows that including the residual DMO term affects
the images of objects with the depth smaller than the half-offset
$h$. This term complicates the residual migration operator with cusps.

\sideplot{vlcve3}{width=\textwidth}{Summation paths of prestack
  residual migration for a series of depth diffractors. Residual
  slowness $v/v_d$ is 1.2; half-offset $h$ is 1 km. This figure
  reproduces Etgen's Figure 2.4.}

\sideplot{vlcve4}{width=\textwidth}{Summation paths of prestack
  residual migration for a series of depth diffractors. Residual
  slowness $v/v_d$ is 0.8; half-offset $h$ is 1 km. This figure
  reproduces Etgen's Figure 2.5.}

\section{FROM KINEMATICS TO DYNAMICS}
%%%%%%%%%%%%%%%%%%%%%%%%%%%%%%%%%%%
The theory of characteristics \cite[]{kurant2} states that if a partial
differential equation has the form
\begin{equation}
\sum_{i,j=1}^{n}\,\Lambda_{ij}(\xi_1,\ldots,\xi_n)\,
{{\partial^2 P} \over {\partial \xi_i\,\partial \xi_j}} +
F\left(\xi_1,\ldots,\xi_n,P,
{{\partial P} \over {\partial \xi_1}},\ldots,
{{\partial P} \over {\partial \xi_n}}\right) = 0\;,
\label{eq:gequation}
\end{equation}
where F is some arbitrary function, and if the eigenvalues of the
matrix $\Lambda$ are nonzero, and one of them is different in sign
from the others, then equation (\ref{eq:gequation}) describes a
wave-type process, and its kinematic counterpart is the characteristic
equation
\begin{equation}
\sum_{i,j=1}^{n}\,\Lambda_{ij}(\xi_1,\ldots,\xi_n)\,
{{\partial \psi} \over {\partial \xi_i}}\, 
{{\partial \psi} \over {\partial \xi_j}} = 0
\label{eq:charequation}
\end{equation}
with the characteristic surface 
\begin{equation}
\psi(\xi_1,\ldots,\xi_n) = 0
\label{eq:charsurface}
\end{equation}
corresponding to the wavefront. In velocity continuation problems,
it is appropriate to choose the variable $\xi_1$ to denote the time
$t$, $\xi_2$ to denote the velocity $v$, and the rest of the
$\xi$-variables to denote one or two lateral coordinates $x$. Without
loss of generality, let us set the characteristic surface to be
\begin{equation}
\psi = t - \tau(x;v) = 0\;,
\label{eq:chartau}
\end{equation}
and use the theory of characteristics to reconstruct the main
(second-order) part of the dynamic differential equation from the
corresponding kinematic equations. As in the preceding section, it is
convenient to consider separately the three different components of the prestack
velocity continuation process.

\subsection{Dynamics of Zero-Offset Velocity Continuation}
%%%%%%%%%%%%%%%%%%%%%%%%%%%%%%%%%%%%%%%%%%%%%%%%%%%%%
In the case of zero-offset velocity continuation, the characteristic
equation is reconstructed from equation (\ref{eq:POMeikonal}) to have
the form
\begin{equation}
{{\partial \psi} \over {\partial v}}\,
{{\partial \psi} \over {\partial t}} +
v\,t\,\left({{\partial \psi} \over {\partial x}}\right)^2 = 0\;,
\label{eq:POMchar} 
\end{equation}
where $\tau$ is replaced by $t$ according to
equation~(\ref{eq:chartau}).  According to
equation~(\ref{eq:gequation}), the corresponding dynamic equation is
\begin{equation}
{{\partial^2 P} \over {\partial v\, \partial t}} +
v\,t\,{{\partial^2 P} \over {\partial x^2}} +
F\left(x,t,v,P,
{{\partial P} \over {\partial t}},
{{\partial P} \over {\partial v}},
{{\partial P} \over {\partial x}}
\right) = 0\;,
\label{eq:POMequation} 
\end{equation}
where the function $F$ remains to be defined. The simplest case of $F$
equal to zero corresponds to Claerbout's velocity continuation
equation \cite[]{Claerbout.sep.48.79}, derived in a different way.
\cite{Levin.sep.48.101} provides the dispersion-relation derivation,
conceptually analogous to applying the method of characteristics.

In high-frequency asymptotics, the wavefield $P$ can be
represented by the ray-theoretical (WKBJ) approximation,
\begin{equation}
P(t,x,v) \approx A(x,v)\,f\left(t - \tau(x,v)\right)\;, 
\label{eq:WKB} 
\end{equation}
where $A$ is the amplitude, $f$ is the short (high-frequency) wavelet,
and the function $\tau$ satisfies the kinematic equation
(\ref{eq:POMeikonal}). Substituting approximation (\ref{eq:WKB}) into
the dynamic velocity continuation equation (\ref{eq:POMequation}),
collecting the leading-order terms, and neglecting the $F$ function
leads to the partial differential equation for amplitude transport:
\begin{equation}
{\partial A \over \partial v} = v\,\tau\,\left(2\,
{\partial A \over \partial x}\,
{\partial \tau \over \partial x} + A\,
{\partial^2 \tau \over \partial x^2}\right)\;.
\label{eq:PAMPequation} 
\end{equation}
The general solution of equation (\ref{eq:PAMPequation}) follows from the
theory of characteristics. It takes the form
\begin{equation}
A(x,v) = A(x_0,0)\,\exp{\left(\int_0^{v}\,u\,\tau(x,u)\,
{\partial^2 \tau(x,u) \over \partial x^2}\,du\right)}\;,
\label{eq:PAMPsolution} 
\end{equation}
where the integral corresponds to the
curvilinear integration along the corresponding velocity ray, \new{and $x_0$
corresponds to the
starting point of the ray.}
In the case of a plane dipping reflector, the image of the reflector remains
plane in the velocity continuation process. Therefore, the second
traveltime derivative ${\partial^2 \tau(x,u) \over \partial x^2}$ in
(\ref{eq:PAMPsolution}) equals zero, and the exponential is equal to
one. This means that the amplitude of the image does not change with
the velocity along the velocity rays. This fact does not agree with the
theory of conventional post-stack migration, which suggests
downscaling the image by the ``cosine'' factor $\tau_0 \over
\tau$ \cite[]{GEO46-05-07170733,Levin.sep.48.147}. The simplest way to
include the cosine factor in the velocity continuation equation is to
set the function $F$ to be ${1 \over t}\,{\partial P \over \partial
v}$. The resulting differential equation
\begin{equation}
{{\partial^2 P} \over {\partial v\, \partial t}} +
v\,t\,{{\partial^2 P} \over {\partial x^2}} +
{1 \over t}\,{\partial P \over \partial v} = 0
\label{eq:POMequation2} 
\end{equation}
has the amplitude transport
\begin{equation}
A(x,v) = {\tau_0 \over \tau}\,A(x_0,0)\,
\exp{\left(\int_0^{v}\,u\,\tau(x,u)\,
{\partial^2 \tau(x,u) \over \partial x^2}\,du\right)}\;,
\label{eq:PAMPsolution2} 
\end{equation}
corresponding to the differential equation
\begin{equation}
{\partial A \over \partial v} = v\,\tau\,\left(2\,
{\partial A \over \partial x}\,
{\partial \tau \over \partial x} + A\,
{\partial^2 \tau \over \partial x^2}\right) - 
A\,{1 \over \tau}\,{\partial \tau \over \partial v}\;.
\label{eq:PAMPequation2} 
\end{equation}
Appendix C proves that the time-and-space solution of the dynamic
velocity continuation equation (\ref{eq:POMequation2}) coincides with the
conventional Kirchhoff migration operator.

\begin{comment}
The finite-difference implementation of zero-offset velocity
continuation resembles the implementation of Claerbout's
15-degree equation in a retarded coordinate system
\cite[]{Claerbout.blackwell.76}. This implementation is discussed in
more detail in Appendix C. 
\end{comment}

\subsection{Dynamics of Residual NMO}
%%%%%%%%%%%%%%%%%%%%%%%%%%%%%%%%
According to the theory of characteristics, described in the beginning
of this section, the kinematic residual NMO equation
(\ref{eq:ResNMOeikonal}) corresponds to the dynamic equation of the form
\begin{equation}
{{\partial P} \over {\partial v}} + 
{{h^2} \over {v^3\,t}}\,{{\partial P} \over {\partial t}} 
+ F(h,t,v,P) = 0
\label{eq:ResNMOdyn} 
\end{equation}
\new{with the undetermined function $F$. In the case of $F=0$}, the 
general solution
is easily found to be
\begin{equation}
P(t,h,v) = \phi\left(t^2 + {h^2 \over v^2}\right)\;.
\label{eq:ResNMOsol} 
\end{equation}
where $\phi$ is an arbitrary smooth function.
The combination of dynamic equations (\ref{eq:ResNMOdyn}) and
(\ref{eq:POMequation2}) leads to an approximate prestack velocity
continuation with the residual DMO effect neglected. To accomplish the
combination, one can simply add the term ${{h^2} \over
{v^3\,t}}\,{{\partial^2 P} \over {\partial t^2}}$ from
equation~(\ref{eq:ResNMOdyn}) to the left-hand
side of equation (\ref{eq:POMequation2}). This addition changes the
kinematics of velocity continuation, but does not change the amplitude
properties embedded in the transport equation (\ref{eq:PAMPsolution2}).

\cite{GEO38-04-06350642} and \cite{Hale.sepphd.36} \new{advocate using
  an amplitude correction term in the NMO step. This term can be
  easily added by selecting an appropriate function $F$ in
  equation~(\ref{eq:ResNMOdyn}). The choice
  $F=\frac{h^2}{v^3\,t^2}\,P$ results in the equation}
\begin{equation}
{{\partial P} \over {\partial v}} + 
{{h^2} \over {v^3\,t^2}}\,\left(t\,{{\partial P} \over {\partial t}} 
+ P\right) = 0
\label{eq:ResNMOdyn2} 
\end{equation}
with the general solution
\begin{equation}
P(t,h,v) = \frac{1}{t}\,\phi\left(t^2 + {h^2 \over v^2}\right)\;,
\label{eq:ResNMOsol} 
\end{equation}
\new{which has the Dunkin-Levin amplitude correction term.}

\subsection{Dynamics of Residual DMO}
%%%%%%%%%%%%%%%%%%%%%%%%%%%%%%%%
The case of residual DMO complicates the building of a dynamic
equation because of the essential nonlinearity of the kinematic
equation (\ref{eq:ResDMOeikonal}). One possible way to linearize the
problem is to increase the order of the equation. In this case, the
resultant dynamic equation would include a term that has the
second-order derivative with respect to velocity $v$. Such an equation
describes two different modes of wave propagation and requires
additional initial conditions to separate them. Another possible way
to linearize equation (\ref{eq:ResDMOeikonal}) is to approximate it at
small dip angles.
%For example, one
%can obtain a recursively accurate approximation by a continued
%fraction expansion of the square root in equation (\ref{eq:ResDMOeikonal}),
%analogously to Muir's method in conventional finite-difference
%migration \cite[]{Claerbout.blackwell.85}. 
In this case, the dynamic
equation would contain only the first-order derivative with respect to
the velocity and high-order derivatives with respect to the other
parameters. The third, and probably the most attractive, method is to
change the domain of consideration. For example, one could switch from
the common-offset domain to the domain of offset dip. This
method implies a transformation similar to slant stacking of
common-midpoint gathers in the post-migration domain in order to
obtain the local offset dip information. Equation (\ref{eq:ResDMOeikonal})
transforms, with the help of the results from Appendix A, to the form
\begin{equation}
v^3\,{{\partial \tau} \over {\partial v}} = 
{{\tau\,\sin^2{\theta}} \over
{\cos^2{\alpha} - \sin^2{\theta}}}\;,
\label{eq:noh} 
\end{equation}
with
\begin{equation}
\cos^2{\alpha} = \left(1 + v^2 \,
\left({{\partial \tau} \over {\partial x}}\right)^2\right)^{-1}\;,
\label{eq:cos2a} 
\end{equation}
and
\begin{equation}
\sin^2{\alpha} = v^2\,
\left({{\partial \tau} \over {\partial h}}\right)^2\,
\left(1 + v^2 \,
\left({{\partial \tau} \over {\partial h}}\right)^2\right)^{-1}\;.
\label{eq:sin2g} 
\end{equation}
For a constant offset dip $\tan{\theta} = v\,{{\partial \tau} \over
{\partial h}}$, the dynamic analog of equation (\ref{eq:noh}) is the
third-order partial differential equation
\begin{equation}
v\,  \cot^2{\theta}\,{{\partial^3 P} \over {\partial t^2\, \partial v}} -
v^3\,{{\partial^3 P} \over {\partial x^2\, \partial v}}
+ t\,{{\partial^3 P} \over {\partial t^2\, \partial v}} +
v^2\,t\,{{\partial^3 P} \over {\partial x^2\, \partial t}} = 0\;.
\label{eq:ResDMOdyn} 
\end{equation}
Equation (\ref{eq:ResDMOdyn}) does not strictly comply with the theory of
second-order linear differential equations. Its properties and
practical applicability require further research.

\section{Conclusions}
%%%%%%%%%%%%%%%%%%%%%
I have derived kinematic and dynamic equations for residual time migration
in the form of a continuous velocity continuation process. This
derivation explicitly decomposes prestack
velocity continuation into three parts corresponding to
zero-offset continuation, residual NMO, and residual DMO. These three
parts can be treated separately both for simplicity of theoretical
analysis and for practical purposes. It is important to note that in
the case of a three-dimensional migration, all three components of
velocity continuation have different dimensionality. Zero-offset
continuation is fully 3-D. It can be split into two 2-D continuations
in the in- and cross-line directions. Residual DMO is a
two-dimensional common-azimuth process. Residual NMO is a 1-D
single-trace procedure.

The dynamic properties of zero-offset velocity continuation are
precisely equivalent to those of conventional post-stack migration
methods such as Kirchhoff migration. Moreover, the Kirchhoff migration
operator coincides with the integral solution of the velocity
continuation differential equation for continuation from the zero
velocity plane.

This rigorous theory of velocity continuation gives us new insights into the
methods of prestack migration velocity analysis. Extensions to the case of
depth migration in a variable velocity background are developed by
\cite{hong} and \cite{adler}. \new{A practical application of
  velocity continuation to migration velocity analysis is demonstrated in the
  companion paper} \cite[]{second}, \new{where the general theory is used to
  design efficient and practical algorithms.}

\section{Acknowledgments}
%%%%%%%%%%%%%%%%%%%%%%%%%
This work was completed when the author was a member of the Stanford
Exploration Project (SEP) at Stanford University. The financial
support was provided by the SEP sponsors.

I thank Bee Bednar, Biondo Biondi, Jon Claerbout, Sergey Goldin, Bill Harlan,
David Lumley, and Bill Symes for useful and stimulating discussions.
\new{Paul Fowler, Hugh Geiger, Samuel Gray, and one anonymous reviewer
  provided valuable suggestions that improved the quality of the paper.}

\bibliographystyle{seg}
\bibliography{SEP2,paper,spec,velcon,SEG}

\append{DERIVING THE KINEMATIC EQUATIONS}
%%%%%%%%%%%%%%%%%%%%%%%%%%%%%%%%%%%%%%%%
The main goal of this appendix is to derive the partial differential
equation describing the image surface in a
depth-midpoint-offset-velocity space.

\inputdir{XFig}

\sideplot{vlcray}{width=0.9\textwidth}{Reflection rays in a constant
  velocity medium (a scheme).}

The derivation starts with observing a simple geometry of reflection
in a constant-velocity medium, shown in Figure \ref{fig:vlcray}. The
well-known equations for the apparent slowness
\begin{equation}
{{\partial t} \over {\partial s}} \,=\,
{ {\sin{\alpha_1}} \over {v}}\;,
\label{eq:snell1}
\end{equation}
\begin{equation}
{{\partial t} \over {\partial r}} \,=\, 
{{\sin{\alpha_2}} \over {v}}  
\label{eq:snell2}
\end{equation} 
relate the first-order traveltime derivatives for the reflected waves
to the emergence angles of the incident and reflected rays. Here $s$
stands for the source location at the surface, $r$ is the receiver
location, $t$ is the reflection traveltime, $v$ is the constant
velocity, and $\alpha_1$ and $\alpha_2$ are the angles shown in Figure
\ref{fig:vlcray}. Considering the traveltime derivative with respect to
the depth of the observation surface $z$ shows that the
contributions of the two branches of the reflected ray, added
together, form the equation
\begin{equation}
- {{\partial t} \over {\partial z}} \,=\,
{{\cos{\alpha_1}} \over {v}} +
{{\cos{\alpha_2}} \over {v}}\;.
\label{eq:snell3}
\end{equation}
It is worth mentioning that the elimination of angles from equations
(\ref{eq:snell1}), (\ref{eq:snell2}), and (\ref{eq:snell3}) leads to
the famous {\em double-square-root equation,}
\begin{equation}
- v\,{{\partial t} \over {\partial z}} \,=\,
\sqrt{1 -  v^2\,\left({{\partial t} \over {\partial s}}\right)^2} +
\sqrt{1 -  v^2\,\left({{\partial t} \over {\partial r}}\right)^2}\;,
\label{eq:DSR}
\end{equation}
published in the Russian literature by \cite{alekseev} and commonly
used in the form of a pseudo-differential dispersion relation
\cite[]{Clayton.sep.14.21,Claerbout.blackwell.85} for prestack
migration \cite[]{Yilmaz.sepphd.18,Popovici.sep.84.53}. Considered
locally, equation (\ref{eq:DSR}) is independent of the constant velocity
assumption and enables \new{recursive} prestack downward
continuation of reflected waves in heterogeneous \new{isotropic}
media.

Introducing the midpoint coordinate $x = {{s+ r} \over 2}$ and half-offset
$h = {{r - s} \over 2}$, one can apply the chain rule and elementary
trigonometric equalities to formulas (\ref{eq:snell1}) and
(\ref{eq:snell2}) and transform these formulas to 
\begin{equation}
{{\partial t} \over {\partial x}} \,=\, 
{{\partial t} \over {\partial s}} + 
{{\partial t} \over {\partial r}} \,=\, 
{ {2 \sin{\alpha}\,\cos{\theta}} \over {v}}\;,
\label{eq:snells1}
\end{equation}
\begin{equation}
{{\partial t} \over {\partial h}} \,=\,
{{\partial t} \over {\partial r}} - 
{{\partial t} \over {\partial s}} \,=\, 
{ {2 \cos{\alpha}\,\sin{\theta}} \over {v}} \;,
\label{eq:snells2}
\end{equation}
where $\alpha = {{\alpha_1 + \alpha_2} \over 2}$ is the dip angle, and
$\theta = {{\alpha_2 - \alpha_1} \over 2}$ is the reflection angle
\cite[]{Clayton.sep.14.21,Claerbout.blackwell.85}. Equation
(\ref{eq:snell3}) transforms analogously to
\begin{equation}
- {{\partial t} \over {\partial z}} \,=\,
{{2 \cos{\alpha} \cos{\theta}} \over {v}}\;. 
\label{eq:snells3}
\end{equation}
This form of equation (\ref{eq:snell3}) is used to describe the stretching
factor of the waveform distortion in depth migration \cite[]{Tygel}.

Dividing (\ref{eq:snells1}) and (\ref{eq:snells2}) by
(\ref{eq:snells3}) leads to
\begin{equation}
{{\partial z} \over {\partial x}} \,=\,
- \tan{\alpha}\;, 
\label{eq:snellz1}
\end{equation}
\begin{equation}
{{\partial z} \over {\partial h}} \,=\,
- \tan{\theta}\;.
\label{eq:snellz2}
\end{equation}
Equation~(\ref{eq:snellz2}) is the basis of the angle-gather construction of
\cite{sandf}.
Substituting formulas (\ref{eq:snellz1}) and (\ref{eq:snellz2}) into equation
(\ref{eq:snells3}) yields yet another form of the double-square-root equation:
\begin{equation}
- {{\partial t} \over {\partial z}} \,=\, {2 \over {v}}\,
\left[\sqrt{1 + \left({\partial z} \over {\partial x}\right)^2}\,
\sqrt{1 + \left({\partial z} \over {\partial h}\right)^2}\right]^{-1}\;, 
\label{eq:snellz3}
\end{equation}
which is analogous to the dispersion relationship of Stolt prestack
migration \cite[]{GEO43-01-00230048}.
 
The law of sines in the triangle formed by the incident and reflected
ray leads to the explicit relationship between the traveltime and the
offset:
\begin{equation}
v\,t = 2\,h\,  {{\cos{\alpha_1}+ \cos{\alpha_2}} \over
\sin{\left(\alpha_2-\alpha_1\right)}} = 2\,h\,{\cos{\alpha} \over
\sin{\theta}} \;.
\label{eq:length} 
\end{equation}
An algebraic combination of formulas (\ref{eq:length}), (\ref{eq:snells1}), and
(\ref{eq:snells2}) forms the basic kinematic equation of the offset
continuation theory \cite[]{ofcon}:
\begin{equation}
{{\partial t} \over {\partial h}} \,
\left(t^2 + {{4\,h^2} \over {v^2}}\right)\,=\,
h\,t\,\left({4 \over {v^2}} + 
\left({{\partial t} \over {\partial h}}\right)^2\,-
\left({{\partial t} \over {\partial x}}\right)^2\right)\;.
\label{eq:OCeikonal}
\end{equation}

Differentiating (\ref{eq:length}) with respect to the velocity $v$ yields
\begin{equation}
- v^2\,{{\partial t} \over {\partial v}} = 
2\,h\,{\cos{\alpha} \over \sin{\theta}}\;.
\label{eq:dtdv} 
\end{equation}
Finally, dividing (\ref{eq:dtdv}) by (\ref{eq:snells3}) produces
\begin{equation}
v\,{{\partial z} \over {\partial v}} = 
{h \over {\cos{\theta}\,\sin{\theta}}}\;.
\label{eq:dzdv} 
\end{equation}
Equation (\ref{eq:dzdv}) can be written in a variety of ways with the help
of an explicit geometric relationship between the half-offset $h$ and
the depth $z$, 
\begin{equation}
h = z\,
{{\sin{\theta}\,\cos{\theta}} \over
{\cos^2{\alpha}-\sin^2{\theta}}}\;, 
\label{eq:z2h} 
\end{equation}
which follows directly from the trigonometry of the triangle in Figure
\ref{fig:vlcray} \cite[]{ofcon}. For example, equation (\ref{eq:dzdv}) can
be transformed to the form obtained by \cite{GEO60-01-01420153}:
\begin{equation}
v\,{{\partial z} \over {\partial v}} = 
{z \over{\cos^2{\alpha}-\sin^2{\theta}}} =
{z \over{\cos{\alpha_1}\,\cos{\alpha_2}}}\;.
\label{eq:liu} 
\end{equation}
In order to separate different factors contributing to the velocity
continuation process, one can transform this equation to the form
\begin{eqnarray}
\nonumber
v\,{{\partial z} \over {\partial v}} & = & 
{z \over {\cos^2{\alpha}}} +
{{h^2} \over z}\,\left(1-\tan^2{\alpha}\,\tan^2{\theta}\right) \\
& = & z\,\left(1 + \left({{\partial z} \over {\partial x}}\right)^2\right) +
{{h^2} \over z}\,\left(1-\left({{\partial z} \over {\partial x}}\right)^2\,
\left({{\partial z} \over {\partial h}}\right)^2\right)\;.
\label{eq:zeikonal} 
\end{eqnarray}
Rewritten in terms of the vertical traveltime $\tau = z/v$, it further
transforms to equation 
\begin{equation}
{{\partial \tau} \over {\partial v}} = 
v\,\tau\,\left({{\partial \tau} \over {\partial x}}\right)^2 +
{{h^2} \over {v^3\,\tau}}\,\left(1 - 
v^4\,
\left({{\partial \tau} \over {\partial x}}\right)^2\,
\left({{\partial \tau} \over {\partial h}}\right)^2\right)\;,
\label{eq:eikonal0} 
\end{equation}
equivalent to equation~(\ref{eq:eikonal}) in the main text. Yet
another form of the kinematic velocity continuation equation follows
from eliminating the reflection angle $\theta$ from equations
(\ref{eq:dzdv}) and (\ref{eq:z2h}). The resultant expression takes the
following form:
\begin{equation}
v\,{{\partial z} \over {\partial v}} = 
{{2\,(z^2 + h^2)} \over
{\sqrt{z^2 + h^2 \sin^2{2\,\alpha}} + z\,\cos{2\,\alpha}}} =
{z \over {\cos^2{\alpha}}} + {{2\,h^2} \over
{\sqrt{z^2 + h^2 \sin^2{2\,\alpha}} + z}}\;.
\label{eq:notheta} 
\end{equation}

\append{Derivation of the residual DMO kinematics}

This appendix derives the kinematical laws for the residual NMO+DMO
transformation in the prestack offset continuation process.

The direct solution of equation (\ref{eq:DMONMOeikonal}) is
nontrivial. A simpler way to obtain this solution is to decompose
residual NMO+DMO into three steps and to evaluate their contributions
separately. Let the initial data be the zero-offset reflection event
$\tau_0(x_0)$. The first step of the residual NMO+DMO is the inverse
DMO operator. One can evaluate the effect of this operator by means of
the offset continuation concept \cite[]{ofcon}. According to this
concept, each point of the input traveltime curve $\tau_0(x_0)$
travels with the change of the offset from zero to $h$ along a special
trajectory, which I call a {\em time ray}. Time rays are parabolic
curves of the form
\begin{equation}
x\left(\tau\right) =  x_0+{{\tau^2-\tau_0^2\left(x_0\right)} \over
{\tau_0\left(x_0\right)\,\tau_0'\left(x_0\right)}}\;,
\label{eq:xoftau}
\end{equation}
with the final points constrained by the equation
\begin{equation}
h^2 = \tau^2\,{{\tau^2-\tau_0^2\left(x_0\right)} \over
{\left(\tau_0\left(x_0\right)\,\tau_0'\left(x_0\right)\right)^2}}\;,
\label{eq:hoftau}
\end{equation}
where $\tau_0'\left(x_0\right)$ is the derivative of 
$\tau_0\left(x_0\right)$.
The second step of the cumulative residual NMO+DMO process is the
residual normal moveout. According to equation (\ref{eq:ResNMO}), residual
NMO is a one-trace operation transforming the traveltime $\tau$ to
$\tau_1$ as follows:
\begin{equation}
\tau_1^2 = \tau^2 + h^2\,d\;,
\label{eq:ResNMO2} 
\end{equation}
where
\begin{equation}
d = \left({1 \over v_0^2} - {1 \over v_1^2}\right)\;.
\label{eq:formers}
\end{equation}
The third step is dip moveout corresponding to the new velocity
$v_1$. DMO is the offset continuation from $h$ to zero
offset along the redefined time rays \cite[]{ofcon}
\begin{equation}
x_2\left(\tau_2\right) =  
x + {{h\,X} \over {\tau_1^2\,H}}\,\left(\tau_1^2-\tau_2^2\right)\;,
\label{eq:xoftau2}
\end{equation}
where $H = {{\partial \tau_1} \over {\partial h}}$, and $X =
{{\partial \tau_1} \over {\partial x}}$. 
The end points of the time rays (\ref{eq:xoftau2}) are defined by the
equation
\begin{equation}
\tau_2^2 = - \tau_1^2\,{{\tau_1\,H} \over {h\,X^2}}\;.
\label{eq:tauofh2}
\end{equation}
The partial derivatives of the common-offset traveltimes are
constrained by the offset continuation kinematic equation
\begin{equation}
h\,(H^2 - X^2) = \tau_1\,H\;,
\label{eq:OCequation}
\end{equation}
which is equivalent to equation (\ref{eq:OCeikonal}) in Appendix
A. Additionally, as follows from equations (\ref{eq:ResNMO2}) and the ray
invariant equations from \cite[]{ofcon},
\begin{equation}
\tau_1\,X = \tau\,{{\partial \tau} \over {\partial x}} = 
{{\tau^2\,\tau_0'\left(x_0\right)} \over {\tau_0\left(x_0\right)}}\;.
\label{eq:invariant}
\end{equation}
Substituting~(\ref{eq:xoftau}-\ref{eq:formers}) and
(\ref{eq:OCequation}-\ref{eq:invariant}) into equations
(\ref{eq:xoftau2}) and (\ref{eq:tauofh2}) and performing the algebraic
simplifications yields the parametric expressions for velocity
rays of the residual NMO+DMO process:
\begin{equation}
\left\{
\begin{array}{rcl}
x_2(d) & = & \displaystyle{x_0 + {{h^2\,\tau_0'(x_0)} \over T}\,
\left(1 - {T^2 \over T_2^2(d)}\right)}\;,
\\ 
\\
\tau(d) & = & \displaystyle{{{\tau_1^2(d)} \over {T_2(d)}}}\;,
\end{array}
\right.
\label{eq:ResDMO2}
\end{equation}
where the function
$T\left(h,\tau_0(x_0),\tau_0'\left(x_0\right)\right)$ is defined by
\begin{equation}
T\left(h,\tau,\tau_x\right) = 
{{\tau + \sqrt{\tau^2 + 4\,h^2\,\tau_x^2}} \over 2}\;,
\label{eq:CapT}
\end{equation}
\begin{equation}
T_2(d) = \sqrt{T\left(h,\tau_1^2(d),\tau_0'\left(x_0\right)\,
T\left(h,\tau_0(x_0),\tau_0'\left(x_0\right)\right)\right)}\;,
\label{eq:CapT2}
\end{equation}
and
\begin{equation}
\tau_1^2(d) = \tau_0\,T + d\,h^2\;.
\label{eq:tauofs}
\end{equation}

The last step of the cascade of inverse DMO, residual NMO, and DMO is
illustrated in Figure \ref{fig:vlcvoc}. The three plots in the figure show
the offset continuation to zero offset of the inverse DMO impulse
response shifted by the residual NMO operator. The middle plot
corresponds to zero NMO shift, for which the DMO step collapses the
wavefront back to a point.  Both positive (top plot) and negative
(bottom plot) NMO shifts result in the formation of the specific
triangular impulse response of the residual NMO+DMO operator. As
noticed by \cite{Etgen.sepphd.68}, the size of the triangular
operators dramatically decreases with the time increase.  For large
times (pseudo-depths) of the initial impulses, the operator collapses
to a point corresponding to the pure NMO shift.

\inputdir{Sage}

\sideplot{vlcvoc}{width=0.9\textwidth}{Kinematic residual NMO+DMO
  operators constructed by the cascade of inverse DMO, residual NMO,
  and DMO. The impulse response of inverse DMO is shifted by the
  residual NMO procedure. Offset continuation back to zero offset
  forms the impulse response of the residual NMO+DMO operator. Solid
  lines denote traveltime curves; dashed lines denote the offset
  continuation trajectories (time rays). Top plot: $v_1/v_0 = 1.2$.
  Middle plot: $v_1/v_0 = 1$; the inverse DMO impulse response
  collapses back to the initial impulse. Bottom plot: $v_1/v_0 = 0.8$.
  The half-offset $h$ in all three plots is 1 km.}

\append{INTEGRAL VELOCITY CONTINUATION AND KIRCHHOFF MIGRATION}
%%%%%%%%%%%%%%%%%%%%%%%%%%%%%%%%%%%%%%%%%%%%%%%%%%%%%%%%%%%%%%
The main goal of this appendix is to prove the equivalence between the
result of zero-offset velocity continuation from zero velocity and
conventional post-stack migration. After solving the velocity
continuation problem in the frequency domain, I transform the solution
back to the time-and-space domain and compare it with the conventional
Kirchhoff migration operator \cite[]{GEO43-01-00490076}. The frequency-domain
solution has its own value, because it forms the basis for an efficient
spectral algorithm for velocity continuation \cite[]{second}. 
 
Zero-offset migration based on velocity continuation is the solution
of the boundary problem for equation (\ref{eq:POMequation2}) with the
boundary condition
\begin{equation}
\left.P\right|_{v=0} = P_0\;,
\label{eq:POMbound} 
\end{equation}
where $P_0(t_0,x_0)$ is the zero-offset seismic section, and
$P(t,x,v)$ is the continued wavefield. In order to find the solution
of the boundary problem composed of (\ref{eq:POMequation2}) and
(\ref{eq:POMbound}), it is convenient to apply the function
transformation $R(t,x,v) = t\,P(t,x,v)$, the time coordinate
transformation $\sigma = t^2/2$, and, finally, the double Fourier
transform over the squared time coordinate $\sigma$ and the spatial
coordinate $x$:
\begin{equation}
\widehat{R}(v) = \int \int\,P(t,x,v)\,
\exp(i \Omega \sigma - i k x )\,t^2\,dt\,dx\;.
\label{eq:FTK} 
\end{equation}
With the change of domain, equation (\ref{eq:POMequation2}) transforms
to the ordinary differential equation
\begin{equation}
{{d\,\widehat{R}} \over {d\,v}} = 
i\,{k^2 \over \Omega}\,v\,\widehat{R}\;,
\label{eq:ODE} 
\end{equation}
and the boundary condition (\ref{eq:POMbound}) transforms to the initial
value condition
\begin{equation}
\widehat{R}(0) = \widehat{R}_0\;, 
\label{eq:ODEbound} 
\end{equation}
where 
\begin{equation}
\widehat{R}_0 = \int \int\,P_0(t_0,x_0)\,
\exp(i \Omega \sigma_0 - i k x_0 )\,t_0^2\,dt_0\,dx_0\;,
\label{eq:FTK0}
\end{equation}
and $\sigma_0 = t_0^2/2$.  The unique solution of the initial value
(Cauchy) problem (\ref{eq:ODE}) - (\ref{eq:ODEbound}) is easily found to be
\begin{equation}
\widehat{R}(v) = \widehat{R}_0\,
\exp\left( i\,{{k^2} \over {2\,\Omega}}\,v^2\right)\;.
\label{eq:ODEsolution} 
\end{equation}

In the transformed domain, velocity continuation appears to be a unitary
phase-shift operator. An immediate consequence of this remarkable fact is the
cascaded migration decomposition of post-stack migration
\cite[]{GEO52-05-06180643}:
\begin{equation}
\exp\left( i\,{{k^2} \over {2\,\Omega}}\,
(v_1^2 +  \cdots + v_n^2)\right) =
\exp\left( i\,{{k^2} \over {2\,\Omega}}\,v_1^2\right)\,\cdots\,
\exp\left( i\,{{k^2} \over {2\,\Omega}}\,v_n^2\right)\;.
\label{eq:cascaded} 
\end{equation}
Analogously, three-dimensional post-stack migration is decomposed
into the two-pass procedure \cite[]{GPR31-01-00340056}:
\begin{equation}
\exp\left( i\,{{k_1^2+k_2^2} \over {2\,\Omega}}\,v^2\right) =
\exp\left( i\,{{k_1^2} \over {2\,\Omega}}\,v^2\right)\,
\exp\left( i\,{{k_2^2} \over {2\,\Omega}}\,v^2\right)\;.
\label{eq:two-pass}
\end{equation}

The inverse double Fourier transform of both sides of equality
(\ref{eq:ODEsolution}) yields the integral (convolution) operator
\begin{equation}
P(t,x,v) = \int\int\,P_0(t_0,x_0)\,K(t_0,x_0;t,x,v)\,dt_0\,dx_0\;,
\label{eq:convolution}
\end{equation}
with the kernel $K$ defined by
\begin{equation}
K = {{t_0^2/t} \over {(2\,\pi)^{m+1}}}\,
\int\int\,\exp\left(
i\,{{k^2} \over {2\,\Omega}}\,v^2 + ik\,(x - x_0) - 
{{i\Omega} \over 2}\,(t^2 - t_0^2)
\right)\,dk\,d\Omega\;,
\label{eq:kernel}
\end{equation}
where $m$ is the number of dimensions in $x$ and $k$ ($m$ equals $1$
or $2$). The inner integral on the wavenumber axis $k$ in formula
(\ref{eq:kernel}) is a known table integral \cite[]{grad}. Evaluating this
integral simplifies equation (\ref{eq:kernel}) to the form
\begin{equation}
K = {{t_0^2/t} \over {(2\,\pi)^{m/2+1}\,v^m}}\,
\int\,(i\Omega)^{m/2}\,\exp\left[
{{i\Omega} \over 2}\,
\left(t_0^2 - t^2 - {{(x - x_0)^2} \over v^2}\right)\right]\,
d\Omega\;.
\label{eq:skernel}
\end{equation}
The term $(i\Omega)^{m/2}$ is the spectrum of the anti-causal
derivative operator ${d \over {d\sigma}}$ of the order $m/2$. Noting
the equivalence
\begin{equation}
\left({\partial \over {\partial \sigma}}\right)^{m/2} =
\left({1 \over t}\,{\partial \over {\partial t}}\right)^{m/2} =
\left({1 \over t}\right)^{m/2}\,
\left({\partial \over {\partial t}}\right)^{m/2}\;,
\label{eq:halfdif}
\end{equation}
which is exact in the 3-D case ($m=2$) and asymptotically correct in
the 2-D case ($m=1$), and applying the convolution theorem
transforms operator (\ref{eq:convolution}) to the form
\begin{equation}
P(t,x,v) = {1 \over {(2\,\pi)^{m/2}}}\,\int\,
{{\cos{\alpha}} \over {(v\,\rho)^{m/2}}}\,
\left(- {\partial \over {\partial t_0}}\right)^{m/2}
P_0\left({\rho \over v},x_0\right)\,dx_0\;,
\label{eq:Kirchhoff}
\end{equation}
where $\rho = \sqrt{v^2\,t^2 + (x - x_0)^2}$, and $\cos{\alpha} =
t_0/t$. Operator (\ref{eq:Kirchhoff}) coincides with the Kirchhoff operator
of conventional post-stack time migration \cite[]{GEO43-01-00490076}.

\newpage
%%% Local Variables: 
%%% mode: latex
%%% TeX-master: t
%%% TeX-master: t
%%% TeX-master: t
%%% TeX-master: t
%%% TeX-master: t
%%% TeX-master: t
%%% TeX-master: t
%%% End: 
