\published{SEP report, 80, 467-476 (1994)}

\lefthead{Bevc and Lumley}
\righthead{Anti-aliasing?}
\footer{SEP--80}
\title{When is anti-aliasing needed in Kirchhoff migration?}
\author{Dimitri Bevc and David E. Lumley}
\maketitle

\begin{abstract}We present criteria to determine when numerical integration 
of seismic data will incur operator aliasing. Although there are
many ways to handle operator aliasing, they add expense to the
computational task. This is especially true in three dimensions.
A two-dimensional Kirchhoff migration example illustrates that 
the image zone of interest may not always require anti-aliasing and 
that considerable cost may be spared by not incorporating it.
\end{abstract}

\section{Introduction}
In this paper we establish some rules of thumb as to when anti-aliasing
is required in Kirchhoff migration. The same criteria are applicable to
other processes such as DMO, velocity analysis, and wave-equation datuming.

There are many methods of handling operator aliasing.
\cite{Gray.gp.92} presented a method which involves
low-pass filtering data traces with a variety of pass bands
and then selecting input data from these sets of traces so that 
operator aliasing does not occur. Spatial trace interpolation is another
method of dealing with the operator aliasing problem \cite[]{SDP00-00-05260526}.
A draw back of the latter two methods is increased
data volumes.  Methods which limit the dip or aperture of the operator reduce
aliasing  without increasing the data volume, but at the expense of losing 
high-angle and wide-aperture information.  An attractive and computationally 
efficient method of handling operator aliasing 
has been implemented by \cite{Claerbout.sep.73.371}. 
His dip-dependant triangular weighting method does not require multiple
copies of the data to be kept in memory since the weights are
generated and applied quickly on-the-fly. 

Claerbout's triangular weighting method has been demonstrated to be
efficient for 2-D \cite[]{Bevc.sep.75.91,Bevc.sep.77.295} and 
3-D \cite[]{Lumley.sep.77.1,Lumley.sep.80.447} 
Kirchhoff time and depth migration.  
It has also been successfully adapted to DMO and 
wave-equation datuming operators \cite[]{Blondel.sep.77.49,Bevc.sep.75.137}.
Even though the triangular weighting method is very efficient, it 
still involves an extra computational cost.
%than using no anti-aliasing at all.
When the anti-aliased algorithm is implemented on the
Connection Machine in FORTRAN 90,
calls to an indirect addressing subroutine are required to extract
data points from individual traces for summing into output locations.
These calls turn out to be a bottleneck.
In order to perform an anti-aliased migration with linear interpolation,
six calls to the indirect addressing subroutine are required for each
input trace location. For a 3-D migration, the indirect addressing
is substantial.

Because this anti-aliasing is currently expensive on the CM5,
we are motivated to
determine when we can get away with not using it. While doing away 
with anti-aliasing is generally not a good idea, there are situations in which
we may be able to live without it. For example, if we are running 
trial migrations to determine velocity models we may concentrate
our efforts on portions of the data where operator aliasing is not a
factor. 

After developing criteria which link frequency and dip content of seismic data,
we migrate a 2-D  salt dome data set with and without anti-aliasing.
The examples illustrate the effects of operator aliasing, how it can be 
ameliorated by aperture limitation and triangle weighted migration, and
when anti-aliasing is unnecessary.

\section{OPERATOR ALIASING}
Operator aliasing most often occurs when operator moveout across adjacent
traces exceeds the time sampling rate. Cycle skips can occur 
when the operator is
aliased.  For a moveout curve with slope $dt/dx$, and data with a spatial
Nyquist frequency of $k_n$, temporal frequencies above
$$
\omega = {k_n\over{dt/dx}}
$$
are aliased.
In terms of the mesh spacing $\triangle x$ and operator slope $dt/dx$,
operator aliasing will occur for all frequencies above $f_{op}$, where
$f_{op}$ is given by:
\begin{equation}
f_{op} \; = \; \frac{1}{2 (\frac{dt}{dx}) \triangle x}.
\end{equation}

Defining the maximum stepout as $p = \delta x /\delta t$, the
highest dip frequency in the data is given by 
\begin{equation}
f_d \; = \; \frac{1}{2 p \triangle x}.
\end{equation}
When the stepout is captured by the mesh 
spacing, $\delta x = \triangle x$, and $\delta t = \triangle t$, the
highest unaliased dip frequency is equal to the 
Nyquist frequency $f_n = 1/2\triangle t$. In areas of
economic interest, steep dips are often
present in the data and $f_d > f_n$.

Anti-aliasing is called for when the frequency content of the data, $f_s$, 
falls between 
\begin{equation}
\label{eqn:crit}
f_{op} \; \leq \; f_s \; \leq \; f_d.
\end{equation}
This situation is illustrated in Figure~\ref{fig:spectrum}.

\inputdir{XFig}
\sideplot{spectrum}{width=3in}{Operator aliasing, event dip, and
frequency content of the data.}

\section{GULF OF MEXICO SALT DOME EXAMPLES}
In this section, we migrate a data set from the Gulf of Mexico to illustrate
when anti-aliasing is not required.  Migration is performed using
an implementation of Claerbout's triangle weighted anti-aliasing 
scheme.  Corresponding examples of standard Kirchhoff migration 
with and without angle limitation, are computed for
comparison.

\subsection{Anti-aliased migration}
\inputdir{gulf}
The diffractions from the salt flanks are 
obvious in the near-offset section (Figure~\ref{fig:Gulfaa}a) and can be seen 
to be spatially aliased.  The data exhibit some overall 
speckling because of temporal aliasing. This temporal aliasing is due to 
recording or processing which was performed on the original data before
it arrived at SEP.
The anti-aliased migration is displayed in Figure~\ref{fig:Gulfaa}b. The 
salt flanks are nicely imaged and there is no evidence of operator aliasing.
There are some artifacts due to the temporal aliasing:
temporal aliasing is a different phenomena than operator aliasing and
cannot be ameliorated by modifying the operator.

\plot{Gulfaa}{height=7in}{(a) Near-offset section from the Gulf of
Mexico.  (b) Kirchhoff migration with triangle weighted anti-aliasing.}

\subsection{Aliased migration}
In Figure~\ref{fig:Gulfaper}a the data are migrated without triangular weighting.
The effect of operator aliasing is most evident in the seafloor arrivals.
We see precursors to the actual event. The top of 
the salt dome (earlier than $t = 0.6$ s) is poorly imaged and there is more
overall speckling than in Figure~\ref{fig:Gulfaa}b, suggesting that
the effects of data aliasing are compounded by operator aliasing.
Other prominent operator aliasing artifacts are seen at about
midpoints 13000 to 14000 and time 0.6 s to 0.9 s as cross-cutting dipping
events.  
Figure~\ref{fig:Zoom2} is a comparison of the anti-aliased 
migration and the aliased migration 
(Figure~\ref{fig:Zoom2}a is a closeup of Figure~\ref{fig:Gulfaa}b, and
Figure~\ref{fig:Zoom2}b is a closeup of Figure~\ref{fig:Gulfaper}a).
The seafloor precursor artifacts before $t=0.18$ s and the cross-cutting 
dipping event artifacts are marked.

The operator aliasing has been somewhat contained by limiting the
migration aperture to $45^\circ$ in Figure~\ref{fig:Gulfaper}b;
however, the seafloor event still has
a precursor, the top of the salt dome is still poorly imaged, and there 
is more coherent noise than in Figure~\ref{fig:Gulfaa}b.

\plot{Gulfaper}{height=7in}{(a) Kirchhoff migration
without any aperture limitation or anti-aliasing. The effect of operator
aliasing is noticeable at the seafloor where migration velocity is slow and
where there is significant operator dip. At later times, the migration velocity
is fast and there is not much operator dip, so there is no operator
aliasing.
(b) Kirchhoff migration of Gulf of
Mexico data with $45^\circ$ aperture limitation.  Limiting the aperture
reduces some, but not all, of the operator aliasing at the seafloor.}

\plot{Zoom2}{height=7.5in,width=6.0in}{ Close up of (a) the
anti-aliased migration, and (b) the standard Kirchhoff migration without 
anti-aliasing. The seafloor operator aliasing artifacts and the
dipping artifacts are marked.}

\subsection{Anti-aliasing is not needed to image the salt flank}
The interesting thing about the images migrated without
anti-aliasing is that in all cases the 
salt flank is nicely imaged at late times. This is because in this region
the migration velocity is fast and the operator does not have much dip, so
that $f_{op} > f_s$ and operator aliasing {\it is not} a problem.  
At earlier times, the migration velocity is lower and the operator has
significant dip so that $f_{op} < f_s$ and operator aliasing {\it is} a problem.
%({\it maybe I'll show this by sticking in some numbers some day})

\subsection{Migration with low velocity}
In Figure~\ref{fig:Miglow} the data have been deliberately
migrated with an unrealistic
low velocity of 1000 m/s in order to confirm that operator aliasing
will occur when the criteria of equation~\ref{eqn:crit} are met.
In these examples $f_{op} < f_s$ so that the operator is aliased.
The diffraction from the salt flank is much better imaged in the
anti-aliased migration (Figure~\ref{fig:Miglow}a) than in the aliased
migration (Figure~\ref{fig:Miglow}b).  All of the same operator aliasing 
artifacts  that were visible at early times in Figure~\ref{fig:Gulfaa} are
still present, but now the events at late times also suffer. The
right hand diffraction in Figure~\ref{fig:Miglow}b is weaker and less continuous
than the anti-aliased diffraction in Figure~\ref{fig:Miglow}a.
Some of the diffractions on the left side of Figure~\ref{fig:Miglow}b are
completely lost in the aliased migration.

\plot{Miglow}{height=7in}{Kirchhoff migration of Gulf of Mexico data
with an unrealistically low velocity: (a) with anti-aliasing, and (b) without
anti-aliasing.}

\section{CONCLUSIONS}
We have presented a simple inequality which can be used to determine
whether operator aliasing is a factor in Kirchhoff migration. The same
criteria can be used for DMO, velocity analysis, wave-equation datuming
or any other integral operator which is applied to seismic data.
The salt dome example illustrates that sometimes some portions of a data set
may be sampled adequately, so that operator aliasing is not a problem.
If these are the regions of interest, computational effort
and time can be reduced by not undertaking the added expense 
of anti-alising.

\section{ACKNOWLEDGEMENTS}
The salt dome data were graciously provided by Halliburton Geophysical Services.
We thank Mihai Popovici for obtaining the data and making it available to us.

\bibliographystyle{seg}
\bibliography{SEP2,SEG,myrefs}

%\newpage
