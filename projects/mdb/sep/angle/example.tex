\inputdir{agmig}

We created some simple synthetic models with constant velocity backgrounds
to test our angle-gather migration method.  One model is a simple dome 
(Figure~\ref{fig:data-dome}).  The other has a series of flat reflectors of 
various dips (Figure~\ref{fig:data-lines}).  Both of these figures also
show the corresponding data that will be generated by Kirchhoff methods
for zero and far offsets.

\plot{data-dome}{width=6.in}{Left: Model. Center: Data at zero offset. 
Right: Data at far offset.}

\plot{data-lines}{width=6.in}{Left: Model. Center: Data at zero offset. 
Right: Data at far offset.}

\par
\subsection{Dome model}

This model contains a wide range of geologic dips across the dome as well
as having a flat reflector at the base of the dome.  
Figure~\ref{fig:offset-dome} shows the resulting common offset sections from
traditional Kirchhoff migration.  As is expected for such a simple model,
the near and far offset sections are very similar and the stacked section
is almost perfect.  We are more interested in the result of the angle-gather
migration.

Figure~\ref{fig:angle-dome} shows the zero and large angle sections as well
as the stack for angle-gather Kirchhoff migration.  The zero-angle section
is weak but clearly shows the correct shape and position.  The large-angle
section is actually only for $\gamma=25^{\circ}$. 
The reason for this is clear if you consider Figure~\ref{fig:rays}.  At 
greater depths, the rays associated with large reflection angles ($\gamma$)
will not emerge at the surface within the model space.  Therefore at angles
greater than $25^{\circ}$ (the maximum useful angle), the information at later 
times disappears.  

We expect the stacked sections for the offset method and the angle method 
to be identical.  Although we sum over different paths for the offset-domain 
migration (Figure~\ref{fig:coffset}) and the angle-domain migration 
(Figure~\ref{fig:cangle}), the stack should sum all of the same information
together for both methods.  Fortunately, a comparison of the stacked sections
in Figures~\ref{fig:offset-dome}~and~\ref{fig:angle-dome} show that the
results are identical as expected.
 
\plot{offset-dome}{width=6.in}{Left: Migrated offset section at zero offset.
Center: Migrated offset section at far offset.  Right: Stack.}

\plot{angle-dome}{width=6.in}{Left: Migrated angle section at small angle.
Center: Migrated offset section at large angle.  Right: Stack.}

\par
\subsection{Dipping reflectors model}

This model contains fewer dips than the dome model but it allows us to 
see what is happening at later times.  Figure~\ref{fig:offset-lines} 
shows the common offset sections  and stacked section from offset-domain 
Kirchhoff migration.  Once again, they are practically perfect.  The only 
problem is near the bottom of the section where we lose energy because the
data was truncated.

The zero-angle and large-angle sections from the angle-domain migration are
in Figure~\ref{fig:angle-lines}, along with the stacked section.  Once 
again, the zero angle section is very weak and the large angle section only
contains information down to a time of $\approx .85$ seconds, for the same
reason as explained for the dome model.  

Once again, we expect the stacked sections in 
Figures~\ref{fig:offset-lines}~and~\ref{fig:angle-lines} to be the same. 
Although the angle-domain stack is slightly lower amplitude throughout the
section, it is clear that this is a simple scale factor so our expectations
remain intact. 

\plot{offset-lines}{width=6.in}{Left: Migrated offset section at zero offset.
Center: Migrated offset section at far offset.  Right: Stack.}

\plot{angle-lines}{width=6.in}{Left: Migrated angle section at zero angle.
Center: Migrated angle section at large angle.  Right: Stack.}

\par
\subsection{Reflectivity variation with angle}

Amplitude variation with offset (AVO) would not be expected to be very
interesting for the simple models just shown.  Consider
Figure~\ref{fig:agather-dome} which contains an offset gather and a
reflection angle gather taken from space location zero from the dome
model in Figure~\ref{fig:data-dome}.  The offset gather shows exactly
what we expect for such a model - no variation.  The angle gather also
shows no variation for angles less than the maximum useful angle
($25^{\circ}$) as discussed in the previous two subsections.  However,
when the angle exceeds the maximum useful angle, the event increases
in amplitude and width.  This is the phenomenon seen in
\cite{GEO55-09-12231234}. 

\plot{agather-dome}{width=6.in}{Gathers taken from space location zero
in the dome model.  Left: Offset domain.  Center: Angle domain less than
$25^{\circ}$.  Right: Angle domain.}

\subsection{Velocity sensitivity}

When dealing with real data we almost never know what the true velocity of 
the subsurface is.  Therefore it is important to understand the effects of
velocity on our angle-gather time migration algorithm.  To do this we
simply created data for the dome model in Figure~\ref{fig:data-dome} at a
fairly high velocity (3 km/s) and migrated it using a low velocity (1.5 km/s).
The results are in Figure~\ref{fig:agather-dome-fast}.  For angles less
than the maximum useful angle ($\gamma=25^{\circ}$), the angle-domain gather
behaves exactly as the offset-domain gather does.  Beyond the maximum
useful angle, the events become even more curved and the amplitudes begin
to change.

The behavior of the angle-gather migration is very similar to that of 
offset-domain migration as long as the limitation of the maximum useful
angle is recognized.  Therefore, we can probably expect angle-gather
migration to behave like offset-domain migration in $v(z)$ media also.  

\plot{agather-dome-fast}{width=6.in}{Gathers taken from space location
zero inthe dome model and migrated at too low a velocity.  Left: Offset
domain. Center: Angle domain less than $25^{\circ}$.  Right: Angle domain.}  
