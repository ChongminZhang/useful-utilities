%\shortnote

\lefthead{}
\righthead{}
\published{Geophysics, 68, 577-588 (2003)}
\title{Multidimensional recursive filter preconditioning \\
  in geophysical estimation problems}

%\renewcommand{\author}[1]{%
%\begin{centering}
%  \textbf{#1}
%\end{centering} 
%}

\author{Sergey Fomel\/\footnotemark[1]
and Jon F. Claerbout\/\footnotemark[2]} 

\footnotetext[1]{\emph{Bureau of Economic Geology, 
Jackson School of Geosciences,
The University of Texas at Austin,
University Station, Box X,
Austin, TX 78713-8972, USA}}
\footnotetext[2]{\emph{Stanford Exploration Project, Mitchell Bldg., 
        Department of Geophysics, 
        Stanford University, Stanford, California 94305-2215, USA}}

\maketitle
    
\begin{abstract}
Constraining ill-posed inverse problems often requires
      regularized optimization. We consider two alternative approaches
      to regularization. The first approach involves a column operator
      and an extension of the data space. It requires a regularization
      operator which enhances the undesirable features of the model.
      The second approach constructs a row operator and expands the
      model space. It employs a preconditioning operator, which
      enforces a desirable behavior, such as smoothness, of the model.
      In large-scale problems, when iterative optimization is incomplete,
      the second method is preferable, because it often leads to
      faster convergence. We propose a method for constructing
      preconditioning operators by multidimensional recursive
      filtering. The recursive filters are constructed by imposing
      helical boundary conditions. Several examples with
      synthetic and real data demonstrate an order of magnitude
      efficiency gain achieved by applying the proposed technique to
      data interpolation problems. 
\end{abstract}

\section{Introduction}

Regularization is a method of imposing additional conditions for
solving inverse problems with optimization methods. When model
parameters are not fully constrained by the problem (i.e. the inverse
problem is mathematically ill-posed), regularization restricts the
variability of the model and guides iterative optimization to the
desired solution by using additional assumptions about the model power,
smoothness, predictability, etc. In other words, it constrains the
model null space to an \emph{a priori} chosen pattern.  A thorough
mathematical theory of regularization has been introduced by works of
Tikhonov's school \cite[]{tikh}.
\par
In this paper, we discuss two alternative formulations of regularized
least-squares inversion problems. The first formulation, which we call
\emph{model-space}, extends the data space and constructs a composite
column operator. The second, \emph{data-space}, formulation extends
the model space and constructs a composite row operator. This second
formulation is intrinsically related to the concept of model
preconditioning \cite[]{GEO59-05-08180829}. We illustrate the general
regularization theory with simple synthetic examples.
\par
Though the final results of the model-space and data-space
regularization are theoretically identical, the behavior of iterative
gradient-based methods, such as the method of conjugate gradients, is
different for the two cases. The obvious difference is in the case
where the number of model parameters is significantly larger than the
number of data measurements. In this case, the dimensions of the
inverted matrix in the case of the data-space regularization are
smaller than those of the model-space matrix, and the convergence of
the iterative conjugate-gradient iteration requires correspondingly
smaller number of iterations.  But even in the case where the number
of model and data parameters are comparable, preconditioning changes
the iteration behavior. This follows from the fact that the objective
function gradients with respect to the model parameters are different
in the two cases.
%The first iteration of
%the model-space regularization yields $\mathbf{L}^T \mathbf{d}$ as the
%model estimate regardless of the regularization operator~$\mathbf{D}$,
%while the first iteration of the data-space regularization yields
%$\mathbf{C L}^T \mathbf{d}$, which is an already ``simplified'' version of
%the model.  
Since iteration to the exact solution is rarely achieved in large-scale
geophysical applications, the results of iterative optimization may turn out
quite differently.  \cite{harlan} points out that the goals of the
model-space regularization conflict with each other: the first one emphasizes
``details'' in the model, while the second one tries to smooth them out. He
describes the advantage of preconditioning as follows:
\begin{quote}
  The two objective functions produce different results when
  optimization is incomplete. A descent optimization of the original
  (\emph{model-space}) objective function will begin with complex
  perturbations of the model and slowly converge toward an
  increasingly simple model at the global minimum.  A descent
  optimization of the revised (\emph{data-space}) objective function
  will begin with simple perturbations of the model and slowly
  converge toward an increasingly complex model at the global minimum.
  \ldots A more economical implementation can use fewer iterations.
  Insufficient iterations result in an insufficiently complex model,
  not in an insufficiently simplified model.
\end{quote}

In this paper, we illustrate the two approaches on synthetic and real data
examples from simple environmental data sets. All examples show that when we
solve the optimization problem iteratively and take the output only after a
limited number of iterations, it is preferable to use the preconditioning
approach. When the regularization operator is convolution with a filter, a
natural choice for preconditioning is inverse filtering (recursive
deconvolution).  We show how to extend the method of preconditioning by
recursive filtering to multiple dimensions. The extention is based on
modifying the boundary conditions with the helix transform \cite[]{helix}.
%Invertible multidimensional filters can be
%created by helical spectral factorization.

\section{Review of regularization in estimation problems}

The basic linear system of equations for least-squares optimization
can be written in the form
\begin{equation}
  \mathbf{d - L m \approx 0}\;,
\label{eqn:main}
\end{equation}
where $\mathbf{m}$ is the vector of model parameters, $\mathbf{d}$ is the
vector of experimental data values, $\mathbf{L}$ is the modeling
operator, and the approximate equality implies finding the solution by
minimizing the power of the left-hand side.  The goal of linear
estimation is to estimate an optimal~$\mathbf{m}$ for a
given~$\mathbf{d}$.

\subsection{Model-space regularization}

Model-space regularization implies adding equations to
system~(\ref{eqn:main}) to obtain a fully constrained (well-posed)
inverse problem. The additional equations take the form
\begin{equation}
\epsilon \mathbf{D m \approx 0} \;,
\label{eqn:mreg}
\end{equation}
where $\mathbf{D}$ is a linear operator that represents additional
requirements for the model, and $\epsilon$ is the scaling parameter.
In many applications, $\mathbf{D}$ can be thought of as a filter,
enhancing undesirable components in the model, or as the operator of
a differential equation that we assume the model should satisfy.

The full system of equations (\ref{eqn:main}-\ref{eqn:mreg}) can be
written in a short notation as
\begin{equation}
  \mathbf{G_m m} = \left[\begin{array}{c} \mathbf{L} \\ \epsilon \mathbf{D}
    \end{array}\right] \mathbf{m} \approx
  \left[\begin{array}{c} \mathbf{d} \\ \mathbf{0} \end{array}\right] = 
  \hat{\mathbf{d}}\;,
\label{eqn:main-m}
\end{equation}
where $\hat{\mathbf{d}}$ is the augmented data vector:
\begin{equation}
\hat{\mathbf{d}} = \left[\begin{array}{c} \mathbf{d} \\ \mathbf{0} 
  \end{array}\right]\;,
\label{eqn:dhat}
\end{equation}
and $\mathbf{G_m}$ is a \emph{column} operator:
\begin{equation}
\mathbf{G_m} = \left[\begin{array}{c} \mathbf{L} \\ \epsilon \mathbf{D}
  \end{array}\right]\;.
\label{eqn:gm}
\end{equation}

The estimation problem (\ref{eqn:main-m}) is fully constrained. We can
solve it by means of unconstrained least-squares optimization,
minimizing the least-squares norm of
the compound residual vector
\begin{equation}
\hat{\mathbf{r}} = \hat{\mathbf{d}} - \mathbf{G_m m} =
\left[\begin{array}{c} \mathbf{d - L m}\\ - \epsilon \mathbf{D m}
  \end{array}\right]\;.
\label{eqn:rhat}
\end{equation}
The formal solution of the regularized optimization problem has the
known form \cite[]{parker}
\begin{equation}
  <\!\!\mathbf{m}\!\!> = 
  \left(\mathbf{L}^T\,\mathbf{L} +
    \epsilon^2\,\mathbf{D}^T\,\mathbf{D}\right)^{-1}\,\mathbf{L}^T\,\mathbf{d}\;,
  \label{eqn:minv1}  
\end{equation}
where $<\!\!\mathbf{m}\!\!>$ denotes the least-squares estimate of $\mathbf{m}$, 
and $\mathbf{L}^T$ denotes the adjoint operator.
One can carry out the optimization iteratively with the help of the
conjugate-gradient method \cite[]{Hestenes.1952} or its analogs
\cite[]{Paige.acm.8.195}.

In the next subsection, we describe an alternative formulation of the
optimization problem.

\subsection{Data-space regularization (model preconditioning)}

The data-space regularization approach is closely connected with the
concept of \emph{model preconditioning}.
We can introduce a new model $\mathbf{p}$ with
the equality
\begin{equation}
\mathbf{m = P p}\;,
\label{eqn:prec}
\end{equation}
where $\mathbf{P}$ is a preconditioning operator.
The residual vector $\mathbf{r}$ for the data-fitting
goal~(\ref{eqn:main}) can be defined by the relationship
\begin{equation}
\epsilon \mathbf{r = d - L m = d - L P p}\;,
\label{eqn:r}
\end{equation}
where $\epsilon$ is the same scaling parameter as in~(\ref{eqn:mreg})
-- the reason for this choice will be clear from the analysis that
follows.  Let us consider a compound model $\hat{\mathbf{p}}$, composed
of the preconditioned model vector $\mathbf{p}$ and the residual
$\mathbf{r}$:
\begin{equation}
\label{eqn:hatp}
\hat{\mathbf{p}} = \left[\begin{array}{c} \mathbf{p} \\ \mathbf{r} \end{array}\right]\;.
\end{equation}
With respect to the compound model, we can rewrite equation
(\ref{eqn:r}) as
\begin{equation}
 \left[\begin{array}{cc} \mathbf{L P} & \epsilon \mathbf{I} \end{array}\right]
\left[\begin{array}{c} \mathbf{p} \\ \mathbf{r} \end{array}\right] = 
\mathbf{G_d} \hat{\mathbf{p}} = \mathbf{d}\;,
\label{eqn:main-d}
\end{equation}
where $\mathbf{G_d}$ is a \emph{row} operator:
\begin{equation}
\mathbf{G_d} = \left[\begin{array}{cc} \mathbf{L P} & 
\epsilon \mathbf{I} \end{array}\right]\;,
\label{eqn:gd}
\end{equation}
and $\mathbf{I}$ represents the data-space identity operator.

Equation (\ref{eqn:main-d}) is clearly underdetermined with respect to
the compound model $\hat{\mathbf{p}}$. If from all possible solutions of
this system we seek the one with the minimal power $\hat{\mathbf{p}}^T
\hat{\mathbf{p}}$, the formal result takes the well-known form
\begin{equation}
<\!\!\hat{\mathbf{p}}\!\!> = \left[\begin{array}{c} 
<\!\!\mathbf{p}\!\!> \\ <\!\!\mathbf{r}\!\!>
\end{array} \right] =
\mathbf{G_d}^T \left(\mathbf{G_d}\,\mathbf{G_d}^T\right)^{-1} \mathbf{d} =
\left[\begin{array}{c}
\mathbf{P}^T \mathbf{L}^T  
\left(\mathbf{L P P}^T \mathbf{L}^T + \epsilon^2 \mathbf{I}\right)^{-1} \mathbf{d} \\
\epsilon 
\left(\mathbf{L P P}^T \mathbf{L}^T + \epsilon^2 \mathbf{I}\right)^{-1} \mathbf{d}
\end{array} \right]\;.
\label{eqn:dsol-i}
\end{equation}
Applying equation~(\ref{eqn:prec}), we obtain the corresponding
estimate $<\!\!\mathbf{m}\!\!>$ for the initial model $\mathbf{m}$, as
follows:
\begin{equation}
  \label{eqn:dinv1}
  <\!\!\mathbf{m}\!\!> =  \mathbf{P}\,\mathbf{P}^T\,\mathbf{L}^T\,\left(
    \mathbf{L}\,\mathbf{P}\,\mathbf{P}^T\,\mathbf{L}^T +
    \epsilon^2\,\mathbf{I}\right)^{-1}\, \mathbf{d}\;.
\end{equation}
It is easy to show algebraically that estimate~(\ref{eqn:dinv1}) 
is equivalent to estimate~(\ref{eqn:minv1}) under the condition
\begin{equation}
\mathbf{C} = \mathbf{P}\,\mathbf{P}^T = \left(\mathbf{D}^T\,\mathbf{D}\right)^{-1}\;,
\label{eqn:cond}
\end{equation}
where we need to assume the self-adjoint operator
$\mathbf{D}^T\,\mathbf{D}$ to be invertible.  

To prove the equivalence, consider the operator
\begin{equation}
\mathbf{G} = \mathbf{L}^T\,\mathbf{L}\,\mathbf{C}\,\mathbf{L}^T + \epsilon^2\,\mathbf{L}^T\;,
\label{eq:g}
\end{equation}
which is a mapping from the data space to the model space.
Grouping the multiplicative factors in two different ways, we can
obtain the equality
\begin{equation}
\mathbf{G} = \mathbf{L}^T\,\left(\mathbf{L}\,\mathbf{C}\,\mathbf{L}^T + 
\epsilon^2\,\mathbf{I}\right) = \left(\mathbf{L}^T\,\mathbf{L} + 
\epsilon^2 \mathbf{C}^{-1}\right)\,\mathbf{C}\,\mathbf{L}^T\;,
\label{eq:gmd}
\end{equation}
or, in another form,
\begin{equation}
\mathbf{C}\,\mathbf{L}^T\,\left(\mathbf{L}\,\mathbf{C}\,\mathbf{L}^T + 
\epsilon^2\,\mathbf{I}\right)^{-1} \equiv \left(\mathbf{L}^T\,\mathbf{L} + 
\epsilon^2\,\mathbf{C}^{-1}\right)^{-1}\,\mathbf{L}^T\;.
\label{eq:md}
\end{equation}
The left-hand side of equality (\ref{eq:md}) is exactly the projection
operator from formula (\ref{eqn:dinv1}), and the right-hand side is the
operator from formula (\ref{eqn:minv1}).

This proves the legitimacy
of the alternative data-space approach to data regularization: the
model estimation is reduced to a least-squares minimization of the
specially constructed compound model $\hat{\mathbf{p}}$ under the
constraint~(\ref{eqn:r}).

%\begin{latexonly}
We summarize the differences between the model-space and data-space
regularization in Table 1.

\tabl{table}{Comparison between model-space and data-space regularization}
{\centering
\begin{tabular}{|l||p{2in}|p{2in}|} \hline
  \multicolumn{3}{c}{\rule[-3mm]{0mm}{8mm}} \\
  \hline \rule[-3mm]{0mm}{8mm} 
  & \emph{Model-space} & \emph{Data-space} \\ \hline \hline
  \rule[-8mm]{0mm}{17mm} effective model & $\mathbf{m}$ & $\hat{\mathbf{p}} =
  \left[\begin{array}{c} \mathbf{p} \\ \mathbf{r} \end{array}\right]$ \\ \hline
  \rule[-8mm]{0mm}{17mm} effective data & $\hat{\mathbf{d}} =
  \left[\begin{array}{c} \mathbf{d} \\ \mathbf{0}
    \end{array}\right]$ & $\mathbf{d}$ \\ \hline \rule[-8mm]{0mm}{17mm}
  effective operator & $\mathbf{G_m} = \left[\begin{array}{c} \mathbf{L} \\
      \epsilon \mathbf{D} \end{array}\right]$ & $\mathbf{G_d} =
  \left[\begin{array}{cc} \mathbf{LP} & \epsilon \mathbf{I} \end{array}\right]$ \\
  \hline \rule[-3mm]{0mm}{8mm} optimization problem & minimize
  $\hat{\mathbf{r}}^T \hat{\mathbf{r}}$, \newline where \newline
  \rule[-3mm]{0mm}{8mm} $\hat{\mathbf{r}} = \hat{\mathbf{d}} - \mathbf{G_m m}$ & 
  minimize
  $\hat{\mathbf{p}}^T \hat{\mathbf{p}}$ \newline under the constraint
  \newline \rule[-3mm]{0mm}{8mm} $\mathbf{G_d} \hat{\mathbf{p}} = \mathbf{d}$ \\
  \hline \rule[-5mm]{0mm}{11mm} formal estimate for $\mathbf{m}$ &
  $\left(\mathbf{L}^T \mathbf{L} + 
    \epsilon^2 \mathbf{C}^{-1}\right) \mathbf{L}^T \mathbf{d}$, \newline
  \rule[-5mm]{0mm}{11mm} where $\mathbf{C}^{-1} = \mathbf{D}^T \mathbf{D}$ & 
  $\mathbf{C L}^T (\mathbf{L C L}^T
  + \epsilon^2 \mathbf{I})^{-1} \mathbf{d}$, 
  \newline \rule[-5mm]{0mm}{11mm} where $\mathbf{C}
  = \mathbf{P P}^T$. \\ \hline
\end{tabular}
}
%\end{latexonly}

Although the two approaches lead to similar theoretical results, they
behave quite differently in the process of iterative optimization.  In
the next section, we illustrate this fact with many examples and show
that in the case of incomplete optimization, the second
(preconditioning) approach is generally preferable.

\section{One-dimensional synthetic examples}

\inputdir{oned}

\plot{data}{width=6in}{ The input data (right) are
  irregularly spaced samples of a sinusoid (left).}

In the first example, the input data $\mathbf{d}$ were randomly
subsampled (with decreasing density) from a sinusoid (Figure
\ref{fig:data}). The forward operator $\mathbf{L}$ in this case is
linear interpolation. In other words, we seek a regularly sampled
model $\mathbf{m}$ on 200 grid points that could predict the data with a
forward linear interpolation. Sparse irregular distribution of the
input data makes the regularization necessary.  We applied convolution
with the simple $(1,-1)$ difference filter as the operator $\mathbf{D}$
that forces model continuity (the first-order spline). An appropriate
preconditioner $\mathbf{P}$ in this case is the inverse of $\mathbf{D}$,
which amounts to recursive causal integration \cite[]{gee}. Figures
\ref{fig:im1} and \ref{fig:fm1} show the results of inverse
interpolation after exhaustive 300 iterations of the
conjugate-direction method. The results from the model-space and
data-space regularization look similar except for the boundary
conditions outside the data range.  As a result of using the causal
integration for preconditioning, the rightmost part of the model in
the data-space case stays at a constant level instead of decreasing to
zero. If we specifically wanted a zero-value boundary condition, we
could easily implement it by adding a zero-value data point at the
boundary.

\plot{im1}{width=3in}{ Estimation of a continuous
  function by the model-space regularization. The difference operator
  $\mathbf{D}$ is the derivative operator (convolution with $(1,-1)$).}

\plot{fm1}{width=3in}{Estimation of a continuous
  function by the data-space regularization. The preconditioning
  operator $\mathbf{P}$ is causal integration.}

The model preconditioning approach provides a much faster rate of convergence.
We measured the rate of convergence using the power of the model residual,
defined as the least-squares distance from the current model to the final
solution.  Figure \ref{fig:schwab1} shows that the preconditioning (data
regularization) method converged to the final solution in about 6 times fewer
iterations than the model regularization. Since the cost of each iteration for
both methods is roughly equal, the computational economy is evident.  Figure
\ref{fig:early1} shows the final solution, and the estimates from model- and
data-space regularization after only 5 iterations of conjugate directions. The
data-space estimate appears much closer to the final solution.

\plot{schwab1}{width=3in}{Convergence of the iterative
  optimization, measured in terms of the model residual.  The ``d''
  points stand for data-space regularization; the ``m'' points for
  model-space regularization.  }

\plot{early1}{width=4in}{ The top figure is the exact
  solution found in 250 iterations. The middle is with data-space
  regularization after 5 iterations. The bottom is with model-space
  regularization after 5 iterations.  }

Changing the preconditioning operator changes the regularization
result. Figure \ref{fig:fm6} shows the result of data-space
regularization after a triangle smoother is applied as the model
preconditioner. Triangle smoother is a filter with the $Z$-transform
$\frac{\left(1-Z^N\right)\left(1-Z^{-N}\right)}{(1-Z)\left(1-Z^{-1}\right)}$
\cite[]{Claerbout.blackwell.92}. We chose the filter length $N=6$.

\sideplot{fm6}{width=3in}{Estimation of a smooth function
  by the data-space regularization. The preconditioning operator
  $\mathbf{P}$ is a triangle smoother.}

If, instead of looking for a smooth interpolation, we want to limit the number
of frequency components, then the optimal choice for the model-space
regularization operator $\mathbf{D}$ is a prediction-error filter (PEF). To
obtain a mono-frequency output, we can use a three-point PEF, which has the
$Z$-transform representation $D (Z) = 1 + a_1 Z + a_2 Z^2$. In this case, the
corresponding preconditioner $\mathbf{P}$ can be the three-point
\emph{recursive} filter $P (Z) = 1 / (1 + a_1 Z + a_2 Z^2)$. To test this
idea, we estimated the PEF $D (Z)$ from the output of inverse linear
interpolation (Figure \ref{fig:fm1}), and ran the data-space regularized
estimation again, substituting the recursive filter $P (Z) = 1/ D(Z)$ in place
of the causal integration.  We repeated this two-step procedure three times to
get a better estimate for the PEF. The result, shown in Figure \ref{fig:pm1},
exhibits the desired mono-frequency output. One can accommodate more frequency
components in the model using a longer filter.

\sideplot{pm1}{width=3in}{Estimation of a mono-frequency
  function by the data-space regularization. The preconditioning
  operator $\mathbf{P}$ is a recursive filter (the inverse of PEF).}

\subsection{Regularization after binning: missing data interpolation}

\inputdir{mis1}

One of the factors affecting the convergence of iterative data
regularization is clustering of data points in the output bins. When
least-squares optimization assigns equal weight to each data point, it
may apply inadequate effort to fit a cluster of data points with
similar values in a particular output bin. To avoid this problem, we
can replace the regularized optimization with a less accurate but more
efficient two-step approach: data binning followed by missing data
interpolation.

Missing data interpolation is a particular case of data
regularization, where the input data are already given on a regular
grid, and we need to reconstruct only the missing values in empty
bins.  The basic principle of missing data interpolation is formulated
as follows \cite[]{Claerbout.blackwell.92}:
\begin{quote}
  A method for restoring missing data is to ensure that the restored
  data, after specified filtering, has minimum energy.
\end{quote}
Mathematically, this principle can be expressed by the simple
equation
\begin{equation}
\mathbf{D m \approx 0}\;,
\label{eqn:basic}
\end{equation}
where $\mathbf{m}$ is the model vector and $\mathbf{D}$ is the specified
filter.  Equation~(\ref{eqn:basic}) is completely equivalent to
equation~(\ref{eqn:mreg}).  The approximate equality sign means that
equation (\ref{eqn:basic}) is solved by minimizing the squared norm
(the power) of its left side.  Additionally, the known data values
must be preserved in the optimization scheme. Introducing the mask
operator $\mathbf{K}$, which is a diagonal matrix with
zeros at the missing data locations and ones elsewhere, we can rewrite
equation (\ref{eqn:basic}) in the extended form
\begin{equation}
\mathbf{D (I-K) m} \approx - \mathbf{D K m} = - \mathbf{D m}_k\;,
\label{eqn:rigor}
\end{equation}
where $\mathbf{I}$ is the identity operator and $\mathbf{m}_k$
represents the known portion of the data. It is important to note that
equation (\ref{eqn:rigor}) corresponds to the
regularized linear system
\begin{equation}
\left\{\begin{array}{l}
\mathbf{K m} = \mathbf{m}_k\;, \\
\epsilon \mathbf{D m} \approx \mathbf{0}
\end{array}\right.
\label{eqn:reg}
\end{equation}
in the limit of the scaling coefficient $\epsilon$ approaching zero.
System~(\ref{eqn:reg}) is equivalent to
system~(\ref{eqn:main}-\ref{eqn:mreg}) with the masking
operator~$\mathbf{K}$ playing the role of the forward interpolation
operator~$\mathbf{L}$. Setting $\epsilon$ to zero implies associating
more weight on the first equation in (\ref{eqn:reg}) and using the
second equation only to constrain the null space of the solution.
Applying the general theory of data-space regularization from
the previous section, we can immediately transform system
(\ref{eqn:reg}) to the equation
\begin{equation}
\mathbf{K P p} \approx  \mathbf{m}_k\;,
\label{eqn:prec2}
\end{equation}
where $\mathbf{P}$ is a preconditioning operator and $\mathbf{p}$ is the
preconditioning variable, connected with $\mathbf{m}$ by the simple
relationship~(\ref{eqn:prec}).  According to equations~(\ref{eqn:dinv1})
and~(\ref{eqn:minv1}) from the previous section, equations
(\ref{eqn:prec2}) and (\ref{eqn:rigor}) have exactly the same
solutions if condition~(\ref{eqn:cond}) is satisfied.  If $\mathbf{D}$ is
represented by a discrete convolution, the natural choice for
$\mathbf{P}$ is the corresponding deconvolution (inverse recursive
filtering) operator:
\begin{equation}
\mathbf{P}  = \mathbf{D}^{-1}\;.
\label{eqn:decon}
\end{equation}

We illustrate the missing data problem with a simple 1-D synthetic data
test taken from \cite{gee}.  Figure~\ref{fig:mall} shows the
interpolation results of the unpreconditioned technique with two
different filters $\mathbf{D}$. For comparison with the preconditioned scheme, we
changed the boundary convolution conditions from internal to truncated
transient convolution. As in the previous example, the system was
solved with a conjugate-gradient iterative optimization.

\plot{mall}{width=6in,height=4in}{Unpreconditioned
  interpolation with two different regularization filters. Left plot:
  the top shows the input data; the middle, the result of
  interpolation; the bottom, the filter impulse response. The right
  plot shows the convergence process for the first four iterations.}

As depicted on the right side of the figures, the interpolation
process starts with a ``complicated'' model and slowly extends
it until the final result is achieved. 

Preconditioned interpolation behaves differently (Figure
\ref{fig:sall}). At the early iterations, the model is simple. As the
iteration proceeds, new details are added into the model.  After a
surprisingly small number of iterations, the output closely resembles
the final output. The final output of interpolation with
preconditioning by recursive deconvolution is exactly
the same as that of the original method.

\plot{sall}{width=6in,height=4in}{Interpolation with
  preconditioning. Left plot: the top shows the input data; the
  middle, the result of interpolation; the bottom, the filter
  impulse response. The right plot shows the convergence process for
  the first four iterations.}

The next section extends the idea of preconditioning by inverse
recursive filtering to multiple dimensions.

\section{Multidimensional recursive filter preconditioning}

\inputdir{Math}

\cite{helix} proposed a \emph{helix} transform for mapping
multidimensional convolution operators to their one-dimensional
equivalents.  This transform proves the feasibility of
multidimensional deconvolution, an issue that has been in question for
more than 25 years \cite[]{Claerbout.blackwell.76}.  By mapping discrete
convolution operators to one-dimensional space, the inverse filtering
problem can be conveniently recast in terms of recursive filtering, a
well-known part of the digital filtering theory.

\plot{helix}{width=5in}{Helix transform of
  two-dimensional filters to one dimension (a scheme).  The
  two-dimensional filter (plot \texttt{a}) is equivalent to the
  one-dimensional filter in (plot \texttt{d}), assuming that a shifted
  periodic condition is imposed on one of the axes (plots \texttt{b}
  and \texttt{c}.)}

The helix filtering idea is schematically illustrated in
Figure~\ref{fig:helix}. The left plot (labeled ``\texttt{a}'' in the
figure) shows a two-dimensional digital filter overlayed on the
computational grid. A two-dimensional convolution computes its output
by sliding the filter over the plane. If we impose helical boundary
conditions on one of the axes, the filter will slide to the beginning
of the next trace after reaching the end of the previous one
(plot~``\texttt{b}''). As evident from plots~``\texttt{c}''
and~``\texttt{d}'', this is completely equivalent to one-dimensional
convolution with a long 1-D filter with internal gaps.  For
efficiency, the gaps are simply skipped in a helical convolution
algorithm. The gain is not in the convolution
itself, but in the ability to perform recursive inverse filtering
(deconvolution) in multiple dimensions. A multi-dimensional filter is
mapped to its 1-D analog by imposing helical boundary conditions on
the appropriate axes. After that, inverse filtering is applied
recursively in a one-dimensional manner.  Neglecting parallelization
and indexing issues, the cost of inverse filtering is equivalent to
the cost of convolution. It is proportional to the data size and to
the number of non-zero filter coefficients.

\inputdir{hlx}

\plot{waves}{width=4in,height=2in}{ Illustration of 2-D
  deconvolution with helix transform.  Left is the input: two spikes
  and two filters.  Right is the output of deconvolution.}

An example of two-dimensional recursive filtering is shown in
Figure~\ref{fig:waves}. The left plot contains two spikes and two
filter impulse responses with different polarity. After deconvolution
with the given filter, the filter responses turn into spikes, and the
initial spikes turn into long-tailed inverse impulse responses (right
plot in Figure~\ref{fig:waves}). Helical wrap-around, visible on the
top and bottom boundaries, indicates the direction of the helix.
\cite{gee} presents more examples and discusses all the issues of
multidimensional helical deconvolution in detail.

%As is known from the one-dimensional theory \cite[]{Claerbout.fgdp.76},
%a stable recursive filtering requires a minimum-phase filter, which
%can be constructed with a spectral factorization algorithm. 

\section{Multidimensional examples}

\inputdir{seab}

Our first multidimensional example is the SeaBeam dataset, a result of
water bottom measurements from a single day of acquisition. SeaBeam is
an apparatus for measuring water depth both directly under a ship and
somewhat off to the sides of the ship's track.  The dataset has been
used at the Stanford Exploration Project for benchmarking different
strategies of data interpolation.  The left plot in Figure
\ref{fig:seabdat} shows the original data. The right plot shows the
result of (unpreconditioned) missing data interpolation with the
Laplacian filter after 200 iterations.  The result is unsatisfactory,
because the Laplacian filter does not absorb the spatial frequency
distribution of the input dataset. We judge the quality of an
interpolation scheme by its ability to hide the footprints of the
acquisition geometry in the final result. The ship track from the
original acquisition pattern is clearly visible in the Laplacian
result, which is an indication of a poor interpolation
method.

\plot{seabdat}{width=6in,height=2.5in}{On the left, the
  SeaBeam data: the depth of the ocean under ship tracks; on the
  right, an interpolation with the Laplacian filter.}
\par
We can obtain a significantly better image (Figure \ref{fig:seabold})
by replacing the Laplacian filter with a two-dimensional
prediction-error filter estimated from the input data.  The
result in the left plot of Figure \ref{fig:seabold} was obtained after
200 conjugate-gradient iterations. If we stop after 20 iterations, the
output (the right plot in Figure \ref{fig:seabold}) shows only a small
deviation from the input data. Large areas of the image remain
unfilled. At each iteration, the interpolation process progresses only
to the length of the filter.

\plot{seabold}{width=6in,height=2.5in}{SeaBeam interpolation
  with the prediction-error filter. The left plot was taken after 200
  conjugate-gradient iterations; the right, after 20 iterations.}
\par
Inverting the PEF convolution with the help of the helix transform, we
can now apply the inverse filtering operator to precondition the
interpolation problem. As expected, the result after 200 iterations
(the left plot in Figure \ref{fig:seabnew}) is similar to the result
of the corresponding unpreconditioned (model-space) interpolation.
However, the output after just 20 iterations (the right plot in Figure
\ref{fig:seabnew}) is already fairly close to the solution.

\plot{seabnew}{width=6in,height=2.5in}{SeaBeam interpolation
  with the inverse prediction-error filter. The left plot was taken
  after 200 conjugate-gradient iterations; the right, after 20
  iterations.}

\inputdir{cube}

For a more practical test, we chose the North Sea seismic reflection
dataset, previously used for testing azimuth moveout and
common-azimuth migration \cite[]{GEO63-02-05740588,SEG-1997-1375}.
Figure~\ref{fig:cmp-win} shows the highly irregular midpoint geometry
for a selected in-line and cross-line offset bin in the data. The data
irregularity is also evident in the bin fold map, shown in
Figure~\ref{fig:fold-win}.  The goal of data regularization is to
create a regular data cube at the specified bins from the irregular
input data, which have been preprocessed by normal moveout without
stacking.

\sideplot{cmp-win}{width=3.5in}{ Midpoint distribution for a 50 by 50 m
  offset bin in the 3-D North Sea dataset.}
 
\sideplot{fold-win}{width=3.5in}{Map of the fold distribution for the
  3-D data test.}

The data cube after normalized binning is shown in
Figure~\ref{fig:bin-win}.  Binning works reasonably well in the areas of
large fold but fails to fill the zero fold gaps and has an overall
limited accuracy.

\plot{bin-win}{width=6in}{3-D data after normalized binning.}

For efficiency, we perform regularization on individual time slices.
Figure~\ref{fig:smo2-win} shows the result of regularization using
bi-linear interpolation and smoothing preconditioning (data-space
regularization) with the minimum-phase Laplacian filter \cite[]{wilsonburg}.
The empty bins are filled in a consistent manner but the data quality
is distorted because simple smoothing fails to characterize the
complicated data structure.  Instead of continuous events, we see
smoothed blobs in the time slices. The events in the in-line and
cross-line sections are also not clearly pronounced.

\plot{smo2-win}{width=6in}{3-D data regularized with bi-linear
  interpolation and smoothing preconditioning.}

We can use the smoothing regularization result to estimate the local
dips in the data, design invertible local plane-wave destruction
filters \cite[]{Fomel.sepphd.107}, and repeat the regularization process.  Inverse
interpolation with plane-wave data-space regularization is shown in
Figure~\ref{fig:int4-win}. The result is noticeably
improved: the continuous reflection events become clearly visible in
the time slices.  Despite the irregularities in the input data, the
regularization result preserves both flat reflection events and
steeply-dipping diffractions. Preserving diffractions is important for
correct imaging of sharp edges in the subsurface structure
\cite[]{GEO63-02-05740588}.

For simplicity, we assumed only a single local dip component in the
data. This assumption degrades the result in the areas of multiple
conflicting dips, such as the intersections of plane reflections and
hyperbolic diffractions in Figure~\ref{fig:int4-win}. One could
improve the image by considering multiple local dips.
\cite{ofcon2} describes an alternative offset-continuation
approach, which uses a physical connection between neighboring offsets
instead of assuming local continuity in the midpoint domain.

\plot{int4-win}{width=6in}{3-D data regularized with cubic B-spline
  interpolation and local plane-wave preconditioning.}

The 3-D results of this paper were obtained with an efficient 2-D
regularization in time slices. This approach is computationally
attractive because of its easy parallelization: different slices can
be interpolated independently and in parallel.
Figure~\ref{fig:winslice} shows the interpolation result for four
selected time slices. Local plane waves, barely identifiable after
binning (left plots in Figure~\ref{fig:winslice}), appear clear and
continuous in the interpolation result (right plots in
Figure~\ref{fig:winslice}). The time slices are assembled
together to form the 3-D cube shown in Figure~\ref{fig:int4-win}.

\plot{winslice}{width=6in}{Selected time slices of the 3-D dataset.
  Left: after binning. Right: after plane-wave data regularization.
  The data regularization program identifies and continues local
  plane waves in the data.}

\section{Conclusions}

Regularization is often a necessary part of geophysical estimation.
Its goal is to impose additional constraints on the model and
to guide the estimation process towards the desired solution.

We have considered two different regularization methods. The first,
model-space approach involves a convolution operator that enhances the
undesired features in the model. The second, data-space, approach
involves inverse filtering (deconvolution) to precondition the model.
Although the two approaches lead to the theoretically equivalent
results, their behavior in iterative estimation methods is quite
different. Using several synthetic and real data examples, we have
demonstrated that the second, preconditioning approach is generally
preferable because it shows a significantly faster convergence at 
early iterations.

We suggest a constructive method for preconditioning multidimensional
estimation problems using the helix transform. Applying inverse
filtering operators constructed this way, we observe a significant
(order of magnitude) speed-up in the optimization convergence. Since
inverse recursive filtering takes almost the same time as forward
convolution, the acceleration translates straightforwardly into
computational time savings.
\par
For simple test problems, these savings are hardly noticeable. On the
other hand, for large-scale (seismic-exploration-size) problems, the
achieved acceleration can have a direct impact on the mere feasibility
of iterative least-squares estimation.

\section{Acknowledgments}
We are grateful to Jim Berryman, Bill Harlan, Dave Nichols, Gennady
Ryzhikov, and Bill Symes for insightful discussions about
preconditioning and optimization problems.

The financial support for this work was provided by the sponsors of
the Stanford Exploration Project (SEP). The 3-D North Sea dataset was
released to SEP by Conoco and its partners, BP and Mobil. The SeaBeam
dataset is courtesy of Alistair Harding of the Scripps Institution of
Oceanography.

%\newpage
\bibliographystyle{seglike}
\bibliography{SEG,SEP2,sergey}

%\APPENDIX{A}

%\plot{name}{width=6in,height=}{caption}
%\sideplot{name}{height=1.5in,width=}{caption}
%
%\begin{equation}
%\label{eqn:}
%\end{equation}
%
%\ref{fig:}
%(\ref{eqn:})

%%% Local Variables: 
%%% mode: latex
%%% TeX-master: t
%%% TeX-master: t
%%% TeX-master: t
%%% End: 
