\published{SEP report, 84, 25-37 (1995)}

\lefthead{Fomel \& Biondi} \righthead{t-x AMO} 
\footer{SEP--84}

\title{The time and space formulation \\ of azimuth moveout}
%\keywords{DMO, three-dimensional, prestack }

\email{sergey@sep.stanford.edu, biondo@sep.stanford.edu} 
\author{Sergey Fomel and Biondo L. Biondi}
\maketitle

\begin{abstract}
%%%%%%
Azimuth moveout (AMO) transforms 3-D prestack seismic data from one
common azimuth and offset to different azimuths and offsets.
AMO in the time-space domain is represented by a three-dimensional
integral operator. The operator components are the summation path,
the weighting function, and the aperture. To determine the summation path and
the weighting function, we derive the AMO operator by cascading dip
moveout (DMO) and inverse DMO for different azimuths in the time-space
domain. To evaluate the aperture, we apply a geometric approach,
defining AMO as the result of cascading prestack migration (inversion)
and modeling. The aperture limitations provide a consistent
description of AMO for small azimuth rotations (including zero) and justify the
economic efficiency of the method. 
\end{abstract}

\section{Introduction}

%%%%%%%%%%%%%%%%%%%
Azimuth moveout (AMO) is by definition an operator that transforms
common-azimuth common-offset seismic reflection data to different
azimuths and offsets\footnote{Here azimuth corresponds to the
direction of a source-receiver pair, and offset is the distance
between the source and the receiver.}. A constructive approach to 
AMO was proposed by \cite{Biondi.sep.80.125}.  According to this approach, an AMO
operator is built by cascading the dip moveout (DMO) operator that
transforms the input common-azimuth data to zero offset, and the inverse
DMO that transforms the zero-offset data to a new offset and azimuth.
Evaluating the cascade of the frequency-domain DMO and  inverse DMO
operators by means of the stationary phase technique produces the
integral (Kirchhoff-type) 3-D AMO operator in the time-space domain. 
\par
The first part of this paper applies an analogous idea to construct the AMO
operator from the time-space domain DMO and
achieves the same result in a simpler way.
Cascading DMO and inverse DMO allows us to evaluate the AMO operator's
summation path and the corresponding weighting function. However, it
is not sufficient for evaluating the third major component of the
integral operator, that is, its aperture (range of integration). To solve this
problem, we apply an 
alternative approach, that 
defines AMO as a cascade of 3-D migration (inversion) for 
particular common-azimuth  
and common-offset data and 3-D modeling for a different azimuth and
offset. This   
definition resembles the viewpoint on DMO developed by \cite{GPR29-03-03740406}. As with the DMO case, the
migration and modeling  approach  reveals the physics of the AMO
aperture and limits its boundaries. 
It is the aperture limitation
that allows us to overcome the paradoxical inconsistency between 2-D
and 3-D AMO operators discussed by \cite{Biondi.sep.80.125}. 
If the aperture is chosen properly, the AMO operator converges to the 2-D
offset continuation limit as the azimuth rotation approaches zero.  
This remarkable fact supports the proof of economical
efficiency of AMO in comparison with the prestack migration operator,
which is known to have an unlimited aperture. 

\section{CASCADING DMO AND INVERSE DMO
IN TIME-SPACE DOMAIN}
%%%%%%%%%%%%%%%%%%%%%%%%%%%%%%%%%%%%%%%%%%%%%%%%%%%%%%%%%%%%%%%%%%%%%%%%%
In this section, we present a new version of the AMO derivation.
Since 
the entire derivation is performed in the time-space domain, it is more
straightforward than the stationary phase technique developed for the
same purpose by \cite{Biondi.sep.80.125}.
\par
Let $P_1\left({\bf x_1},t_1;{\bf h_1}\right)$ be the in\-put of an AMO
ope\-ra\-tor  (com\-mon-azi\-muth and com\-mon-off\-set seismic
reflection data after normal moveout correction) and 
$P_2\left({\bf x_2},t_2;{\bf h_2}\right)$ be the output. Here 
${\bf x_i}\; (i=1,2)$ are midpoint locations on the surface:
${\bf x_i}=\left\{x_i,y_i\right\}$, and 
${\bf h_i}$ are half-offset vectors.
The 3-D AMO operator has the following general form:
\begin{equation} 
P_2\left({\bf x_2},t_2;{\bf h_2}\right)  =  {\bf D}_{t_2} \int\int 
w_{12}\left({\bf x_1;x_2,h_2},t_2\right)\,
P_1\left({\bf x_1},
t_2\,\theta_{12}\left({\bf x_1;x_2, h_2}\right);\,
{\bf h_1}\right)\,d{\bf x_1}\;,
\label{eqn:AMO}
\end{equation}
where ${\bf D}$ is the differentiation operator $\left({\bf
D}_t\equiv{d \over dt}\right)$, $t_2\,\theta_{12}$ is the summation
path, and $w_{12}$ is the weighting function. In this section we will
evaluate $\theta_{12}$ and $w_{12}$ using the cascade of integral 3-D DMO and
inverse DMO operators in the time-space domain. The idea of this
derivation originated in Biondi and Chemingui's paper
\cite[]{Biondi.sep.80.125}, where it was  
applied with the frequency-domain DMO and inverse DMO operators. In
the next section, 
we apply a new geometric approach to evaluate the AMO aperture (range of
integration in (\ref{eqn:AMO})). 
\par
To derive (\ref{eqn:AMO}) in the time-space domain, 
an integral
(Kirchoff-type) DMO operator of the form
\begin{equation}
P_0\left({\bf x_0},t_0;{\bf 0}\right)={\bf D}_{-t_0}^{1/2}\,\int 
w_{10}\left({\bf x_1;x_0,h_1},t_0\right)\,
P_1\left({\bf x_1},t_0\,\theta_{10}\left({\bf x_1;x_0,h_1}\right);
{\bf h_1}\right)\,d\hat{x}_1
\label{eqn:DMO}
\end{equation}
is cascaded with an inverse DMO of the form
\begin{equation}
P_2\left({\bf x_2},t_2;{\bf h_2}\right)={\bf D}_{t_2}^{1/2}\,\int 
w_{02}\left({\bf x_0;x_2, h_2},t_2\right)\,
P_0\left({\bf x_0},t_2\,\theta_{02}\left({\bf x_0;x_2, h_2}\right);
{\bf 0}\right)\,d\hat{x}_0\;,
\label{eqn:IDMO}
\end{equation}
where ${\bf D}_t^{1/2}$ stands for the operator of half-order differentiation
(equivalent to $(i \omega)^{1/2}$ multiplication in Fourier domain),
$t_0\, \theta_{10}$ and $t_2\,\theta_{02}$ are the summation paths of
the DMO and inverse DMO  
operators \cite[]{GPR29-03-03740406}:
\begin{eqnarray}
\theta_{10}({\bf x_1;x_0, h_1}) & = &
\left(1-{\bf \left(x_1-x_0\right)^2 \over h_1^2}\right)^{-1/2}\;,
\label{eqn:DMOtt}\\
\theta_{02}({\bf x_0;x_2, h_2}) & = &
\left(1-{\bf \left(x_0-x_2\right)^2 \over h_2^2}\right)^{1/2}\;,
\label{eqn:IDMOtt}
\end{eqnarray}
$w_{10}$ and $w_{02}$ are the corresponding weighting functions (amplitudes of
impulse responses), $\hat{x}_1$ is the component of ${\bf x_1}$ along the
${\bf h_1}$ azimuth, and $\hat{x}_0$ is the component of ${\bf x_0}$
along the
${\bf h_2}$ azimuth. 
Integral operators (\ref{eqn:DMO}) and (\ref{eqn:IDMO}) correspond to
the high-frequency asymptotic (the geometrical seismic) description of
the wave field. As shown by \cite{myDMO}, 
operator (\ref{eqn:IDMO}) has an asymptotically equivalent
form
\begin{equation}
P_2\left({\bf x_2},t_2;{\bf h_2}\right)=\int 
\tilde{w}_{02}\left({\bf x_0;x_2, h_2},t_2\right)\,
{\bf D}_{-t_0}^{1/2}P_0\left({\bf x_0},t_2\,\theta_{02}
\left({\bf x_0;x_2, h_2}\right);
{\bf 0}\right)\,d\hat{x}_0\;,
\label{eqn:tIDMO}
\end{equation}
where $\tilde{w}_{02}=w_{02}\,\sqrt{\theta_{02}}$.
\par
Both DMO and inverse DMO operate on 3-D seismic data
as 2-D operators, since their apertures are defined on a line. This
implies that for a given input midpoint ${\bf x_1}$, the corresponding
location of ${\bf x_0}$ must belong to the
line going through ${\bf x_1}$, with the azimuth defined by the input
offset ${\bf h_1}$. 
Similarly, ${\bf x_0}$ must be on the line going through 
${\bf x_2}$ with the azimuth of ${\bf h_2}$ (Figure \ref{fig:amox12}).
These theoretical facts lead us to the following conclusion: 
\begin{quote}
{\em For a given pair of input and output midpoints ${\bf x_1}$ and ${\bf x_2}$
of the AMO operator, the corresponding midpoint ${\bf x_0}$ on
the intermediate zero-offset gather is determined by the  intersection of 
two lines drawn
through ${\bf x_1}$ and ${\bf x_2}$
in the offset directions.} 
\end{quote}
Applying the geometric connection among the three midpoints, we can
find the cascade 
of the DMO
and inverse DMO operators in one step.
For this purpose, it is convenient to choose an orthogonal coordinate
system $\{x,y\}$ on the 
surface in such a way that the direction of the $x$ axis corresponds
to the input azimuth 
(Figure \ref{fig:amox12}).
In this case the connection between the three midpoints is given by
\begin{eqnarray}
y_0=y_1\;;\;x_0=x_2-\left(y_2-y_1\right) \cot{\varphi}\;,
\label{eqn:x012}\\
d\hat{x}_1=dx_1\;;\;d\hat{x}_0=dy_1 \csc{\varphi}\;.
\label{eqn:jacob}
\end{eqnarray}

\inputdir{XFig}
\plot{amox12}{width=5.in,height=2.5in}{Geometric
relationships between input and output midpoint locations in AMO.}
\par
Substituting (\ref{eqn:DMO}) into (\ref{eqn:tIDMO}) and taking into account
(\ref{eqn:jacob}) produces the 3-D integral AMO operator (\ref{eqn:AMO}),
where
\begin{eqnarray}
\theta_{12}\left({\bf x_1;x_2, h_2}\right) & = &
\theta_{02}\left({\bf x_0;x_2, h_2}\right)\,
\theta_{10}\left({\bf x_1;x_0, h_1}\right)=
{\left|{\bf h_1 \over h_2}\right|}\,
\sqrt{{\bf {h_2^2-\left(x_2-x_0\right)^2} \over
{h_1^2-\left(x_1-x_0\right)^2}}}
\nonumber \\
& = &
{\left|{\bf h_1 \over h_2}\right|}\,
\sqrt{{{\bf h_2^2}\,\sin{\varphi}^2-\left(y_2-y_1\right)^2} \over
{{\bf h_1^2}\,\sin{\varphi}^2-
\left(\left(x_2-x_1\right)\,\sin{\varphi}-
\left(y_2-y_1\right)\,\cos{\varphi}\right)^2}}\;,
\label{eqn:AMOtt}
\end{eqnarray}
\begin{eqnarray}
\nonumber
w_{12}\left({\bf x_1;x_2, h_2},t_2\right) = \\
w_{02}\left({\bf x_0;x_2, h_2},t_2\right)\,
w_{10}\left({\bf x_1;x_0, h_1},t_2\,
\theta_{02}\left({\bf x_0;x_2, h_2}\right)\right)\,
{\csc{\varphi}\over \sqrt{\theta_{02}\left({\bf x_0;x_2, h_2}\right)}}\;,
\label{eqn:AMOwf}
\end{eqnarray}
$d{\bf x_1}=dx_1\,dy_1$. Equation 
$t_1=t_2\,\theta_{12}\left({\bf x_1;x_2, h_2}\right)$ is the same
as equation (4) in \cite[]{Biondi.sep.80.125} except for a different
notation. The weighting function of the derived AMO operator
$\left(w_{12}\right)$ 
depends on the weighting functions of DMO and inverse DMO that are involved in
the construction. In Appendix A, we apply equation (\ref{eqn:AMOwf}) to two
popular versions of the DMO weighting functions that
correspond to Hale's \cite[]{Hale} and Zhang's
\cite[]{Zhang.sep.59.201} DMO operators.
\par
Deriving formula (\ref{eqn:AMOtt}), we have to assume
that the input 
and output offset azimuths are different ($\varphi \neq 0$). In the case
of equal azimuths, AMO reduces to 2-D offset continuation (OC). The
location of ${\bf x_0}$ in this case is not constrained by the input
and output midpoints and can take different values on the line.
Therefore the superposition of DMO and inverse DMO for offset
continuation is a convolution on that line. To find the summation path
of the OC operator, we should consider the envelope of the family of traveltime
curves (where $x_0$ is the parameter of a curve in the family):
\begin{equation}
t_1=t_2\,\theta_{12}\left(x_1;x_2, h_2\right) =
t_2\,{\left| h_1 \over h_2\right|}\,
\sqrt{ {h_2^2-\left(x_2-x_0\right)^2} \over
{h_1^2-\left(x_1-x_0\right)^2}}\;.
\label{eqn:OCfamily}
\end{equation} 
Solving the envelope condition
\begin{equation}
{\partial \theta_{12} \over \partial x_0}=0
\label{eqn:OCenvelop}
\end{equation}
with respect to $x_0$ produces
\begin{equation}
x_0={{\left(\Delta\,x\right)^2+h_2^2-h_1^2+
\mbox{sign}\left(h_1^2-h_2^2\right)\,
\sqrt{\left(\left(\Delta\,x\right)^2-h_1^2-h_2^2\right)^2-4\,h_1^2\,h_2^2}}
\over {2\,\left(\Delta\,x\right)}}\;,
\label{eqn:OCx0}
\end{equation}
where $\Delta\,x=x_1-x_2$.
Substituting (\ref{eqn:OCx0}) into (\ref{eqn:OCfamily}), we get the explicit
expression of the OC summation path:
%\samepage{
\begin{eqnarray}
t_1 & = & \nonumber \\
& & {t_2 \over \left|h_2\right|}\,\sqrt{{U+V} \over 2}
\;\mbox{for $h_2 > h_1$}\;, \nonumber \\
& & {t_2 \left|h_1\right|}\,\sqrt{2 \over {U+V}}
\;\mbox{for $h_2 < h_1$}\;, 
\label{eqn:OCtt}
\end{eqnarray}  
where $U=h_1^2+h_2^2-\left(\Delta\,x\right)^2$, and
$V=\sqrt{U^2-4\,h_1^2h_2^2}$.
%}
Equation (\ref{eqn:OCtt}) corresponds to formula (6) in
\cite[]{Biondi.sep.80.125} (with a typo corrected). The same
expression was obtained in a different way by \cite{myDMO}. 
The apparent difference between the 2-D and 3-D solutions introduces the
problem of finding a consistent description valid for both cases.
Such a description is especially important for practical applications
dealing with small angles of azimuth rotation, e.g. cable feather
correction in marine seismics. The next section develops
a way of solving
this problem, which
refers to the kinematic theory of AMO and follows the ideas that
\cite{GPR29-03-03740406} applied
to DMO-type operators.

\section{AMO APERTURE: CASCADING MIGRATION AND MODELING}
%%%%%%%%%%%%%%%%%%%%%%%%%%%%%%%%%%%%%%%%%%%%%%%%%%%%%%%%%%%%%%%%%%%%%%%
The impulse response of the AMO operators corresponds to a spike on
the initial constant-offset constant-azimuth gather. Such a spike
can physically occur in the case of a focusing ellipsoidal
reflector whose focuses are coincident with the initial source and
receiver locations (the impulse response of prestack common-offset
migration). Therefore, the impulse response of AMO corresponds 
kinematically to a reflection from this ellipsoid. These
considerations allow us to define AMO as the cascade of the 3-D
common-offset common-azimuth migration and the 3-D modeling for a different
azimuth and offset. An analogous point of
view was
developed for the 2-D case
by \cite{GPR29-03-03740406}.      
\par
Let's consider the general symmetric ellipsoid equation
\begin{equation}
z(x,y)=\sqrt{R^2-\beta\,\left(x-x_1\right)^2-\left(y-y_1\right)^2}\;,
\label{eqn:ellips}
\end{equation}
where $z$ stands for the depth coordinate, $R$ is the small semi-axis
of the ellipsoid, and $\beta$ is a nondimensional parameter describing
the stretching of the ellipse $(\beta < 1)$. \cite{GPR29-03-03740406}
derived the following connections between the geometric properties of
the reflector and the coordinates of the corresponding spike in the
data:
\begin{equation}
R={{v\,t_1}\over 2}\;;\;
\beta={t_1^2 \over t_1^2+{{4\,{\bf h_1}^2}\over v^2}}\;,
\label{eqn:rbeta}
\end{equation}
where $v$ is the wave velocity. 
The center of the ellipsoid is at the initial midpoint ${\bf x_1}$.
\par
This section addresses the kinematic problem of reflection
from the ellipsoid defined by (\ref{eqn:ellips}). In particular, we are looking for
the answer to the following question: {\em For a given elliptic
reflector defined by the input midpoint, offset, and time coordinates,
what points on the surface can form a source-receiver pair valid for a
reflection?} If a point in the output midpoint-offset space
cannot be related to a reflection pattern, we should exclude it from the AMO
impulse response defined in (\ref{eqn:AMO}).
\par 
Fermat's principle provides a general method of solving the kinematic reflection problems. Consider a formal expression for the two-point
reflection traveltime 
\begin{equation}
t={\sqrt{
(\mbox{\boldmath{$s - \xi$}})^2+z^2(\xi_x,\xi_y)}
\over v}+
{\sqrt{
(\mbox{\boldmath{$r - \xi$}})^2+z^2(\xi_x,\xi_y)}
\over v}\;,
\label{eqn:reflecttt}
\end{equation}  
where \boldmath$\xi=$\unboldmath$\left\{\xi_x,\xi_y\right\}$ is the vertical
projection of the reflection 
point to the surface, ${\bf s}=\left\{s_x,s_y\right\}={\bf x_2-h_2}$
is the source location, and 
${\bf r}=\left\{r_x,r_y\right\}={\bf x_2+h_2}$ is the receiver location.
According to Fermat's principle, the reflection ray path between two
fixed points must correspond to the
extremum value of the traveltime. Hence, in the vicinity of a
reflected ray,
\begin{equation}
{\partial t \over \partial \xi_x}=0\;;\;
{\partial t \over \partial \xi_y}=0\;.
\label{eqn:fermat}
\end{equation}   
Solving the system of equations (\ref{eqn:fermat}) for $\xi_x$ and $\xi_y$
allows us to find the reflection ray path for a given source-receiver
pair on the surface. The solution is derived in Appendix B to be
\begin{equation}
\xi_x={{x_0-\beta\,x_1}\over{1-\beta}}\;,
\label{eqn:x02xi}
\end{equation} 
\begin{equation}
\xi_y=y_1+\left(x_0-\xi_x\right)\,\cot{\varphi}-
{{\left(y_2-y_1\right)\,\left[
\left(x_0-\xi_x\right)^2-\beta\,\left(x_1-\xi_x\right)^2+R^2
\right]}\over
{{\bf h_2^2}\sin^2{\varphi}-\left(y_2-y_1\right)^2}}\;,
\label{eqn:xiy}
\end{equation} 
where $x_0$ has the same meaning as in the preceding section
and is defined by (\ref{eqn:x012}).
\par
Since the reflection point is contained inside the ellipsoid,
its projection obeys the evident inequality
\begin{equation}
\left(\xi_y-y_1\right)^2\leq R^2-\beta\,\left(\xi_x-x_1\right)^2\;.
\label{eqn:xiyleq}
\end{equation}
It is inequality (\ref{eqn:xiyleq}) that defines the aperture of the AMO
operator.

\inputdir{app}
\plot{amoapp}{width=6.in,height=6.in}{The AMO impulse
response traveltime. Parameters: $\left|{\bf h1}\right|=1000$ m,
$\left|{\bf h2}\right|=750$ m, $t_1=1$ sec. The top plots illustrate
the case of an unrealistically low velocity ($v=10$ m/s); on the bottom,
$v=2000$ m/s. On the left side the azimuth rotation
$\varphi=30^{\circ}$; on the right, $\varphi=3^{\circ}$ }
\par
The AMO operator's contours for
different azimuth rotation angles are shown in  
Figure \ref{fig:amoapp}. 
Comparing the results for the case of an unrealistically low velocity
(the top two plots in Figure \ref{fig:amoapp}) and the case of a realistic
velocity (the bottom two plots) clearly demonstrates  
the gain in the reduction of the aperture size 
achieved by the aperture limitation.
The gain is
especially spectacular for small azimuths. When the azimuth rotation
approaches zero, the area of the 3-D aperture monotonously shrinks to a
line, and the limit of the traveltime of the AMO impulse response
(the inverse of (\ref{eqn:AMOtt})) approaches the offset continuation operator
(\ref{eqn:OCtt}) (Figures \ref{fig:amocom}). This means that
taking into account 
the aperture limitations of AMO provides a consistent description
valid for small azimuth rotations including zero (the offset 
continuation case). Obviously, the cost of an integral operator is
proportional to its size. The size of the offset continuation
operator cannot extend the difference between the offsets ${\bf
\left|\left|h_1\right|-\left|h_2\right|\right|}$. If we applied DMO and
inverse DMO explicitly, the total size of the two operators would be
about ${\bf\left|h_1\right|+\left|h_2\right|}$, which is
substantially greater. This fact proves that in the case of small
azimuth rotations the AMO price is less than those of not only 3-D prestack
migration, but also 3-D DMO and inverse DMO combined \cite[]{anat}.
Figure \ref{fig:amoavs} shows the saddle shape of the AMO operator impulse
response in a 3-D AVS display. 

\inputdir{Sage}
\sideplot{amocom}{width=3.5in}{Traveltime curves
of the impulse responses. The dashed lines indicate the AMO impulse
response with an azimuth rotation of 3 degrees (projection on the $x$
plane); the solid lines, the 2-D offset continuation impulse response.}

\inputdir{.}
\plot{amoavs}{width=5in}{AMO impulse
response traveltime in three dimensions (the AVS display). Parameters:
$\left|{\bf h1}\right|=1000$ m,  
$\left|{\bf h2}\right|=750$ m, $t_1=1$ sec, $v=2000$ m/s,
$\varphi=30^{\circ}$.}

%\pagebreak
\section{Conclusions}

We have applied two different theoretical approaches to AMO to find a
complete definition of the integral operator
(\ref{eqn:AMO}). \cite{Biondi.sep.80.125} proposed cascading the DMO
and inverse DMO operators to define AMO in the frequency domain. The
same approach is repeated here in a simpler way by transferring the
analysis to the natural time-space domain. A new contribution to the
evaluation of the AMO operator follows from applying a different
approach, which extends the geometric theory of DMO
\cite[]{GPR29-03-03740406} to the AMO case. Cascading prestack
migration and modeling allows us to evaluate the AMO operator aperture.
The compactness of the AMO aperture indicates that the integral operator can be
performed at a low cost and therefore
promises economic benefits for its practical implementation.
%\nopagebreak
\bibliographystyle{seg}
\bibliography{SEP2,SEG,paper}

%\pagebreak
\appendix
\section{APPENDIX A: AMO AMPLITUDE}
The weighting function of the AMO operator can be determined from 
cascading the DMO and inverse DMO operators by means of equation
(\ref{eqn:AMOwf}). In the case of Hale's DMO \cite[]{Hale} and
its adjoint \cite[]{GEO52-07-09730984},
\begin{equation}
w_{10}\left({\bf x_1;x_0, h_1},t_0\right) = 
\sqrt{t_0 \over 2\,\pi}\,
{{\bf {\left|h_1\right|} \over {h_2^2-\left(x_1-x_0\right)^2}}}\;,
\label{eqn:HDMOwf}
\end{equation}
\begin{equation}
w_{02}\left({\bf x_0;x_2, h_2},t_2\right)  = 
\sqrt{t_2 \over 2\,\pi}\,
{{\bf {\left|h_2\right|} \over {h_2^2-\left(x_0-x_2\right)^2}}}\;.
\label{eqn:HIDMOwf}
\end{equation}  
As follows from (\ref{eqn:HDMOwf}),(\ref{eqn:HIDMOwf}), and (\ref{eqn:AMOwf}),
\begin{eqnarray}
\nonumber
w_{12}\left({\bf x_1;x_2, h_2},t_2\right) = {t_2 \over {2\,\pi}}\,
\times \\
{{{\bf \left|h_1\right|\,\left|h_2\right|}\,\sin{\varphi}}\over
{\left({\bf h_1^2}\,\sin{\varphi}^2-
\left(\left(x_2-x_1\right)\,\sin{\varphi}-
\left(y_2-y_1\right)\,\cos{\varphi}\right)^2\right)\,
\left({\bf h_2^2}\,\sin{\varphi}^2-\left(y_2-y_1\right)^2\right)}}\;.
\label{eqn:HAMOwfdf}
\end{eqnarray}
In the case of the so-called {\em true-amplitude} DMO
\cite[]{black} and its asymptotic inverse,
\begin{equation}
w_{10}\left({\bf x_1;x_0, h_1},t_0\right)  = 
\sqrt{t_0 \over 2\,\pi}\,
{\bf {h_1^2+\left(x_1-x_0\right)^2} \over
\left|h_1\right|\,\left(h_1^2-\left(x_1-x_0\right)^2\right)}\;,
\label{eqn:DMOwf}
\end{equation}
\begin{equation}
w_{02}\left({\bf x_0;x_2, h_2},t_2\right)  = 
\sqrt{t_2 \over 2\,\pi}\,
{{\bf {\left|h_2\right|} \over {h_2^2-\left(x_0-x_2\right)^2}}}\;.
\label{eqn:IDMOwf}
\end{equation}  
Inserting (\ref{eqn:DMOwf}) and (\ref{eqn:IDMOwf}) into (\ref{eqn:AMOwf}) yields
%\pagebreak
%\samepage{
\begin{eqnarray}
\nonumber
w_{12}\left({\bf x_1;x_2, h_2},t_2\right) = {t_2 \over {2\,\pi}}\,
{\left|{\bf h_2 \over h_1}\right|}\,\times \\
{{{\bf h_1^2}\,\sin{\varphi}^2+
\left(\left(x_2-x_1\right)\,\sin{\varphi}-
\left(y_2-y_1\right)\,\cos{\varphi}\right)^2}\over
{\left({\bf h_1^2}\,\sin{\varphi}^2-
\left(\left(x_2-x_1\right)\,\sin{\varphi}-
\left(y_2-y_1\right)\,\cos{\varphi}\right)^2\right)\,
\left({\bf h_2^2}\,\sin{\varphi}^2-\left(y_2-y_1\right)^2\right)}}\;.
\label{eqn:AMOwfdf}
\end{eqnarray}
%}
%\pagebreak
\appendix
\section{APPENDIX B: DERIVING THE AMO APERTURE}
\inputdir{XFig}
\sideplot{amosym}{width=3.5in}{Reflection from the
ellipsoid of a prestack migration impulse response (a scheme). Top: Map view.
Bottom: Section of the ellipsoid with the plane drawn through the
central line and the reflection point.}

%\samepage{
This appendix describes the derivation of the main formulas for the
aperture evaluation that follow from the Fermat principle (\ref{eqn:fermat}).
In order to avoid the algebraic
complications  of (\ref{eqn:fermat}), we simplify the
problem by taking into account the cylindrical symmetry of the ellipsoidal
reflector (\ref{eqn:ellips}).
%\nopagebreak
\par
Consider a plane drawn through the reflection point and the central line
of the ellipsoid (the axis of the cylindrical symmetry). This plane
has to contain the central (normally reflected) ray from the
reflector. This conclusion follows from the fact that all the normal
reflections emerge at the central line because of the cylindrical
symmetry, as shown in Figure \ref{fig:amosym}. The intersection of the 3-D
reflector and the plane is the 
2-D ellipse
\begin{equation}
\hat{z}(x)=\sqrt{R^2-\beta\,\left(x-x_1\right)^2}\;.
\label{eqn:ellips2D}
\end{equation}
The connection between the emergence point of the normal ray $x_0$ and
the $x$ coordinate of the reflection point $\xi_x$ can be derived from the
relationship evident in Figure \ref{fig:amosym}, as follows:
\begin{equation}
x_0=\xi_x-\hat{z}\left(\xi_x\right)\,\tan{\alpha}=
\xi_x+\hat{z}\left(\xi_x\right)\hat{z}'\left(\xi_x\right)=\xi_x\,(1-\beta)+
\beta\,x_1\,.
\label{eqn:xi2x0}
\end{equation}
Equation (\ref{eqn:xi2x0}) allows us to evaluate $\xi_x$ in terms of
$x_0$ and get (\ref{eqn:x02xi}). The emergence point of the normal ray $x_0$
corresponds to the 
midpoint on an imaginary zero-offset section ( with a coincident
source and receiver). Therefore, the location of this point is
determined for given input 
and output midpoints in accordance with expression (\ref{eqn:x012}).
%} 
\par
Obviously, the reflection point has to be inside the ellipse
(\ref{eqn:ellips2D}). Therefore, its projection obeys the inequality
\begin{equation}
\left|\xi_x-x_1\right| \leq {R\over \sqrt{\beta}}\;.
\label{eqn:xileq}
\end{equation} 
As follows from (\ref{eqn:xileq}), (\ref{eqn:xi2x0}), and (\ref{eqn:rbeta}),
\begin{equation}
\left|x_0-x_1\right| \leq {R\,(1-\beta)\over \sqrt{\beta}}=
{{\bf h_1^2}\over \sqrt{{{v^2\,t_1^2}\over 2}+{\bf h_1^2}}}\;.
\label{eqn:x0leq}
\end{equation}
Inequality (\ref{eqn:x0leq}) is the known aperture limitation of the
DMO operator (\ref{eqn:DMO}) found by \cite{GPR29-03-03740406}. The
equality in (\ref{eqn:x0leq}) is achieved when the reflection point is
on the surface, where the reflector dip increases to 90 degrees.
%\nopagebreak
\par
Now the only unknown left in our problem is the $y$-coordinate of the
reflection point $\xi_y$. To find this unknown, we substitute
(\ref{eqn:x02xi}) into (\ref{eqn:reflecttt}), choosing the convenient
parameterization
\begin{equation}
{\bf s  =  x_0+ h_s\;;\; r  =   x_0+ h_r}\;, 
\label{eqn:x02sr}
\end{equation}
where ${\bf h_r-h_s}=2\,{\bf h_2}$, and ${\bf h_r+h_s}=2\ {\bf
\left(x_1-x_0\right)}$ (Figure \ref{fig:amosym}). The 
two-point traveltime function in (\ref{eqn:reflecttt}) transforms to the form
\begin{eqnarray}
t & = & {\sqrt{\left(x_0-\xi_x\right)^2-\beta\,\left(x_1-\xi_x\right)^2+R^2+
{\bf h_s^2}+2\,{\bf h_s^2}\cdot\left(\mbox{\boldmath{$x_0-\xi$}}\right)}
\over v}+
\nonumber \\
& & + {\sqrt{\left(x_0-\xi_x\right)^2-\beta\,\left(x_1-\xi_x\right)^2+R^2+
{\bf h_r^2}+2\,{\bf h_r^2}\cdot\left(\mbox{\boldmath{$x_0-\xi$}}\right)}
\over v}\;.
\label{eqn:reflectrt}
\end{eqnarray} 
Applying the second equation from (\ref{eqn:fermat}), we get a simple linear
equation for $\xi_y$, which has the explicit solution (\ref{eqn:xiy}).
From (\ref{eqn:x02xi}) and (\ref{eqn:xiy}) one can find the reflection
point location for given midpoint and offset. To find the limits of
possible output midpoint locations, we constrain the reflection
point to be inside the ellipsoid (\ref{eqn:ellips}) similarly to the way we did
in two dimensions when deriving (\ref{eqn:x0leq}). First, let's consider the
case of $y_2=y_1$ (the output midpoint ${\bf x_2}$ is on the line drawn
through ${\bf x_1}$ in the direction of the input azimuth). In this
case, combining expression (\ref{eqn:xiy}) and inequality (\ref{eqn:xiyleq})
produces
\begin{equation}
\left|x_0-x_1\right| \leq {R\,(1-\beta)\over 
\sqrt{\beta+\beta^2\,\cot^2{\varphi}}}\;.
\label{eqn:x0leq3D}
\end{equation}
For any azimuth rotation angle $\varphi$ less than 90 degrees, the
limitation (\ref{eqn:x0leq3D}) is smaller than that of the DMO operator
(\ref{eqn:x0leq}). The difference increases with the decrease of the
azimuth rotation, since the AMO aperture section
on the line $y_2=y_1$ monotonously shrinks to a point $x_2=x_0=x_1$
when $\varphi$ approaches zero. To extend this conclusion to the whole
3-D aperture, we can find the contour of the aperture by putting the
reflection point 
at the edge of the ellipsoid (\ref{eqn:ellips}), as follows:  
\begin{equation}
\left(\xi_y-y_1\right)^2= R^2-\beta\,\left(\xi_x-x_1\right)^2
\label{eqn:edge}
\end{equation}
and solving (\ref{eqn:xiy}) for $y_2$. The aperture contour can then be defined by
the system of parametric expressions
\begin{equation}
y_2\left(\xi_x\right)  = 
y_1+d\left(\xi_x\right)\,\sin{\varphi}\;,
\label{eqn:ycon}
\end{equation}
\begin{equation}
x_2\left(\xi_x\right)  =  \xi_x\,(1-\beta)+\beta\,x_1+
d\left(\xi_x\right)\,\cos{\varphi}\;,
\label{eqn:xcon}
\end{equation}
where
\begin{equation}
d\left(\xi_x\right)= 
{{d_y^2+d_x^2-
\sqrt{\left(d_y^2+d_x^2\right)^2+
4\,{\bf h_2^2}\,\left(
d_y\,\sin{\varphi}+d_x\,\cos{\varphi}\right)^2}}\over
{2\,\left(d_y\,\sin{\varphi}+d_x\,\cos{\varphi}
\right)}}\;,
\label{eqn:dxi}
\end{equation}
$d_x\left(\xi_x\right)=\xi_x-x_0=\beta\,\left(\xi_x-x_1\right)$,
and $d_y\left(\xi_x\right)=\xi_y-y_1$ is defined by (\ref{eqn:edge}).
