Clearly singularities are one of the more important features of
slowness surfaces; how do they change when the underlying elastic
constants are perturbed?
\label{Sec-Sing}

\Tactiveplot{Sing-Crack}{height=7.50in}{\figdir}
{How the TI kiss singularity splits.}
{
\small
Successive perturbations away from transverse isotropy. (The plots
are ordered from left to right and then top to bottom.)
The first (top left) plot corresponds to the TI medium shown
in the top plot in Figure~\protect\ref{Separ3-Wavedef}.
Only the outer {\qS2} wave is shown here.
The view is from the ``North Pole'': the $+k_z$ axis points
directly out of the page, the $+k_x$ axis points to the bottom of the
page and the $+k_y$ axis points to the right.
For the second (top right) and third (middle left) plot
successively more cracks are added in the $x$-$z$ plane.
Starting with the fourth (middle right) plot the first set of cracks
are held constant and more and more cracks are added
in the $y$-$z$ plane instead. For the fifth plot (lower left)
the amount of $x$-$z$ and $y$-$z$ cracking are the same.
\index{shear singularities | example}
\index{shear singularities | cracks}
\index{slowness surface | 3D example}
}

Figure~\ref{Sing-Crack} shows a succession of {\qS2} slowness surfaces
as the elastic constants are successively perturbed away from
transverse isotropy. The first plot is transversely isotropic.
Cracks are then added in the $x$-$z$ plane
to perturb the elastic
constants to become orthorhombic
\refer{Nichols, Muir, and Schoenberg}
{Elastic properties of rocks with multiple sets of fractures}
{1989}.
The TI kiss singularity at the center of the plot splits into
two point singularities on the $k_y$ axis,
a typical configuration for orthorhombic media
\refer{Crampin and Kirkwood}
{Velocity variations in systems of anisotropic symmetry}
{1981}.
Starting with the fourth plot in the sequence the $x$-$z$ cracks
are held constant and another crack set is
added in the $y$-$z$ plane.
The singularities move back towards the center and converge
back into a single kiss singularity when the amount of cracking
in the two perpendicular directions is equal. As the cracking in
the $y$-$z$ plane becomes dominant in the last plot,
the kiss singularity bifurcates again, this time along the $k_x$ axis.

\Tactivesideplot{Sing-Funny}{width=3.60in}{\figdir}
{Another view of Figure~\protect\ref{Separ3-Funnyshear}.}
{
Another view of the upper plot in Figure~\protect\ref{Separ3-Funnyshear}.
A Saskatchewan shaped portion of the slowness surface
containing the three point singularities visible near the $+k_y$ axis
has been cut out and rotated to the front.
(Ideally, Saskatchewan is bounded by two lines of constant
latitude and two lines of constant longitude.)
\index{shear singularities | example}
\index{slowness surface | 3D example}
}
\Tactiveplot{Sing-Sweep}{width=5.75in}{\figdir}
{How a singularity ``dipole'' appears.}
{
This sequence of four plots (left to right and then top to bottom)
shows how the particle-motion directions change as an orthorhombically
anisotropic medium is perturbed to become generally anisotropic.
Two singularities of opposite sign appear together out of nowhere
and then separate.
Figure~\protect\ref{Sing-Funny} forms the fifth plot in the sequence
and shows the orientation of all of the plots.
The elastic constants are given in Note~\protect\ref{Const3-Funny}
on page~\protect\pageref{Const3-Funny}.
\index{shear singularities | classification of}
\index{shear singularities | example}
\index{slowness surface | 3D example}
}

This behavior suggests to me an alternative notation for singularities
that has some advantages over the ``\{point, kiss, intersection\}''
terminology now in common use.
(This notation was first introduced by
Crampin and Yedlin~\referna{Crampin and Yedlin}
{Shear-wave singularities of wave propagation in anisotropic media}
{1981}, and more recently has been discussed by
Winterstein~\referna{Winterstein}
{Velocity anisotropy terminology for geophysicists}
{1990}.)
I suggest that singularities could be classified
by the number of half-loops the particle-motion direction vector
completes in one loop around the singularity, reminiscent of
integration around residuals in complex analysis.
\index{shear singularities | classification of}
In particular it is possible
for the particle-motion vector to rotate in the same
sense as the loop is traversed ($+$) or in the opposite sense ($-$).
This seems to be a fundamental property of singularities
that is robust against
perturbations in the elastic constants; the sum about a group of
singularities also appears to be conserved when
two of them merge or a double one splits.

Figure~\ref{Separ3-Blowup} shows a canonical example of an order $+1$
singularity; the particle-motion direction vector performs
a half flip in the same direction as the traverse around the singularity.
In Figure~\ref{Sing-Crack} the order $+2$ singularity on the $k_z$ axis
at the center of the plot splits into two order $+1$ singularities.
Although harder to see, the infinity of order $0$ intersection singularities
(the dark circle halfway out from the center in the first plot
in Figure~\ref{Sing-Crack}) splits into 4 order $+1$ singularities
(on the $k_x$ and $k_y$ axes) and 4 order $-1$ singularities (in between).

The labeled ``point'' singularity in Figure~\ref{Separ3-Funnyshear}
shows a clearer example of an order $-1$ singularity (the two adjacent
ones are each of order $+1$).\footnote{A closely related classification
scheme for ``umbilic points'' on surfaces is described by
Berry and Hannay~\referna{Berry and Hannay}
{Umbilic points on Gaussian random surfaces}
{1977}. My order $-1$ singularity they would call a ``Star''
singularity, which has ``index $-1/2$''.
My order $+1$ singularity they would
call a ``Lemon'' singularity, which has ``index $+1/2$''.
They also describe one more possible singularity type, the ``Monstar'',
which also has index $+1/2$.
I have not been able to generate an example of
a ``monstar'' shear singularity.}
Figure~\ref{Sing-Sweep} shows how this singularity spontaneously
appears as part of a singularity-antisingularity pair as originally
orthorhombic elastic constants are perturbed more and more.
The original order $+1$ singularity of the orthorhombic medium moves
slightly but cannot split, since the particle-motion direction vector
must complete an integral number of half-flips.
