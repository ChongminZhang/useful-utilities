\def\myrotate{0}
%%
\def\captionB#1{
the image obtained by downward continuation in
Cartesian coordinates with the #1 equation (a);
%%
the image in panel (a) interpolated 
to ray coordinates (b);
%%
image obtained by extrapolation in 
ray coordinates with the #1 equation (d);
%%
the image in panel (d) interpolated 
to Cartesian coordinates (c).
}

\def\captionA#1{
Panels (a) and (c) correspond to
Cartesian coordinates, and
panels (b) and (d) correspond to 
ray coordinates.
%%
Velocity model with an overlay of the 
ray coordinate system initiated by a #1 
source at the surface (a);
%%
image obtained by downward continuation in 
Cartesian coordinates with the 
$15^\circ$ equation (c);
%%
velocity model with an overlay of the 
ray coordinate system (b);
%%
image obtained by wavefield extrapolation in 
ray coordinates with the 
$15^\circ$ equation (d).
}

\def\captionMarmousiZoom{
Velocity model (a);
image obtained by wavefield extrapolation 
in ray coordinates using
the $15^\circ$ equation (b) and 
the split-step equation (c);
image obtained using downward continuation 
in Cartesian coordinates
with the $45^\circ$ equation (d),
the $15^\circ$ equation (e) and the 
split-step equation (f).}

\def\captionZiggyZoom{
Velocity (a);
finite-difference solution to the
two-way acoustic wave equation for a point source at 
$x=16000$~m (b);
image obtained by downward continuation in
Cartesian coordinates with the $45^\circ$ equation (c);
image obtained by wavefield extrapolation in 
ray coordinates with the $15^\circ$ equation (d).
}

\section{Examples}
We illustrate our method with several synthetic examples of
various degrees of complexity. In all examples, we use 
extrapolation in {2-D} orthogonal Riemannian spaces
(ray coordinates), and compare the results with extrapolation
in Cartesian coordinates.
We present images obtained by migration of synthetic datasets
represented by events equally spaced in time.
In our examples, we use synthetic data from point 
sources located on the surface. After we migrate these data 
in depth, we obtain images which are representations of 
Green's functions from the chosen source point.
In all examples, $(x,z)$ are the Cartesian
space coordinates, $(\t,x_0)$ are the ray coordinates
for plane wave sources, and $(\t,\g)$ are ray
coordinates for point sources.
$\x_0$ stands for surface coordinate, 
$\g$ for shooting angle, and
$\t$ for one-way traveltime.
%%--------------------------------------------------------------
%% b-b example
%%--------------------------------------------------------------
\par
Our first example is designed to illustrate our method 
in a fairly simple model. We use a {2-D} model with 
horizontal and vertical gradients
$\vv(\x,\z)=250+0.2\;x+0.15\;\z$~m/s which gives waves
propagating from a point source a pronounced 
tendency to overturn 
(Figure~\ref{fig:RCsi1.com.ps}).
The model also contains a diffractor located
around $x=3800$~m and $z=3000$~m.
\par
We use ray tracing to create an orthogonal 
ray coordinate system corresponding to a point 
source on the surface at $x=6000$~m.
Figure~\ref{fig:RCsi1.com.ps}(a) shows the velocity
model and the rays in the original Cartesian 
coordinate system ($\x,\z$).
Figure~\ref{fig:RCsi1.com.ps}(b) shows the one-to-one
mapping of the velocity model from 
Cartesian coordinates ($\x,\z$) into ray coordinate ($\t,\g$)
using the functions 
$\x(\t,\g)$ and 
$\z(\t,\g)$ obtained by ray tracing.
The diffractor is mapped to $\t=2.4$~s and 
$\g=-18^\circ$ measured from the vertical.
%%\par
The synthetic data we use is represented by
impulses at the source location at every $0.25$~s.
In ray coordinates, this source is represented by a 
plane-wave evenly distributed over all 
shooting angles $\g$.
Ideally, an image obtained by migrating such a dataset
is a representation of the acoustic wavefield 
produced by a source that pulsates periodically.
\par
Figure~\ref{fig:RCsi1.com.ps}(c)
shows the image obtained by downward continuation 
in Cartesian coordinates
using the standard $15^\circ$ equation.
Figure~\ref{fig:RCsi1.com.ps}(d)
shows the image obtained by wavefield extrapolation
using the ray-coordinate $15^\circ$ equation.
The overlays in panels (c) and (d) are
wavefronts at every $0.25$~s and rays shot at
every $20^\circ$ to facilitate
comparisons between the images in ray and Cartesian
coordinates.
\par
Figure~\ref{fig:RCsi1.f15.ps} is a direct comparison
of the results obtained by extrapolation in the two
coordinate systems.
The image created by extrapolation in Cartesian 
coordinates (a) is mapped to ray coordinates (b).
The image created by extrapolation in ray coordinates (d)
is mapped to Cartesian coordinates (c).
%%\par
Since we use the same velocity for ray tracing and
for wavefield extrapolation, we expect the
wavefields and the overlain wavefronts to be in 
agreement. The most obvious mismatch occurs in regions
where the $15^\circ$ equation fails to extrapolate
correctly at steep dips $\g=-20^\circ\dots-50^\circ$.
This is not surprising since, as its name indicates, 
this equation is only accurate up to $15^\circ$.
%%\par
However, this limitation is eliminated in
ray coordinates, because the coordinate system
\uline{
re-orients the extrapolation closer to the 
direction of wave propagation.}
\sout{
brings the extrapolator in a reasonable position 
and at a good angle,
although the extrapolator uses an equation of a 
similar order of accuracy.}
\par
Another interesting observation in 
Figures~\ref{fig:RCsi1.f15.ps} (a) and (c)
concerns the diffractor we introduced in the velocity
model. When we extrapolate in Cartesian coordinates,
the diffraction is only accurate to a small angle
relative to the extrapolation direction (vertical).
In contrast, the diffraction develops relative to the 
propagation direction when computed in ray coordinates,
thus being more accurate after mapping to Cartesian 
coordinates.
%%
\plot{RCsi1.com.ps}{angle=\myrotate,width=6.0in}
{Simple linear gradient model: \captionA{point}}
%%
\plot{RCsi1.f15.ps}{angle=\myrotate,width=6.0in}
{Simple linear gradient model: \captionB{$15^\circ$}}
%%
We can also observe that the diffractions created
by the anomaly in the velocity model are not at all
limited in the ray coordinate domain.
In a beam-type approach, such diffraction would not 
develop beyond the extent of the beam
in which it arises. Neighboring extrapolation 
beams are completely insensitive to the 
velocity anomaly. 
%%--------------------------------------------------------------
%% Gaussian anomaly example
%%--------------------------------------------------------------
\par
The second example is a smooth velocity 
with a negative Gaussian anomaly that creates
a triplication of the ray coordinate system
(Figure~\ref{fig:RCga1.com.ps}).
Everything other than the velocity model is 
identical to its counterpart in the preceding 
example.
%%\par
Similarly to Figure~\ref{fig:RCsi1.com.ps},
panels (a) and (c) correspond to Cartesian 
coordinates, and panels (b) and (d) correspond
to ray coordinates. Using regularization of the
ray coordinates parameters, we are able to 
extrapolate through the triplication. 
The discrepancy between the wavefields and
the corresponding wavefronts highlight
the decreasing accuracy in the caustic region
caused by the parameter regularization.
%%
\plot{RCga1.com.ps}{angle=\myrotate,width=6.0in}
{Gaussian anomaly model: \captionA{point}}
%%
\plot{RCga1.f15.ps}{angle=\myrotate,width=6.0in}
{Gaussian anomaly model: \captionB{$15^\circ$}}
%%
The ``butterfly'' in Figure~\ref{fig:RCga1.f15.ps} (b) 
is another indication that the ray coordinate system
is triplicating and that the 
\sout{Cartesian coordinates are multi-valued 
functions of ray coordinates}
\uline{
ray coordinates are multi-valued functions 
of Cartesian coordinates}.
None of this happens when
we extrapolate in ray coordinates (d) and 
interpolate to Cartesian coordinates (c),
since the mappings $\x(\t,\g)$ and 
$\z(\t,\g)$ are single-valued.
\par
Comparing panels (a) and (c) of Figure~\ref{fig:RCga1.f15.ps},
we notice that the triplication tails at, for example,
$\x\approx7000$~m and $\z\approx4000$~m extend farther
with the Cartesian extrapolator (a) than with the 
Riemannian extrapolator (c). 
The triplications create internal boundaries in the coordinate
system which are better avoided.
\uline{
Figures~\ref{fig:RCga3.com.ps} and \ref{fig:RCga3.f15.ps} are 
similar to
Figures~\ref{fig:RCga1.com.ps} and \ref{fig:RCga1.f15.ps},
except that the velocity used to create the coordinate system
is different from the one used for extrapolation.
In this case, the coordinate system does not triplicate,
therefore the internal boundaries of the extrapolation domain
do not exist.
The triplication is completely described by extrapolation
in ray coordinates, and the angular accuracy is limited
by the extrapolation kernel accuracy. In this example,
the angle between the wave propagation direction 
and the extrapolation direction is small, such that
a $15^\circ$ kernel is accurate enough.
}
%%
\plot{RCga3.com.ps}{angle=\myrotate,width=6.0in}
{Gaussian anomaly model: \captionA{point}
Compare with Figure~\ref{fig:RCga1.com.ps}.}
%%
\plot{RCga3.f15.ps}{angle=\myrotate,width=6.0in}
{Gaussian anomaly model: \captionB{$15^\circ$}
Compare with Figure~\ref{fig:RCga1.f15.ps}.}
%%
%%--------------------------------------------------------------
%% Marmousi example
%%--------------------------------------------------------------
\par
Our next example is the more complicated Marmousi
model. Figure~\ref{fig:RCma1.com.ps} shows
the velocity models mapped into the two
different domains, and the wavefields obtained
by extrapolation in each one of them.
We create the ray coordinate system by 
ray tracing in a smooth version of the model,
and extrapolate in the rough version.
The source is located on the surface at $x=5000$~m.
\par
In this example, the wavefields triplicate in 
both domains (Figure~\ref{fig:RCma1.f15.ps}).
Since we are using a $15^\circ$ equation,
extrapolation in Cartesian coordinates is only
accurate for the small incidence angles, as
observed in panels (a) and (b).
In contrast, extrapolating in ray coordinates
(d) does not have the same angular limitation,
which can be seen after mapping back to 
Cartesian coordinates (c).
%%
\plot{RCma1.com.ps}{angle=\myrotate,width=6.0in}
{Marmousi model: \captionA{point}}
%%
\plot{RCma1.f15.ps}{angle=\myrotate,width=6.0in}
{Marmousi model: \captionB{$15^\circ$}}
%%
\plot{RCma1.zom.ps}{angle=\myrotate,width=6.0in}
{Marmousi model: \captionMarmousiZoom}
%%
\par
Figure~\ref{fig:RCma1.zom.ps} is a close-up
comparison of the wavefields obtained by 
extrapolation with different methods in different
domains.
Panel (a) is a window of the velocity model for
reference. Panels (b) and (c) are obtained
by extrapolation in ray coordinates using
the $15^\circ$ and split-step equations, respectively.
Panels (d), (e) and (f) are obtained
by downward continuation in Cartesian coordinates
using the $45^\circ$, $15^\circ$ and split-step
equations, respectively.
The ray-coordinate extrapolation results are similar
to the Cartesian coordinates results in the regions
where the wavefields propagate mostly vertically,
but are much improved in the regions where the 
wavefields propagate almost horizontally.
\par
Figure~\ref{fig:RCga2.kin.ps} illustrates the difference
between wavefield extrapolation using \req{oneway.3d}, panel (b)
and wavefield extrapolation using \req{oneway.3d.kinematic}, panel (c).
Kinematically, the two images are equivalent and the main changes
are related to amplitudes.
Panels (b) and (c) have the same clip to highlight the point that
only the amplitudes change but not the kinematics.
%%
\plot{RCga2.kin.ps}{width=6.0in}{
The effect of neglecting the first order 
terms in Riemannian wavefield extrapolation.
From left to right 
the velocity model with an overlay of the ray coordinate system (a), 
extrapolation with \req{oneway.3d} including the first order terms (b), and
extrapolation with the simplified \req{oneway.3d.kinematic} (c).
}
%%--------------------------------------------------------------
%% Marmousi example 2
%%--------------------------------------------------------------
\par
Figure~\ref{fig:RCma2.fdm.ps} shows a comparison between 
time domain acoustic finite-difference modeling (a), and
Riemannian wavefield extrapolation (b) for a point
source. The Marmousi velocity model is smoothed to avoid
backscattered energy in panel (a) in order to facilitate
a comparison with the one-way wavefield extrapolator in 
panel (b).
\par
Despite being computed with a one-way extrapolator,
the wavefield in panel (b) captures accurately all the
important features of the reference wavefield
depicted in panel (a), including triplications and
amplitude variations. 
Some of the diffractions in panel
(b) are not as well developed as their counterparts in 
panel (a) due to the limited angular accuracy of the
$15^\circ$ approximation.
Regardless of accuracy, the computed
Riemannian wavefield could not be achieved
with Cartesian-based downward continuation.
%%
\plot{RCma2.fdm.ps}{width=6.0in}{
A comparison of wavefields computed by 
time-domain acoustic finite-difference modeling (a), and
wavefields computed by Riemannian wavefield extrapolation (b).
}
%%
