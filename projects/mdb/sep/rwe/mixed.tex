\section{Mixed-domain solutions to the one-way wave-equation}
We can use \req{oneway.3d} to construct a numerical solution
to the one-way wave equation in the mixed 
$\omega-k$, $\omega-x$ domain.
The extrapolation wavenumber described in \req{oneway.3d}
is, in general, a function dependent on several quantities
\beq
\kz = \kz \lp \ss,\c_j\rp \;,
\eeq
where $\ss \ofpo$ is slowness, and
$\c_j\ofpo=\lc \cx,\cy,\cz,\cxx,\cyy,\czz,\cxy \rc$ are 
coefficients computed numerically from the definition
of the coordinate system, as indicated by 
equations~(\ref{eqn:coefs.3d}).
Specifying a coordinate system, indirectly defines all $\c_j$.
\par
Next, we write the extrapolation wavenumber $\kz$ as a 
first-order Taylor expansion relative to a reference medium:
\beq \label{eqn:mixed.3d}
\kz = {\kz}_0 + 
            \left . \frac{\partial \kz}{\partial\ss }\right |_{ \ss_0,{\c_j}_0} \lp \ss  - \ss_0    \rp +
\sum_{\c_j} \left . \frac{\partial \kz}{\partial\c_j}\right |_{ \ss_0,{\c_j}_0} \lp \c_j - {\c_j}_0 \rp \;,
\eeq
where $\ss\ofpo$ and  $\c_j\ofpo$
represent the spatially variable slowness and coordinate 
system parameters, and  $\ss_0$ and ${\c_j}_0$ are the 
constant reference values at every 
extrapolation level.
\par
As usual, the first part of \req{mixed.3d}, 
corresponding to the extrapolation wavenumber in the 
reference medium ${\kz}_0$,
is implemented in the Fourier domain,
while the second part, corresponding to the spatially variable
medium coefficients, is implemented in the space domain.
\par
If we make the further simplifying assumptions that 
$\kx \approx 0$ and $\ky \approx 0$, 
we can write
\beq \label{eqn:mixed.3d.explicit}
\kz = {\kz}_0 +
\left . \frac{\partial\kz}{\partial\czz}\right |_{0} \lp \ss  -  \ss_0   \rp +
\left . \frac{\partial\kz}{\partial\czz}\right |_{0} \lp \czz - {\czz}_0 \rp + 
\left . \frac{\partial\kz}{\partial\cz }\right |_{0} \lp \cz  - {\cz}_0  \rp \;,
\eeq
\par
where
\beqa
\left . \frac{\partial\kz}{\partial\ss}\right |_{0} &=&
\frac {2\ww \lp \ww\ss_0\rp }
      {            \sqrt {4{\czz}_0\lp\ww\ss_0\rp^2 - {\cz}_0^2} } \;,
\nonumber \\
\left . \frac{\partial\kz}{\partial\czz}\right |_{0} &=&
- \frac{i{\cz}_0}{ 2{\czz}_0^2} +
\frac {  {\cz}_0^2-2{\czz}_0 \lp\ww\ss_0\rp^2 }
      {2{\czz}_0^2 \sqrt {4{\czz}_0\lp\ww\ss_0\rp^2 - {\cz}_0^2} } \;,
\nonumber \\
\left . \frac{\partial\kz}{\partial\cz}\right |_{0} &=&
  \frac{i       }{ 2{\czz}_0  } -
\frac {  {\cz}_0 }
      {2{\czz}_0   \sqrt {4{\czz}_0\lp\ww\ss_0\rp^2 - {\cz}_0^2} } \;.
\eeqa
%%Equation~\reqo{mixed.3d.explicit} is motivated by 
%%a wavefront normal propagation approximation.
By ``0'', we denote the reference medium ($\ss_0,{\c_j}_0)$.
We could also use many reference media, followed by 
interpolation, similarly to the 
Phase-Shift Plus Interpolation (PSPI) technique of
\cite{GEO49.02.01240131}.
\par
For the particular case of Cartesian coordinates
($\cz=0, \czz=1$),
\req{mixed.3d.explicit} reduces to
\beq 
\kz = {\kz}_0 + \ww \lp \ss - \ss_0 \rp \;,
\eeq
which corresponds to the popular Split-Step Fourier (SSF) 
extrapolation method \cite[]{GEO55-04-04100421}.



