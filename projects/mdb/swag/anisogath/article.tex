\hyphenation{Fo-mel}

\def\nn{{\textbf{n}}}
\def\nx{n_x}
\def\ny{n_y}
\def\nz{n_z}
\def\na{\lp \nx^2 + \ny^2 \rp}

\def\nnm{\nn_{m}}
\def\nmz{n_{mz}}
\def\nmx{n_{mx}}
\def\nmy{n_{my}}
\def\nma{\lp \nmx^2 + \nmy^2 \rp}

\def\vm{v_{m}}
\def\td{t_d}



\def\cn{\lp \cc \cdot \nn \rp}

\def\nsx{n_{sx}}
\def\nsz{n_{sz}}
\def\nrx{n_{rx}}
\def\nrz{n_{rz}}
\def\ps{\textbf p_\ss}
\def\pr{\textbf p_\rr}
\def\ns{\textbf n_\ss}
\def\nr{\textbf n_\rr}

%\def\pm{\textbf p_\xx}
\def\ph{\textbf p_\hh}
\def\km{\textbf k_\xx}
\def\kh{\textbf k_\hh}

\def\sinc{\sin \theta}
\def\cosc{\cos \theta}
\def\tanc{\tan \theta}

\def\vq{{\textbf{q}}}

\def\vt{{v(\theta) \tau}}

\def\geosout#1{ \sout{#1} }
\def\geosout#1{}

\def\geouline#1{ \uline{#1} }
\def\geouline#1{#1}

\def\geosou2#1{ \sout{#1} }
\def\geosou2#1{}

\def\geoulin2#1{ \uline{#1} }
\def\geoulin2#1{#1}

\def\geosoum#1{ \sout{#1} }
\def\geosoum#1{}

\def\geoulinm#1{ \uline{#1} }
\def\geoulinm#1{#1}

\def\beqa{\begin{eqnarray}}
\def\eeqa{\end{eqnarray}}


\published{Geophysical Prospecting, 59, 422-431, (2011)}

\righthead{Angle gathers for VTI media}
\title{Angle gathers in wave-equation imaging for transversely isotropic media}

\author{Tariq Alkhalifah, King Abdullah University for Science and Technology, \\
       Sergey Fomel, The University of Texas at Austin}
\maketitle

\begin{abstract}
  In recent years, wave-equation imaged data are often presented in common-image angle-domain gathers as a decomposition in scattering
  angle at the reflector, which provide a natural access to analyzing
  migration velocities and amplitudes. In the case of anisotropic
  media, the importance of angle gathers is enhanced by the need to
  properly estimate multiple anisotropic parameters for a proper
  representation of the medium.  We extract angle gathers for each
  downward-continuation step from converting offset-frequency
  planes into angle-frequency planes simultaneously with applying the
  imaging condition in a transversely isotropic \geoulin2{with a vertical symmetry axis} (VTI) medium. The
  analytic equations, though cumbersome, are exact within the
  framework of the acoustic approximation.  They are also easily
  programmable and show that angle gather mapping in the case \geouline{of}
  anisotropic media differs from its isotropic counterpart, with the difference
  depending mainly on the strength of anisotropy. Synthetic examples demonstrate
  the importance of including anisotropy in the angle gather generation as mapping of
  the energy is negatively altered otherwise. In the case of a titled axis of symmetry (TTI), 
the same VTI formulation is applicable but requires a rotation
  of the wavenumbers.

\end{abstract}

\section{Introduction}

Angle gathers have gained prominence as they provide  wave equation imaging methods with an outlet
to perform velocity analysis. Angle gathers also alleviate the limitations that offset gathers have
\geoulin2{in handling} \geosou2{with} multi pathing \cite[]{GEO55-09-12231234,SEG-2000-08300833,stolk}. An angle gather decomposition for 
anisotropic media will allow us to export these features to the anisotropic world, \geouline{and this is especially important considering the
number of parameters we need to deal with in anisotropic media and the prevalent multi-pathing that takes place in such media.}

Downward wave extrapolation provides an accurate method of seismic
imaging in structurally complex areas. Downward wave extrapolation is
also naturally formulated to produce angle gathers
\cite[]{GEO55-09-12231234,SEG-2000-08300833,GEO67-03-08830889,wu,GEO68-03-10651074,SEG-2003-08890892,new,coord}.
 \cite{SEG-2004-10531056} showed that
structural dependence can be removed in a depth-slice approach to
extracting angle gathers. Specifically, one can generate gathers at
each depth level, converting offset-space-frequency planes into
angle-space planes and applying simultaneously the imaging
condition. The improved mapping retains velocity dependence but
removes the effect of the structure. Because of its
ray-parameter-based (Fourier) formulation, this approach lends itself
naturally to an anisotropic phase-velocity extension.

Migration velocity analysis in anisotropic media remains a challenging
 and open issue. Multiple parameters are needed to represent the
 anisotropic model. For a transversely isotropic medium with vertical
 axis of symmetry (VTI media), only NMO velocity ($v$) and the
 non-elliptic parameter ($\eta$) predominantly influence imaging
 \cite[]{GEO60-05-15501566,tariq-lat}. Vertical velocity ($v_z$)
 controls mainly placement of the image in depth.  Nevertheless,
 estimating even two parameters that can vary laterally and vertically
 from image gathers is difficult. Angle gathers provide an opportunity
 to use residual moveout \geosout{and amplitude information} to help update
 anisotropic parameters. \cite{biondianis} suggested an approach to
 extracting angle gathers in anisotropic media from post-migration
 data. Biondi's formulation is based on numerical calculation of angle gathers and relies on ray information that is hard to examine analytically. 

In this paper, we develop an analytical formulation for extracting
angle gathers in VTI \geouline{2D} media.  We use the depth-slice approach to angle
gathers as a platform for the extension \geoulin2{to VTI.} Angle gather mapping depends
strongly on anisotropic parameters. We analyze this dependency using
numerical computations. Next, \geoulin2{we} show synthetic data examples that confirm the theoretical analysis. Finally, we explain the possible extension of our approach to TTI (tilted transversely isotropic) media.
\geouline{An extension to 3D can be achieved along the lines of} \cite{SEG-2004-10531056}.

\section{The depth slice approach for VTI media}
\inputdir{XFig}

Relations between image coordinates and reflection (scattering) angles
at reflecting interfaces can be extracted by analyzing the geometry of
reflections in the simple case of a dipping reflector in a locally
homogeneous medium \cite[]{SEG-2004-10531056}.  The geometry of the
reflection ray paths in 2-D is depicted in Figure~\ref{fig:raysr}.

\multiplot{2}{raysr,rayparameters}{width=0.45\textwidth}{ (a) A schematic plot showing angle $\theta$. Although the model depicts a homogeneous setting, the
development will rely on the ray parameters defined in the immediate
vicinity of the the reflection point, as shown in b.  
(b) A schematic plot depicting the relation between the source and receiver ray-parameter vectors ($\mathbf{p}_s$ and $\mathbf{p}_g$)
 and the offset and midpoint vectors ($\mathbf{p}_h$ and $\mathbf{p}_m$)}
According to elementary rules of geometry for the ray configuration in
Figures~\ref{fig:raysr} and~\ref{fig:rayparameters}, with the
wavenumber vector given by $\mathbf{k} =\omega\,\mathbf{p}$ as it relates to the
ray-parameter vector for a given angular frequency $\omega$, \geosou2{reflection} \geoulin2{opening (scattering)}
\geouline{phase} angle $\theta$ is represented by the following relation \cite[]{SEG-2004-10531056,coord}
\begin{equation}
  \label{eq:theta}
  k_{\text{hx}}^2+k_{\text{hz}}^2 = k_s^2+k_r^2 - 2 k_r k_s \cos (\theta )\;,
\end{equation}
where $k_{\text{hx}}$ and $k_{\text{hz}}$ are horizontal and vertical components of the offset
wave number, and $k_s$ and $k_r$ are source and receiver wavenumber amplitudes related to their components as follows:
$k_s^2 \equiv k_{\text{sx}}^2+k_{\text{sz}}^2$,
$k_r^2 \equiv k_{\text{rx}}^2+k_{\text{rz}}^2$,
with
$k_{\text{hx}} \equiv k_{\text{rx}} - k_{\text{sx}}$,
$k_{\text{mx}} \equiv k_{\text{rx}} + k_{\text{sx}}$,
as suggested by Figure~\ref{fig:rayparameters}, where $k_{\text{mx}}$ is the horizontal component of the midpoint wavenumber.

To complete the system of equations necessary to relate angle $\theta$ to midpoint and offset \geoulin2{horizontal} 
wavenumbers, we use the dispersion
relation developed by \cite{GEO63-02-06230631} to define each of $k_{\text{sz}}$ and $k_{\text{rz}}$ as follows:
\begin{eqnarray}
  \label{eq:kzs}
  k_{\text{sz}}^2  \equiv & (\omega \frac{\partial t_{\text{s}}}{\partial z})^2=\frac{\omega ^2}{v_z^2}-\frac{v^2 \omega ^2
   \left(k_{\text{hx}}-k_{\text{mx}}\right){}^2}{2 v_z^2 \left(2 \omega ^2-v^2 \eta 
   \left(k_{\text{hx}}-k_{\text{mx}}\right){}^2\right)}\;,
  \\ \label{eq:kzr}
  k_{\text{rz}}^2  \equiv & (\omega \frac{\partial t_{\text{r}}}{\partial z})^2=\frac{\omega ^2}{v_z^2}-\frac{v^2 \omega ^2
   \left(k_{\text{hx}}+k_{\text{mx}}\right){}^2}{2 v_z^2 \left(2 \omega ^2-v^2 \eta 
   \left(k_{\text{hx}}+k_{\text{mx}}\right){}^2\right)}\;,
\end{eqnarray}
\geoulin2{where $v$ is the NMO velocity. Using equation~(\ref{eq:theta}) in its expanded form and} after some manipulation 
and collecting terms with the same power of  $\cos \theta$, we end up with the following quadratic equation:
\begin{equation}
  \label{eq:quad}
  a \cos ^4(\theta )+b \cos ^2(\theta )+c=0,
\end{equation}
with solutions given by
\begin{equation}
 \theta =  \cos^{-1}\left(\pm \sqrt{\frac{-b\pm \sqrt{b^2-4 a
   c}}{2 a}}\right)\;.
  \label{eq:sol}
\end{equation}
Analytical representation of the coefficients is shown in
Table~\ref{tbl:equations}.  The four solutions of
equation~(\ref{eq:sol}) are controlled by the sign of the offset
wavenumber and its magnitude compared with the midpoint wavenumber.
In the frequency-wavenumber domain, equation~(\ref{eq:sol}) can be
used to map offset \geoulin2{(horizontal)} wavenumbers to angle gathers for a specific
frequency, midpoint \geoulin2{(horizontal)} wavenumber, and depth slice. A description of an
algorithm to use with the mapping equation, in the case of an isotropic
medium, is given by \cite{SEG-2004-10531056}.

Setting $\eta=0$ yields mapping for elliptical anisotropy with
coefficients of equation~(\ref{eq:sol}) given by
Table~\ref{tbl:equationsE}. The coefficients are represented by much
simpler formulas. In the isotropic case, $\eta=0$ and $v_z=v$,
Table~\ref{tbl:equations} reduces to Table~\ref{tbl:equationsI} and,
if substituted into the mapping formula of equation~(\ref{eq:sol}), is
equivalent to the corresponding mapping equation of
\cite{SEG-2004-10531056}.

\section{Numerical tests: The anisotropy influence}
\inputdir{Math}

Using equation~(\ref{eq:sol}) we evaluated angle gathers as a function
of offset and midpoint wavenumbers for a given frequency. We tested
such mapping for various models using different strengths of
anisotropy as we varied $\eta$, $v_z$, and the NMO velocity $v$.

Figures~\ref{fig:AnglesEta0}-\ref{fig:AnglesEta} show contour
density plots of angle \geosout{gathers} as a function of offset and
midpoint wavenumbers, for a 60-Hz frequency slice. \geosout{ and a single depth
step during downward continuation.} In Figure~\ref{fig:AnglesEta0} the
medium is isotropic, with a velocity of 2 km/s. Clearly, for
$k_{\text{hx}}=0$, the angle is zero regardless of the midpoint
wavenumber, which is expected, because for zero-offset the scattering or
opening angle is equal to zero. Also, we observe that angles
decrease with dip \geouline{(or $k_{mx}$)} for a given offset wavenumber, which is also
expected, because for any offset a scattering angle becomes zero in the case of a
vertical reflector. \geouline{The areas given in white in the Figures~\ref{fig:AnglesEta0}-\ref{fig:Anglesdiffv} 
\geosou2{and throughout} correspond to regions where the 
$k_{sz}$ or $k_{rz}$ become complex, and thus represent evanescent waves.}
\sideplot{AnglesEta0}{width=0.8\textwidth}{Constant-depth constant-frequency (60 Hz)
  slice mapped to \geosou2{reflection} \geoulin2{opening} angles for an isotropic medium with velocity equal to 2~km/s. 
Zero-offset wavenumber maps to zero (normal incidence)
  angle. The four blank corners represent evanescent regions. \geoulin2{Negative angles correspond to a switch in the source-receiver direction,
and thus, the result is symmetric based on the principal of reciprocity}}

In anisotropic media, as illustrated in Figure~\ref{fig:AnglesEta},
for $\eta$ equal to 0.1 and 0.3,
the angles decrease with dip for a constant offset wavenumber faster
than in the isotropic case. \geouline{In the example, considering that} \geosout{Considering that} $v_z$ is lower in the
anisotropic models, the higher horizontal velocities given by the
larger $\eta$ resulted in smaller scattering angles because reflection
occurs more \geosout{toward the} updip \geosout{side} for larger $\eta$.
\sideplot{AnglesEta}{width=0.9\textwidth}{Constant-depth constant-frequency (60 Hz)
  slice mapped to \geosou2{reflection} \geoulin2{opening} angles as in Figure~\protect{\ref{fig:AnglesEta0}}, but
for a VTI model with $v_z$=1.8 km/s, $v$=2 km/s, and $\eta=0.1$ (left) and $\eta=0.3$ (right).}

%\sideplot{AnglesEta3}{width=0.45\textwidth}{Constant-depth constant-frequency (60 Hz)
%  slice mapped to reflection angles as in Figure~\protect\ref{fig:AnglesEta0}, but
%for a more extreme VTI model with the $v_z$=1800 m/s, $v$=2 km/s, and $\eta=0.3$.}
Whereas the influence of $\eta$ is clearly \geouline{large}, the
change in vertical velocity has a minor influence on the angles as a
function of the midpoint wavenumber (or dip), as demonstrated by the
difference plot in Figure~\ref{fig:Anglesdiffvz}. A 0.6~km/s
difference in vertical velocity of an elliptical isotropic model with $\eta$=0 (left) and a VTI model with
$\eta$=0.3 resulted in differences mainly in the offset wavenumber
direction, because depth change caused by the different vertical
velocity provides variations in angles with offset.

\sideplot{Anglesdiffvz}{width=0.9\textwidth}{Left: The difference between 
\geosou2{reflection} \geoulin2{opening} angles for an elliptical anisotropic model 
 with $v_z$=1.8 km/s, $v$=2 km/s and that of a similar
model, with $v_z$=1.2 km/s. Right: The difference between\geosou2{reflection} \geoulin2{opening} angles for a VTI model 
of Figure~\protect\ref{fig:AnglesEta} (right) for
$v_z$=1.8 km/s, $v$=2 km/s, and $\eta=0.3$ and that of a similar
model, with $v_z$=1.2 km/s.}
In comparison, if we change the NMO velocity, $v$, the angles hardly
change at all, especially around small dips and small offsets. This
fact is evident in Figure~\ref{fig:Anglesdiffv}, where we change
NMO velocity 0.6 km/s, and the general difference is small until we get
to large offset and midpoint wavenumbers. This difference implies that
the mapping is practically NMO-velocity independent. This is the case for $\eta=0$ (left) and $\eta=0.3$ (right)
in Figure~\ref{fig:Anglesdiffv}, which implies, \geosout{among other things} that for a given elliptical anisotropic model
one can find an isotropic model that produces similar mapping granted that the velocity of the isotropic model
is equal to the vertical velocity for the elliptical anisotropy.
\plot{Anglesdiffv}{width=0.9\textwidth}{Left: The difference between reflection angles for an elliptical anisotropic model 
 with $v_z$=1.8 km/s, $v$=2 km/s and that of a similar
model, with $v$=1.4 km/s. Right: The difference between reflection angles for a VTI model 
of Figure~\protect\ref{fig:AnglesEta} (right) for
$v_z$=1.8 km/s, $v$=2 km/s, and $\eta=0.3$ and that of a similar
model, with $v$=1.4 km/s.}

%\sideplot{AnglesEta0vdiff}{width=0.45\textwidth}{Difference between reflection angles for the isotropic model given in Figure~\protect\ref{fig:AnglesEta0} for
%$v=v_z$=2 km/s, and that for $v$=2 km/s and $v_z$=1400 m/s (elliptical anisotropy, $\eta=0$).}

\section{Synthetic Example}
\inputdir{synth}

In the following example, we use a homogeneous model for simplicity, although nothing in the development requires that.
It is convenient so we can isolate the anisotropy influence on angle gathers decomposition. \geouline{To follow convention, we display angle gathers in the following examples using 
half the opening (or scattering) angle ($=\frac{\theta}{2}$)}.
We consider the reflector model in Figure~\ref{fig:ref}, which is
made up of a number of dome-like anticlines. This model allows us to focus on an angle gather located at 8 km that includes many dips. For a velocity of 2 km/s
and $\eta=0.2$, we generate the prestack synthetic dataset shown in Figure~\ref{fig:adata}. We use Kirchhoff modeling to obtain the synthetic data \cite[]{GEO60-04-11391150}. 
\geouline{As
a reference, we show in Figure~\ref{fig:agath} the isotropic migration of the isotropic version of the data. 
In this figure, we observe the extension of reflections that are acquired with a limited offset in the angle representation.}

Conventional phase shift downward continuation requires that no lateral velocity variation is present. Since the synthetic model has no lateral (or even vertical) velocity
variation, we use a VTI version of the DSR (double-square-root) phase-shift migration \cite[]{GEO65-04-11791194} to migrate the data. However, prior to applying the zero-time imaging condition we map the offset wavenumbers 
to angle, and thus, obtain angle gathers. Figure~\ref{fig:agath2} shows the isotropically migrated section at near zero angle. It also shows on the right hand side the 
angle gather for an isotropic angle
gather mapping. Clearly, the angle gather includes residuals resulting from ignoring anisotropy. \geouline{These residuals start at the top with deviations at large
angles with a fourth-order moveout (known as nonhyperbolic)
often associated with the semi-horizontal 
reflectors to second-order strong deviations typically associated with dipping reflectors} \cite[]{GEO60-05-15501566}. The top plot in Figure~\ref{fig:agath2} is 
a slice of a constant depth of 2 km and includes some residual information spanning other angle gathers.

If we downward continue using an anisotropic phase-shift
migration followed by an isotropic angle gather mapping, 
we image the data accurately as shown in Figure~\ref{fig:agath3}. Even the angle gathers, despite using an isotropic mapping,
show no residuals as we have imaged that data accurately. However, though not immediately obvious, most of the migrated energy is mapped to the wrong angle.
On the other hand, an anisotropic mapping of angle gathers places reflections at their true angles (Figure~\ref{fig:agath4}). This fact can be realized from comparing
the extension of angle gathers of Figures~\ref{fig:agath3} and~\ref{fig:agath4}. The lower-than-actual horizontal velocity treatment in the isotropic mapping
places energy at smaller reflection angle values. This phenomenon can be directly attributed to the difference between phase and group velocities. Specifically, for
horizontal reflections, the isotropic angle gathers map phase angles (ignoring the difference), while anisotropic ones map
ray angles. When the horizontal velocity is higher in a VTI medium, the ray angle, measured from vertical, tend to be higher than the phase angle.

\begin{comment}
Finally, a mix of isotropic downward continuation
and anisotropic angle-gathers mapping reveals more of the features of the new mapping equation. Figure~\ref{fig:agath55} displays the result and
clearly the angle gather is different than that shown in Figure~\ref{fig:agath2} with the isotropic mapping. For one moveouts are clearly more stretched out over 
angles.
\end{comment}

\plot{ref}{width=0.9\textwidth}{ A reflector model containing 5 reflections in a dome like formation.
	We focus on angle gathers at location 8 km, at which several reflection dips are represented.}

\plot{adata}{width=0.9\textwidth}{Prestack synthetic data generated using Kirchhoff modeling for a VTI model with velocity (NMO and vertical) equal 2 km/s and $\eta=0.2$.}

\plot{agath}{width=0.9\textwidth}{Migrated section after an isotropic migration with velocity of 2 km/s of an equivalent isotropic
 synthetic data. The angle gathers obtained using an isotropic mapping at 8 km location is displayed on the right. 
The top section shows a depth slice as a function of angle gather at 
depth 2 km.}

\plot{agath2}{width=0.9\textwidth}{Migrated section after an isotropic migration with velocity of 2 km/s of the VTI synthetic data in Figure~\protect\ref{fig:adata}.
Again, the angle gathers obtained using an isotropic mapping at 8 km location is displayed on the right, and 
the top section shows a depth slice as a function of angle gather at 
depth 2 km.}

\plot{agath3}{width=0.9\textwidth}{Migrated section after a VTI migration with velocity of 2 km/s and $\eta$=0.2 of the VTI synthetic data in Figure~\protect\ref{fig:adata}.
Again, the angle gathers obtained using an isotropic mapping at 8 km location is displayed on the right, 
and the top section shows a depth slice as a function of angle gather at 
depth 2 km.}

\plot{agath4}{width=0.9\textwidth}{Migrated section after a VTI migration with velocity of 2 km/s and $\eta$=0.2 of the VTI synthetic data in Figure~\protect\ref{fig:adata}.
The angle gathers obtained, now, using the VTI mapping at 8 km location is displayed on the right. The top section shows a depth slice as a function of angle gather at 
depth 2 km.}

\begin{comment}
\plot{agath55}{width=0.9\textwidth}{Migrated section after an isotropic migration with velocity of 2 km/s of the VTI synthetic data in Figure~\protect\ref{fig:adata}.
The angle gathers obtained using the VTI mapping  at 8 km location is display on the right. The top section shows a depth slice as a function of angle gather at 
depth 2 km.}
\end{comment}

\geouline{This synthetic test the importance of anisotropic angle gather mapping to place reflections at their true
scattering angles.} In practice, the velocity model building process uses residuals along angle gathers to estimate the required velocity update.
Proper definition of the angle gather residuals will simplify the update process. This is especially true if the update is based on reflection tomography. 


\section{The TTI case}


In the case of a tilt in the angle of symmetry of the TI (TTI) medium, the dispersion relations~\ref{eq:kzs} and~\ref{eq:kzr}
must be altered to reflect the tilt. Specifically, the wavenumbers should be transformed
to the direction of the tilt. In fact, at the reflection point all equations used to develop the mapping in equation~(\ref{eq:quad}) 
hold regardless of the direction of tilt. This implies that the
quadratic solution~(\ref{eq:sol}) applies with $a$, $b$, and $c$ given by Table 1 granted that the wavenumbers are transformed in the direction of the tilt.
Considering that $\phi$ is the tilt angle measured from vertical in 2-D, the horizontal (conventional) wavenumbers given by the surface-recorded data
are given by 
\begin{equation}
k_{sc} \equiv k_{sx} \cos\phi - k_{sz} \sin\phi,
  \label{eq:kst}
\end{equation}
and
\begin{equation}
k_{rc} \equiv k_{rx} \cos\phi - k_{rz} \sin\phi.
\label{eq:krt}
\end{equation}
\inputdir{Math}
\geouline{where $k_{sx}$ and $k_{rx}$ now correspond to the normal-to-the-tilt wavenumber direction and they are related to $k_{sz}$ and $k_{rz}$ (tilt direction wavenumbers), 
respectively using equations~\ref{eq:kzs} and~\ref{eq:kzr}.}
\geouline{Based on the above equations, to solve for $k_{sx}$ and $k_{rx}$ needed for the angle gather mapping, we are required to solve a quartic equation that can be represented, with pain, analytically 
or solved numerically. Alternatively, the formulations for a transversely isotropic medium with tilt constrained to the dip (DTI), introduced by} \cite{tariq-sava}, \geouline{is 
simpler than those introduced here for a general TI medium, and thus can be used at the velocity model building stage. However, when the assumption of the tilt being normal to the reflector dip
fails, for example at salt
flank reflections where the tilt is generally not normal to the Salt flank, we will need a general formulation similar to the one developed here.}

\section{Conclusions}

We have developed analytical relationships to generate angle gathers
using wave-equation migration in VTI media. These relations are based
on an approach for generating gathers at each depth level by
converting offset-space-frequency planes into angle-space planes while
simultaneously applying the imaging condition. Although the angle
gathers depend on medium parameters, they are
independent of the structure, which provides an opportunity for a
simple and practical implementation.  Comparing the mapping equation
for VTI media with those for the isotropic case demonstrates the large
influence that the anisotropy parameter $\eta$ has on the mapping
process. On the other hand, the influence of vertical velocity is
confined to the offset direction, and the influence of NMO velocity is
small.
Synthetic data applications demonstrate the importance of accurate mapping of energy
in the angle domain. In the case of a TTI medium, the required modification to the mapping equations
is given by the transformation of the wavenumbers in the dispersion relation 
to the direction of of the symmetry axis and, as a result, a similar analytic mapping
exists.

\section{Acknowledgments}

We are grateful to KACST, KAUST, and the Bureau of Economic Geology, University of Texas at
  Austin for their support. We thank Gilles Lambare, Jeff Shragge and Jun Cao for their critical and helpful review of the paper.

\tabl{equations}{Exact analytical equations for the coefficients of equation~(\ref{eq:quad}).}
 {
    \begin{center}
     \begin{tabular}{|c|c|}
      \hline $a$ &
      $\begin{array}{c} \frac{1}{2} \left(k_h^2-k_m^2\right)^4 \left(v^2 \eta 
   \left(k_h-k_m\right)^2-2 \omega ^2\right)^3\,
   \left(v^2 (4 \eta +1) \left(k_h-k_m\right)^2-4 \omega ^2\right)^2\,
   \left(v^2 \eta \left(k_h+k_m\right)^2-2 \omega ^2\right)^4 \\
   \left[v_z^4 \left(k_h^2-k_m^2\right)^2 \left(-2 v^2 \eta  k_h^2 \left(v^2 \eta k_m^2+2 \omega ^2\right) +v^4 \eta ^2 k_h^4+\left(v^2 \eta  k_m^2-2
   \omega ^2\right)^2\right) \right. \\ 
   \left. +4 \omega ^4 \left(2 v^2 (4 \eta +1) k_h^2 \left(v^2 (4 \eta +1) k_m^2+4 \omega ^2\right) 
    -v^4 (4 \eta +1)^2  k_h^4-\left(v^2 (4 \eta +1) k_m^2-4 \omega ^2\right)^2\right)\right]^2\end{array}$ \\
    \hline $b$ & 
    $\begin{array}{c} \left(k_h^2-k_m^2\right)^4 \left(-\left(v^2 \eta \left(k_h-k_m\right)^2 -2 \omega ^2\right)^3\right)  
    \left(v^2 (4 \eta +1) \left(k_h-k_m\right)^2-4 \omega ^2\right)^2 \left(v^2 \eta \left(k_h+k_m\right)^2-2 \omega ^2\right)^4 \\  
    \left[v_z^4 \left(k_h^2-k_m^2\right)^2 \left(-2 v^2 \eta  k_h^2 \left(v^2 \eta k_m^2+2 \omega ^2\right) +v^4 \eta ^2 k_h^4  +\left(v^2 \eta  k_m^2-2
       \omega ^2\right)^2\right) \right. \\ \left.
      +4 \omega ^4 \left(-2 v^2 (4 \eta +1) k_h^2
       \left(v^2 (4 \eta +1) k_m^2+4 \omega ^2\right) +v^4 (4 \eta +1)^2 k_h^4+\left(v^2 (4 \eta +1) k_m^2-4 \omega ^2\right)^2\right)\right] \\
   \left(\left(k_h-k_m\right)^2 \left(v_z^2 \left(v^2 \eta  \left(k_h-k_m\right)^2-2 \omega ^2\right) +2 v^2 (4 \eta +1) \omega^2\right) -8 \omega ^4\right) \\
   \left(\left(k_h+k_m\right)^2 \left(v_z^2 \left(v^2 \eta  \left(k_h+k_m\right)^2-2 \omega ^2\right) +2 v^2 (4 \eta +1) \omega^2\right) -8 \omega ^4\right)\end{array}$ \\
   \hline $c$ &
      $\begin{array}{c} \frac{1}{2} \left(k_h^2-k_m^2\right)^4 \left(v^2 \eta \left(k_h-k_m\right)^2-2 \omega ^2\right)^3 
      \left(v^2 (4 \eta +1) \left(k_h-k_m\right)^2-4 \omega ^2\right)^2 \left(v^2 \eta \left(k_h+k_m\right)^2-2 \omega ^2\right)^4 \\ 
      \left[v_z^4 \left(k_h^2-k_m^2\right)^2 \left(-2 v^2 \eta  k_h^2 \left(v^2 \eta k_m^2+2 \omega ^2\right) 
      + v^4 \eta ^2 k_h^4 + \left(v^2 \eta  k_m^2-2 \omega ^2\right)^2\right) \right. \\ \left. 
      +4 \omega ^4 \left(2 v^2 (4 \eta +1) k_h^2 \left(v^2 (4 \eta +1) k_m^2+4 \omega ^2\right) 
      -v^4 (4 \eta +1)^2 k_h^4-\left(v^2 (4 \eta +1) k_m^2-4 \omega^2\right)^2\right)\right]^2\end{array}$ \\
    \hline
    \end{tabular} 
   \end{center}
} 

\tabl{equationsE}{Exact analytical equations for the coefficients of equation~(\ref{eq:quad}) in the case of elliptic anisotropy ($\eta=0$).}
 {
    \begin{center}
     \begin{tabular}{|c|c|}
      \hline $a$ &
      $\begin{array}{c} -\left(k_h^2-k_m^2\right){}^4 \left(v k_h-v k_m+2 \omega
   \right){}^2 \left(-v k_h+v k_m+2 \omega \right){}^2 \\
\left(-2 k_h^2 \left(k_m^2
   \left(v^4-v_z^4\right)+4 v^2 \omega ^2\right)+k_h^4 \left(v^4-v_z^4\right)+\left(v^2
   k_m^2-4 \omega ^2\right){}^2-k_m^4 v_z^4\right){}^2 \end{array}$ \\
    \hline $b$ & 
    $\begin{array}{c} 2 \left(k_h^2-k_m^2\right){}^4 \left(v k_h-v k_m+2 \omega
   \right){}^2 \left(-v k_h+v k_m+2 \omega \right){}^2 \left(4 \omega
   ^2-\left(v^2-v_z^2\right) \left(k_h-k_m\right){}^2\right) \\
\left(4 \omega
   ^2-\left(v^2-v_z^2\right) \left(k_h+k_m\right){}^2\right) \\
\left(-2 k_h^2 \left(k_m^2
   \left(v^4+v_z^4\right)+4 v^2 \omega ^2\right)+k_h^4 \left(v^4+v_z^4\right)+\left(v^2
   k_m^2-4 \omega ^2\right){}^2+k_m^4 v_z^4\right) \end{array}$ \\
   \hline $c$ &
      $\begin{array}{c} -\left(k_h^2-k_m^2\right){}^4 \left(v k_h-v k_m+2 \omega
   \right){}^2 \left(-v k_h+v k_m+2 \omega \right){}^2 \\
\left(-2 k_h^2 \left(k_m^2
   \left(v^4-v_z^4\right)+4 v^2 \omega ^2\right)+k_h^4 \left(v^4-v_z^4\right)+\left(v^2
   k_m^2-4 \omega ^2\right){}^2-k_m^4 v_z^4\right){}^2 \end{array}$ \\
    \hline
    \end{tabular} 
   \end{center}
}

\tabl{equationsI}{Exact analytical equations for the coefficients of equation~(\ref{eq:quad}) in the case of isotropy ($\eta=0$, $v_z=v$).}
 {
    \begin{center}
     \begin{tabular}{|c|c|}
      \hline $a$ &
      $\begin{array}{c} -\left(k_h^2-k_m^2\right){}^4 \left(v k_h-v k_m+2 \omega
   \right){}^2 \left(-v k_h+v k_m+2 \omega \right){}^2 \left(v^2
   \left(k_h^2+k_m^2\right)-2 \omega ^2\right){}^2 \end{array}$ \\
    \hline $b$ & 
    $\begin{array}{c} \left(k_h^2-k_m^2\right){}^4 \left(v k_h-v k_m+2 \omega
   \right){}^2 \left(-v k_h+v k_m+2 \omega \right){}^2 \\
	\left(v^2 \left(-2 k_h^2 \left(v^2
   k_m^2+2 \omega ^2\right)+v^2 k_h^4+v^2 k_m^4-4 \omega ^2 k_m^2\right)+8 \omega
   ^4\right) \end{array}$ \\
   \hline $c$ &
      $\begin{array}{c} -\left(k_h^2-k_m^2\right){}^4 \left(v k_h-v k_m+2 \omega
   \right){}^2 \left(-v k_h+v k_m+2 \omega \right){}^2 \left(v^2
   \left(k_h^2+k_m^2\right)-2 \omega ^2\right){}^2 \end{array}$ \\
    \hline
    \end{tabular} 
   \end{center}
} 

\bibliographystyle{seg}
\bibliography{SEG,SEP2,agath}
