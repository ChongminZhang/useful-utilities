\published{Geophysics, 77, no. 2, V61-V69, (2012)}
\title{Random noise attenuation using $f$-$x$ regularized nonstationary autoregression}

\author{Guochang Liu\footnotemark[1], Xiaohong Chen\footnotemark[1], Jing Du\footnotemark[2], Kailong Wu\footnotemark[1]}

\ms{GEO-2011-0117}

\address{
\footnotemark[1] China University of Petroleum (Beijing), \\
National Engineering Laboratory for Offshore Oil Exploration, \\
Beijing, China. E-mail: guochang.liu@yahoo.com; chenxh@cup.edu.cn; \\
wuyanping220@163.com
\footnotemark[2]China Petroleum \& Chemical Corporation, \\
Shengli Geophysical Research Institute, China. \\
E-mail: dujing_421@yahoo.com
}

\lefthead{Liu et al.}
\righthead{Noise attenuation using $f$-$x$ RNA}

\maketitle

\begin{abstract}
We propose a novel method for random noise attenuation in seismic 
data by applying regularized nonstationary autoregression (RNA) in 
frequency-space ($f$-$x$) domain. The method adaptively predicts the 
signal with spatial changes in dip or amplitude using $f$-$x$ RNA. The 
key idea is to overcome the assumption of linearity and stationarity 
of the signal in conventional $f$-$x$ domain prediction technique. The 
conventional $f$-$x$ domain prediction technique uses short temporal 
and spatial analysis windows to cope with the nonstationary of the 
seismic data. The proposed method does not require windowing 
strategies in spatial direction. We implement the algorithm by 
iterated scheme using conjugate gradient method. We constrain the 
coefficients of nonstationary autoregression (NA) to be smooth 
along space and frequency in $f$-$x$ domain. The shaping regularization 
in least square inversion controls the smoothness of the coefficients 
of $f$-$x$ RNA. There are two key parameters in the proposed method: 
filter length and radius of shaping operator. Synthetic and field 
data examples demonstrate that, compared with $f$-$x$ domain and 
time-space ($t$-$x$) domain prediction methods, $f$-$x$ RNA can be more 
effective in suppressing random noise and preserving the signals, 
especially for complex geological structure. 

\end{abstract}

\section{Introduction}
Random noise attenuation in seismic data can be implemented in the 
frequency-space ($f$-$x$) and time-space ($t$-$x$) domain using prediction filters \cite[]{Abma1995}.
Linear prediction filtering assumes that the 
signal can be described by an autoregressive (AR) model. When the data are 
contaminated by random noise, the signal is considered to be predicted by 
the AR filter and the noise is the residual \cite[]{Bekara2009}. 
A number of approaches in $f$-$x$ domain have been proposed and been used for 
attenuating random noise. The $f$-$x$ prediction technique was introduced by 
\cite{Canales1984} and further developed by \cite{Gulunay1986}. The $f$-$x$ domain 
prediction technique is also referred as $f$-$x$ deconvolution by \cite{Gulunay1986}. 
\cite{Sacchi2001} utilized the autoregressive-moving average (ARMA) 
structure of the signal to estimate a prediction error filter (PEF) and 
the noise sequence is estimated by self-deconvolving the PEF from the 
filtered data. \cite{Hodgson2002} presented a novel method of noise
 attenuation for 3D seismic data, which applies a smoothing filter 
(e.g. 2D median filter) to each targeted frequency slice and allows 
targeted filtering of selected parts of the frequency spectrum. The 
conventional $f$-$x$ domain prediction uses windowing strategies to avoid 
that the seismic events are not linear. The data are assumed to be 
piecewise linear and stationary in an analysis temporal and spatial 
window. To overcome the potentially low performance of $f$-$x$ deconvolution
that arises with processing structural complex data, \cite{Bekara2009} 
proposed a new filtering technique for random and 
coherent noise attenuation in seismic data by applying empirical 
mode decomposition (EMD) \cite[]{Huang1998} on constant-frequency
slices in the $f$-$x$ domain and removing the first intrinsic mode 
function. In addition, in the research field of seismic data 
interpolation, \cite{Naghizadeh2009} proposed an adaptive
$f$-$x$ prediction filter, which was used to interpolate waveforms 
that have spatially variant dips. The $f$-$x$ domain prediction 
technique can be implemented in the frequency slice and also
in pyramid domain \cite[]{Sun1996}. The implemented in 
pyramid domain makes the operators more efficient because one
only needs to estimate one prediction filter from many different 
frequencies \cite[]{Sun1996, Hung2004, Guitton2010}.

The prediction process can be also achieved in $t$-$x$ domain 
\cite[]{Claerbout1992}. \cite{Abma1995} discussed $f$-$x$ 
and $t$-$x$ approaches to predict linear events and concluded 
that $f$-$x$ prediction is equivalent to $t$-$x$ prediction with a 
long time length. \cite{Crawley1999} proposed smooth 
nonstationary PEFs with micropatches and radial smoothing
in the application of seismic interpolation, which 
typically produces better results than the rectangular
patching approach. \cite{Izquierdo2006} proposed a 
technique for structural noise reduction in ultrasonic 
nondestructive examination using time-varying prediction 
filter. \cite{Sacchi2009} proposed an algorithm 
to compute time and space variant prediction filters for 
noise attenuation, which is implemented by a recursive 
scheme where the filter is continuously adapted to 
predict the signal.

\cite{Fomel2009} developed a general method of nonstationary 
regression with shaping regularization \cite[]{Fomel2007}. 
Shaping regularization has an advantage of a fast 
iterative convergence. Regularized nonstationary 
regression (RNA) has been used in multiple subtraction 
\cite{Fomel2009}, time-frequency analysis \cite[]{Liu2009}, 
and nonstationary polynomial fitting \cite[]{Liu2011}. 
\cite{Liu2010} introduced an adaptive PEFs using 
RNA in $t$-$x$ domain which has been used for trace interpolation. 

In this paper, we investigate the $f$-$x$ domain prediction 
technique and propose $f$-$x$ domain RNA to attenuate
random noise in seismic data. Firstly, we review 
the theory of $f$-$x$ stationary autoregression. Then, 
we describe the $f$-$x$ RNA and extend to complex number 
domain. Next we provide the methodology of random 
noise attenuation using $f$-$x$ RNA. Finally, we use 
synthetic and real data examples to evaluate and compare 
the proposed method with other noise attenuation techniques,
such as $f$-$x$ domain and $t$-$x$ domain prediction techniques.

\section{Review of $f$-$x$ domain stationary autoregression}

We first consider a seismic section $S(t,x)$ that consists of a single 
linear event with the slope $p$ and constant amplitude. 
The frequency domain representation of $S(t,x)$ is given by
      \begin{equation}
          S(f,x)=A(f){{e}^{j2\pi fxp}},
        \label{eq:eq1}
      \end{equation} 
where $A(f)$is the wavelet spectrum, $f$is the temporal frequency, 
and $x$is the spatial variable. We assume $x=n\Delta x$, where 
$n=1,2,...,N$, $N$is the number of traces in the whole section. 
The relationship between the n-th trace and (n-1)-th trace can be easily 
shown as
      \begin{equation}
          {{S}_{n}}(f)={{a}_{1}}(f){{S}_{n-1}}(f),
        \label{eq:eq2}
      \end{equation} 
where ${{a}_{1}}=\exp (j2\pi fp\Delta x)$. This recursion is a first-order 
differential equation also known as an AR model of order 1 and represents a 
single complex-valued harmonic \cite[]{Bekara2009}. If there are $M$ linear
events in x-t domain, we can have a difference equation of order $M$ \cite[]{Sacchi2001}
      \begin{equation}
          {{S}_{n}}(f)=\sum\limits_{i=1}^{M}{{{a}_{i}}(f){{S}_{n-i}}(f)}.
        \label{eq:eq3}
      \end{equation} 
The recursive filter $\left\{ {{a}_{i}}(f) \right\}$can be found for predicting a 
noise-free superposition of complex harmonics. Considering seismic data with additive 
random noise and non-causal prediction with order $2M$which includes both forward 
and backward prediction equations \cite[]{Spitz1991, Naghizadeh2009}, we can obtain
      \begin{equation}
         {{\varepsilon }_{n}}(f)={{S}_{n}}(f)-\sum\limits_{i=1}^{M}{{{a}_{i}}{{S}_{n-i}}}(f)-\sum\limits_{i=-1}^{-M}{{{a}_{i}}{{S}_{n-i}}(f)},
        \label{eq:eq4}
      \end{equation} 
where ${{\varepsilon }_{n}}(f)$is a complex noise sequence. \cite{Canales1984} argues 
a causal estimate of signal $\sum\limits_{i=1}^{M}{{{a}_{i}}(f){{S}_{n-i}}(f)}$ 
is the predictable part of data obtained by an AR model. This operation is usually 
called $f$-$x$ deconvolution \cite[]{Gulunay1986}. Noise-free events that are linear in 
the $t$-$x$ domain manifest as a superposition of harmonics in the $f$-$x$ domain and these 
harmonics can be perfectly predicted using AR filter. If seismic events are not linear, 
or the amplitudes of wavelet are varying from trace to trace, they no longer follow 
Canales’s assumptions \cite[]{Canales1984}. One needs to perform $f$-$x$ deconvolution over a 
short sliding window in time and space. This leaves the choice of window parameters 
(window size and length of overlapping between adjacent windows). \cite{Bekara2009} 
discuss some limitations of conventional $f$-$x$ deconvolution in detail.

\section{$f$-$x$ domain regularized nonstationary autoregression}

Nonstationary autoregression has been developed and used in signal processing 
\cite[]{Bakrim1994, Aboutajdine1996, Izquierdo2006} and seismic data processing
\cite[]{Sacchi2009}. \cite{Fomel2009} developed a general method of nonstationary 
autoregression using shaping regularization technology and applied it to multiples
 subtraction. In this paper, we extend the RNA method to $f$-$x$ domain for complex numbers 
and apply it to seismic random noise attenuation. 

Consider two adjacent seismic traces ${{S}_{n}}(f)$ and ${{S}_{n-1}}(f)$ in 
$f$-$x$ domain of a seismic section that consists of a single nonlinear event with the 
slope ${{p}_{n}}$ and varying amplitude ${{B}_{n}}(f)$. Similar to equation ~\ref{eq:eq1}, 
we can write
      \begin{equation}
          [{{S}_{n}}(f)=\frac{{{B}_{n}}(f)}{{{B}_{n-1}}(f)}{{e}^{j2\pi f\Delta x{{p}_{n}}}}{{S}_{n-1}}(f)={{a}_{1}}(f){{S}_{n-1}}(f).
        \label{eq:eq5}
      \end{equation}
 
From equation ~\ref{eq:eq5} we can find that the coefficients of AR is the function of space index 
$n$ and frequency index $f$. Therefore, we can use nonstationary autoregression to 
describe this problem. Equation 5 describes the relation between two traces. 
If we consider multiple traces, the nonstationary autoregression can be defined as \cite[]{Fomel2009}
      \begin{equation}
         {{\varepsilon }_{n}}(f)={{S}_{n}}(f)-\sum\limits_{i=1}^{M}{{{a}_{n,i}}(f){{S}_{n-i}}(f)},
        \label{eq:eq6}
      \end{equation} 
where $n$ and $f$are the coordinate of space and frequency, respectively. If considering the situation
 of non-causal nonstationary autoregression, we can rewrite the Nonstationary autoregression models
 (equation ~\ref{eq:eq6}) as
      \begin{equation}
          {{\varepsilon }_{n}}(f)={{S}_{n}}(f)-\sum\limits_{i=1}^{M}{{{a}_{n,i}}(f){{S}_{n-i}}(f)}-\sum\limits_{i=-1}^{-M}{{{a}_{n,i}}(f){{S}_{n-i}}(f)}.
        \label{eq:eq7}
      \end{equation}
Equation ~\ref{eq:eq7} indicates that one trace noise-free in $f$-$x$ domain can 
be estimated by weighted stacking adjacent traces with the weights ${{a}_{n,i}}(f)$, 
which is varying along the space and frequency. Note that the difference between equations ~\ref{eq:eq4} and ~\ref{eq:eq6}
 is that the coefficients are varying with space coordinate $n$ in equation ~\ref{eq:eq7}. 
To obtain the coefficients ${{a}_{n,i}}(f)$ from equation 7, we can transform equation ~\ref{eq:eq7} 
to the following least square problem:
      \begin{equation}
          \min_{a_{n,k}(f)}||{{S}_{n}}(f)-\sum\limits_{i=1}^{M}{{{a}_{n,i}}(f){{S}_{n-i}}(f)}-\sum\limits_{i=-1}^{-M}{{{a}_{n,i}}(f){{S}_{n-i}}(f)}||_{2}^{2},
        \label{eq:eq8}
      \end{equation} 
where $\left\| {} \right\|_{2}^{2}$denotes the squared L-2 norm. Note that both the data 
${{S}_{n}}(f)$ and the coefficients ${{a}_{n,i}}(f)$ are in complex numbers domain.

The problem of the minimization in equation ~\ref{eq:eq7} is ill-posed because it 
has more unknown variables than constraints. To obtain the spatial-varying coefficients
in nonstationary autoregression, several methods can be employed \cite[]{Aboutajdine1996}. 
Some of these are related to the expansion of the spatial-varying coefficients in terms of a
given sets of orthogonal basis functions and estimation of the coefficients of the expansion 
by the least-square method \cite[]{Izquierdo2006}. \cite{Naghizadeh2009} used exponentially 
weighted recursive least square (EWRLS) to solve the adaptive problem and applied it to 
seismic trace interpolation. \cite{Fomel2009} proposed regularized nonstationary autoregression 
in which shaping regularization technology \cite[]{Fomel2007} is used to constrain the 
smoothness of the coefficients of nonstationary autoregression. In this paper, we also adopt
shaping regularization to solve the under-constrained problem equation ~\ref{eq:eq8}.
 
With the addition of a regularization term, equation ~\ref{eq:eq8} can be written as
      \begin{equation}
          \min_{a_{n,i}{f}}||{{S}_{n}}(f)-\sum\limits_{i=-M,i\ne 0}^{M}{{{a}_{n,i}}(f){{S}_{n-i}}(f)}||_{2}^{2}+R[{{a}_{n,i}}(f)],
        \label{eq:eq9}
      \end{equation} 
where $R$ denotes the shaping regularization operator. \cite{Fomel2009} compared classic 
Tikhonov’s regularization with shaping regularization in RNA problem. 
Shaping regularization \cite{Fomel2007} provides a particularly convenient method of 
enforcing smoothness in iterative optimization schemes. Shaping regularization has clear 
advantages of a more intuitive selection of regularization parameters and a faster iterative 
convergence \cite[]{Fomel2009}. In the shaping regularization, we assume the initial 
value for the estimated model ${{a}_{n,i}}(f)$ is zero. If we choose a more appropriate 
initial value, we can have a fast iterative convergence of the conjugate-gradient iteration
in shaping regularization.

Note that the RNA equation in this paper (equation ~\ref{eq:eq9}) is in complex 
number domain while the RNA used in multiples subtraction \cite[]{Fomel2009}
is in real number domain. Analogous to RNA in real number domain, 
we force the complex coefficients ${{a}_{n,i}}(f)$ in equation ~\ref{eq:eq9} 
to have a desired behavior, such as smoothness. In shaping regularization technology, 
we need to choose a shaping operator $\mathbf{S}$ \cite[]{Fomel2007}.
In this paper, we choose shaping operator $\mathbf{S}$ as Gaussian smoothing with adjustable radius $r$. 
\cite{Fomel2007} indicated Gaussian smoothing can be implemented by repeated triangle smoothing operator. 
Both the data ${{S}_{n}}(f)$ and the coefficients ${{a}_{n,i}}(f)$ are complex numbers, but the shaping 
operator $\mathbf{S}$ is real. Therefore, shaping operator $\mathbf{S}$ is operated in real and imaginary 
parts of the complex coefficients respectively and the L-2 norm in equation ~\ref{eq:eq9} is the norm of complex numbers. 
When using the algorithm of conjugate-gradient iterative inversion with shaping regularization proposed by \cite{Fomel2007}
 to solve the complex RNA, we only need to replace transpose of real number by conjugate transpose of complex number.

Once we obtain the complex coefficients of RNA, we can achieve an estimation of signal
      \begin{equation} 
        {{\tilde{S}}_{n}}(f)=\sum\limits_{i=-M,i\ne 0}^{M}{{{a}_{n,i}}(f){{S}_{n-i}}(f)}.
        \label{eq:eq10}
      \end{equation} 
When we use $f$-$x$ RNA to noise attenuation, we first select a time window in $t$-$x$ domain and 
transform the data to $f$-$x$ domain. The usage of time window is to guarantee that the data 
is approximately stationary in time. The $f$-$x$ NAR can deal with spatial nonstationary data 
in $f$-$x$ domain. Then we use equation ~\ref{eq:eq9} with shaping regularization to compute the coefficients
 ${{a}_{n,i}}(f)$ and use equation ~\ref{eq:eq10} to estimate the signal in $f$-$x$ domain. Finally, 
we transform data back to the $t$-$x$ domain and repeat for the next time window.

The computational cost of the proposed method is $O\left( {{N}_{d}}{{N}_{f}}{{N}_{iter}} \right)$, 
where ${{N}_{d}}$ is data size, ${{N}_{f}}$is filter size, ${{N}_{iter}}$is the number of 
conjugate-gradient iterations. If ${{N}_{iter}}$and ${{N}_{f}}$ are small, this is a 
fast method. In practical implementation, we can choose the range of computed 
frequency to reduce the computation cost. In addition, if we can simplify the 
coefficients not to be frequency-dependent, we can apply RNA in frequency slice. 
In that case the different-frequency slices can be processed in parallel. 
And the computation cost and memory requirements can be further reduced.

\inputdir{simple}
 
\multiplot{6}{sin,nsin,tpefpatch,fxpatch,npre1,npre2}{width=0.3\columnwidth}{(a) Synthetic data. (b) Noisy data (SNR=1.53). 
(c) Result of $f$-$x$ domain prediction (SNR=2.53). (d) Result of $t$-$x$ domain prediction (SNR=3.12). (e) Result of $f$-$x$ RNA 
without constraint of smoothness along the frequency axis (SNR=4.87). (f) Result of $f$-$x$ RNA with ${{r}_{f}}=3$ (SNR=5.06). 
We decimate the data for display purpose.
}
  

\plot{npar2}{height=0.5\textheight}{The real part of coefficients at a given shift ${{a}_{n,i=1}}(f)$.}

\multiplot{3}{fxdiff,mpapatch,ndiff2}{width=0.25\columnwidth}{Difference sections of $f$-$x$ domain prediction (a), 
$t$-$x$ domain prediction (b), and $f$-$x$ RNA (c).
}

Consider a simple section (Figure ~\ref{fig:sin}), which includes one event, 501 traces. 
The event is obtained by convolution with Ricker wavelet. Both the dip and amplitude 
of the event are space-varying. The travel time is a sine function and the amplitude of 
the Ricker wavelet is multiplied by$B(x)=0.2{{(x-2.5)}^{2}}+0.5$. This event is obviously 
nonstationary in space. We add some random noise to it (Figure ~\ref{fig:nsin}). 
The event is greatly contaminated by random noise, especially in the middle part, 
because the signal of the middle part is poorer than the sideward and the noise levels 
are the same in the whole section. We use three methods, $f$-$x$ domain prediction, $t$-$x$ 
domain prediction, and $f$-$x$ domain RNA, to suppress random noise. The $f$-$x$ domain and $t$-$x$ 
domain prediction methods we used in this paper are discussed by \cite[]{Abma1995}. 
The $f$-$x$ domain prediction is implemented over a sliding window of 20 traces width with 50\% 
overlap and the filter length is 4, $M=2$ in equation 4 and the $t$-$x$ domain prediction is 
implemented over the same sliding window and the filter length in space and time are 4 and 5 
respectively. The $f$-$x$ RNA is implemented with the parameters: the filter length is 4, $M=2$ 
in equation ~\ref{eq:eq8}; the smoothing radiuses in space and frequency axes of shaping regularization are 
respectively 20 and 3, ${{r}_{x}}=20$, ${{r}_{f}}=3$. Comparing the results of $f$-$x$ RNA (Figure ~\ref{fig:npre2}) 
with other prediction methods (Figures ~\ref{fig:tpefpatch} and ~\ref{fig:fxpatch}, we find that the 
$f$-$x$ RNA is more effective in random noise attenuation for this simple nonstationary data. From the 
difference sections (Figure ~\ref{fig:npar2}), we find that all the three methods remove 
some effective signals. However, the proposed method removes fewest signals than other two methods. 
Note that the removed signals by $t$-$x$ and $f$-$x$ prediction methods are bigger in the sideward 
part than middle part, because the filters are the same in a sliding time window while 
the amplitude and slope of the event are different.
 
In order to quantitatively evaluate the effect of denoising between different methods, 
we use signal-to-noise ratio (SNR) to compare these three methods. The SNR is estimated by
      \begin{equation} 
          SNR=10{{\log }_{10}}(\frac{\sum\limits_{t,n}{{{[{{d}_{n}}(t)]}^{2}}}}{\sum\limits_{t,n}{{{[{{d}_{n}}(t)-{{r}_{n}}(t)]}^{2}}}}),
        \label{eq:eq11}
      \end{equation} 
where ${{r}_{n}}(t)$ is the noisy section and ${{d}_{n}}(t)$is the noisy-free section or 
the section after noise attenuation. In this simple example, the SNR of the input data without processing is 1.53. 
After noise attenuation, the SNR of $f$-$x$ RNA is 5.06 dB, and the SNRs of $f$-$x$ domain and $t$-$x$ domain prediction are 
respectively 2.53 dB and 3.12 dB. The $f$-$x$ RNA can improve SNR more greatly.

To test the sensitivity to the constraint of smoothness along the frequency axis, we give the result of $f$-$x$ RNA without constraint of smoothness along the frequency axis in this simple example (Figure ~\ref{fig:npre1}). The result without constraint along frequency axis 
(Figure ~\ref{fig:npre1}) is a little worse than that with constraint along frequency. 
But both of them are better than the results of conventional $f$-$x$ domain and $t$-$x$ domain prediction. 
Therefore, to reduce the computation cost, we can simplify the coefficients not to be frequency-dependent 
when the input seismic data is huge. It is a tradeoff between computation cost and effect of noise attenuation.

Because the dip and amplitude of the event are varying smoothly, $f$-$x$ RNA can predict the signal with smooth 
coefficients ${{a}_{n,i}}(f)$. To specify the coefficients, we display the real parts of the complex 
coefficients at a given shift ${{a}_{n,i=1}}(f)$ (Figure ~\ref{fig:npar2}). The reason of displaying real parts not 
imaginary parts is that the signs of real parts of coefficients are the same for forward and 
backward prediction. We find that the real parts of the complex coefficients are smooth. 
The smoothing radius controls the smoothness of the coefficients. If we only use one adjacent 
trace to predict the trace, the coefficients should have the expression
      \begin{equation} 
          {{a}_{n,i=1}}(f)=\frac{{{B}_{n}}(f)}{{{B}_{n-1}}(f)}{{e}^{j2\pi f\Delta x{{p}_{n}}}}.
        \label{eq:eq12}
      \end{equation} 
From equation ~\ref{eq:eq12} we can note that if the dip and amplitude are smoothly varying, 
the coefficients are smooth. Therefore, we use Gaussian shaping regularization to constrain 
the coefficients when solving the least squares equation ~\ref{eq:eq9}.

\section{Examples}

We demonstrate the effectiveness of the proposed $f$-$x$ RNA on a synthetic shot gather and a field poststack dataset. 

\subsection{Synthetic shot gather}

\inputdir{shot}

\multiplot{6}{para,npara,tpefpatch,fxpatch,npre,npar}{width=0.3\columnwidth}{(a) Synthetic shot gather. 
(b) Noisy gather. (c) Result of $f$-$x$ domain prediction (SNR=0.98). (d) Result of $t$-$x$ domain prediction (SNR=1.25). 
(e) Result of $f$-$x$ RNA (SNR=3.12). (f) The real part of coefficients at a given shift ${{a}_{n,i=1}}(f)$. 
We decimate the data in (a)-(e) for display purpose.
}

\multiplot{3}{fxdiff,mpapatch,ndiff}{width=0.25\columnwidth}{Difference sections of $f$-$x$ domain prediction (a), 
$t$-$x$ domain prediction (b), and $f$-$x$ RNA (c).
}

Figure ~\ref{fig:para} shows a synthetic shot gather with four hyperbolic events, 
501 traces. Some random noise is added to this gather. We do not use windows in time 
for this example. For $f$-$x$ RNA, the length of filter is $M=4$ and the smoothing radiuses in 
space and frequency axes are respectively 20 and 3, ${{r}_{x}}=20$, ${{r}_{f}}=3$. The $f$-$x$ domain prediction 
is implemented over a sliding window of 20 traces width with 50\% overlap and the filter length is 6, $M=3$ and 
the $t$-$x$ domain prediction is implemented over the same sliding window and the filter length in space and 
time are 6 and 5 respectively. The estimated nonstationary coefficients by the proposed $f$-$x$ RNA are shown 
in Figure ~\ref{fig:npar}. Note that the middle coefficient is bigger than the sideward, which is because the dip of 
the middle is smaller than the sideward. The results of three methods are shown in Figures ~\ref{fig:fxpatch}-~\ref{fig:npre}, 
respectively. The $f$-$x$ RNA achieves a similar result to $f$-$x$ domain and $t$-$x$ domain prediction methods. 
However, we use equation ~\ref{eq:eq11} to compute the SNRs of the results of three methods. The SNRs of three 
methods are 0.98 dB, 1.25 dB, 1.67 dB, respectively. The $f$-$x$ RNA can improve SNR more greatly. The 
$f$-$x$ RNA solves the nonstationary case by allowing the coefficients smoothly varying, while $f$-$x$ domain 
or $t$-$x$ domain prediction method uses windowing strategies. From the difference sections (Figure ~\ref{fig:fxdiff}-~\ref{fig:ndiff}), we 
find that $f$-$x$ domain and $t$-$x$ domain prediction methods damage more signals than $f$-$x$ RNA. If we use 
windows in time for this example, we can obtain better results. This example shows that $f$-$x$ RNA can 
be used for random noise attenuation in shot gather. 

\subsection{Field poststack dataset}

\inputdir{real}

\multiplot{4}{data,fxm,tx,npre}{width=0.4\columnwidth}{ (a) A field marine data set. 
(b) The result of $f$-$x$ domain prediction. (c) The result of $t$-$x$ domain prediction. 
(d) The result of $f$-$x$ RNA.
}

\multiplot{3}{fxdiff,txdiff,nprediff}{width=0.5\columnwidth}{Difference sections of $f$-$x$ domain prediction (a), 
$t$-$x$ domain prediction (b), and $f$-$x$ RNA (c).
}

\multiplot{4}{zdata,zfxm,ztx,znpre}{width=0.4\columnwidth}{Zoomed sections. (a) Original data. 
(b) The result of $f$-$x$ domain prediction. (c) The result of $t$-$x$ domain prediction. 
(d) The result of $f$-$x$ RNA.
}

\multiplot{4}{dataf,fxf,txf,npref}{width=0.4\columnwidth}{Comparison on spectra of one trace 
at 6000 m in field marine data (Figure ~\ref{fig:data}-~\ref{fig:npre}). (a)-(d) are the amplitude spectra of 
one trace at 6000 m in Figures ~\ref{fig:data}-~\ref{fig:npre}, respectively.
}

Figure ~\ref{fig:data} is a seismic image from marine data after time migration. 
The preprocessing, such as bandpass filtering and migration, has removed some noise. 
However, some noise still exists in this image (indicated by arrows). This dataset is 
not structurally too complex and the noise seems random. Therefore, we can use $t$-$x$ 
domain or $f$-$x$ domain prediction methods to attenuate the random noise. 
Figures ~\ref{fig:fxm}-~\ref{fig:npre} show the results of random noise attenuation using 
$f$-$x$ domain prediction, $t$-$x$ domain prediction and $f$-$x$ RNA, respectively. The length of 
time window is 512 ms in all the three methods. For $f$-$x$ RNA, the filter length is $M=4$, 
and the smoothing radiuses in space and frequency axes are respectively 20 and 3, 
${{r}_{x}}=20$, ${{r}_{f}}=3$. The $f$-$x$ domain prediction is implemented over a 
sliding window of 20 traces width with 50\% overlap and the filter length is $M=4$ 
and the $t$-$x$ domain prediction is implemented over the same sliding window and the 
filter length in space and time are 6 and 5 respectively. The $f$-$x$ domain and $t$-$x$ 
domain prediction methods removes random noise well in the case that the events 
are approximately linear. In the area of complex structure, however, both of the 
$f$-$x$ domain and $t$-$x$ domain prediction methods can not obtain a good result. Compared 
to $f$-$x$ domain and $t$-$x$ domain prediction methods, $f$-$x$ RNA removes more noise and 
preserves signals (Figure ~\ref{fig:fxdiff}-~\ref{fig:nprediff}). 
Note that $f$-$x$ domain and $t$-$x$ domain prediction methods remove some signals, 
especially for complex structure (indicated by arrows in Figures ~\ref{fig:fxdiff} and ~\ref{fig:txdiff}). 
We display the zoomed section in Figure ~\ref{fig:zdata}-~\ref{fig:znpre}. 
The zoomed sections show the proposed method is more effective than other methods, 
especially in the area of complex structure indicated by arrow. The $f$-$x$ RNA gives 
a good result not only for linear events but also for curving events 
(indicated by arrows in Figure ~\ref{fig:zdata}-~\ref{fig:znpre}). 
From the comparison on the spectra of a trace randomly chosen as shown in 
Figure ~\ref{fig:dataf}-~\ref{fig:npref}, we can see that $f$-$x$ domain and $t$-$x$ 
domain prediction methods greatly attenuate frequency components in [10,30] Hz, 
which includes effective signals. Thus, the difference sections of $f$-$x$ domain and 
$t$-$x$ domain prediction methods (Figures ~\ref{fig:fxdiff} and ~\ref{fig:txdiff}) 
include more signals than $f$-$x$ RNA (Figure ~\ref{fig:nprediff}). For high frequency random noise, 
all the three methods can achieve a similar result (Figure ~\ref{fig:dataf}-~\ref{fig:npref}).

\section{Conclusions}

We have proposed a novel method for random noise attenuation using $f$-$x$ 
domain regularized nonstationary autoregression. $f$-$x$ RNA uses shaping 
regularization to constrain the complex nonstationary coefficients to 
be smooth along space and frequency axes. Contrary to conventional 
noise-reduction technology, $f$-$x$ domain and $t$-$x$ domain prediction, $f$-$x$ 
RNA invokes no piecewise-stationary assumption. The parameters used in 
$f$-$x$ RNA are intuitive because the parameters directly control the 
smoothness of complex coefficients. The proposed method has two key 
parameters: filter length and smoothing radius of shaping operator. 
Filter length is related to the number of events and smoothing radius 
is related to the smoothness of desired RNA complex coefficients. As 
the smoothing radius increases, the result of RNA approaches the result 
of stationary autoregression. This approach does not require breaking 
the input data into local windows along space axis, although it is 
conceptually analogous to sliding spatial windows with maximum overlap. 
Both synthetic and field data examples confirm that the proposed approach 
can be significantly more effective than other noise-reduction methods 
in improving signal-to-noise ratio and preserving the signals. A 
comparison with the recently published $t$-$x$ RNA method has not been 
attempted, but remains of interest for further investigation. The 
proposed method is easy to extend to the 3D case ($f$-$x$-$y$ domain). 
One only needs to add a space dimension in the equation 9 when 
applied in 3D case. Besides random noise attenuation, $f$-$x$ RNA may 
have other applications in seismic data processing, such as seismic 
trace interpolation.

\section{Acknowledgments}

We would like to thank Sergey Fomel, Jingye Li and Yang Liu for inspiring 
discussions and help with the code in the Madagascar software package. 
We also thank the associate editor, the reviewer Clement Kostov and one 
anonymous reviewer for their constructive comments, which improved the
 quality of the paper. This work is financially supported by Important 
National Science and Technology Specific Projects of China 
(grant 2011ZX05023-005-005).

%\section{Note on reproducibility}

%All computational experiments described in this paper can be
%reproduced using the \emph{Madagascar} open-source software package
%available at \url{http://rsf.sourceforge.net}.


\bibliographystyle{seg}
\bibliography{paper}


