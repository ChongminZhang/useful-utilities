\title{Industrial seismology sampler}
\author{Jon Claerbout}
\maketitle
\label{paper:rez}
\plot{line}{width=6in, height=4.5in}{
	A 2-D seismic survey line.
	Left half is layers.
	Right half is a salt dome.
	Salt flows upwards, dragging hence bending upwards the adjacent layers.
	There are no reflections inside the salt.
	In the salt are only artefacts of data processing.
	}
Industrial seismology is a big consumer
of technologies developed in this book.
This book steers away from seismology because of its complexity
(and because I have written other books devoted to seismology).
Figure~\ref{fig:line} is a traditional single survey line
of the kind that dominated the industry in the 1960s.

\par
This book is merely a ``warm up'' to today's industry.
In earlier chapters you saw tiny data
sets manageable in a small desktop computer.
Industrial seismology measures hydrophone voltage
in a five dimensional data space,
two surface coordinates $(x_s,y_s)$ for each source pop,
two more  $(x_r,y_r)$ for each receiver,
and the echo delay time $t$.
It has the 3-D model space of our world $(x,y,z)$,
though on the cube here we do not see $z$,
but $t$, the vertical seismic travel time.
\par

\plot{rivers}{width=6in,height=5in}{
	At $t=$1.387s (about 1.4km depth):
	The upper right circular corner is a salt dome.
	River meanders from about a million years ago.
	River meanders are a common sight in 3-D reflection seismic images.
	Rivers typically migrate significant distances
	in the 7000 years between our resolution slices.
	Some depth ranges contain no rivers.
	Such correspond to eras when these layers
	were being laid down lay beneath the sea.
	}

\par
Illustrations here may look like data,
but they are slices from model space,
On figure \ref{fig:line} the alternating voltages in the seismic microphone
suggest black-white physical layering in the earth.
While this surely indicative,
higher frequency filtration would yield more layers.
Keep this in mind as you examine Figure \ref{fig:rivers},
a horizontal slice inside the earth at a constant depth
(travel-time depth $t=1.387$sec).
Local outline shapes are truly meaningful here while
black/white polarities hardly so.
Whether a river is white in a black background,
or black in a white background is an accidental function
of overall travel time and spectrum.
What is significant is the rings surrounding the dome.
These are a consequence of the upward bending layers
you saw in figure  \ref{fig:line}.

\plot{rivers2}{width=6in, height=5in}{
	At $t$=0.888s:
	The $(x,y)$ plane shown here is grabbed
	from a volume of slices separated by 6ms, about 18 feet.
	Slice to slice represents about 7,000 years of
	sedimentary deposition in the Gulf of Mexico.
	Top to bottom is about a million years
	(about the age of the human species).
	Think of the creatures in all those rivers, their ancient worlds.
	Awesome, isn't it?
	}

\par
Seismic waves here are a little faster than 2 km/sec,
but they must go both down into the earth and up again,
so the bottom of the time axis is a little more than 2 km deep.
A ship sails from west to east creating an $x$-axis,
22 kilofeet long, a little under 5 miles.
Where the vertical axis is not north-south it is travel time.
Typically that axis might run to 5 sec.
Here for space limitations, it runs less than 2s.
All the planes you see in this chapter come from one
$292\times 451\times 551 $ cube of 72 megapixels,
a subset of a larger volume of model space.

\par
\plot{bigcanyon}{width=6in, height=5in}{
	At $t$=1.830s, a rarely seen image:
	Embedded in this map view
	at $(x,y)=$(6-13, \, 14-16)
	are many drainage tributaries (a dendritic pattern)
	to a central canyon on a lower slice (not shown).
	An artist might see it as a tree root,
	feeders going off to the lower left.
	The fault in Figure~\ref{fig:fault}
	is here again seen emerging southward
	from the salt-dome at $x$=14.
	}

\par
You may be seeing paper or images of what's on paper,
but what you see is merely two-dimensional slices thru the 3-D model space.
I can plunge into these volumes, panning and zooming.
Thanks to my colleague Bob Clapp and others like him
after some years we may escape the constraints of PDF files
and deliver such experiences to readers outside our lab.

The upper right corner of the constant depth slices shows a circular region.
This is salt.  Salt, like ice, seems brittle, but under pressure it flows like a liquid.
Before the past million years ago before the sediments
of this cube were laid down
there was a salt lake here
that eventually dried and was buried beneath the sand, shale,
and carbonates that became this cube.
Salt is lighter than rock,
and so eventually it erupted like a pimple on the face of the earth,
a pimple two miles wide.
No oil in here, but the bent up layers aside it seen
in Figure \ref{fig:line} are excellent prospects.
Salt flow is a dominant feature in the Gulf of Mexico.

\plot{worms}{width=6in, height=5in}{
	At $t$=1.938s:
	To the east of the fault noted already in Figure \ref{fig:bigcanyon}
	is a broken up layer with a ``wormy'' appearance.
	I do not know what it is.
	Curiously it is found only on one side of the fault.
	}

\plot{slump}{width=6in, height=5in}{
	At $t$=1.014s:
	A full-page slash upon the earth of strikingly mathematical perfection,
	that of a hyperbola.
	Increasing the slice depth I found it shifted southward,
	persisting only about 6 of these 18 foot slices, roughly 120' 
	top to bottom.
	I interpret this as a land slump.
	Sediment accumulates on the water bottom increasing its weight
	until suddenly with an earthquake
	it slides down toward deeper water.
%
	Seeing a hyperbola in a solid material was
	startling to us seismic data analysts.
	We see many hyperbolas on the time axis, but never on a space axis.
	Our first thought was, ``This must be a data processing artifact.''
	Now we feel we have eliminated that possibility.
	Something about the presumed stress to earthquake process
	could make this shape.
%
	I first suggested to a petroleum geologist it might be a beach.
	He said beaches move rapidly in geologic time
	as land and water levels rise and fall.
	He suggested an area covered in parallel lines.
	I believe he was correct, but the example we found
	was sufficiently imperfect that I'm not showing it.
	}

\par
This data cube (actually model space) is about 20 years old.
It came from Chevron via David Lumley to James Rickett.
It is textbook quality 3-D data from the Gulf of Mexico.
It would have taken the survey company about a month to acquire,
and it would cost the oil company (group maybe) about ten million dollars.

\plot{karst}{width=6in, height=5in}{
	At $t$=1.578s:
	A speckled area about 6 kilofeet square is centered about $(4.,6.)$.
	I interpret this as karst, limestone in its varied forms,
	very rugged. Some circular dots might be sink holes.
	At increasing travel-time depths,
	this area drifted towards the southeast,
	the drift being evidence of changing sea level.
	}

\par
A ship with an air gun towed
a 7km long cable with a thousand hydrophones.
Today there would be several gun boats.
The recording ship would trail about a dozen streamers
separated about 150 meters.
World-wide there are about 50 marine survey teams working continuously.
The half dozen largest seismic survey companies together sell
about ten billion dollars of surveys per year to oil companies, private and national.
Data is also recorded on land with more varied equipment types.
\par

\plot{fault}{width=6in, height=4in}{
	Under about $x=$ 14 kilofeet
	is a prominent near-vertical fault in this 2-D line.
	Faults are more common than river meanders,
	especially prolific in 2-D seismic display.
	This data cube is unusual in that it shows only
	this one prominent fault.
	Fascinating detail easily apparent on the time slices
	are unintelligible on the 2-D $(x,t)$ lines
	simply showing as mere irregularities on the layers,
	seeming merely noise.
	Back in the 1960s, when this was the only kind of data we had,
	we might see many faults per mile,
	but never imagined the wealth of geologic detail
	you have seen here on the time slices..
	}


\clearpage
