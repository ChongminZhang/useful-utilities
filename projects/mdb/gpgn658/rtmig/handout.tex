\author{Carl Friedrich Gauss}
\title{GPGN658: Seismic migration}{Reverse-time migration}

% ------------------------------------------------------------
In this homework, you will use a finite-differences modeling code
to implement
basic reverse time migration. I do not expect you to be concerned with
the efficiency of your implementation at this time. This
implementation of reverse-time migration does not require that you
write any new C code. You will use pre-existing Madagascar programs,
but you will modify the \texttt{SConstruct} file to combine those
programs.

% ------------------------------------------------------------

\section{Prerequisites}

Log into your class account and prepare for the assignment:

\begin{enumerate}

\item Update the Madagascar library and

\texttt{cd \$RSF/book/gpgn658/rtmig}.

You should have in your 
account the following files:\\
\texttt{rtmig/SConstruct} \\
\texttt{rtmig/handout.tex} \\
\texttt{rtmig/exercise/SConstruct} \\
\texttt{rtmig/exercise/fdm.py}

\item 
Open the file \texttt{handout.tex} in your favorite text editor
and change the author name at the top of the file.

\item 
\texttt{cd exercise} \par
to change directory to the programming exercise.

\item 
Run \texttt{scons view} to build all targets described in the
\texttt{SConstruct} file. Watch for figures popping-up on your screen.
Close each figure by typing \texttt{q} with the cursor inside the window.

\item
Run \texttt{scons lock} to prepare your figures to be included in the
main document.

\item 
\texttt{cd rtmig} and run \texttt{scons read} to build and 
read your answer. The following figures are included in your document:

% ------------------------------------------------------------
\inputdir{exercise}
% ------------------------------------------------------------

\begin{itemize}

\item The velocity model (Figure~\ref{fig:vo}).
\sideplot{vo}{width=\textwidth}{Velocity.}

\item The density model (Figure~\ref{fig:ra}).
\sideplot{ra}{width=\textwidth}{Density.}

\item The recorded data (Figures~\ref{fig:dr0} and \ref{fig:dr1}).
\sideplot{dr0}{width=\textwidth}{Receiver data (case 0).}
\sideplot{dr1}{width=\textwidth}{Receiver data (case 1).}

\end{itemize}

\end{enumerate}
% ------------------------------------------------------------
\section{Exercise}

Using the finite-differences modeling function \texttt{awefd2d},
construct an image of the subsurface. This function takes the
following parameters: \\
\texttt{awefd(odat,owfl,idat,velo,dens,sou,rec,custom,par)}
\begin{itemize}
\item \texttt{odat}: output data $d\left( x,t \right)$
\item \texttt{owfl}: output wavefield $u \left( z,x,t \right)$
\item \texttt{idat}: input data (wavelet)
\item \texttt{velo}: velocity model $v \left( z,x \right)$
\item \texttt{dens}: density model $\rho \left( z,x \right)$
\item \texttt{sou}: source coordinates
\item \texttt{rec}: receiver coordinates
\item \texttt{custom}: custom parameters
\item\texttt{par}: parameter dictionary
\end{itemize}

Design an imaging procedure following the generic scheme developed in
class. Your task is to identify the useful Madagascar programs for
this task and generate the appropriate Flows in the
\texttt{SConstruct} to build an image. Explain in detail how your 
imaging procedure works. What processing steps do you take? Why do you
do that?

Your assignment has two parts:
\begin{enumerate}
\item 
Use your imaging procedure to generate images based on recorded data
in Figures~\ref{fig:dr0} and \ref{fig:dr1}. For this exercise, use the
constant density \texttt{rb.rsf} for imaging. Include those two images
in this document. Are the images different from each-other? Why?

\item
Use your imaging procedure to generate images based on recorded data
in Figures~\ref{fig:dr0} and \ref{fig:dr1}. For this exercise, use the
constant density \texttt{ra.rsf} for imaging. Include those two images
in this document. Are the images different from each-other? Why? How
do your images compare with the ones from the preceding exercise?

\end{enumerate}

% ------------------------------------------------------------
\section{Wrap-up}

After you are satisfied that your document looks ok, print it from the
PDF viewer and bring it to class.

% ------------------------------------------------------------
\newpage
\section{SConstruct}
\tiny
\lstinputlisting{exercise/SConstruct}
\normalsize


