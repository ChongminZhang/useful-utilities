\title{BPait Model}
\author{Trevor Irons}
\maketitle
\lstset{language=python,numbers=left,numberstyle=\tiny,showstringspaces=false}

\noindent
\textbf {Data Type:} \emph{2D acoustic synthetic model}\\
\textbf {Source:} \emph{British Petroleum}\\
\textbf {Location:} \emph{http://www.software.seg.org}\\
\textbf {Format:} \emph{Native} \\
\textbf{Date of origin:} \emph{Model was produced for an EAGE convention in 2004}\\

\section{Introduction}
The Bpait model was constructed as a sort of test of current imaging techniques.  BP produced the dataset and provided it to interested parties for processing.  The results were presented at the 2004 EAGE workshop.  The velocity models were later made available and the dataset is now publically released courtesy of BP.  

\tabl{FILES}{A list of files contained within the Madagascar Bpait dataset repository}
{
\tiny
\lstinputlisting[frame=single]{FILES}
\normalsize
}

\section{Velocity Model}
The 2D velocity model is 67 km across and 12 km in depth. There are several parameters known about the model.  
First, it contains a water layer with a velocity of 1486 m/s.  Secondly, there are two salt bodies; 
the left most one has a velocity of 4510 m/s while the body on the right hand side has a velocity of 4790 m/s.  

In a geologic, and imaging, sense the model can be broken into three distinct areas.  
The left side of the model has a simple backround with a complex rugose salt body with several sub salt 
low velocity anomalies.  This part of the model emulates deep water prospect areas in the Gulf of Mexico.  The middle part of the model has a deeply sourced steeply dipping salt stalk structure and emulates scenarios found in the Gulf of Mexico and offshore West Africa.  The right side of the model has no salt structures but contains many shallow gas prospects and targets similiar to environments found in the North Sea.        

In order to process the data using Madagascar headers should be configured as shown in table \ref{tbl:modelHeader}.  
The \emph{bpaitvel.hh} file found in the repository is already formatted in this fashion aside from labels and units.    

\tabl{modelHeader}{Bpaint header information for velocity model}
{
\begin{tabular}{|llllll|}
        \hline
    n1=1944    &     d1=6.25   &        o1=0  &        label= Depth & unit1=m &  \\
    n2=5395   &     d2=12.5   &        o2=0  &        label2=X     & unit2=m &  \\
        \hline
\end{tabular}
}

The file \emph{bpait/model/SConstruct} shown in figure \ref{fig:velSConstruct} fetches the necessary files, 
appends the header slightly and produces an image of the velocity model which is reproduced in figure \ref{fig:bpaitvel}.

\tabl{velSConstruct}{\emph{SConstruct} script generating a BPait velocity model image}
{
\tiny
\lstinputlisting[frame=single]{model/SConstruct}
\normalsize
}

\inputdir{model}
\plot{bpaitvel}{width=\textwidth}{Velocity model}


\section{Shot Records}
The synthetic data were generated using a 2D time-domain, acoustic, finite-difference
modeling algorithm. The data were recorded with a free-surface, and all free-surface 
related multiples were left in the data set.

This required a rigorous effort on de-multiple/multiple suppression
before migration and velocity analysis could be applied. In hindsight, we see that this caused
problems for some of the academic institutions that did not have access to proper de-multiple
software. One the other hand, we may see future benefits in this since the dataset could be
used in testing of novel de-multiple techniques.
The data were generated with a streamer configuration, using a 15km streamer with 12.5m
group interval and a 50m shot interval. Minimum offset in the data is 0m. We recorded 14
seconds of data with a 6ms sampling interval. The dominant frequency is 27Hz and data can
be whitened up to 54Hz. The low-cut frequency is 0.5Hz. The wavelength is causal and has
not been zero-phased. A total of 1340 shots were generated each with 1201 receivers.

