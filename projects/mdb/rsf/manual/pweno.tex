\section{1-D ENO power-p interpolation (pweno.c)}




\subsection{{sf\_pweno\_init}}
Initializes an object of type \texttt{sf\_pweno}.

\subsubsection*{Call}
\begin{verbatim}ent = sf_pweno sf_pweno_init (int order, n);\end{verbatim}

\subsubsection*{Definition}
\begin{verbatim}
sf_pweno sf_pweno_init (int order /* interpolation order */,
                        int n     /* data size */)
/*< Initialize interpolation object >*/
{
   ...
}
\end{verbatim}

\subsubsection*{Input parameters}
\begin{desclist}{\tt }{\quad}[\tt order]
   \setlength\itemsep{0pt}
   \item[order] order of interpolation (\texttt{int}). 
   \item[n] size of the data (\texttt{int}).  
\end{desclist}

\subsubsection*{Output}
\begin{desclist}{\tt }{\quad}[\tt ]
   \setlength\itemsep{0pt}
   \item[ent] object of type \texttt{sf\_pweno}.
\end{desclist}




\subsection{{sf\_pweno\_close}}
Frees the space allocated for the \texttt{sf\_pweno} object by \texttt{sf\_pweno\_init}.

\subsubsection*{Call}
\begin{verbatim}sf_pweno_close (ent);\end{verbatim}

\subsubsection*{Definition}
\begin{verbatim}
void sf_pweno_close (sf_pweno ent)
/*< Free internal storage >*/
{
   ...
}
\end{verbatim}

\subsubsection*{Input parameters}
\begin{desclist}{\tt }{\quad}[\tt ]
   \setlength\itemsep{0pt}
   \item[ent] object of type \texttt{sf\_pweno}.
\end{desclist}




\subsection{{powerpeno}}
Calculates the Power-p limiter for eno method using the input numbers \texttt{x} and \texttt{y}.

\subsubsection*{Call}
\begin{verbatim}power = powerpeno (x, y, p);\end{verbatim}

\subsubsection*{Definition}
\begin{verbatim}
float powerpeno (float x, float y, int p /* power order */)
/*< Limiter power-p eno >*/
{
   ...
}
\end{verbatim}

\subsubsection*{Input parameters}
\begin{desclist}{\tt }{\quad}[\tt ]
   \setlength\itemsep{0pt}
   \item[x] an input number (\texttt{float}). 
   \item[y] an input number (\texttt{float}). 
   \item[p] power order (\texttt{int}).  
\end{desclist}

\subsubsection*{Output}
\begin{desclist}{\tt }{\quad}[\tt ]
   \setlength\itemsep{0pt}
   \item[mins * power] limiter power-p. It is of type \texttt{float}.
\end{desclist}




\subsection{{sf\_pweno\_set}}
Sets the interpolation undivided difference table.

\subsubsection*{Call}
\begin{verbatim}void sf_pweno_set (sf_pweno ent, float* c /* data [n] */, int p);\end{verbatim}

\subsubsection*{Definition}
\begin{verbatim}
void sf_pweno_set (sf_pweno ent, float* c /* data [n] */, int p /* power order */)
/*< Set the interpolation undivided difference table. c can be changed or freed 
afterwards >*/
{
   ...
}
\end{verbatim}

\subsubsection*{Input parameters}
\begin{desclist}{\tt }{\quad}[\tt ent]
   \setlength\itemsep{0pt}
   \item[ent] the interpolation object. Must be of type \texttt{sf\_pweno}. 
   \item[c] input data (\texttt{float*}).  
   \item[p] power order (\texttt{int}).  
\end{desclist}




\subsection{{sf\_pweno\_apply}}
Applies the interpolation.

\subsubsection*{Call}
\begin{verbatim}sf_pweno_apply (ent, i, x, f, f1, what);\end{verbatim}

\subsubsection*{Definition}
\begin{verbatim}
void sf_pweno_apply (sf_pweno ent, 
                int i     /* grid location */, 
                float x   /* offset from grid */, 
                float *f  /* output data value */, 
                float *f1 /* output derivative */, 
                derr what /* flag of what to compute */) 
/*< Apply interpolation >*/
{
   ...
}
\end{verbatim}

\subsubsection*{Input parameters}
\begin{desclist}{\tt }{\quad}[\tt what]
   \setlength\itemsep{0pt}
   \item[i]    location of the grid (\texttt{int}).  
   \item[x]    offset from the grid (\texttt{float}). 
   \item[f]    output data value (\texttt{float*}).  
   \item[f1]   output derivative (\texttt{float*}).  
   \item[what] flag of what to compute. Must be of type \texttt{derr}.  
\end{desclist}





