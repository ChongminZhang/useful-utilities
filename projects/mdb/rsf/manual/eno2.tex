\section{ENO interpolation in 2-D (eno2.c)}\label{sec:eno2.c}




\subsection{{sf\_eno2\_init}}\label{sec:sf_eno2_init}
Initializes an object of type \texttt{sf\_eno}2 for interpolation of 2D data.

\subsubsection*{Call}
\begin{verbatim}pnt =  sf_eno2_init (order, n1, n2);\end{verbatim}

\subsubsection*{Definition}
\begin{verbatim}
sf_eno2 sf_eno2_init (int order      /* interpolation order */, 
                      int n1, int n2 /* data dimensions */)
/*< Initialize interpolation object >*/
{
    sf_eno2 pnt;
    int i2;
    
   ...
    return pnt;
}
\end{verbatim}

\subsubsection*{Input parameters}
\begin{desclist}{\tt }{\quad}[\tt order]
   \setlength\itemsep{0pt}
   \item[order] interpolation order (\texttt{int}).  
   \item[n1]    first dimension of the data (\texttt{int}).  
   \item[n2]    second dimension of the data (\texttt{int}).  
\end{desclist}

\subsubsection*{Output}
\begin{desclist}{\tt }{\quad}[\tt ]
   \setlength\itemsep{0pt}  
   \item[pnt] object for interpolation. It is of type \texttt{sf\_eno}2.
\end{desclist}




\subsection{{sf\_eno2\_set}}
Sets the interpolation table for the 2D data in a 2D array.

\subsubsection*{Call}
\begin{verbatim}sf_eno2_set (pnt, c);\end{verbatim}

\subsubsection*{Definition}
\begin{verbatim}
void sf_eno2_set (sf_eno2 pnt, float** c /* data [n2][n1] */)
/*< Set the interpolation table. c can be changed or freed afterwords. >*/
{
   ...    
}
\end{verbatim}

\subsubsection*{Input parameters}
\begin{desclist}{\tt }{\quad}[\tt pnt]
   \setlength\itemsep{0pt}
   \item[pnt] object for interpolation. It is of type \texttt{sf\_eno}2.
   \item[c]   the data (\texttt{float**}).  
\end{desclist}





\subsection{{sf\_eno2\_set1}}
Sets the interpolation table for the 2D data in a 1D array, which is of size \texttt{n1*n2}.

\subsubsection*{Call}
\begin{verbatim}sf_eno2_set1 (pnt, c);\end{verbatim}

\subsubsection*{Definition}
\begin{verbatim}
void sf_eno2_set1 (sf_eno2 pnt, float* c /* data [n2*n1] */)
/*< Set the interpolation table. c can be changed or freed afterwords. >*/
{
   ...
}
\end{verbatim}

\subsubsection*{Input parameters}
\begin{desclist}{\tt }{\quad}[\tt pnt]
   \setlength\itemsep{0pt}
   \item[pnt] object for interpolation. It is of type \texttt{sf\_eno}2.
   \item[c]   the data (\texttt{float*}).  
\end{desclist}




\subsection{{sf\_eno2\_close}}
Frees the space allocated for the internal storage by \hyperref[sec:sf_eno2_init]{\texttt{sf\_eno2\_init}}.

\subsubsection*{Call}
\begin{verbatim}sf_eno2_close (pnt);\end{verbatim}

\subsubsection*{Definition}
\begin{verbatim}
void sf_eno2_close (sf_eno2 pnt)
/*< Free internal storage >*/
{
   ...
}
\end{verbatim}




\subsection{{sf\_eno2\_apply}}
Interpolates the 2D data.

\subsubsection*{Call}
\begin{verbatim}sf_eno2_apply (pnt, i, j, x, y, f, f1, what);\end{verbatim}

\subsubsection*{Definition}
\begin{verbatim}
void sf_eno2_apply (sf_eno2 pnt, 
                    int i, int j     /* grid location */, 
                    float x, float y /* offset from grid */, 
                    float* f         /* output data value */, 
                    float* f1        /* output derivative [2] */,
                    der what         /* what to compute [FUNC,DER,BOTH] */)
/*< Apply interpolation. >*/
{
   ...
}
\end{verbatim}

\subsubsection*{Input parameters}
\begin{desclist}{\tt }{\quad}[\tt ]
   \setlength\itemsep{0pt}
   \item[pnt]  an object for interpolation. It is of type \texttt{sf\_eno}2.
   \item[i]    location of the grid for first dimension (\texttt{int}).
   \item[j]    location of the grid for second dimension (\texttt{int}).
   \item[x]    offset from the grid for the first dimension (\texttt{float}).  
   \item[y]    offset from the grid for the second dimension (\texttt{float}).  
   \item[f]    output data value (\texttt{float*}).  
   \item[f1]   output derivative (\texttt{float*}).  
   \item[what] whether the function value or the derivative or both are required. Must be of type \texttt{der}.  
\end{desclist}


