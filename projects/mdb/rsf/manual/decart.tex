\section{Conversion between line and Cartesian coordinates of a vector (decart.c)}




\subsection{{sf\_line2cart}}\label{sec:sf_line2cart}
Converts the line coordinates to Cartesian coordinates.


\subsubsection*{Call}
\begin{verbatim}sf_line2cart(dim, nn, i, ii);\end{verbatim}


\subsubsection*{Definition}
\begin{verbatim}
void sf_line2cart(int dim       /* number of dimensions */, 
                  const int* nn /* box size [dim] */, 
                  int i         /* line coordinate */, 
                  int* ii       /* cartesian coordinates [dim] */)
/*< Convert line to Cartesian >*/
{
   ...
}
\end{verbatim}


\subsubsection*{Input parameters}
\begin{desclist}{\tt }{\quad}[\tt dim]
   \setlength\itemsep{0pt}
   \item[dim] number of dimensions (\texttt{int}).  
   \item[nn]  box size (size of the data file) (\texttt{const int*}).  
   \item[i]   the line coordinate (\texttt{int}).  
   \item[ii]  the Cartesian coordinates (\texttt{int*}).  
\end{desclist}




\subsection{{sf\_cart2line}}
Converts the Cartesian coordinates to line coordinate.


\subsubsection*{Call}
\begin{verbatim}int sf_cart2line(dim, nn, ii);\end{verbatim}


\subsubsection*{Definition}
\begin{verbatim}
int sf_cart2line(int dim       /* number of dimensions */, 
                 const int* nn /* box size [dim] */, 
                 const int* ii /* cartesian coordinates [dim] */) 
/*< Convert Cartesian to line >*/
{
   ...
}
\end{verbatim}


\subsubsection*{Input parameters}
\begin{desclist}{\tt }{\quad}[\tt dim]
   \setlength\itemsep{0pt}
   \item[dim] number of dimensions (\texttt{int}).  
   \item[nn]  box size (size of the data file) (\texttt{const int*}).  
   \item[ii]  the Cartesian coordinates (\texttt{int*}).  
\end{desclist}

\subsubsection*{Output}
\begin{desclist}{\tt }{\quad}[\tt ]
   \setlength\itemsep{0pt}
   \item[i] line coordinate. It is of type \texttt{int}.
\end{desclist}




\subsection{{sf\_first\_index}}\label{sec:sf_first_index}
Returns the first index for a particular dimension.

\subsubsection*{Call}
\begin{verbatim}sf_first_index (i, j, dim, n, s);\end{verbatim}


\subsubsection*{Definition}
\begin{verbatim}
int sf_first_index (int i        /* dimension [0...dim-1] */, 
                    int j        /* line coordinate */, 
                    int dim      /* number of dimensions */, 
                    const int *n /* box size [dim] */, 
                    const int *s /* step [dim] */)
/*< Find first index for multidimensional transforms >*/
{
   ... 
}
\end{verbatim}


\subsubsection*{Input parameters}
\begin{desclist}{\tt }{\quad}[\tt dim]
   \setlength\itemsep{0pt}
   \item[i]   the dimension (\texttt{int}).  
   \item[j]   the line coordinate (\texttt{int}).  
   \item[dim] number of dimensions (\texttt{int}).  
   \item[n]   box size (size of the data file) (\texttt{const int*}).  
   \item[s]   the step size (\texttt{const int*}).  
\end{desclist}

\subsubsection*{Output}
\begin{desclist}{\tt }{\quad}[\tt ]
   \setlength\itemsep{0pt}
   \item[i0] first index for the given dimension. It is of type \texttt{int}.
\end{desclist}




\subsection{{sf\_large\_line2cart}}
Converts the line coordinate to Cartesian coordinates. It works exactly like \hyperref[sec:sf_line2cart]{\texttt{sf\_line2cart}} but in this one the line and Cartesian coordinates are of type \texttt{off\_t}, which means that they are given in terms of the offset in bytes in the data file.

\subsubsection*{Call}
\begin{verbatim}sf_large_line2cart(dim, nn, i, ii);\end{verbatim}

\subsubsection*{Definition}
\begin{verbatim}
void sf_large_line2cart(int dim         /* number of dimensions */, 
                        const off_t* nn /* box size [dim] */, 
                        off_t i         /* line coordinate */, 
                        off_t* ii       /* cartesian coordinates [dim] */)
/*< Convert line to Cartesian >*/
{
   ...
}
\end{verbatim}


\subsubsection*{Input parameters}
\begin{desclist}{\tt }{\quad}[\tt dim]
   \setlength\itemsep{0pt}
   \item[dim] number of dimensions (\texttt{int}). 
   \item[nn]  box size (size of the data file) (\texttt{const off\_t*}).  
   \item[i]   the line coordinate (\texttt{off\_t}). 
   \item[ii]  the Cartesian coordinates (\texttt{off\_t*}).  
\end{desclist}




\subsection{{sf\_large\_cart2line}}
Converts the Cartesian coordinates to line coordinate. It works exactly like \hyperref[sec:sf_line2cart]{\texttt{sf\_line2cart}} but in this one the line and Cartesian coordinates are of type \texttt{off\_t}, which means that they are given in terms of the offset in bytes in the data file.


\subsubsection*{Call}
\begin{verbatim}sf_large_cart2line(int, nn, ii);\end{verbatim}


\subsubsection*{Definition}
\begin{verbatim}
off_t sf_large_cart2line(int dim         /* number of dimensions */, 
                         const off_t* nn /* box size [dim] */, 
                         const off_t* ii /* cartesian coordinates [dim] */) 
/*< Convert Cartesian to line >*/
{
   ...
}
\end{verbatim}


\subsubsection*{Input parameters}
\begin{desclist}{\tt }{\quad}[\tt dim]
   \setlength\itemsep{0pt}
   \item[dim] number of dimensions (\texttt{int}). 
   \item[nn]  box size (size of the data file) (\texttt{const off\_t*}).  
   \item[ii]  the Cartesian coordinates (\texttt{const off\_t*}).  
\end{desclist}

\subsubsection*{Output}
\begin{desclist}{\tt }{\quad}[\tt ]
   \setlength\itemsep{0pt}
   \item[i] line coordinate. It is of type \texttt{off\_t}.
\end{desclist}




\subsection{{sf\_large\_first\_index}}
Returns the first index for a particular dimension. It works exactly like \hyperref[sec:sf_first_index]{\texttt{sf\_first\_index}} but in this one the line coordinate, box size, step size and the first index are of type \texttt{off\_t}, which means that they are given in terms of the offset in bytes in the data file.

\subsubsection*{Call}
\begin{verbatim}sf_large_first_index (i, j, dim, n, s);\end{verbatim}


\subsubsection*{Definition}
\begin{verbatim}
off_t sf_large_first_index (int i          /* dimension [0...dim-1] */, 
                            off_t j        /* line coordinate */, 
                            int dim        /* number of dimensions */, 
                            const off_t *n /* box size [dim] */, 
                            const off_t *s /* step [dim] */)
/*< Find first index for multidimensional transforms >*/
{
   ...    
}
\end{verbatim}


\subsubsection*{Input parameters}
\begin{desclist}{\tt }{\quad}[\tt dim]
   \setlength\itemsep{0pt}
   \item[i ]  the dimension (\texttt{int}). 
   \item[j]   the line coordinate (\texttt{off\_t}). 
   \item[dim] number of dimensions (\texttt{int}). 
   \item[n]   box size (size of the data file) (\texttt{const off\_t*}).  
   \item[s]   the step size (\texttt{const off\_t*}).  
\end{desclist}

\subsubsection*{Output}
\begin{desclist}{\tt }{\quad}[\tt ]
   \setlength\itemsep{0pt}
   \item[i0] first index for the given dimension. It is of type \texttt{int}.
\end{desclist}




