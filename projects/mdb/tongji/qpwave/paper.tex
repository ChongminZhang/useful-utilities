\published{Geophysics, 79, no. 1,  C1-C18, (2014)}

\title{Simulating propagation of separated wave modes in general anisotropic media, Part I: qP-wave propagators}

\author{Jiubing Cheng\footnotemark[1] and Wei Kang\footnotemark[2]}

\address{
\footnotemark[1] State Key Laboratory of Marine Geology, Tongji University, Shanghai, China. E-mail: cjb1206@tongji.edu.cn\\
\footnotemark[2] School of Ocean and Earth Science, Tongji University, Shanghai, China. E-mail: w.kang\_1986@hotmail.com
}

\lefthead{Cheng \& Kang}
\righthead{Propagation of separated qP-wave mode}

\maketitle

\begin{abstract}
Wave propagation in an anisotropic medium is inherently described by elastic wave equations, with P- and S-wave modes
intrinsically coupled. We present an approach to simulate propagation of separated wave modes for
 forward modeling,
 migration, waveform inversion and other applications in general anisotropic media.
The proposed approach consists of two cascaded computational steps. First, we simulate equivalent
 elastic anisotropic wavefields with a minimal second-order coupled system 
(that we call here a pseudo-pure-mode wave equation),
which describes propagation of all wave modes with a partial wave mode separation. Such a system for qP-wave
 is derived from the inverse Fourier transform of the Christoffel equation after a
 similarity transformation, which aims to project the original vector displacement wavefields onto isotropic
 references of the polarization directions of qP-waves. It accurately describes the kinematics
 of all wave modes and enhances qP-waves when the pseudo-pure-mode wavefield components are summed.
 The second step is a filtering to further project the pseudo-pure-mode wavefields onto the polarization directions
 of qP-waves so that residual qS-wave energy is removed and scalar qP-wave fields are accurately separated
 at each time step during wavefield extrapolation.
As demonstrated in the numerical examples, pseudo-pure-mode wave equation plus
correction of projection deviation provides a robust and flexible
tool for simulating propagation of separated wave modes in
transversely isotropic and orthorhombic media.
The synthetic example of Hess VTI model shows that
the pseudo-pure-mode qP-wave equation can be used in prestack reverse-time migration (RTM) applications.
\end{abstract}

\section{Introduction}
All anisotropy arises from ordered heterogeneity much smaller than the wavelength \cite[]{winterstein}.
 With the increased resolution of seismic data and because of wider seismic acquisition aperture
(both with respect to offset and azimuth), there is a growing awareness that an 
isotropic description of the Earth may no longer be adequate. Anisotropy appears to be a near-ubiquitous
 property of earth materials, and its effects on seismic data must be quantified.

 Wave equation is the central ingredient in characterizing wave propagation for seismic imaging and elastic
parameters inversion. In isotropic media, it is common to use scalar acoustic wave 
equations to describe the propagation of seismic data as representing only P-wave energy \cite[]{yilmaz}.
 Compared to the elastic wave equation, the acoustic wave equation is simpler and
 more efficient to use, and does not yield S-waves for modeling of P-waves.
 Anisotropic media are inherently described by elastic wave equations with P- and S-wave modes
 intrinsically coupled.
It is well known that a S-wave passing through an anisotropic medium splits into two mutually
 orthogonal waves \cite[]{crampin:1984}. Generally the P-wave and the two S-waves are not polarized parallel 
 and perpendicular to the wave vector, thus are called quasi-P (qP) and qausi-S (qS) waves.
 However, most anisotropic migration implementations
 do not use the full elastic anisotropic wave equation because of the high computational cost involved, and
 the difficulties in separating wavefields into different wave modes. Although an acoustic wave does not
 exist in anisotropic media, \cite{alkhalifah:2000} introduced a pseudo-acoustic
 approximate dispersion relation for vertically transverse isotropic (VTI) media by setting the shear 
velocity along the axis of symmetry to zero, which leads to a fourth-order partial differential equation (PDE)
 in space-time domain.
Following the same procedure, he also presented a pseudo-acoustic wave equation of sixth-order in
vertical orthorhombic (ORT) anisotropic media \cite[]{alkhalifah:2003}. Many authors have
 implemented pseudo-acoustic VTI modeling and migration based on
 various coupled second-order PDE systems derived
 from Alkhalifah's dispersion relation \cite[]{alkhalifah:2000,klie,zhou:2006eage,hestholm}.
Alternatively, coupled first-order and second-order systems are derived starting from first principles
 (the equations of motion and Hooke's law) under the pseudo-acoustic approximation for
 VTI media \cite[]{duveneck:2011} and recently for orthorhombic media 
as well \cite[]{fowler.king:2011,zhang.zhang:2011}.
 The pseudo-acoustic tilted transversely isotropic (TTI) or tilted orthorhombic wave equations
 can be obtained from their pseudo-acoustic VTI (or pseudo-acoustic vertical orthorhombic)
counterparts by simply performing a coordinate rotation according to the directions of the
 symmetry axes \cite[]{zhou:2006seg,fletcher:2009}. 
Pseudo-acoustic wave equations have been
 widely used for RTM in transversely isotropic (TI) media because
they describe the kinematic signatures of qP-waves
 with sufficient accuracy and are simpler than their elastic 
counterparts, which leads to computational savings in practice.

On the other hand, the pseudo-acoustic approximation may result in some problems in characterizing wave propagation
 in anisotropic media. First, to enhance computational stability,
pseudo-acoustic approximations reduce the freedom
 to choose the material parameters compared with their elastic counterparts \cite[]{grechka:2004}.
Practitioners often observe instability in practice when the pseudo-acoustic
 equations are used in complex TI media \cite[]{fletcher:2009,zhang:2011,bube:2012}.
 Stable RTM implementations in TTI media 
can be achieved by using pseudo-acoustic wave equations derived directly from first
 principles \cite[]{duveneck:2011} 
using self-adjoint or covariant derivative operators \cite[]{macesanu,zhang:2011}. 
Second, the widely-used pseudo-acoustic approximation
 still results in significant shear wave presence in both modeling data and RTM images \cite[]{zhang:2003,
 grechka:2004, jin}. It is not easy to get rid of qSV-waves from
 the wavefields simulated by the pseudo-acoustic
  wave equations when a full waveform modeling for 
qP-wave is required. Placing both sources and receivers into an artificial isotropic or elliptic
 anisotropic acoustic layer could eliminate many of the undesired
qSV-wave energy \cite[]{alkhalifah:2000}, but the propagated qP-wave 
may get converted to qSV-wave and the qSV-wave might get
 converted back to qP-wave in other portions of the model.
 A projection filtering based on an approximate representation of characteristic-waveform
of qP-waves was suggested to suppress undesired 
qSV-wave energy at each output time step \cite[]{zhang:2009}.
But qS-wave artifacts still remain and
 qP-wave amplitudes may be not correctly restored due to the
 approximation introduced in the used wave equation. To avoid the qSV-wave energy completely,
different approaches have recently been proposed to model the pure qP-wave
 propagation in VTI and TTI media. The optimized separable approximation \cite[]{liu:2009, zhang.zhang:2009,
 du:2010}, the pseudo-analytical method \cite[]{etgen:2009}, the low-rank
 approximation \cite[]{fomel:2013}, the Fourier finite-difference method \cite[]{song.fomel:2011} and the
 rapid expansion method \cite[]{pestana.stoffa:2010} belong to this category.

 In fact, anisotropic phenomena are especially noticeable in shear and mode-converted wavefields.
 Therefore, modeling of anisotropic shear waves
may be important both on theoretical and practical aspects. \cite{liu:2009}
 factorized the pseudo-acoustic VTI dispersion relation
 and obtained two pseudo-partial differential (PPD) equations, of which the qP-wave equation
is well-posed for any value of the anisotropic parameters, but the qSV-wave
 equation becomes well-posed only when the condition $\epsilon > \delta$ is satisfied. These PPD equations are very hard
 to solve in heterogeneous media unless further approximations are introduced \cite[]{liu:2009,
 chu:2011} or recently developed FFT-based approaches are used \cite[]{pestana:2011,song.fomel:2011,fomel:2013}.
Note that some of the above efforts to model pure-mode wavefields suffer from accuracy
 loss more or less due to the approximations to the phase velocities or dispersion relations. Furthermore,
these pure-mode propagators only consider the phase term in wave propagation, so they are appropriate
 for seismic migration but not necessarily for accurate seismic modeling, which may require taking account 
of amplitude effects and other elastic phenomena such as mode conversion.

In kinematics, there are various forms equivalent to the original first- or second-order elastic wave equations.
Mathematically, analysis of the dispersion relation as matrix eigenvalue system allows one to generate equivalent
coupled linear second-order systems by similarity transformations \cite[]{fowler:2010}.
Accordingly, \cite{kang.cheng:2011a} proposed new coupled second-order systems for both
qP- and qS-waves
 in TI media by applying specified similarity transformations to the Christoffel equation.
 Their coupled system for qP-waves represents dominantly
the energy propagation of scalar qP-waves while that for qSV-waves 
propagates dominantly the scalar qSV-wave
 energy. However, each of the two systems still contains
 relatively weak residual energy of the other mode. 
\cite{cheng.kang:2012} and \cite{kang.cheng:2012} 
 called such coupled systems ``pseudo-pure-mode wave equations'' and further proposed an approach
to get separated qP- or qS-wave data from the pseudo-pure-mode wavefields in general anisotropic media.
In the two articles of this series, we demonstrate how to simulate propagation of separated wave-modes based on 
a new simplified description of wave propagation in general anisotropic media.
We shall focus on qP- and qS-waves in each article separately.

The first paper is structured as follows: First, we revisit the plane-wave analysis of the full elastic anisotropic
 wave equation. Then we introduce a similarity transformation to the Christoffel equation required to
 derive the pseudo-pure-mode qP-wave equation, and give the simplified forms under pseudo-acoustic or/and isotropic
 approximations to illustrate the physical interpretation.
 After that, we discuss how to obtain separated qP-wave
 data from the extrapolated wavefields coupled with residual qS-waves.
Finally, we show numerical examples
to demonstrate the features and advantages of our approach in wavefield modeling and RTM in 
TI and orthorhombic media.

\section{PSEUDO-PURE-MODE COUPLED SYSTEM FOR qP-WAVES}

\subsection{Plane-wave analysis of the elastic wave equation}
Vector and component notations are used alternatively throughout the paper. The wave equation in general
 heterogeneous anisotropic media can be expressed as \cite[]{carcione:2001},
\begin{equation}
\label{eq:elastic}
\rho\frac{\partial^2\mathbf{u}}{\partial t^2} = [{\bigtriangledown}{\mathbf{C}{\bigtriangledown}^{T}}]\mathbf{u} + \mathbf{f},
\end{equation}
where $\mathbf{u}=(u_x,u_y,u_z)^{T}$ is the particle displacement vector, $\mathbf{f}=(f_x,f_y,f_z)^{T}$ represents
 the force term, $\rho$ is the density, $\mathbf{C}$ is the matrix representing the stiffness tensor in a
 two-index notation called the “Voigt recipe”, and the symmetric gradient operator has
 the following matrix representation:
\begin{equation}
\label{eq:grad}
\tensor{\bigtriangledown} =
\begin{pmatrix}\frac{\partial}{\partial x} &0 &0 &0& \frac{\partial}{\partial z} & \frac{\partial}{\partial y} \cr
0& \frac{\partial}{\partial y} &0 & \frac{\partial}{\partial z}  &0 & \frac{\partial}{\partial x} \cr
0& 0& \frac{\partial}{\partial z} & \frac{\partial}{\partial y} & \frac{\partial}{\partial x} &0 \end{pmatrix}.
\end{equation}
\new{Assuming that the material properties vary sufficiently slowly so that spatial derivatives of the stiffnesses 
can be ignored, e}\old{E}quation 1 can be \old{rewritten} \new{simplified} as
\begin{equation}
\label{eq:elastic1}
\rho\frac{\partial^2\mathbf{u}}{\partial t^2} = \Gamma\mathbf{u} + \mathbf{f},
\end{equation}
where $\Gamma$ is the $3\times3$ symmetric Christoffel differential-operator matrix, of which the elements
 are given for locally smooth media as follows \cite[]{auld:1973},
\begin{equation}
\label{eq:gama}
\begin{split}
\Gamma_{11} &= 
              C_{11}\frac{\partial^2}{\partial x^2}
            + C_{66}\frac{\partial^2}{\partial y^2}
            + C_{55}\frac{\partial^2}{\partial z^2} 
            + 2C_{56}\frac{\partial^2}{{\partial y}{\partial z}} 
            + 2C_{15}\frac{\partial^2}{{\partial x}{\partial z}}
            + 2C_{16}\frac{\partial^2}{{\partial x}{\partial y}}, \\
\Gamma_{22} &= 
              C_{66}\frac{\partial^2}{\partial x^2}
            + C_{22}\frac{\partial^2}{\partial y^2}
            + C_{44}\frac{\partial^2}{\partial z^2}
            + 2C_{24}\frac{\partial^2}{{\partial y}{\partial z}}
            + 2C_{46}\frac{\partial^2}{{\partial x}{\partial z}}
            + 2C_{26}\frac{\partial^2}{{\partial x}{\partial y}}, \\
\Gamma_{33} &= 
              C_{55}\frac{\partial^2}{\partial x^2}
            + C_{44}\frac{\partial^2}{\partial y^2}
            + C_{33}\frac{\partial^2}{\partial z^2}
            + 2C_{34}\frac{\partial^2}{{\partial y}{\partial z}}
            + 2C_{35}\frac{\partial^2}{{\partial x}{\partial z}}
            + 2C_{45}\frac{\partial^2}{{\partial x}{\partial y}}, \\
\Gamma_{23} &= 
              C_{56}\frac{\partial^2}{\partial x^2}
            + C_{24}\frac{\partial^2}{\partial y^2}
            + C_{34}\frac{\partial^2}{\partial z^2}
            +(C_{44}+C_{23})\frac{\partial^2}{{\partial y}{\partial z}}
            +(C_{36}+C_{45})\frac{\partial^2}{{\partial x}{\partial z}} \\
            &+(C_{25}+C_{46})\frac{\partial^2}{{\partial x}{\partial y}}, \\
\Gamma_{13} &= 
              C_{15}\frac{\partial^2}{\partial x^2}
            + C_{46}\frac{\partial^2}{\partial y^2}
            + C_{35}\frac{\partial^2}{\partial z^2}
            +(C_{45}+C_{36})\frac{\partial^2}{{\partial y}{\partial z}}
            +(C_{13}+C_{55})\frac{\partial^2}{{\partial x}{\partial z}} \\
            &+(C_{14}+C_{56})\frac{\partial^2}{{\partial x}{\partial y}}, \\
\Gamma_{12} &= 
              C_{16}\frac{\partial^2}{\partial x^2}
            + C_{26}\frac{\partial^2}{\partial y^2}
            + C_{45}\frac{\partial^2}{\partial z^2} 
            +(C_{46}+C_{25})\frac{\partial^2}{{\partial y}{\partial z}}
            +(C_{14}+C_{56})\frac{\partial^2}{{\partial x}{\partial z}} \\
            &+(C_{12}+C_{66})\frac{\partial^2}{{\partial x}{\partial y}}.
\end{split}
\end{equation}
For the most important types of seismic anisotropy such as transverse isotropy and \old{orhtorhombic} \new{orthorhombic} anisotropy,
 some terms in equation~\ref{eq:gama} vanish
because the corresponding stiffness coefficients become zeros.

Neglecting the source term, a plane-wave analysis of the elastic anisotropic wave equation yields the 
Christoffel equation,
\begin{equation}
\label{eq:chris1}
\widetilde{\mathbf{\Gamma}}\widetilde{\mathbf{u}} = \rho{\omega}^2\widetilde{\mathbf{u}},
\end{equation}
or
\begin{equation}
\label{eq:chris2}
(\widetilde{\mathbf{\Gamma}} - \rho{\omega}^2\mathbf{I})\widetilde{\mathbf{u}} = \mathbf{0},
\end{equation}
where $\omega$ is the frequency, $\widetilde{\mathbf{u}}=(\widetilde{u}_x,\widetilde{u}_y,\widetilde{u}_z)^{T}$
 is the wavefield in Fourier domain, $\widetilde{\Gamma} = \widetilde{\mathbf{L}}\mathbf{C}\widetilde{\mathbf{L}}^{T}$ 
is the symmetric Christoffel matrix, $\mathbf{I}$ is a $3\times3$ identity matrix.
To support the sign notation in equations~\ref{eq:chris1} and \ref{eq:chris2}, 
we remove the imaginary unit $i$ of the wavenumber-domain counterpart of the gradient operator
 $\tensor{\bigtriangledown}$ and thus express matrix $\widetilde{\mathbf{L}}$ as: 
\begin{equation}
\label{eq:wavenumber}
\widetilde{\mathbf{L}}=
\begin{pmatrix}k_x & 0 &0 &0 & k_z & k_y \cr
         0 & k_y & 0 & k_z &0 & k_x \cr
         0 & 0 & k_z & k_y & k_x &0\end{pmatrix}.
\end{equation}
Setting the determinant of 
$\widetilde{\mathbf{\Gamma}} - \rho{\omega}^2\mathbf{I}$ in equation 6 to zero gives the
 characteristic equation, and expanding that determinant gives the (angular) dispersion relation. For a given
 spatial direction specified by a wave vector $\mathbf{k} = (k_x, k_y, k_z)^{T}$, the characteristic equation
 poses a standard $3\times3$ eigenvalue problem. The three eigenvalues correspond to the phase velocities of
 the qP-wave and two qS waves. Inserting one of the eigenvalues back into the Christoffel equation gives
 ratios of the components of $\mathbf{\widetilde{u}}$, from which the polarization or displacement direction
 can be determined for the given wave mode. In general, these directions are neither parallel nor
 perpendicular to the wave vector, and depend on the local material parameters for the anisotropic
 medium. For a given wave vector or slowness direction, the polarization vectors of the three wave modes are
 always mutually orthogonal.

Applying an inverse Fourier transform to the dispersion relation yields a high-order PDE in time and space
 and contains mixed space and time derivatives. Setting the shear velocity along the axis of symmetry to zero
while using Thomsen's parameter notation yields the pseudo-acoustic dispersion relation and wave equation
 in VTI media \cite[]{alkhalifah:2000}. Most published methods instead
 have used coupled PDEs (derived from the pseudo-acoustic dispersion relation) that are only second-order in time and
 eliminate the mixed space-time derivatives, e.g., \cite{zhou:2006eage}. Many kinematically equivalent
 coupled second-order systems can be generated from the dispersion relation
 by similarity transformations \cite[]{fowler:2010}. In the next section, we present a particular similarity
 transformation to the Christoffel equation in order to derive a minimal second-order coupled system,
 which is helpful for simulating propagation of separated qP-waves in anisotropic media.

\subsection{Pseudo-pure-mode qP-wave equation}
To describe propagation of separated qP-waves in anisotropic media, we first revisit the classical wave mode
 separation theory. In isotropic media, scalar P-wave can be separated from the extrapolated
 vector wavefield $\mathbf{u}$ by applying a divergence operation: $P = \bigtriangledown\cdot{\mathbf{u}}$.
 In the wavenumber domain,
this can be equivalently expressed as a dot product that essentially projects the wavefield
	$\widetilde{\mathbf{u}}$ onto the wave vector $\mathbf{k}$, i.e.,
\begin{equation}
\label{eq:PSep}
\widetilde{P} = i\mathbf{k}\cdot{\widetilde{\mathbf{u}}},
\end{equation}
Similarly, for an anisotropic medium, scalar qP-waves can be separated
by projecting the vector wavefields onto the true polarization
 directions of qP-waves by
\cite[]{dellinger.etgen:1990},
\begin{equation}
\label{eq:qPSep}
q\widetilde{P} = i\mathbf{a_{p}}\cdot{\widetilde{\mathbf{u}}},
\end{equation}
where $\mathbf{a}_{p}=(a_{px},a_{py},a_{pz})^{T}$ represents the polarization vector for qP-waves.
For heterogeneous models, this scalar projection can be performed using
nonstationary spatial filtering depending on local material parameters \cite[]{yan.sava:2009}.

To provide more flexibility for characterizing wave propagation in anisotropic media, we suggest to split
the one-step projection into two steps, of which the first step
implicitly implements partail wave mode separation (like in equation~\ref{eq:PSep}) during wavefield
extrapolation with a transformed wave equation, while the second step is designed to correct the
projection deviation implied by equations~\ref{eq:PSep} and \ref{eq:qPSep}.
We achieve this on the base of the following observations:
 the difference of the polarization between an ordinary anisotropic medium and its isotropic reference 
at a given wave vector direction is usually
 small, though exceptions are possible \cite[]{thomsen:1986, tsvankin.chesnokov:1990}; The wave vector can be
taken as the isotropic reference of the polarization vector for qP-waves; It is a material-independent
operation to project the elastic wavefield onto the wave vector.

Therefore, we introduce a similarity transformation to the Christoffel matrix, i.e.,
\begin{equation}
\label{eq:tansChrisM}
\widetilde{\overline{\mathbf{\Gamma}}} = \mathbf{M_p}\widetilde{\mathbf{\Gamma}}\mathbf{M_p}^{-1},
\end{equation}
with a invertible $3\times3$ matrix $\mathbf{M}_{p}$ related to the wave vector:
\begin{equation}
\label{eq:tansM}
\mathbf{M_p}=
\begin{pmatrix}i{k_x} & 0 &0 \cr
         0 & i{k_y} &0 \cr
         0 & 0 & i{k_z}\end{pmatrix}.
\end{equation}
Accordingly, we derive an equivalent Christoffel equation,
\begin{equation}
\label{eq:tansChris}
\widetilde{\overline{\mathbf{\Gamma}}}\widetilde{\overline{\mathbf{u}}} = \rho{\omega}^2\widetilde{\overline{\mathbf{u}}},
\end{equation}
for a transformed wavefield:
\begin{equation}
\label{eq:similarT}
\widetilde{\overline{\mathbf{u}}} = \mathbf{M_p}\widetilde{\mathbf{u}}.
\end{equation}
The above similarity transformation does not change the eigeinvalues of the Christoffel matrix and thus 
introduces no kinematic errors for the wavefields. By the way, we can obtain 
the same transformed Christoffel equation if matrix $\mathbf{M}_{p}$ is constructed using the normalized wavenumbers
to ensure all spatial frequencies are uniformly scaled.
For a locally smooth medium, applying an inverse Fourier transform to
equation~\ref{eq:tansChris}, we obtain a coupled
 linear second-order system kinematically equivalent to the original elastic wave equation:
\begin{equation}
\label{eq:tansElastic}
\rho\frac{\partial^2\overline{\mathbf{u}}}{\partial t^2} = \overline{\mathbf{\Gamma}}\overline{\mathbf{u}},
\end{equation}
where $\overline{\mathbf{u}}$ represents the time-space domain wavefields, and $\overline{\mathbf{\Gamma}}$
 represents the Christoffel differential-operator matrix after the similarity transformation.

For the transformed elastic wavefield in the wavenumber-domain, we have
\begin{equation}
\label{eq:sumKdomain}
\widetilde{\overline{u}} = \widetilde{\overline{u}}_x + \widetilde{\overline{u}}_y + \widetilde{\overline{u}}_z
             = i\mathbf{k}\cdot{\widetilde{\mathbf{u}}}.
\end{equation}
And in space-domain, we also have
\begin{equation}
\label{eq:sum}
\overline{u} = \overline{u}_x + \overline{u}_y + \overline{u}_z
             = \bigtriangledown\cdot{\mathbf{u}},
\end{equation}
with
\begin{equation}
\label{eq:Deriv}
\overline{u}_x = \frac{\partial u_x}{\partial x},\qquad \overline{u}_y = \frac{\partial u_y}{\partial y},\qquad \mbox{and} \qquad \overline{u}_z=\frac{\partial u_z}{\partial z}.
\end{equation}
These imply that the new wavefield components essentially represent the spatial derivatives of the original
components of the displacement wavefield, and the transformation (equation~\ref{eq:similarT})
plus the summation of the transformed wavefield components (like in equation~\ref{eq:sumKdomain} or~\ref{eq:sum})
essentially finishes a scalar projection of the displacement wavefield onto the wave vector. 
For isotropic media, such a projection directly produces scalar P-wave data. In an anisotropic medium,
 however, only a partial wave-mode separation is achieved becuase there is usually a direction deviation 
between the wave vector and the polarization vector of qP-wave.
Generally, this deviation turns out to be small and
its maximum value rarely exceeds $20^\circ$ for typical anisotropic earth media\mbox{\cite[]{psencik:1998}}.
Because of the orthognality of qP- and qS-wave polarizations, the projection deviations of qP-waves are generally
far less than those of the qSV-waves when the elastic wavefields are projected onto the isotropic references
of the qP-wave's polarization vectors.
As demonstrated in the synthetic examples of various symmetry and strength of anisotropy, 
	the scalar wavefield $\overline{u}$ represents dominantly
the energy of qP-waves but contains some weak residual qS-waves.
This is why we call the coupled system (equation~\ref{eq:tansElastic}) a pseudo-pure-mode wave equation
for qP-wave in anisotropic media.

Substituting the corresponding stiffness matrix into the above derivations, we get the extended expression of
pseudo-pure-mode qP-wave equation for any anisotropic media.
As demonstrated in Appendix A, pseudo-pure-mode qP-wave equation in vertical TI and orthorhombic media can be expressed as
\begin{equation}
\label{eq:pseudo}
\begin{split}
\rho\frac{\partial^2\overline{u}_x}{\partial t^2} &= C_{11}\frac{\partial^2{\overline{u}_x}}{\partial x^2}
                                                  + C_{66}\frac{\partial^2{\overline{u}_x}}{\partial y^2}
                                                  + C_{55}\frac{\partial^2{\overline{u}_x}}{\partial z^2}
                                                  +(C_{12}+C_{66})\frac{\partial^2{\overline{u}_y}}{\partial x^2}
                                                  +(C_{13}+C_{55})\frac{\partial^2{\overline{u}_z}}{\partial x^2}, \\
\rho\frac{\partial^2\overline{u}_y}{\partial t^2} &= C_{66}\frac{\partial^2{\overline{u}_y}}{\partial x^2}
                                                  + C_{22}\frac{\partial^2{\overline{u}_y}}{\partial y^2}
                                                  + C_{44}\frac{\partial^2{\overline{u}_y}}{\partial z^2}
                                                  +(C_{12}+C_{66})\frac{\partial^2{\overline{u}_x}}{\partial y^2}
                                                  +(C_{23}+C_{44})\frac{\partial^2{\overline{u}_z}}{\partial y^2}, \\
\rho\frac{\partial^2\overline{u}_z}{\partial t^2} &= C_{55}\frac{\partial^2{\overline{u}_z}}{\partial x^2}
                                                  + C_{44}\frac{\partial^2{\overline{u}_z}}{\partial y^2}
                                                  + C_{33}\frac{\partial^2{\overline{u}_z}}{\partial z^2}
                                                  +(C_{13}+C_{55})\frac{\partial^2{\overline{u}_x}}{\partial z^2}
                                                  +(C_{23}+C_{44})\frac{\partial^2{\overline{u}_y}}{\partial z^2}.
\end{split}
\end{equation}
Note that, unlike the original elastic wave equation, pseudo-pure-mode wave equation dose not contain mixed partial
derivatives.
This is a good news because it takes more computational cost to compute the mixed partial derivatives
using \new{a} finite-difference algorithm with required accuracy.
In the forthcoming text, we focus on demonstration of pseudo-pure-mode qP-wave equations for TI media while
briefly supplement similar derivation for orthorhombic media in Appendix B.

\subsubsection{Pseudo-pure-mode qP-wave equation in VTI media}
For a VTI medium, there are only five independent parameters: $C_{11}$, $C_{33}$, $C_{44}$, $C_{66}$ and $C_{13}$, 
with $C_{12}=C_{11}-2C_{66}$, $C_{22}=C_{11}$, $C_{23}=C_{13}$ and $C_{55}=C_{44}$.
 So we rewrite equation~\ref{eq:pseudo} as,
\begin{equation}
\label{eq:pseudoVTI0}
\begin{split}
\rho\frac{\partial^2{\overline{u}_x}}{\partial t^2} &= C_{11}\frac{\partial^2{\overline{u}_x}}{\partial x^2}
                                         + C_{66}\frac{\partial^2{\overline{u}_x}}{\partial y^2}
                                         + C_{44}\frac{\partial^2{\overline{u}_x}}{\partial z^2}
                                         +(C_{11}-C_{66})\frac{\partial^2{\overline{u}_y}}{\partial x^2}
                                         +(C_{13}+C_{44})\frac{\partial^2{\overline{u}_z}}{\partial x^2}, \\
\rho\frac{\partial^2{\overline{u}_y}}{\partial t^2} &= C_{66}\frac{\partial^2{\overline{u}_y}}{\partial x^2}
                                         + C_{11}\frac{\partial^2{\overline{u}_y}}{\partial y^2}
                                         + C_{44}\frac{\partial^2{\overline{u}_y}}{\partial z^2} 
                                         +(C_{11}-C_{66})\frac{\partial^2{\overline{u}_x}}{\partial y^2}
                                         +(C_{13}+C_{44})\frac{\partial^2{\overline{u}_z}}{\partial y^2}, \\
\rho\frac{\partial^2{\overline{u}_z}}{\partial t^2} &= C_{44}\frac{\partial^2{\overline{u}_z}}{\partial x^2}
                                         + C_{44}\frac{\partial^2{\overline{u}_z}}{\partial y^2}
                                         + C_{33}\frac{\partial^2{\overline{u}_z}}{\partial z^2} 
                                         +(C_{13}+C_{44})\frac{\partial^2{\overline{u}_x}}{\partial z^2}
                                         +(C_{13}+C_{44})\frac{\partial^2{\overline{u}_y}}{\partial z^2}.
\end{split}
\end{equation}
Since a TI material has cylindrical symmetry in its elastic properties, it is safe to sum the first two equations
 in equation~\ref{eq:pseudoVTI0} to yield a simplified form for wavefield modeling and RTM, namely
\begin{equation}
\label{eq:pseudoVTI1}
\begin{split}
\rho\frac{\partial^2{\overline{u}_{xy}}}{\partial t^2} &=
                   C_{11}(\frac{\partial^2}{\partial x^2}+\frac{\partial^2}{\partial y^2}){\overline{u}_{xy}}
                  + C_{44}\frac{\partial^2{\overline{u}_{xy}}}{\partial z^2}
                  +(C_{13}+C_{44})(\frac{\partial^2}{\partial x^2}+\frac{\partial^2}{\partial y^2}){\overline{u}_z}, \\
\rho\frac{\partial^2{\overline{u}_z}}{\partial t^2} &= 
                   C_{44}(\frac{\partial^2}{\partial x^2}+\frac{\partial^2}{\partial y^2}){\overline{u}_z}
                  + C_{33}\frac{\partial^2{\overline{u}_z}}{\partial z^2} 
                  +(C_{13}+C_{44})\frac{\partial^2{\overline{u}_{xy}}}{\partial z^2},
\end{split}
\end{equation}
where $\overline{u}_{xy}=\overline{u}_{x}+\overline{u}_{y}$ represents the sum of the two horizontal components.
Pure SH-waves horizontally polarize in the isotropic planes of VTI media
with the polarization given by $(-k_{y}, k_{x}, 0)$, which implies $ik_{x}\widetilde{u}_{x}+ik_{y}\widetilde{u}_{y}=0$,
i.e., $\overline{u}_{xy}=0$, for the SH-wave.
Therefore, the above partial summation (after the first-step projection) completes divergence operation and removes the SH-waves from
the three-component pseudo-pure-mode qP-wave fields.
As a result, there are no terms related to $C_{66}$ any more in equation~\ref{eq:pseudoVTI1}.
Compared with original elastic wave equation, equation~\ref{eq:pseudoVTI1} further reduces the compuational
costs for 3D wavefield modeling and RTM for VTI media.

Applying the Thomsen notation \cite[]{thomsen:1986},
\begin{equation}
\label{eq:ThomsenVTI}
\begin{split}
C_{11} &= (1+2\epsilon)\rho{v_{p0}^2}, \\
C_{33} &= \rho{v_{p0}^2}, \\
C_{44} &= \rho{v_{s0}^2}, \\
(C_{13}+C_{44})^2 &= \rho^2({v_{p0}^2}-{v_{s0}^2})({v_{pn}^2}-{v_{s0}^2}),
\end{split}
\end{equation}
the pseudo-pure-mode qP-wave equation can be expressed as,
\begin{equation}
\label{eq:pseudoVTIxy}
\begin{split}
\frac{\partial^2\overline{u}_{xy}}{\partial t^2} & =
 {v_{px}^2}(\frac{\partial^2}{\partial x^2}+\frac{\partial^2}{\partial y^2}){\overline{u}_{xy}}
+{v_{s0}^2}\frac{\partial^2{\overline{u}_{xy}}}{\partial z^2}
+ \sqrt{({v_{p0}^2}-{v_{s0}^2})({v_{pn}^2}-{v_{s0}^2})}
(\frac{\partial^2}{\partial x^2}+\frac{\partial^2}{\partial y^2}){\overline{u}_z}, \\
\frac{\partial^2\overline{u}_{z}}{\partial t^2} & =
{v_{s0}^2}(\frac{\partial^2}{\partial x^2}+\frac{\partial^2}{\partial y^2}){\overline{u}_{z}}
+ {v_{p0}^2}\frac{\partial^2{\overline{u}_z}}{\partial z^2}
+ \sqrt{({v_{p0}^2}-{v_{s0}^2})({v_{pn}^2}-{v_{s0}^2}) }
  \frac{\partial^2\overline{u}_{xy}}{\partial z^2},
\end{split}
\end{equation}
where $v_{p0}$ and $v_{s0}$ represent the vertical velocities of qP- and qSV-waves,
 $v_{pn} = v_{p0}\sqrt{1+2\delta}$ represents the interval
NMO velocity, $v_{px} = v_{p0}\sqrt{1+2\epsilon}$ 
represents the horizontal velocity of qP-waves,
$\delta$ and $\epsilon$ are the other two Thomsen coefficients.
 Unlike other coupled second-order systems derived from the dispersion relation 
of VTI media \cite[]{zhou:2006eage}, the wavefield components in
 equations~\ref{eq:pseudoVTI1} and \ref{eq:pseudoVTIxy}
have clear physical meaning and their summation automatically produces scalar wavefields dominant of qP-wave energy.
 Equation~\ref{eq:pseudoVTIxy} is also similar to a minimal coupled system (equation 30 in their paper) 
demonstrated by Fowler
 et al. (2010), except that it is now derived from a significant similarity transformation that helps to
enhance qP-waves and suppress qS-waves (after summing the transformed wavefield components). 
 This is undoubtedly useful for migration of conventional seismic data representing mainly qP-wave data.

We can also obtain a pseudo-acoustic coupled system by setting $v_{s0}=0$ in equation~\ref{eq:pseudoVTIxy}, namely:
\begin{equation}
\label{eq:acoustic}
\begin{split}
\frac{\partial^2\overline{u}_{xy}}{\partial t^2} & =
 (1+2\epsilon){v_{p0}^2}(\frac{\partial^2}{\partial x^2}+\frac{\partial^2}{\partial y^2}){\overline{u}_{xy}}
+ \sqrt{1+2\delta}{v_{p0}^2}(\frac{\partial^2}{\partial x^2}+\frac{\partial^2}{\partial y^2}){\overline{u}_z}, \\
\frac{\partial^2\overline{u}_{z}}{\partial t^2} & =
{v_{p0}^2}\frac{\partial^2{\overline{u}_z}}{\partial z^2}
+ \sqrt{1+2\delta}{v_{p0}^2}\frac{\partial^2\overline{u}_{xy}}{\partial z^2}.
\end{split}
\end{equation}
The pseudo-acoustic approximation does not significantly
 affect the kinematic signatures but may distort the reflection,
 transmission and conversion coefficients (thus the amplitudes) of waves in elastic media.

If we further apply the isotropic assumption (seting $\delta=0$ and $\epsilon=0$) and sum the two equations in
 equation~\ref{eq:acoustic}, we get the familar constant-density acoustic wave equation:
\begin{equation}
\frac{\partial^2\overline{u}}{\partial t^2} = 
{v_{p}^2}(\frac{\partial^2}{\partial x^2}+\frac{\partial^2}{\partial y^2}+\frac{\partial^2}{\partial z^2}){\overline{u}},
\end{equation}
where $\overline{u}=\overline{u}_{xy}+\overline{u}_{z}$ represents the acoustic pressure wavefield, and $v_{p}$ is
 the propagation velocity of isotropic P-wave.

\subsubsection{Pseudo-pure-mode qP-wave equation in TTI media}

In the case of transversely isotropic media with a tilted symmetry axis, the elastic tensor loses its simple
 form. Written in Voigt notation, it contains nonzero entries in all four quadrants if expressed in global
 Cartesian coordinates $\mathbf{x}=(x,y,z)$. The generalization of pseudo-pure-mode wave equation to a tilted symmetry
 axis involves no additional physics but greatly complicates the algebra. One strategy to derive the wave
 equations in TTI media is to locally rotate the coordinate system so that its third axis coincides with
 the symmetry axis, and make use of the simple form in VTI media.

We introduce a transformation to a rotated coordinate system $\widehat{\mathbf{x}}=(\widehat{x},\widehat{y},\widehat{z})$,
\begin{equation}
\widehat{\mathbf{x}}=\mathbf{R}^{T}\mathbf{x},
\end{equation}
where the rotation matrix $\mathbf{R}$ is dependent on the tilt angle $\theta$ and the azimuth $\varphi$ of the
 symmetry axis, namely,
\begin{equation}
\mathbf{R}=
\begin{pmatrix}r_{11} & r_{12} &r_{13} \cr
         r_{21} & r_{22} &r_{23} \cr
         r_{31} & r_{32} &r_{33}\end{pmatrix}
=\begin{pmatrix}\cos{\varphi} & -\sin{\varphi} &0 \cr
          \sin{\varphi} & \cos{\varphi} &0 \cr
          0 & 0 & 1\end{pmatrix}
\begin{pmatrix}\cos{\theta} & 0 & -\sin{\theta} \cr
          0 & 1 & 0 \cr
          \sin{\theta} & 0 & \cos{\theta}\end{pmatrix}.
\end{equation}
So,
\begin{equation}
\begin{split}
r_{11}&=\cos{\theta}\cos{\varphi}, \\
r_{12}&=-\sin{\varphi}, \\
r_{13}&=-\sin{\theta}\cos{\varphi}, \\
r_{21}&=\cos{\theta}\sin{\varphi}, \\
r_{22}&=\cos{\varphi}, \\
r_{23}&=-\sin{\theta}\sin{\varphi}, \\
r_{31}&=\sin{\theta}, \\
r_{32}&=0, \\
r_{33}&=\cos{\theta}.
\end{split}
\end{equation}
Assuming that the rotation operator $\mathbf{R}$ varies slowly so that its spatial derivatives can be ignored,
the second-order differential operators in the rotated coordinate system aligned with the symmetry axis are given as:
\begin{equation}
\label{eq:difoper}
\begin{split}
\frac{\partial^2}{\partial{\widehat{x}^2}} &= {r_{11}^2}\frac{\partial^2}{\partial x^2}
                                           + {r_{21}^2}\frac{\partial^2}{\partial y^2}
                                           + {r_{31}^2}\frac{\partial^2}{\partial z^2}
                                           + 2r_{11}r_{21}\frac{\partial^2}{{\partial x}{\partial y}}
                                           + 2r_{11}r_{31}\frac{\partial^2}{{\partial x}{\partial z}}
                                           + 2r_{21}r_{31}\frac{\partial^2}{{\partial y}{\partial z}}, \\
\frac{\partial^2}{\partial{\widehat{y}^2}} &= {r_{12}^2}\frac{\partial^2}{\partial x^2}
                                           + {r_{22}^2}\frac{\partial^2}{\partial y^2}
                                           + {r_{32}^2}\frac{\partial^2}{\partial z^2}
                                           + 2r_{12}r_{22}\frac{\partial^2}{{\partial x}{\partial y}}
                                           + 2r_{12}r_{32}\frac{\partial^2}{{\partial x}{\partial z}}
                                           + 2r_{22}r_{32}\frac{\partial^2}{{\partial y}{\partial z}}, \\
\frac{\partial^2}{\partial{\widehat{z}^2}} &= {r_{13}^2}\frac{\partial^2}{\partial x^2}
                                           + {r_{23}^2}\frac{\partial^2}{\partial y^2}
                                           + {r_{33}^2}\frac{\partial^2}{\partial z^2}
                                           + 2r_{13}r_{23}\frac{\partial^2}{{\partial x}{\partial y}}
                                           + 2r_{13}r_{33}\frac{\partial^2}{{\partial x}{\partial z}}
                                           + 2r_{23}r_{33}\frac{\partial^2}{{\partial y}{\partial z}}.
\end{split}
\end{equation}
Substituting these differential operators into the pseudo-pure-mode qP-wave equation of VTI media 
 yields the pseudo-pure-mode qP-wave equation for TTI media in the global Cartesian coordinates.
Likewise, the pseudo-pure-mode qP-wave equation in TTI media can be further simplified by applying the pseudo-acoustic approximation.
We must mention that, the above
coordinate rotation in deriving the wave equations for TTI and tilted orthorhombic media (see Appendix B)
 should be improved to enhance numerical stability according to some significant insights provided in recent literatures  
 \cite[]{duveneck:2011,macesanu,zhang:2011,bube:2012}.

\section{CORRECTION OF PROJECTION DEVIATION TO REMOVE qS-WAVES}

According to the theory of wave mode separation in anisotropic media,
 one needs to project the elastic wavefields onto the polarization 
direction to get the separated wavefields of the given mode \cite[]{dellinger.etgen:1990}.
Mathematically, this can be implemented through a dot product of the original vector wavefields
 and the polarization vector in the wavenumber domain  \cite[]{dellinger.etgen:1990}
or applying pseudo-derivative operators
 to the vector wavefields in the space-domain \cite[]{yan.sava:2009}. 
 However, the pseudo-pure-mode qP-wave equations are derived by a similarity transformation
aiming to project the displacement wavefield onto the isotropic reference of
 qP-wave's polarization vector. A partial mode separation
 has been automatically achieved during wavefield extrapolation using the pseudo-pure-mode qP-wave equations.
For typical anisotropic earth media, thanks to the small departure of qP-wave's polarization
	direction from its isotropic reference,
	the resulting pseudo-pure-mode wavefields are dominated by qP-wave energy and contaminated by residual qS-waves.
To achieve a complete mode separation, we should further correct the projection 
deviations resulting from the differences between polarization and its isotropic reference.
In other words, we split the conventional one-step wave mode separation for anisotropic media
\cite[]{dellinger.etgen:1990,yan.sava:2009} into two steps, of which the first one is implicitly achieved
during extrapolating the pseudo-pure-mode wavefields and the second one is implemented after that
using the approach that we will present immediately. 

 Taking VTI as an example, the deviation angle $\zeta$ between the polarization and 
propagation directions has a complicated nonlinear relation
 with anisotropic parameters and the phase angle (see Appendix C).
According to its expression for weak anisotropic VTI media
 \cite[]{rommel:1994,tsvankin:2001}, it seems that the deviation is mainly affected
 by the difference between $\epsilon$ and $\delta$,  the magnitude of $\delta$ (when 
$\epsilon-\delta$ stays the same) and the ratio of vertical velocities of qP- and qS-wave,
 as well as the phase angle.
It is possible to design a filtering algorithm 
to suppress the residual qS-waves using the deviation angle given under the assumption of weak anisotropy. 
To completely remove the residual qS-waves and correctly separate the qP-waves for arbitrary anisotropy, we 
propose an accurate correction approach according to the deviation between polarization and wave vectors.

Considering equations~\ref{eq:PSep}, \ref{eq:qPSep}, \ref{eq:similarT}, and \ref{eq:sumKdomain}, we
first decompose the polarization vector of qP-wave $\mathbf{a}_{p}$ 
 as follows:
\begin{equation}
\label{eq:polDemp}
\mathbf{a}_{p}=\mathbf{E}_{p}\mathbf{k},
\end{equation}
where the deviation operator satisfies,
\begin{equation}
\label{eq:devM}
\mathbf{E}_{p}=
\begin{pmatrix}\frac{a_{px}}{k_{x}} &0 &0 \cr
          0 & \frac{a_{py}}{k_{y}} &0 \cr
          0 & 0 & \frac{a_{pz}}{k_{z}}\end{pmatrix}. 
\end{equation}
This matrix can be constructed once the qP-wave polarization directions
are determined based on the local medium properties at a grid point.
For TI media, there are analytical expressions for the qP-wave polarization vectors \cite[]{dellinger.thesis}. 
For other anisotropic media with lower symmetry (such as orthorhombic media), we have to numerically compute
the polarization vectors using the Christoffel equation. 

Then we correct the pseudo-pure-mode qP-wave fields
 $\widetilde{\overline{\mathbf{u}}}=(\widetilde{\overline{u}}_{x},
 \widetilde{\overline{u}}_{y}, \widetilde{\overline{u}}_{z})^{T}$ using a wavenumber-domain filtering
based on the deviation operator:
\begin{equation} 
\label{eq:correct}
\widetilde{\mathbf{u}}_{p}=\mathbf{E}_{p}\cdot{\widetilde{\overline{\mathbf{u}}}},
\end{equation}
and finally extract the scalar qP-wave data using
\begin{equation}
\label{eq:sepP}
u_{p}=u_{px}+u_{py}+u_{pz}
\end{equation}
after 3D inverse Fourier transforms. 
Here, the magnitude of the deviation operator for a certain wavenumber $k=\sqrt{k^2_{x}+k^2_{y}+k^2_{z}}$ is a constant becuase
this operator is computed by using the normalized wave and polarization vectors in equation~\ref{eq:devM}. This ensures that
for a certain wavenumber, the separated qP-waves are uniformly scaled. More important, this correction step thoroughly
removes the residual qS-wave energy.

In heterogeneous anisotropic media, the polarization directions and thus the deviation operators vary
 spatially, depending on the local material parameters. To account for spatial variability,
 we propose an equivalent expression to equations~\ref{eq:correct} and \ref{eq:sepP} as a nonstationary filtering 
 in the space domain at each location,
\begin{equation}
u_{p}=E_{px}(\overline{u}_{x})+E_{py}(\overline{u}_{y})+E_{pz}(\overline{u}_{z})
\end{equation}
where the pseudo-derivative operators $E_{px}(\cdot)$, $E_{py}(\cdot)$, and $E_{pz}(\cdot)$ represent
 the inverse Fourier transforms of the diagonal elements in the deviation matrix $\mathbf{E}_{p}$.

\inputdir{homovti.eta0.5}
Figure 1 displays the wavenumber-domain operators of projection onto isotropic (reference) and anisotropic polarization vectors 
(namely $\mathbf{k}$ and $\mathbf{a}_{p}$)
as well as the corresponding deviation operator $\mathbf{E}_{p}$
for a 2D homogeneous VTI medium with $v_{p0}=3000m/s$, $v_{s0}=1500m/s$, $\epsilon=0.25$ 
and $\delta=-0.25$. Note that $\mathbf{E}_{p}$ is not simply the difference between
 $\mathbf{k}$ and $\mathbf{a}_{p}$,
and $\mathbf{E}_{p}$ becomes the identity operator in case of an isotropic medium.
In the space-domain, projecting onto isotropic polarization directions
is equivalent to a divergence operation using partial derivative operators, 
while projection onto polarization directions of qP-waves use operators that have 
the character of pseudo-derivative operators, due to anisotropy (see Figure 2).
Figure 3 shows that the variation of the anisotropy changes the deviation operators greatly.
 The weaker the anisotropy, the more compact the deviation operators appear. The observation is 
basically consistent with the equation of polarization deviation angle for VTI media with weak anisotropy.
 The exact pseudo-derivative operators are very long series in 
the discretized space domain. Generally, the far ends of these operators have 
ignorable values even for strong anisotropy. Therefore, in practice, we could truncate the operators
 to make the spatial filters short and computationally efficient.

This procedure to separate qP-waves, although accurate, is computationally expensive,
 especially in 3D heterogeneous media. Like the computational problem in conventional wave mode separation
 from the anisotropic elastic wavefields \cite[]{yan.sava:2009,yan:2012}, the spatial filtering to separate 
qP-waves is significantly more expensive than extrapolating the pseudo-pure-mode wavefields.
In practice, we find that it is not necessary to apply the filtering at every time step.
 A larger time interval is allowed 
to save costs enormously, especially for RTM of multi-shot seismic data. 
According to our experiments, there is little difference between the two migrated images when the filtering is applied
at every one and two time step (if only the filtered wavefields are used in the imaging procedure), 
although about three-forth of the original computational cost
are saved for the filtering in the latter case.
Moreover, filtering at every three time step still produces an acceptable migrated image.
We may further improve the efficiency of the filtering procedure
by using the algorithm that resembles the phase-shift plus interpolation (PSPI) scheme recently used in
anisotropic wave mode separation \cite[]{yan.sava:2011}.
Alternatively, we may greatly reduce the compuational cost but guarantee the accuracy 
using the mixed (space-wavenumber) domain filtering algorithm 
based on low-rank approximation \cite[]{cheng.fomel:2013}.

\multiplot{6}{adxNT,apxNT,apvxNT,adzNT,apzNT,apvzNT}{width=0.25\textwidth}
{
Normalized wavenumber-domain operators of projection onto isotropic (reference) and anisotropic
 polarization vectors of qP-waves, and wavenumber-domain deviation operators in a 2D homogeneous VTI medium: 
$\mathbf{k}$ (left), $\mathbf{a}_{p}$ (middle) and $\mathbf{E}_{p}$ (right);
 Top: x-component, Bottom: z-component.
}
\multiplot{6}{adxx,apxx,apvxx,adzz,apzz,apvzz}{width=0.25\textwidth}
{
Space-domain operators of projecting onto isotropic (left) and anisotropic (middle) polarization vectors, and
the corresponding deviation operators (right): Top: x-component, Bottom: z-component.
Note that the operators are tapered before transforming into space-domain and the same
gain is applied to all pictures to highlight the differences among these operators.
}

\inputdir{comparison.operators}

\multiplot{10}{apvxx1,apvxx2,apvxx3,apvxx4,apvxx5,apvzz1,apvzz2,apvzz3,apvzz4,apvzz5}{width=0.15\textwidth}
{Comparison of the spatial domain deviation operators in VTI media with varied anisotropy strength:
In all cases, $v_{p0}=3000m/s$, $v_{s0}=1500m/s$, and $\epsilon$ is fixed as 0.2.
From left to right, $\delta$ is set as 0.2, 0.1, 0, -0.1, and -0.2, respectively. Top: x-components;
 Bottom: z-components. To highlight the differences,the same gain is applied to all pictures.}

In kinematics, it seems that we can extract scalar qSV-wave fields from the pseudo-pure-mode qP-wave fields
 $\overline{\mathbf{u}}$ by filtering according to the projection deviation defined by qSV-wave's polarization
and wave vector. However, unlike separation of the qP-wave mode, the large projection deviations for
 qSV-wave modes would result in significant discontinuities in the wavenumber-domain correction operators and
 strong tails extending off to infinity in the space domain. Accordingly, this reduces compactness
of the spatial filters, which prohibits applying the same truncation as for qP-wave spatial filters to
 reduce computational cost. Computational tricks such as smoothing may result in distorted and imcomplete separation.
That is why we are developing a similar approach to simulate propagation 
of separated qS-wave modes based on their own pseudo-pure-mode wave equations and the corresponding
projection deviation corrections for anisotropic media \cite[]{kang.cheng:2012}.

\section{EXAMPLES}
\subsection{1. Simulating propagation of separated wave modes}

%%%%%%%%%%%%%%%%%%%%%%%%%%%%%%%%%%%%%%%%%%%%%%%%%%%%%%%%%%%%%%%%%%%%%%%%%%%%%%%%%%%%%%%%%%%%%%%%%%%%%
\subsubsection{1.1 Homogeneous VTI model}
\inputdir{homovti.eta0.05}

For comparison, we first appply the original anisotropic elastic wave equation
 to synthesize wavefields in a homogeneous VTI medium with weak anisotropy, in which
$v_{p0}=3000 m/s$, $v_{s0}=1500 m/s$, $\epsilon=0.1$, and $\delta=0.05$. 
Figure 4a and 4b display the horizontal and vertical components of the displacement wavefields at 0.3 s.
 Then we try to simulate propagation of separated wave modes using the pseudo-pure-mode qP-wave equation 
(equation~\ref{eq:pseudoVTIxy} in its 2D form).
 Figure 4c and 4d display the two components of the pseudo-pure-mode qP-wave fields, and Figure 4e
 displays their summation, i.e., the pseudo-pure-mode scalar qP-wave fields with weak residual qSV-wave energy. 
 Compared with the theoretical wavefront curves (see Figure 4f) calculated
on the base of group velocities
 and angles, pseudo-pure-mode scalar qP-wave fields have correct kinematics for both qP- and qSV-waves.
We finally remove residual qSV-waves and get completely separated scalar qP-wave fields by 
 applying the filtering to correct the projection deviation (Figure 4g).

\multiplot{7}{Elasticx,Elasticz,PseudoPurePx,PseudoPurePz,PseudoPureP,WF,PseudoPureSepP}{width=0.2\textwidth}
{
Synthesized wavefields in a VTI medium with weak anisotropy: (a) x- and
(b) z-components synthesized by original elastic wave equation; (c) x- and
 (d) z-components synthesized by pseudo-pure-mode qP-wave equation; (e) pseudo-pure-mode scalar qP-wave fields; 
(f) kinematics of qP- and qSV-waves; and (g) separated scalar qP-wave fields.
}

%%%%%%%%%%%%%%%%%%%%%%%%%%%%%%%%%%%%%%%%%%%%%%%%%%%%%%%%%%%%%%%%%%%%%%%%%%%%%%%%%%%%%%%%%%%%%%%%%%%%%
\inputdir{homovti.eta0.5}

Then we consider wavefield modeling in a homogeneous VTI medium with strong anisotropy,
 in which $v_{p0}=3000 m/s$, $v_{s0}=1500 m/s$, $\epsilon=0.25$, and $\delta=-0.25$.
 Figure 5 displays the wavefield snapshots at 0.3 s synthesized by using original elastic wave equation
and pseudo-pure-mode qP-wave equation respectively. Note that the pseudo-pure-mode qP-wave fields still accurately
represent the qP- and qSV-waves' kinematics. Although the residual qSV-wave energy becomes stronger when
the strength of anisotropy increases, the filtering step still removes these residual qSV-waves effectively.

\multiplot{7}{Elasticx,Elasticz,PseudoPurePx,PseudoPurePz,PseudoPureP,WF,PseudoPureSepP}{width=0.2\textwidth}
{
Synthesized wavefields in a VTI medium with strong anisotropy: (a) x- and
(b) z-components synthesized by original elastic wave equation; (c) x- and
 (d) z-components synthesized by pseudo-pure-mode qP-wave equation; (e) pseudo-pure-mode scalar qP-wave fields; 
(f) kinematics of qP- and qSV-waves; and (g) separated scalar qP-wave fields.
}

%%%%%%%%%%%%%%%%%%%%%%%%%%%%%%%%%%%%%%%%%%%%%%%%%%%%%%%%%%%%%%%%%%%%%%%%%%%%%%%%%%%%%%%%%%%%%%%%%%%%%
\subsubsection{1.2 Two-layer TI model}
\inputdir{twolayer2dti}

This example demonstrates the approach on a two-layer TI model, in which the first layer is a very
strong VTI medium with $v_{p0}=2500 m/s$, $v_{s0}=1200 m/s$, $\epsilon=0.25$, and $\delta=-0.25$, 
and the second layer is a TTI medium with $v_{p0}=3600 m/s$, $v_{s0}=1800 m/s$, 
$\epsilon=0.2$, $\delta=0.1$, and $\theta=30^{\circ}$. The horizontal interface between the two layers 
is positioned at a depth of 1.167 km.
 Figure 6a and 6d display the horizontal and vertical components of the displacement wavefields at 0.3 s.
 Using the pseudo-pure-mode qP-wave equation, we simulate equivalent wavefields on the same model.
Figure 6b and 6e display the two components of the pseudo-pure-mode qP-wave fields at the same time step.
Figure 6c and 6f display pseudo-pure-mode scalar qP-wave fields and separated qP-wave fields respectively.
Obviously, residual qSV-waves (including transmmited, reflected 
and converted qSV-waves) are effectively removed, and all transmitted, reflected as well as converted
qP-waves are accurately separated after the projection deviation correction.

\multiplot{6}{ElasticxInterf,PseudoPurePxInterf,PseudoPurePInterf,ElasticzInterf,PseudoPurePzInterf,PseudoPureSepPInterf}{width=0.25\textwidth}
{
Synthesized wavefields on a two-layer TI model with strong anisotropy in the first layer and
a tilted symmetry axis in the second layer: (a) x- and 
(d) z-components synthesized by original elastic wave equation; (b) x- and
 (e) z-components synthesized by pseudo-pure-mode qP-wave equation; 
 (c) pseudo-pure-mode scalar qP-wave fields; (f) separated scalar qP-wave fields.
}

%%%%%%%%%%%%%%%%%%%%%%%%%%%%%%%%%%%%%%%%%%%%%%%%%%%%%%%%%%%%%%%%%%%%%%%%%%%%%%%%%%%%%%%%%%%%%%%%%%%%%
\subsubsection{1.3 BP 2007 TTI model}
\inputdir{bptti2007}

Next we test the approach of simulating propagation of the separated qP-wave mode
in a complex TTI model. Figure 7 shows parameters for part of
the BP 2D TTI model. The space grid size is 12.5 m and the time step is 1 ms for 
high-order finite-difference operators. Here the vertical velocities for the qSV-wave are set as
 half of the qP-wave velocities. 
Figure 8 displays snapshots of wavefield components at the time of 1.4s
synthesized by using original elastic wave equation and pseudo-pure-mode qP-wave equation.
The two pictures at the bottom 
represent the scalar pseudo-pure-mode qP-wave and the separated qP-wave fileds, respectively.
The correction appears to remove residual qSV-waves and accurately separate qP-wave data 
including the converted qS-qP waves from
the pseudo-pure-mode wavefields in this complex model.

\multiplot{4}{vp0,epsi,del,the}{width=0.3\textwidth}
{
Partial region of the 2D BP TTI model: (a) vertical qP-wave velocity, Thomsen coefficients
 (b) $\epsilon$ and (c) $\delta$, and (d) the tilt angle $\theta$. 
}

\multiplot{6}{Elasticx,Elasticz,PseudoPurePx,PseudoPurePz,PseudoPureP,PseudoPureSepP}{width=0.3\textwidth}
{
Synthesized elastic wavefields on BP 2007 TTI model using original elastic wave equation and pseudo-pure-mode 
qP-wave equation respectively: (a) x- and 
(b) z-components synthesized by original elastic wave equation; (c) x- and
 (d) z-components synthesized by pseudo-pure-mode qP-wave equation; 
 (e) pseudo-pure-mode scalar qP-wave fields; (f) separated scalar qP-wave fields.
}

%%%%%%%%%%%%%%%%%%%%%%%%%%%%%%%%%%%%%%%%%%%%%%%%%%%%%%%%%%%%%%%%%%%%%%%%%%%%%%%%%%%%%%%%%%%%%%%%%%%%%
\subsubsection{1.4 Homogeneous 3D ORT model}

\inputdir{ort3dhomo}

Figure 9 shows an example of simulating propagation of separated qP-wave fields in a 3D homogeneous
 vertical ORT model, in which $v_{p0}=3000 m/s$,
$v_{s0}=1500 m/s$, $\delta_{1}=-0.1$, $\delta_{2}=-0.0422$, $\delta_{3}=0.125$, $\epsilon_{1}=0.2$,
$\epsilon_{2}=0.067$, $\gamma_{1}=0.1$, and $\gamma_{2}=0.047$.
The first three pictures display wavefield snapshots at 0.5s synthesized by using
 pseudo-pure-mode qP-wave equation, according to equation~\ref{eq:ort}.
As shown in Figure 9d, qP-waves again appear\old{s} to dominate the wavefields in energy when we sum the 
three wavefield components of the pseudo-pure-mode qP-wave fields.
As for TI media, we obtain completely separated qP-wave fields from the
 pseudo-pure-mode wavefields once the correction of projection deviation is finished (Figure 9e).
By the way, in all above examples, we find that the filtering to remove qSV-waves does not
require the numerical dispersion of the qS-waves to be limited. So there is no additional requirement
of the grid size for qS-wave propagation. The effects of grid dispersion for the separation of low velocity
qS-waves will be further investigated in the second article of this series.
\multiplot{5}{PseudoPurePx,PseudoPurePy,PseudoPurePz,PseudoPureP,PseudoPureSepP}{width=0.3\textwidth}
{
Synthesized wavefield snapshots in a 3D homogeneous vertical ORT medium: (a) x-, (b) y- and (c) z-component
 of the pseudo-pure-mode qP-wave fields, (d) pseudo-pure-mode scalar qP-wave fields, (e) separated scalar qP-wave fields.
}

%%%%%%%%%%%%%%%%%%%%%%%%%%%%%%%%%%%%%%%%%%%%%%%%%%%%%%%%%%%%%%%%%%%%%%%%%%%%%%%%%%%%%%%%%%%%%%%%%%%%%
\subsection{2. Reverse-time migration of Hess VTI model}
\inputdir{hessvti}

Our final example shows application of the pseudo-pure-mode qP-wave equation (i.e., equation~\ref{eq:pseudoVTIxy} in its 2D form) 
 to RTM of conventional seismic data representing mainly qP-wave energy using the synthetic data of
 SEG/Hess VTI model (Figure 10).
In the original data set, there is no vertical velocity model for qSV-wave, namely $v_{s0}$.
 For simplicity, we first get this parameter by setting $\frac{v_{s0}}{v_{p0}}=0.5$ anywhere.
Figures 11a and 11b display the two components of the synthesized pseudo-pure-mode qP-wave fields,
 in which the source is located at the center of the windowed region of the original models. 
 We observe that the summed wavefields (i.e., pseudo-pure-mode scalar qP-wave fields) contain quite weak
 residual qSV-wave energy (Figure 11c).
For seismic imaging of qP-wave data, we try the finite nonzero $v_{s0}$ scheme \cite[]{fletcher:2009}
 to suppress qSV-wave artifacts and enhance computation stability.
Thanks to superposition of multi-shot migrated data, we obtain a good RTM result (Figure 12) 
using the common-shot gathers provided at http://software.seg.org, although spatial filtering
has not been used to remove the residual qSV-wave energy. This example shows that the proposed pseudo-pure-mode
qP-wave equation could be directly used for reverse-time migration of conventional single-component seismic data.

\multiplot{3}{hessvp0,hessepsilon,hessdelta}{width=0.3\textwidth}
{
Part of SEG/Hess VTI model with parameters of (a) vertical qP-wave velocity, Thomsen coefficients 
(b) $\epsilon$ and (c) $\delta$.
}

\multiplot{3}{PseudoPurePx,PseudoPurePz,PseudoPureP}
{width=0.3\textwidth}
{
Synthesized wavefields using the pseudo-pure-mode qP-wave equation in SEG/Hess VTI model:
The three snapshots are synthesized by fixing the ratio of $v_{s0}$ to $v_{p0}$ as 0.5.  
The pseudo-pure-mode qP-wave fields (c) are the sum of the (a) x- and (b) z-components 
of the pseudo-pure-mode wavefields.
}

\plot{hessrtm}{width=0.75\textwidth}
{
RTM of Hess VTI model using the pseudo-pure-mode qP-wave equation with nonzero finite $v_{s0}$.
}
\section{Discussion}

For the general anisotropic media, qP- and qS-wave modes are intrinsically coupled.
The elastic wave equation must be solved at once to get correct kinematics and amplitudes
for all modes. The scalar wavefields, however, are widely used with the help of
wave mode separation or by using approximate equations derived from the elastic wave equation 
for many applications such as seismic imaging.
As demonstrated in the theoretical parts, the pseudo-pure-mode wave equation is derived
from the elastic wave equation through a similarity transformation to the Christoffel
equation in the wavenumber domain. The components of the transformed wavefield essentially
represent the spatial derivatives of the displacement wavefield components.
This transformation preserves the kinematics of wave propagation but inevitablely
changes the phases and amplitudes for qP- and qS-waves as the 
elastic wave mode separation procedure using divergence-like and curl-like operations
\cite[]{dellinger.etgen:1990,yan.sava:2009,zhang.mcmechan:2010}.
The filtering step to correct the projection deviation is indispensable for complete
removing the residual qS-waves from the extrapolated pseudo-pure-mode qP-wave fields.
This procedure does not change the phases and amplitudes of the scalar qP-waves because
the deviation operator is computed using the normalized wave and polarization vectors.

In fact, it is not even clear what the correct amplitudes should be for "scalar anisotropy".
Like the anisotropic pseudo-analytic methods \cite[]{etgen:2009,fomel:2013,zhan:2012,song:2013},
the pseudo-pure-mode wave equation may distort the reflection, transmission and conversion
coefficients of the elastic wavefields when there are high-frequency perturbations in the velocity model.
Therefore, the converted qP-waves remaining in the separated qP-wave fileds only have reliable kinematics. 
What happens to the qP-wave's amplitudes and how to
make use of the converted qP-waves (for seismic imaging) 
on the base of the pseudo-pure-mode qP-wave equation need further investigation in our future research.

\section{Conclusions}
% principles, relationships, generalizations inferred from the results
We have proposed an alternative approach 
to simulate propagation of separated wave modes in general anisotropic media.
The key idea is splitting the one-step wave mode separation into two cascaded steps based on the following
observations: First, the Christoffel equation derived from the original elastic wave equation 
accurately represents the kinematics of all wave modes; Second, various coupled 
second-order wave equations can be derived from the Christoffel equation
through similarity transformations. Third, wave mode separation can be achieved by projecting the original
elastic wavefields onto the given mode's polarization directions, which are usually calculated 
based on the local material parameters using the Christoffel equation.
Accordingly, we have derived the pseudo-pure-mode qP-wave equation by applying a similarity transformation 
aiming to project the elastic wavefield onto the wave vector, which
is the isotropic references of qP-wave polarization for an anisotropic medium.
The derived pseudo-pure-mode equations not only describe propagation of all wave modes
but also implicitly achieve partial mode separation once the wavefield components are summed.
As shown in the examples,
the scalar pseudo-pure-mode qP-wave fields
are dominated by qP-waves while the residual
 qS-waves are weaker in energy, because the projection deviations of qP-waves are generally far less than those of the qSV-waves.
Synthetic example of Hess VTI model demonstrates successful application of the pseudo-pure-mode qP-wave
equation to RTM for conventional seismic exploration.
To completely remove the residual qS-waves, a filtering step has been proposed 
to correct the projection
 deviations resulting from the difference between polarization direction and its isotropic reference.
In homogeneous media, it can be efficiently implemented by applying wavenumber domain filtering to each
 wavefield component. In heterogeneous media, nonstationary spatial filtering using 
pseudo-derivative operators are applied to finish the second step for wave mode separation.
In a word, pseudo-pure-mode wave equations plus corrections of projection deviations provide us an efficient
and flexible tool to simulate propagation of separated wave modes in anisotropic media.

In spite of the amplitude properties, this approach has some advantages over the classical solution 
combining elastic wavefield extrapolation and wave mode separation:
First, the pseudo-pure-mode wave equations could be directly used for migration of seismic data recorded
with single-component geophones without computationally expensive wave mode separation (as shown in the last example).
Second, because partial wave mode separation is automatically achieved during wavefield extrapolation and the correction step
to remove the residual qS-waves is optional depending on the strength of anisotropy,
	our approach provides better flexibilty for seismic modeling,
   migration and parameter inversion in practice;
Third, computational cost is reduced at least one third for the 2D cases
if the finite difference algorithms are used thanks to the simpler structure of pseudo-pure-mode wave equations
(i.e., having no mixed derivative terms for VTI and vertically orthorhombic media).
For the 3D TI media, computational cost is further reduced about one third because two instead of three
equations are used to simulate wave propagation.

Unlike the pseudo-acoustic wave equations, pseudo-pure-mode wave equations have no approximation in
kinematics and allow for $\epsilon<\delta$ provided that the stiffness tensor is positive-definite.
Moreover, they provide a possibility to extract artifact-free separated wave mode
during wavefield extrapolation.
Although we focus on propagation of separated qP-waves using the pseudo-pure-mode qP-wave equation,
our approach also works for qS-waves in TI media. This will be demonstrated in the second paper of this series.

\section{ACKNOWLEDGMENTS}
The research leading to this paper was supported by the National Natural Science Foundation of China (No.41074083)
and the Fundamental Research Funds for the Central Universities (No.1350219123).
We would like to thank Sergey Fomel, Paul Fowler, Yu Zhang, Tengfei Wang and Chenlong Wang
for helpful discussions in the later period of this study.
Constructive comments by Joe Dellinger, Faqi Liu, Mirko van der Baan, Reynam Pestana, and an anonymous reviewer
are much appreciated. We thank SEG, HESS Corporation and BP for making the 2D VTI and TTI synthetic data sets available,
and the authors of \emph{Madagascar} for providing this
software platform for reproducible computational experiments.

\append{Pseudo-pure-mode qP-wave equation for vertical TI and orthorhombic media}

For vertical TI and orthorhombic media, the stiffness tensors have the same null components and can
be represented in a two-index notation \cite[]{musgrave} often called the “Voigt notation” as
\begin{equation}
\mathbf{C} =
\begin{pmatrix}C_{11} &C_{12} &C_{13} &0 &0 &0 \cr
         C_{12} &C_{22} &C_{23} &0 &0 &0 \cr
         C_{13} &C_{23} &C_{33} &0 &0 &0 \cr
         0& 0&  0 & C_{44} &0 &0 \cr
         0& 0&  0 &0 & C_{55} &0 \cr
         0& 0&  0 &0 &0 &C_{66}\end{pmatrix}.
\end{equation}
For vertical orthorhombic tensor, the nine coefficients are indepedent, but the VTI tensor has only 
five independent coefficients with $C_{12}=C_{11}-2C_{66}$, $C_{22}=C_{11}$, $C_{23}=C_{13}$ and $C_{55}=C_{44}$.
The stability condition requires these parameters to satisfy the corresponding constraints \cite[]{helbig:1994, tsvankin:2001}.
According to equations~\ref{eq:elastic1} and \ref{eq:gama}, the full elastic wave equation without the source terms is expressed as:
\begin{equation}
\begin{split}
\rho\frac{\partial^2{u_x}}{\partial t^2} &= C_{11}\frac{\partial^2{u_x}}{\partial x^2}
                                         + C_{66}\frac{\partial^2{u_x}}{\partial y^2}
                                         + C_{55}\frac{\partial^2{u_x}}{\partial z^2}
                                         +(C_{12}+C_{66})\frac{\partial^2{u_y}}{{\partial x}{\partial y}}
                                         +(C_{13}+C_{55})\frac{\partial^2{u_z}}{{\partial x}{\partial z}}, \\
\rho\frac{\partial^2{u_y}}{\partial t^2} &= C_{66}\frac{\partial^2{u_y}}{\partial x^2}
                                         + C_{22}\frac{\partial^2{u_y}}{\partial y^2}
                                         + C_{44}\frac{\partial^2{u_y}}{\partial z^2}
                                         +(C_{12}+C_{66})\frac{\partial^2{u_x}}{{\partial x}{\partial y}}
                                         +(C_{23}+C_{44})\frac{\partial^2{u_z}}{{\partial y}{\partial z}}, \\
\rho\frac{\partial^2{u_z}}{\partial t^2} &= C_{55}\frac{\partial^2{u_z}}{\partial x^2}
                                         + C_{44}\frac{\partial^2{u_z}}{\partial y^2}
                                         + C_{33}\frac{\partial^2{u_z}}{\partial z^2} 
                                         +(C_{13}+C_{55})\frac{\partial^2{u_x}}{{\partial x}{\partial z}}
                                         +(C_{23}+C_{44})\frac{\partial^2{u_y}}{{\partial y}{\partial z}}.
\end{split}
\end{equation}
Thus the corresponding Christoffel matrix in wavenumber domain satisfies
\begin{equation}
\widetilde{\mathbf{\Gamma}}=
 \begin{pmatrix} C_{11}{k_x}^2 + C_{66}{k_y}^2 + C_{55}{k_z}^2  & (C_{12}+C_{66}){k_x}{k_y} &(C_{13}+C_{55}){k_x}{k_z} \cr
         (C_{12}+C_{66}){k_x}{k_y} & C_{66}{k_x}^2+C_{22}{k_y}^2+C_{44}{k_z}^2 &(C_{23}+C_{44}){k_y}{k_z} \cr
         (C_{13}+C_{55}){k_x}{k_z} & (C_{23}+C_{44}){k_y}{k_z} & C_{55}{k_x}^2+C_{44}{k_y}^2+C_{33}{k_z}^2\end{pmatrix}.
\end{equation}
According to equation~\ref{eq:tansChrisM}, the Christoffel matrix after the similarity transformation is given as, 
\begin{equation}
\label{eq:transGama}
\widetilde{\overline{\mathbf{\Gamma}}}=
\begin{pmatrix} C_{11}{k_x}^2+C_{66}{k_y}^2+C_{55}{k_z}^2 &(C_{12}+C_{66}){k_x}^2 &(C_{13}+C_{55}){k_x}^2 \cr
         (C_{12}+C_{66}){k_y}^2 & C_{66}{k_x}^2+C_{22}{k_y}^2+C_{44}{k_z}^2 &(C_{23}+C_{44}){k_y}^2 \cr
         (C_{13}+C_{55}){k_z}^2 & (C_{23}+C_{44}){k_z}^2 & C_{55}{k_x}^2+C_{44}{k_y}^2+C_{33}{k_z}^2\end{pmatrix}.
\end{equation}
Finally, we obtain the pseudo-pure-mode qP-wave equation (i.e., equation~\ref{eq:pseudo})
by inserting equation~\ref{eq:transGama} into equation~\ref{eq:tansChris} and applying an
 inverse Fourier transform.

\append{Pseudo-pure-mode qP-wave equation in orthorhombic media}

One of the most common reasons for orthorhombic anisotropy in sedimentary basins is a combination of parallel 
vertical fractures and vertically transverse isotropy in the background medium \cite[]{wild.crampin:1991,
schoenberg.helbig:1997}.
Vertically orthorhombic models have three mutually
orthogonal planes of mirror symmetry that coincide with the coordinate planes $[x_{1},x_{2}]$, $[x_{1},x_{3}]$ and
 $[x_{2},x_{3}]$. Here we assume $x_{1}$ axis is the x-axis (and used as the symmetry axis),
 $x_{2}$ the y-axis, and $x_{3}$ the z-axis.
Using the Thomsen-style notation for orthorhombic media \cite[]{tsvankin:1997b}:
\begin{equation}
\begin{split}
v_{p0}&=\sqrt{\frac{C_{33}}{\rho}}, \\
v_{s0}&=\sqrt{\frac{C_{55}}{\rho}}, \\
\epsilon_{1}&=\frac{C_{22}-C_{33}}{2C_{33}}, \\
\delta_{1}&=\frac{(C_{23}+C_{44})^2-(C_{33}-C_{44})^2}{2C_{33}(C_{33}-C_{44})}, \\
\gamma_{1}&=\frac{C_{66}-C_{55}}{2C_{55}}, \\
\epsilon_{2}&=\frac{C_{11}-C_{33}}{2C_{33}}, \\
\delta_{2}&=\frac{(C_{13}+C_{55})^2-(C_{33}-C_{55})^2}{2C_{33}(C_{33}-C_{55})}, \\
\gamma_{2}&=\frac{C_{66}-C_{44}}{2C_{44}}, \\
\delta_{3}&=\frac{(C_{12}+C_{66})^2-(C_{11}-C_{66})^2}{2C_{11}(C_{11}-C_{66})}, 
\end{split}
\end{equation}
	and
\begin{equation}
\begin{split}
v_{px1}&=v_{p0}\sqrt{1+2\epsilon_{1}}, \\
v_{px2}&=v_{p0}\sqrt{1+2\epsilon_{2}}, \\
v_{sx1}&=v_{s0}\sqrt{1+2\gamma_{1}}, \\
v_{sx2}&=v_{s0}\sqrt{1+2\gamma_{2}}, \\
v_{pn1}&= v_{p0}\sqrt{1+2\delta_{1}}, \\
v_{pn2}&= v_{p0}\sqrt{1+2\delta_{2}}, \\
v_{pn3}&= v_{p0}\sqrt{1+2\delta_{3}},
\end{split}
\end{equation}
the pseudo-pure-mode qP-wave equation (equation~\ref{eq:pseudo}) is rewritten as,
\begin{equation}
\label{eq:ort}
\begin{split}
\frac{\partial^2\overline{u}_x}{\partial t^2} &= {v_{px2}^2}\frac{\partial^2{\overline{u}_x}}{\partial x^2}
                                              + {v_{sx1}^2}\frac{\partial^2{\overline{u}_x}}{\partial y^2}
                                              + {v_{s0}^2}\frac{\partial^2{\overline{u}_x}}{\partial z^2}
                                              + \sqrt{({v_{px2}^2}-{v_{sx1}^2})[(1+2\epsilon_{2}){v_{pn3}^2}-{v_{sx1}^2}]}
                                                \frac{\partial^2{\overline{u}_y}}{\partial x^2} \\
                                              &+ \sqrt{({v_{p0}^2}-{v_{s0}^2})({v_{pn2}^2}-{v_{s0}^2})}
                                                \frac{\partial^2{\overline{u}_z}}{\partial x^2}, \\
\frac{\partial^2\overline{u}_y}{\partial t^2} &= {v_{sx1}^2}\frac{\partial^2{\overline{u}_y}}{\partial x^2}
                                              + {v_{px1}^2}\frac{\partial^2{\overline{u}_y}}{\partial y^2}
                                              + {v_{s12}^2}\frac{\partial^2{\overline{u}_y}}{\partial z^2}
                                              + \sqrt{({v_{px2}^2}-{v_{sx1}^2})[(1+2\epsilon_{2}){v_{pn3}^2}-{v_{sx1}^2}]}
                                                \frac{\partial^2{\overline{u}_x}}{\partial y^2} \\
                                              &+ \sqrt{({v_{p0}^2}-{v_{s12}^2})({v_{pn1}^2}-{v_{s12}^2}) }
                                                \frac{\partial^2{\overline{u}_z}}{\partial y^2}, \\
\frac{\partial^2\overline{u}_z}{\partial t^2} & = {v_{s0}^2}\frac{\partial^2{\overline{u}_z}}{\partial x^2}
                                              + {v_{s12}^2}\frac{\partial^2{\overline{u}_z}}{\partial y^2}
                                              + {v_{p0}^2}\frac{\partial^2{\overline{u}_z}}{\partial z^2}
                                              + \sqrt{({v_{p0}^2}-{v_{s0}^2})({v_{pn2}^2}-{v_{s0}^2}) }
                                                \frac{\partial^2{\overline{u}_x}}{\partial z^2} \\
                                              &+ \sqrt{({v_{p0}^2}-{v_{s12}^2})({v_{pn1}^2}-{v_{s12}^2}) }
                                                \frac{\partial^2{\overline{u}_y}}{\partial z^2},
\end{split}
\end{equation}
where $v_{p0}$ represents the vertical velocity of qP-wave, $v_{s0}$ represents the vertical velocity of qS-waves polarized
in the $x_{1}$ direction, $\epsilon_{1}$, $\delta_{1}$ and $\gamma_{1}$ represent the VTI parameters $\epsilon$, $\delta$
 and $\gamma$ in the $[y,z]$ plane, $\epsilon_{2}$, $\delta_{2}$ and $\gamma_{2}$ represent the VTI parameters
 $\epsilon$, $\delta$ and $\gamma$ in the $[x,z]$ plane, $\delta_{3}$ represents the VTI parameter $\delta$ 
in the $[x,y]$ plane. $v_{px1}$ and $v_{px2}$  are the
 horizontal velocities of qP-wave in the
symmetry planes normal to the x- and y-axis, respectively.
$v_{pn1}$, $v_{pn2}$ and $v_{pn3}$ 
are the interval NMO velocities
 in the three symmetry planes, and $v_{s12}=v_{s0}\sqrt{\frac{1+2\gamma_{1}}{1+2\gamma_{2}}}$.

Setting $v_{s0}=0$, we further obtain the pseudo-acoustic coupled system in a vertically orthorhombic media,
\begin{equation}
\label{eq:ortA}
\begin{split}
\frac{\partial^2\overline{u}_x}{\partial t^2} &= v_{px2}^2\frac{\partial^2{\overline{u}_x}}{\partial x^2}
                                              + (1+2\epsilon_{2})\sqrt{1+2\delta_{3}}{v_{p0}^2}
                                                \frac{\partial^2{\overline{u}_y}}{\partial x^2} 
                                              + \sqrt{1+2\delta_{2}}{v_{p0}^2}
                                                \frac{\partial^2{\overline{u}_z}}{\partial x^2}, \\
\frac{\partial^2\overline{u}_y}{\partial t^2} &= (1+2\epsilon_{2})\sqrt{1+2\delta_{3}}{v_{p0}^2}
                                               \frac{\partial^2{\overline{u}_x}}{\partial y^2}
                                             +(1+2\epsilon_{1}){v_{p0}^2}\frac{\partial^2{\overline{u}_y}}{\partial y^2} 
                                             + \sqrt{1+2\delta_{1}}{v_{p0}^2}
                                               \frac{\partial^2{\overline{u}_z}}{\partial y^2}, \\
\frac{\partial^2\overline{u}_z}{\partial t^2} & = \sqrt{1+2\delta_{2}}{v_{p0}^2}
                                               \frac{\partial^2{\overline{u}_x}}{\partial z^2} 
                                             + \sqrt{1+2\delta_{1}}{v_{p0}^2}
                                               \frac{\partial^2{\overline{u}_y}}{\partial z^2}
                                             + {v_{p0}^2}\frac{\partial^2{\overline{u}_z}}{\partial z^2}.
\end{split}
\end{equation}
Note that this equation does not contain any of the parameters $\gamma_{1}$ and $\gamma_{2}$ that describe
 the shear-wave velocities in the directions of the x- and y-axis, respectively.
 Evidently, kinematic signatures of qP-waves in pseudo-acoustic
 orthorhombic media depend on just five anisotropic coefficients ($\epsilon_{1}$, $\epsilon_{2}$, $\delta_{1}$,
 $\delta_{2}$ and $\delta_{3}$) and the vertical velocity $v_{p0}$.

In the presence of dipping fracture, we need to extend the vertically orthorhombic
 symmetry to a more complex form, i.e. orthorhombic media with tilted symmetry planes.
 Similar to TTI media, we locally rotate the coordinate system to make use of the simple form of the pseudo-pure-mode wave equations
 in vertically orthorhombic media. Since the physical properties are not symmetric in the local $[x_{1};x_{2}]$ plane, we need three angles
 to describe the rotation \cite[]{zhang.zhang:2011}. Two angles, $\theta$ and $\varphi$, are used to define the vertical axis at each
 spatial point as we did for the symmetry axis in TTI model. The third angle $\alpha$ is introduced to rotate the stiffness tensor 
on the local  plane and to represent the orientation of the fracture system in a VTI background or the orientation of the first
 fracture system of two orthogonal ones in an isotropic background.

The second-order differential operators in the rotated coordinate system are expressed in 
the same forms as in equation~\ref{eq:difoper}, but the rotation matrix is now given by,
\begin{equation}
\mathbf{R}=
\begin{pmatrix}\cos{\varphi} & -\sin{\varphi} &0 \cr
          \sin{\varphi} & \cos{\varphi} &0 \cr
          0 & 0 & 1\end{pmatrix}
\begin{pmatrix}\cos{\theta} & 0 & \sin{\theta} \cr
          0 & 1 & 0 \cr
          -\sin{\theta} & 0 & \cos{\theta}\end{pmatrix}
\begin{pmatrix}\cos{\alpha}  &-\sin{\alpha} &0 \cr
          \sin{\alpha} &\cos{\alpha} &0 \cr
          0 & 0 & 1\end{pmatrix},
\end{equation}
where
\begin{equation}
\begin{split}
r_{11}&=\cos{\theta}\cos{\varphi}\cos{\alpha}-\sin{\varphi}\sin{\alpha}, \\
r_{12}&=-\cos{\theta}\cos{\varphi}\sin{\alpha}-\sin{\varphi}\cos{\alpha}, \\
r_{13}&=\sin{\theta}\cos{\varphi},  \\
r_{21}&=\cos{\theta}\sin{\varphi}\cos{\alpha}+\cos{\varphi}\sin{\alpha}, \\
r_{22}&=-\cos{\theta}\sin{\varphi}\sin{\alpha}+\cos{\varphi}\cos{\alpha}, \\
r_{23}&=\sin{\theta}\sin{\varphi}, \\
r_{31}&=-\sin{\theta}\cos{\alpha}, \\
r_{32}&=\sin{\theta}\sin{\alpha}, \\
r_{33}&=\cos{\theta}.
\end{split}
\end{equation}
Substituting the second-order differential operators into the rotated coordinate system for
 those in the pseudo-pure-mode qP-wave equation
of vertically orthorhombic media yields the pseudo-pure-mode qP-wave equation of 
tilted orthorhombic media in the global Cartesian coordinates.

\append{Deviation between phase normal and polarization direction of qP-waves in VTI media}

For VTI media,  \cite{dellinger.thesis} presents an expression of the deviation angle $\zeta$ between the phase normal
(with phase angle $\psi$) and the polarization direction of qP-waves, namely
\begin{equation}
\label{eq:devAngle}
\sin^2(\zeta)=\frac{1}{2}+\frac{[(2s-1)t_{1}-t_{2}]\sqrt{t_{1}^2-t_{2}\chi}}{2(t_{2}\chi-{t_{1}}^2)},
\end{equation}
where
\begin{equation}
\begin{split}
s & =  \sin^2(\psi), \\
t_{1} & =  s(C_{33}+C_{11}-2C_{44})-C_{33}+C_{44} \\
      & =  s\rho[v_{p0}^2+(1+2\epsilon)v_{p0}^2-2v_{s0}^2]-\rho{v_{p0}^2}+\rho{v_{s0}^2}, \\
t_{2} & =  4s(s-1)\chi, \\
\chi & =  C_{13}+C_{44} \\ 
     & =  \rho\sqrt{(v_{p0}^2-v_{s0}^2)(v_{pn}^2-v_{s0}^2)}.
\end{split}
\end{equation}
Equation~\ref{eq:devAngle} indicates that the deviation angle has a complicated nonlinear relation with anisotropic parameters 
and the phase angle. The relationship is rather lengthy and does not easily reveal the features caused by anisotropy.
 Hence we use an alternative expression under a weak anisotropy assumption \cite[]{rommel:1994, tsvankin:2001},
\begin{equation}
\zeta=\frac{[\delta+2(\epsilon-\delta)\sin^2{\psi}]\sin{2\psi}}{2(1-\frac{v_{s0}^2}{v_{p0}^2})}
\end{equation}
It appears that the deviation angle is mainly affected by the difference between $\epsilon$ and $\delta$,
 the magnitude of $\delta$ (when $\epsilon-\delta$ stays the same) and the ratio of vertical velocities of
 qP- and qS-wave, as well as the phase angle.

%\onecolumn
\newpage
\bibliographystyle{seg}
\bibliography{reference}
