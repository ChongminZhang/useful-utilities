\published{Geophysics, 78, no. 6, U89-U101, (2013)}

\title{First-break traveltime tomography with the double-square-root eikonal equation}

\renewcommand{\thefootnote}{\fnsymbol{footnote}}

\author{Siwei Li\footnotemark[1], Alexander Vladimirsky\footnotemark[2] and Sergey Fomel\footnotemark[1]}

\ms{GEO-2012}

\address{
\footnotemark[1]Bureau of Economic Geology \\
John A. and Katherine G. Jackson School of Geosciences \\
The University of Texas at Austin \\
University Station, Box X \\
Austin, TX 78713-8924 \\
\footnotemark[2]Department of Mathematics \\
Cornell University \\
430 Malott Hall \\
Ithaca, NY 14853-4201}

\lefthead{Li et al.}
\righthead{DSR Tomography}
\footer{TCCS-6}

\maketitle

\begin{abstract}
First-break traveltime tomography is based on the eikonal equation. Since the 
eikonal equation is solved at fixed shot positions and only receiver positions can move along the ray-path, 
the adjoint-state tomography relies on inversion to resolve possible contradicting information between 
independent shots. The double-square-root eikonal equation allows not only the receivers but also the shots 
to change position, and thus describes the prestack survey as a whole. Consequently, its linearized tomographic 
operator naturally handles all shots together, in contrast with the shot-wise approach in the traditional 
eikonal-based framework. The double-square-root eikonal equation is singular for the horizontal waves, which 
require special handling. Although it is possible to recover all branches of the solution through 
post-processing, our current forward modeling and tomography focus on the diving wave branch only. We consider 
two upwind discretizations of the double-square-root eikonal equation and show that the explicit scheme is only 
conditionally convergent and relies on non-physical stability conditions. We then prove that an implicit upwind 
discretization is unconditionally convergent and monotonically causal. The latter property makes it possible to 
introduce a modified fast marching method thus obtaining first-break traveltimes both efficiently and 
accurately. To compare the new double-square-root eikonal-based tomography and traditional eikonal-based 
tomography, we perform linearizations and apply the same adjoint-state formulation and upwind 
finite-differences implementation to both approaches. Synthetic model examples 
justify that the proposed approach converges faster and is more robust than the traditional one.

\end{abstract}

\section{Introduction}

The first-break traveltime tomography \cite[]{zhu,osypov,leung,taillandier,noble} has been an established 
tool for estimating near-surface macro-feature seismic velocities. Starting from a prior model, tomographic 
inversion gradually modifies the velocities such that the misfits between predicted and observed first-breaks 
decrease. Since the problem is nonlinear, several linearization iterations may be required until 
convergence. Moreover, inversion must be carried out with careful choice of regularization in 
order to avoid local minima \cite[]{stefani,simmons,engl}. The estimated model has a direct 
influence on subsequent applications, for example static corrections \cite[]{marsden,cox,bergman} where it 
provides a medium-to-long wavelength near-surface model, and waveform tomography \cite[]{sheng,brenders,virieux} 
where it serves as a low-frequency prior.

The traditional first-break traveltime tomography is based on the eikonal equation that arises from 
high-frequency approximation of the wave equation \cite[]{chapman}. During forward modeling, the first-breaks 
computed through the eikonal equation are naturally shot-indexed because only receiver coordinates 
move while the source is fixed. At tomography stage, one may formulate the minimization of cost 
function as a sequence of explicitly linearized problems or directly as a nonlinear optimization problem. 
The first choice \cite[]{zelt,zhu1,dessa,pei} requires computation of Fr\'{e}chet derivatives, which is usually 
carried out by combining an eikonal solver with posterior ray tracing. Then an algorithm such as LSQR 
\cite[]{paige} is applied to solve the linearized tomographic system iteratively. While this approach accounts 
for information from both source and receiver dimensions, it faces computational limitations when the Fr\'{e}chet 
derivative matrix becomes difficult to handle because of a large number of 
model parameters. The nonlinear optimization approach, on the other hand, can be combined with the adjoint-state 
method \cite[]{plessix} and avoids an explicit computation of Fr\'{e}chet derivatives \cite[]{taillandier}. The 
cost of computing gradient is equivalent to twice the solution of the forward modeling problem, regardless of the 
size of input data. However, one major drawback of this approach, as we will show later, is that the resulting 
gradient disregards information available along the shot dimension.

The drawback of eikonal-based adjoint-state tomographies is that they
always face conflicting information that propagates across different
shots. Such conflicts must be resolved during inversion, or else an
erroneous model update may appear. In practice, the inversion may be
less robust and may take more iterations to converge, compared to the
situation where we replace the eikonal equation with another governing
equation that allows both source and receiver positions to change
along ray-paths. The double-square-root (DSR) eikonal equation is a
promising candidate in this regard, because it describes the prestack
data as a whole by linking the evolution of traveltimes to both
sub-surface source and receiver positions. In this paper, we
investigate the feasibility of using the DSR eikonal equation for
first-break traveltime tomography with the adjoint-state method.

DSR eikonal was analyzed previously by \cite{belonosova}, \cite{duchkov} and \cite{tariq}. Ray-tracing methods 
applied to DSR are capable of providing multi-arrivals by extrapolating isochron rays \cite[]{iversen} or using 
perturbation theory, but their extra costs in computing non-first-breaks are not necessary for first-break 
tomography purpose. We first prove that an implicit discretization of the DSR eikonal equation is causal 
and thus can be solved by a Dijkstra-like non-iterative method \cite[]{dijkstra}. The DSR singularity and two DSR 
branches that are non-causal need special treatment. Our current implementation 
employs a modified fast-marching \cite[]{sethian} DSR eikonal 
solver. We first test its accuracy by DSR forward modeling. Next, we linearize the DSR 
eikonal equation and use the resulting operators in adjoint-state tomography. For comparison, we apply an 
analogous linearization and adjoint-state formulation to the traditional tomography based on shot-indexed eikonal 
equation. Then we demonstrate the differences between the proposed and traditional approaches and 
justify advantages of the new method using several synthetic model examples. We conclude by discussing possible 
further improvements and extensions of our method.

\section{Theory}
\subsection{DSR eikonal equation}
\inputdir{.}

The DSR eikonal equation can be derived by considering a ray-path and its segments between two depth levels. 
Figure~\ref{fig:raypath} illustrates a diving ray \cite[]{zhu} in 2-D with velocity $v = v (z,x)$. We denote 
$T (z,r,s)$ as the total traveltime of the ray-path beneath depth $z$, where $r$ and $s$ are \textit{sub-surface} 
receiver and source lateral positions, respectively. 
\plot{raypath}{width=0.6\textwidth}{A diving ray and zoom-in of 
the ray segments between two depth levels.}

At both source and receiver sides the traveltime satisfies the eikonal equation, therefore
\begin{equation}
\label{eq:raypath}
\frac{\partial T}{\partial z} = 
- \sqrt{\frac{1}{v^2 (z,s)}-\left( \frac{\partial T}{\partial s} \right)^2} 
- \sqrt{\frac{1}{v^2 (z,r)}-\left( \frac{\partial T}{\partial r} \right)^2}\;.
\end{equation}
The negative signs before the two square-roots in equation \ref{eq:raypath} correspond to a decrease of 
traveltime with increasing depth, or geometrically a \textit{downward} 
pointing of slowness vectors on both $s$ and $r$ sides. Since the slowness vectors could also be pointing 
\textit{upward} and the directions may be different at $r$ and $s$, the DSR eikonal equation \cite[]{belonosova} 
should account for all the possibilities (Figure~\ref{fig:root}):
\begin{equation}
\label{eq:DSR}
\frac{\partial T}{\partial z} = 
\pm \sqrt{\frac{1}{v^2 (z,s)}-\left( \frac{\partial T}{\partial s} \right)^2} 
\pm \sqrt{\frac{1}{v^2 (z,r)}-\left( \frac{\partial T}{\partial r} \right)^2}\;.
\end{equation}
The boundary condition for DSR eikonal equation is that traveltimes at the subsurface zero-offset plane, i.e. 
$r = s$, are zero: $T (z,r=s) = 0$.

Equation \ref{eq:DSR} has a \textit{singularity} when $\partial T / \partial z = 0$, in which case the slowness 
vectors at $s$ and $r$ sides are both horizontal and equation \ref{eq:DSR} reduces to 
\begin{equation}
\label{eq:singular}
\left( \frac{\partial T}{\partial s} \right)^2 = \frac{1}{v^2 (z,s)}\;;\,\,\,
\left( \frac{\partial T}{\partial r} \right)^2 = \frac{1}{v^2 (z,r)}\;.
\end{equation}
The two independent equations in \ref{eq:singular} are not in conflict according to the source-receiver 
reciprocity, because they are the same with an exchange of $s$ and $r$.

Note that equations \ref{eq:DSR} and \ref{eq:singular} describe $T$ in full prestack domain 
$(z,r,s)$ by allowing not only receivers but also sources to change 
positions. In contrast, the eikonal equation
\begin{equation}
\label{eq:eikonal}
\left( \frac{\partial T}{\partial z} \right)^2 + \left( \frac{\partial T}{\partial x} \right)^2 
= \frac{1}{v^2 (z,x)}
\end{equation}
with boundary condition $T (z=0,x=s) = 0$ can be used only for one fixed source position at a time and thus 
traveltimes of different shots are independent of each other. In equation \ref{eq:eikonal} $s$ is 
\textit{surface} source lateral position. In the 3-D case, the scalars $s$, $r$ and $x$ in 
equations \ref{eq:DSR}, \ref{eq:singular} and \ref{eq:eikonal} become 2-D vectors $\mathbf{s}$, $\mathbf{r}$ and 
$\mathbf{x}$ that contain in-line and cross-line positions. The prestack traveltime is then in a 5-D space. Our 
current work is restricted to 2-D and we consider the 3-D extension in the Discussion section.

\plot{root}{width=0.95\textwidth}{All four branches of DSR 
eikonal equation from different combination of upward or 
downward pointing of slowness vectors. Whether the slowness 
vector is pointing leftward or rightward does not matter 
because the partial derivatives with respect to $s$ and $r$ 
in equation \ref{eq:DSR} are squared. Figure~\ref{fig:raypath} 
and equation \ref{eq:raypath} belong to the last situation.}

Similarly to the eikonal equation, the DSR eikonal equation is a nonlinear first-order partial differential 
equation. Its solutions include in general not only first-breaks but all arrivals, and can 
be computed by solving separate eikonal equations for each sub-surface source-receiver pair followed by 
extracting the traveltime and putting the value into prestack volume. However, such an implementation is 
impractical due to the large amount of computations. Meanwhile, for first-break tomography 
purposes, we are only interested in the first-arrival solutions but require an efficient and 
accurate algorithm. In this regard, a finite-difference DSR eikonal solver analogous to the fast-marching 
\cite[]{sethian} or fast-sweeping \cite[]{zhao} eikonal solvers is preferable.

In upwind discretizations of the DSR eikonal equation on the grid in $(z,r,s)$ domain, one has to make a decision 
about the $z$-slice, in which the finite differences are taken to approximate $\partial T / \partial s$ and 
$\partial T / \partial r$. In Figure~\ref{fig:raypath}, it appears natural to approximate these partial 
derivatives in the $z$-slice below $T (z,r,s)$. We refer to the corresponding scheme as \textit{explicit}, since 
it allows to directly compute the grid value $T (z,r,s)$ based on the already known $T$ values from the 
next-lower $z$. An alternative \textit{implicit} scheme is obtained by approximating $\partial T / \partial s$ 
and $\partial T / \partial r$ in the same $z$-slice as $T (z,r,s)$, which results in a coupled system of 
nonlinear discretized equations. In Appendix A, we prove the following:
\begin{enumerate}
\item The explicit scheme is very efficient to use on a fixed grid, but only conditionally convergent. 
This property is also confirmed numerically in Synthetic Model Examples section.
\item The implicit scheme is \textit{monotone causal}, meaning $T (z,r,s)$ depends on the smaller neighboring 
grid values only. This enables us to apply a Dijkstra-like method \cite[]{dijkstra} to solve the discretized 
system efficiently. Importantly, the DSR singularity requires a special ordering in the selection of upwind 
neighbors, which switches between equations \ref{eq:raypath} and \ref{eq:singular} when necessary. We provide a 
modified fast-marching \cite[]{sethian} DSR eikonal solver along with such an ordering strategy in Numerical 
Implementation section.
\item The causality analysis in Appendix A applies only to the first and last \textit{causal branches} out of all 
four shown in Figure~\ref{fig:root}. Additional post-processings, albeit expensive, can be used to recover the 
rest two non-causal branches as they may be decomposed into summations of the causal ones.
\end{enumerate}

In practice, we find that, for moderate velocity variations, the first-breaks correspond only to causal branches. 
An example in Synthetic Model Examples section serves to illustrate this observation. Therefore, for 
efficiency, we turn off the non-causal branch post-processings in forward modeling and base the 
tomography solely on equations \ref{eq:raypath} and \ref{eq:singular}.

\subsection{DSR tomography}

The first-break traveltime tomography with DSR eikonal equation (DSR tomography) can be established by following 
a procedure analogous to the traditional one with the shot-indexed eikonal equation (standard tomography). 
To further reveal their differences, in this section we will derive both approaches.

For convenience, we use slowness-squared $w \equiv 1/v^2$ instead of velocity $v$ in equations \ref{eq:raypath}, 
\ref{eq:singular} and \ref{eq:eikonal}. Based on analysis in Appendix A, the velocity model $w (z,x)$ and 
prestack cube $T (z,r,s)$ are Eulerian discretized and arranged as column vectors $\mathbf{w}$ of size 
$nz \times nx$ and $\mathbf{t}$ of size $nz \times nx \times nx$. We denote the observed first-breaks by 
$\mathbf{t}^{obs}$, and use $\mathbf{t}^{std}$ and $\mathbf{t}^{dsr}$ whenever necessary to discriminate between 
$\mathbf{t}$ computed from shot-indexed eikonal equation and DSR eikonal equation.

The tomographic inversion seeks to minimize the $l_2$ (least-squares) norm of the data residuals. We define 
an objective function as follows: 
\begin{equation}
\label{eq:objective}
E (\mathbf{w}) = \frac{1}{2} (\mathbf{t}-\mathbf{t}^{obs})^T (\mathbf{t}-\mathbf{t}^{obs})\;,
\end{equation}
where the superscript $T$ represents transpose. A Newton method of inversion can be established by considering an 
expansion of the misfit function \ref{eq:objective} in a Taylor series and retaining terms up to the 
quadratic order \cite[]{bertsekas}:
\begin{equation}
\label{eq:taylor}
E (\mathbf{w} + \delta \mathbf{w}) = E (\mathbf{w}) +
\delta \mathbf{w}^T \nabla_w E (\mathbf{w}) + 
\frac{1}{2} \delta \mathbf{w}^T \mathbf{H} (\mathbf{w}) \delta \mathbf{w} +
O (|\delta \mathbf{w}|^3)\;.
\end{equation}
Here $\nabla_w E$ and $\mathbf{H}$ are gradient vector and Hessian matrix, respectively. We may 
evaluate the gradient by taking partial derivatives of equation \ref{eq:objective} with respect to 
$\mathbf{w}$, yielding
\begin{equation}
\label{eq:linear}
\nabla_w E \equiv \frac{\partial E}{\partial \mathbf{w}} = 
\mathbf{J}^T (\mathbf{t} - \mathbf{t}^{obs})\;,
\end{equation}
where $\mathbf{J}$ is the Frech\'{e}t derivative matrix and 
can be found by further differentiating $\mathbf{t}$ with respect to $\mathbf{w}$. 

We start by deriving the Frech\'{e}t derivative matrix of standard tomography. Denoting
\begin{equation}
\label{eq:difference}
D_m \equiv \frac{\partial}{\partial m}\;;\,\,\,
m = z,x,r,s
\end{equation}
as the partial derivative operator in the $m$'s direction, 
equation \ref{eq:eikonal} 
can be re-written as
\begin{equation}
\label{eq:eikonal2}
D_z t^{std}_k \cdot D_zt^{std}_k  + D_x t^{std}_k \cdot D_x t^{std}_k = w\;;\,\,\,
k = 1,2,3,...,nx\;.
\end{equation}
Here we assume that there 
are in total $nx$ shots and use $t^{std}_k$ for first-breaks of the $k$th shot. Applying 
$\partial / \partial w$ to both sides of equation \ref{eq:eikonal2}, we find
\begin{equation}
\label{eq:stdderiv1}
J^{std}_k \equiv 
\frac{\partial t^{std}_k}{\partial w} = 
\frac{1}{2} (D_z t^{std}_k \cdot D_z + 
D_x t^{std}_k \cdot D_x)^{-1}\;;\,\,\,
k = 1,2,3,...,nx\;.
\end{equation}
Kinematically, each $J^{std}_k$ contains characteristics of the 
$k$th shot. Because shots are independent of each other, the full Frech\'{e}t 
derivative is a concatenation of individual $J^{std}_k$, as follows:
\begin{equation}
\label{eq:stdderiv}
J^{std} = \left[ 
J^{std}_1\;\;\;J^{std}_2\;\;\;\cdots\;\;\;J^{std}_{nx}
\right]^T\;.
\end{equation}
Inserting equation \ref{eq:stdderiv} into equation \ref{eq:linear}, we obtain
\begin{equation}
\label{eq:linearstd}
\nabla_w E = 
\sum_{k=1}^{nx} \left(\mathbf{J}^{std}_k\right)^T (\mathbf{t}^{std}_k - \mathbf{t}^{obs}_k)\;,
\end{equation}
where, similar to $t^{std}_k$, $t^{obs}_k$ stands for the observed first-breaks of the $k$th shot. 

Figure~\ref{fig:cartonstd} illustrates equation \ref{eq:linearstd} schematically, i.e. the 
gradient produced by standard tomography. The first step on the left depicts the transpose of the $k$th Frech\'{e}t 
derivative acting on the corresponding $k$th data residual. It implies a back-projection that takes 
place in the $z - r$ plane of a fixed $s$ position. The second step on the right is simply the summation 
operation in equation \ref{eq:linearstd}. 
\plot{cartonstd}{width=0.95\textwidth}{The gradient produced by standard
tomography. The solid curve indicates a shot-indexed 
characteristic.}

To derive the Frech\'{e}t derivative matrix associated with DSR tomography, we first re-write equation 
\ref{eq:raypath} with definition \ref{eq:difference}
\begin{equation}
\label{eq:raypath2}
D_z t^{dsr} = 
- \sqrt{w_s - D_s t^{dsr} \cdot D_s t^{dsr}} 
- \sqrt{w_r - D_r t^{dsr} \cdot D_r t^{dsr}}\;,
\end{equation}
where $w_s$ and $w_r$ are $w$ at sub-surface source and receiver locations, 
respectively. Note that in equation \ref{eq:raypath2} 
$w$ appears twice. Thus a differentiation of 
$t^{dsr}$ with respect to $w$ must be carried 
out through the chain-rule:
\begin{equation}
\label{eq:chain}
J^{dsr} \equiv \frac{\partial t^{dsr}}{\partial w} = 
\left. \frac{\partial t^{dsr}}{\partial w_s} \right|_{w_r} 
\frac{\partial w_s}{\partial w} + 
\left. \frac{\partial t^{dsr}}{\partial w_r} \right|_{w_s} 
\frac{\partial w_r}{\partial w}\;.
\end{equation}

We recall that $w$ and $t^{dsr}$ are of different lengths. Meanwhile in equation \ref{eq:raypath2}, both $w_s$ 
and $w_r$ have the size of $t^{dsr}$. Clearly in equation \ref{eq:chain} $\partial w_s / \partial w$ and 
$\partial w_r / \partial w$ must achieve dimensionality enlargement. 
In fact, according to Figure~\ref{fig:raypath}, $w_s$ and $w_r$ can be obtained by spraying $w$ such that 
$w_s (z,r,s) = w (z,s)$ and $w_r (z,r,s) = w (z,r)$. Therefore, $\partial w_s / \partial w$ and 
$\partial w_r / \partial w$ are essentially spraying operators and their adjoints perform stackings along $s$ and 
$r$ dimensions, respectively.
\begin{comment}
\plot{relation}{width=0.95\textwidth}{A 2-D velocity model $w$ (left) 
is used in equation \ref{eq:raypath2} for both $w_s$ and $w_r$ (right). 
The plane with red solid lines is the model domain $(z,x)$ while the cube 
with gray dashed lines represents the prestack domain $(z,r,s)$.}
\plot{relation2}{width=0.95\textwidth}{Following 
Figure~\ref{fig:relation}, $w_s$ and $w_r$ are the results of spraying 
$w$ from $(z,x)$ to $(z,r,s)$.}
\end{comment}

In Appendix B, we prove that $J^{dsr}$ has the following form:
\begin{equation}
\label{eq:dsrderiv}
J^{dsr} = B^{-1} (C_s + C_r)\;.
\end{equation}
Combining equations \ref{eq:linear} and \ref{eq:dsrderiv} results in
\begin{equation}
\label{eq:lineardsr}
\nabla_w E = 
\left(\mathbf{C}_s^T + \mathbf{C}_r^T\right) \mathbf{B}^{-T} 
(\mathbf{t}^{dsr} - \mathbf{t}^{obs})\;.
\end{equation}
Note that unlike equation \ref{eq:linearstd}, equation \ref{eq:lineardsr} can not be expressed as an explicit 
summation over shots.

Figure~\ref{fig:cartondsr} shows the gradient of DSR tomography. Similarly to the standard tomography, the 
gradient produced by equation \ref{eq:lineardsr} is a result of two steps. The first step on the left is a 
back-projection of prestack data residuals according to the adjoint of operator $B^{-1}$. Because $B$ contains 
DSR characteristics that travel in prestack domain, this back-projection takes place in $(z,r,s)$ and is 
different from that in standard tomography, although the data residuals are the same for both cases. The second 
step on the right follows the adjoint of operators $C_s$ and $C_r$ and reduces the dimensionality from $(z,r,s)$ 
to $(z,x)$. However, compared to standard tomography this step involves summations in not only $s$ but also $r$.
\plot{cartondsr}{width=0.95\textwidth}{The gradient produced by 
DSR tomography. The solid curve indicates a DSR 
characteristic, which has one end in plane $z = 0$ and the 
other in plane $s = r$. Compare with Figure~\ref{fig:cartonstd}.}

\section{Numerical Implementation}

Following the analysis in Appendix A, we consider an implicit Eulerian discretization. For forward 
modeling, we solve the DSR eikonal equation by a version of the fast-marching method (FMM) \cite[]{sethian}. 
First, a plane-wave with $T = 0$ at subsurface zero-offset $r = s$ is initialized. Next, in the update stage the 
traveltime at a grid point is computed from its upwind neighbors. A priority queue keeps track of the first-break 
wave-front, and the computation is non-recursive.

To properly handle the DSR singularity, we design an ordering of the combination of upwind neighbors during the 
update stage. Assuming that $T^i$ is the upwind neighbor of $T$ in the $i$'s direction for $i = z,r,s$, we 
summarize the ordering as follows:
\begin{enumerate}
\item First try a three-sided update:
\begin{itemize}
\item[] Solve equation \ref{eq:discre3}, return $T$ if $T \geq \max (T^z,T^r,T^s)$;
\end{itemize}
\item Next try a two-sided update: solve equations \ref{eq:discre4}, \ref{eq:discre5} and \ref{eq:discre6} 
and keep the results as $T_{rs}$, $T_{zr}$ and $T_{zs}$, respectively.
\begin{itemize}
\item[] If $T_{zr} \geq \max (T^z,T^r)$ and $T_{zs} \geq \max (T^z,T^s)$, return 
$\min (T_{zr}, T_{zs}, T_{rs})$;
\item[] If $T_{zr} < \max (T^z,T^r)$ and $T_{zs} \geq \max (T^z,T^s)$, return 
$\min (T_{zs}, T_{rs})$;
\item[] If $T_{zr} \geq \max (T^z,T^r)$ and $T_{zs} < \max (T^z,T^s)$, return 
$\min (T_{zr}, T_{rs})$;
\end{itemize}
\item Finally try a one-sided update:
\begin{itemize}
\item[] Solve equation \ref{eq:discre7}, return $\min (T, T_{rs})$.
\end{itemize}
\end{enumerate}
An optional search routine \ref{eq:search} may be added after the update to recover all branches of the DSR 
eikonal equation. The overall cost can be reduced roughly by half by acknowledging the source-receiver 
reciprocity and thus computing only the positive (or negative) subsurface offset region.

For an implementation of linearized tomographic operators \ref{eq:linearstd} and 
\ref{eq:lineardsr}, we choose upwind approximations \cite[]{franklin,li,lelievre} for the difference operators in 
equation \ref{eq:difference}. In Appendix C, we show that the upwind 
finite-differences result in triangularization of matrices \ref{eq:stdderiv} 
and \ref{eq:dsrderiv}. Therefore, the costs of applying $\mathbf{J}^{std}$ and $\mathbf{J}^{dsr}$ and their 
transposes are inexpensive. Moreover, although our implementation belongs to the family of 
adjoint-state tomographies, we do not need to compute the adjoint-state variable as an intermediate product for 
the gradient.

Additionally, the Gauss-Newton approach approximates the Hessian in equation \ref{eq:taylor} by 
$\mathbf{H} \approx \mathbf{J}^T \mathbf{J}$. An update $\delta \mathbf{w}$ at current $\mathbf{w}$ is found by 
taking derivative of equation \ref{eq:taylor} with respect to $\delta \mathbf{w}$, which results in the 
following normal equation:
\begin{equation}
\label{eq:normal}
\delta \mathbf{w} = \left[ \mathbf{J}^T \mathbf{J} \right]^{-1} \mathbf{J}^T (\mathbf{t}^{obs} - \mathbf{t})\;.
\end{equation}
To add model constraints, we combine equation \ref{eq:normal} with Tikhonov regularization \cite[]{tikhonov} 
with the gradient operator and use the method of conjugate gradients \cite[]{hestenes} to solve for the model 
update $\delta \mathbf{w}$.

\section{Synthetic Model Examples}

The numerical examples in this section serve several different purposes. 
The first example will test the accuracy of modified FMM DSR eikonal equation solver (DSR FMM) and show 
the drawbacks of the alternative explicit discretization. The second example will demonstrate effect of 
considering non-causal branches of DSR eikonal equation in forward modeling. The third 
example will compare the sensitivity kernels of DSR tomography and standard tomography in a 
simple model. The last example will present a tomographic inversion 
and demonstrate advantages of DSR method over the traditional method.

Figure~\ref{fig:modl} shows a 2-D velocity model with a constant-velocity-gradient background plus a Gaussian 
anomaly in the middle. We use $\Delta$ to denote the grid spacing in $z$ and $\delta$ in $x$. The 
traveltimes on the surface $z = 0$ km of a shot at $(0,0)$ km are computed by DSR FMM at a gradually refined 
$\Delta$ or $\delta$ while fixing the other one. For reference, we also calculate first-breaks 
by a second-order FMM \cite[]{rickett,popovici} for the same shot at a very fine grid spacing of 
$\Delta = \delta = 1$ m. In Figure~\ref{fig:imp}, a 
grid refinement in both $\Delta$ and $\delta$ helps reducing errors of the implicit discretization, although 
improvements in the $\Delta$ refinement case are less significant because the majority of the ray-paths are 
non-horizontal. The results are consistent with the analysis in Appendix A, which shows 
that the implicit discretization is unconditionally convergent. On the other hand, as shown in 
Figure~\ref{fig:exp}, the explicit discretization is only conditionally convergent when 
$\Delta /\delta \rightarrow 0$ under grid refinement in order to resolve the flatter parts of the ray-paths. This 
explains why its accuracy deteriorates when refining $\delta$ and fixing $\Delta$. A more detailed error analysis 
remains open for future research.
\inputdir{accuracy}
\plot{modl}{width=0.95\textwidth}{The synthetic model used for DSR FMM 
accuracy test. The overlaid curves are rays traced from a shot at 
$(0,0)$ km.}
\plot{imp}{width=0.95\textwidth}{Grid refinement experiment (implicit 
discretization). In both figures, the solid blue curve is the 
reference values and the dashed curves are 
computed by DSR FMM. Top: fixed $\delta = 10$ m and $\Delta = 50$ m 
(cyan), $10$ m (magenta), $5$ m (black). Bottom: fixed $\Delta = 10$ m 
and $\delta = 50$ m (cyan), $10$ m (magenta), $5$ m (black).}
\plot{exp}{width=0.95\textwidth}{Grid refinement experiment (explicit 
discretization). The experiment set-ups are the same as in 
Figure~\ref{fig:imp}.}

Next, we use a smoothed Marmousi model (Figure~\ref{fig:marmsmooth}) and run two DSR FMMs, one with the search 
process for non-causal DSR branches turned-on and the other turned-off. In Figure~\ref{fig:causal}, again we 
compute reference values by a second-order FMM. The three groups of curves are 
traveltimes of shots at $(0,0)$ km, $(0.75,0)$ km and $(1.5,0)$ km, respectively. The maximum absolute 
differences between the two DSR FMMs, for all three shots, are approximately $5$ ms at the largest offset. This 
shows that, if the near-surface model is moderately complex, then the first-breaks are of causal 
types described by equations \ref{eq:raypath} and \ref{eq:singular}, and we therefore can use their 
linearizations \ref{eq:dsrderiv} for tomography.
\inputdir{causal}
\plot{marmsmooth}{width=0.95\textwidth}{A smoothed Marmousi model 
overlaid with rays traced from a shot at $(0,0)$ km. 
Because of velocity variations, 
multi-pathing is common in this model, especially at large offsets.}
\plot{causal}{width=0.95\textwidth}{DSR FMM with non-causal branches. 
The solid black lines are reference values. 
There are two groups of dashed lines, both from DSR FMM but one with 
the optional search process turned-on and the other without. The 
differences between them are negligible and 
hardly visible.}

According to equations \ref{eq:stdderiv} and \ref{eq:dsrderiv}, the sensitivity 
kernels (a row of Frech\'{e}t derivative matrix) of standard tomography and DSR 
tomography are different. Figure~\ref{fig:grad} compares sensitivity kernels for the same 
source-receiver pair in a constant velocity-gradient model. We use a fine model sampling of 
$\Delta = \delta = 2.5$ m. The standard tomography kernel appears to be asymmetric. Its amplitude has a bias 
towards the source side, while the width is broader on the receiver side. These phenomena are related to our 
implementation, as described in Appendix C. Note in the top plot of Figure~\ref{fig:grad}, the curvature of 
first-break wave-front changes during propagation. Upwind finite-differences 
take the curvature variation into consideration and, as a result, back-project 
data-misfit with different weights along the ray-path. Meanwhile, the DSR tomography kernel is symmetric in both 
amplitude and width, even though it uses the same discretization and upwind approximation as 
in standard tomography. The source-receiver reciprocity may suggest averaging the standard tomography kernel with 
its own mirroring around $x = 1$ km, however the result will still be different from the DSR tomography kernel as 
the latter takes into consideration all sources at the same time.
\inputdir{hessian}
\plot{grad}{width=0.75\textwidth}{(Top) model overlaid with traveltime 
contours of a source at $(0,0)$ km and sensitivity kernels of (middle) 
the standard tomography and (bottom) the DSR tomography.}

Finally, Figure~\ref{fig:data} illustrates a prestack first-break traveltime modeling of the Marmousi model by 
DSR FMM. We use a constant-velocity-gradient model as the prior for inversion. There are $287$ 
shots evenly distributed on the surface, each shot has a maximum absolute receiver offset of $6$ 
km. Figure~\ref{fig:marm} shows a zoom-in of the exact model that is within the tomographic aperture. The DSR 
tomography and standard tomography are performed with the same parameters: $10$ conjugate gradient iterations per 
linearization update and $4$ linearization updates in total. Figure~\ref{fig:conv} shows the convergence 
histories. While both inversions converge, the relative $l_2$ data misfits of DSR tomography decreases faster 
than that of standard tomography. Figure~\ref{fig:tomo} compares the recovered models. Although both results 
resemble the exact model in Figure~\ref{fig:marm} at the large scale, the standard tomography model exhibits 
several undesired structures. For example, a near-horizontal structure with a velocity of around $2.75$ km/s at 
location $(0.85,4.8)$ km is false. It indicates the presence of a local minimum that has trapped the standard 
tomography. In practice, it is helpful to tune the inversion parameters so that the standard tomography takes 
more iterations with a gradually reducing regularization. The inversion parameters are usually empirical and hard 
to control. Our analysis in preceeding sections suggests that part of the role of regularization is to deal with 
conflicting information between shots. In contrast, we find DSR tomography less dependent on regularization and 
hence more robust.
\inputdir{marm}
\plot{data}{width=0.95\textwidth}{DSR first-break traveltimes in the Marmousi model. 
The original model is decimated by $2$ in both vertical and lateral 
directions, such that $nz = 376$, $nx = 1151$ and $\Delta = \delta = 8$ m.}
\plot{marm}{width=0.95\textwidth}{(Top) a zoom-in of Marmousi model and 
(bottom) the initial model for tomography.}
\plot{conv}{width=0.6\textwidth}{Convergency history of DSR tomography 
(solid) and standard tomography (dashed). There is no noticeable 
improvement on misfit after the fourth update.}
\plot{tomo}{width=0.95\textwidth}{Inverted model of (top) 
standard tomography and (bottom) DSR 
tomography. Compare with Figure~\ref{fig:marm}.}

The advantage of DSR tomography becomes more significant in the presence of 
noise in the input data. We generate random noise of normal distribution with zero mean and a 
range between $\pm 600$ ms, then threshold the result with a minimum absolute value of $250$ ms. This is to mimic 
the spiky errors in first-breaks estimated from an automatic picker. After adding noise to the data, we run 
inversions with the same parameters as in Figures \ref{fig:conv} and \ref{fig:tomo}. Figures \ref{fig:nconv} and 
\ref{fig:ntomo} show the convergence history and inverted models. Again, the standard tomography seems to provide 
a model with higher resolution, but a close examination reveals that many small scale details are in fact 
non-physical. On the other hand, DSR tomography suffers much less from the added noise. Adopting a $l1$ 
norm in objective function \ref{eq:objective} can improve the inversion, especially for standard tomography. 
However, it also raises the difficulty in selecting appropriate inversion 
parameters.
\plot{nconv}{width=0.6\textwidth}{Inversion with noisy data. Convergency 
history of DSR tomography (solid) and standard tomography (dashed). No 
significant decrese in misfit appears after the fourth update.}
\plot{ntomo}{width=0.95\textwidth}{Inversion with noisy data. Inverted model 
of (top) standard tomography and (bottom) 
DSR tomography. Compare with Figure~\ref{fig:tomo}.}

\section{Discussion}

There are three main issues in the DSR tomography. The first issue comes from a large dimensionality of the 
prestack space, which results in a considerable computational domain size after discretization. The memory 
consumption becomes an immediate problem for 3-D models, where the prestack traveltime belongs to a 5-D space 
and may require distributed storage.

The second issue is related to the computational cost. The FMM DSR we have introduced in this 
paper has a computational complexity of $O (N\,\log N)$, where $N$ is the total number of grid points after 
discretization, $N = nz \times nx^2$. The $\log N$ factor arises in the priority queue used in 
FMM for keeping track of expanding wave-fronts. Some existing works could 
accelerate FMM to an $O (N)$ complexity and may be applicable to the DSR eikonal equation \cite[]{kim,yatziv}. A 
number of other fast methods developed for the eikonal equation might be similarly applicable to the DSR eikonal 
equation. These include fast sweeping \cite[]{zhao}, hybrid two-scale marching-sweeping methods and various 
label-correcting algorithms (see \cite{chacon} and references therein).

The last issue is possible parallelization of the proposed method. Our current 
implementation of the FMM DSR tomography algorithm is sequential, while the traditional tomography could be 
parallelized among different shots. However, we notice that the DSR eikonal equation has a plane-wave source, 
therefore a distributed wave-front propagating at the beginning followed by a subdomain merging is possible. A 
number of parllelizable algorithms for the eikonal equation have been developed 
\cite[]{zhao1,jeong,weber,chacon2,detrixhe}. Extending these methods to the DSR eikonal equation would be the 
first step in parallelizing DSR tomography.

\section{Conclusions}

We propose to use the DSR eikonal equation for the first-break traveltime tomography. 
The proposed method relies on an efficient DSR solver, which is realized by a version of the fast-marching method 
based on an implicitly causal discretization. Since the DSR eikonal equation allows changing of source position 
along the ray-path, its linearization results in a tomographic inversion that naturally handles possible 
conflicting information between different shots. Our numerical tests show that, compared to 
the traditional tomography with a shot-indexed eikonal equation, the DSR tomography converges faster and is more 
robust. Its benefits may be particularly significant in the presence of noise in the data. 

\section{Acknowledgments}

This research was supported in part by Saudi Aramco. The second author's work was also partially supported by the 
NSF grant DMS-1016150. We thank Anton Duchkov, Mauricio Sacchi, J\"{o}rg Schleicher, Aldo Vesnaver, and four 
anonymous reviewers for valuable comments and suggestions. We thank Tariq Alkhalifah and Tim Keho for useful 
discussions. This publication is authorized by the Director, Bureau of Economic Geology, The University of Texas 
at Austin.

\appendix
\section{Appendix A: Causal discretization of DSR eikonal equation}
\inputdir{.}

To simplify the analysis, we consider first the DSR branch as shown in Figure~\ref{fig:raypath} and described by 
equation \ref{eq:raypath}. We assume a rectangular 2-D velocity model $v (z,x)$ and thus a cubic 3-D prestack 
volume $T (z,r,s)$ with $r$ and $s$ axes having the same dimension as $x$. After an Eulerian discretization of 
both $v$ and $T$, we denote the grid spacing in $z$ as $\Delta$, and in $x$, $r$ and $s$ as $\delta$. 
\plot{update1}{width=0.6\textwidth}{An implicit discretization 
scheme. The arrow indicates a DSR characteristic. Its root 
is located in the simplex $T^z T^r T^s$.}

In Figure~\ref{fig:update1}, we study the traveltime at grid point $\mathbf{y} = (z,r,s)$ and its relationship 
with neighboring grid points $\mathbf{y}^z = (z+\Delta,r,s)$, $\mathbf{y}^r = (z,r-\delta,s)$ and 
$\mathbf{y}^s = (z,r,s+\delta)$ with a semi-Lagrangian scheme. According to the geometry in 
Figure~\ref{fig:raypath}, in the $(z,r,s)$ space the DSR characteristic \cite[]{duchkov} is straddled by 
$\mathbf{y}^z \mathbf{y}^r \mathbf{y}^s$. 

In order to compute $T$, we could continue along this characteristic up until its intersection with the simplex 
$\mathbf{y}^z \mathbf{y}^r \mathbf{y}^s$. Suppose the intersection point is 
$\mathbf{\tilde{y}} = (\tilde{z},\tilde{r},\tilde{s})$ and $\alpha_i$'s are its barycentric coordinates, i.e.
\begin{equation}
\label{eq:barycentric}
\alpha_z,\alpha_r,\alpha_s \in [0,1];\,\,\,
\alpha_z + \alpha_r + \alpha_s = 1;\,\,\,
\mathbf{\tilde{y}} = \alpha_z \mathbf{y}^z + \alpha_r \mathbf{y}^r + \alpha_s \mathbf{y}^s\;.
\end{equation}
This leads to the following discretization:
\begin{equation}
\label{eq:discre1}
T \equiv T (\mathbf{y}) = \smash{\displaystyle\min_{\mathbf{\tilde{y}} \in \mathbf{y}^z \mathbf{y}^r \mathbf{y}^s}} 
\left\{ T (\mathbf{\tilde{y}}) + \frac{\sqrt{(z-\tilde{z})^2 + (r-\tilde{r})^2}}{v (z,r)} 
+ \frac{\sqrt{(z-\tilde{z})^2 + (s-\tilde{s})^2}}{v (z,s)} \right\}\;.
\end{equation}
Here we further assume that $v (z,r)$ and $v (z,s)$ are locally constant and that ray-segments between $z$ and 
$z + \Delta$ are well approximated by straight lines. This also means that a linear interpolation in $T$ within 
the simplex $\mathbf{y}^z \mathbf{y}^r \mathbf{y}^s$ is exact, i.e. 
$T (\mathbf{\tilde{y}}) = \alpha_z T^z + \alpha_r T^r + \alpha_s T^s$, where $T^i = T (\mathbf{y}^i)$ for 
$i = z,r,s$. The minimization over all possible intersection points in equation \ref{eq:discre1} guarantees a 
first-arrival traveltime.

Defining the ratio in grid spacing as $\mu \equiv \Delta / \delta$ and denoting $v_r = v (z,r)$ and 
$v_s = v (z,s)$, equation \ref{eq:discre1} can be re-written with the barycentric coordinates in 
\ref{eq:barycentric} as
\begin{equation}
\label{eq:discre2}
T = \smash{\displaystyle\min_{(\alpha_z,\alpha_r,\alpha_s)}}
\left\{ (\alpha_z T^z + \alpha_r T^r + \alpha_s T^s) 
+ \frac{\delta}{v_r} \sqrt{\alpha_r^2 + \mu^2 \alpha_z^2} 
+ \frac{\delta}{v_s} \sqrt{\alpha_s^2 + \mu^2 \alpha_z^2} \right\}\;.
\end{equation}
Figure~\ref{fig:update1} is based on a particular direction of the diving wave: rightward from the source and 
leftward from the receiver, as in Figure~\ref{fig:raypath}. This yields the above positions of $\mathbf{y}^s$ and 
$\mathbf{y}^r$, and the formula \ref{eq:discre2} becomes an update from the 
$\mathbf{y}^z \mathbf{y}^r \mathbf{y}^s$ quadrant. Since in general the direction of a diving wave is not a 
priori known, we compute one such update from each of the lower quadrants and take the smallest amongst them as a 
value of $T$. 

To explore the causal properties of equation \ref{eq:discre2}, we first assume that the minimum is attained at 
some $\mathbf{\tilde{y}}^* = \xi_z \mathbf{y}^z + \xi_r \mathbf{y}^r + \xi_s \mathbf{y}^s$ such that $\xi_i > 0$ 
for $i = z,r,s$. From the Kuhn-Tucker optimality conditions \cite[]{kuhn-tucker}, there exists a Lagrange 
multiplier $\lambda$ such that
\begin{equation}
\label{eq:kkt1}
\lambda = T^z + 
\delta \left( \frac{\mu^2 \xi_z}{v_r \sqrt{\xi_r^2 + \mu^2 \xi_z^2}} 
+ \frac{\mu^2 \xi_z}{v_s \sqrt{\xi_s^2 + \mu^2 \xi_z^2}} \right)\;;
\end{equation}
\begin{equation}
\label{eq:kkt2}
\lambda = T^i + 
\delta \left( \frac{\xi_i}{v_i \sqrt{\xi_i^2 + \mu^2 \xi_z^2}} \right)\;;\,\,\,i = r,s\;.
\end{equation}
Taking a linear combination of 
\ref{eq:kkt1} and \ref{eq:kkt2} to match the right-hand 
side of \ref{eq:discre2}, we find that $\lambda = T$ and thus 
\begin{equation}
\label{eq:kkt10}
T - T^z = 
\delta \left( \frac{\mu^2 \xi_z}{v_r \sqrt{\xi_r^2 + \mu^2 \xi_z^2}} 
+ \frac{\mu^2 \xi_z}{v_s \sqrt{\xi_s^2 + \mu^2 \xi_z^2}} \right) > 0\;;
\end{equation}
\begin{equation}
\label{eq:kkt20}
T - T^i = 
\delta \left( \frac{\xi_i}{v_i \sqrt{\xi_i^2 + \mu^2 \xi_z^2}} \right) > 0\;;\,\,\,i = r,s\;.
\end{equation}
This means that if $T$ defined by equation \ref{eq:discre2} depends on $T^i$ then $T > T^i$ 
for $i = z,r,s$, or
\begin{equation}
\label{eq:causality}
T > \max (T^z, T^r, T^s)\;
\end{equation}
and a Dijkstra-like method \cite[]{dijkstra} is applicable to solve the discretized system. 

A direct substitution from equations \ref{eq:kkt10} and \ref{eq:kkt20} results in 
\begin{equation}
\label{eq:discre3}
\frac{T-T^z}{\Delta} = 
\sqrt{\frac{1}{v_r^2} - \left( \frac{T-T^r}{\delta} \right)^2} + 
\sqrt{\frac{1}{v_s^2} - \left( \frac{T-T^s}{\delta} \right)^2}\;.
\end{equation}
If $T^z$, $T^r$ and $T^s$ are known, then $T$ can be recovered by solving the 4th degree polynomial equation 
\ref{eq:discre3} and choosing the smallest real root that satisfies condition \ref{eq:causality}. This gives a 
three-sided update at $T$. The discretization is \textit{implicitly causal} and provides unconditional 
consistency and convergence. 

If there is no real root or none of the real roots satisfy \ref{eq:causality}, the minimizer $\mathbf{\tilde{y}}$ 
can not lie in the interior of simplex $\mathbf{y}^z \mathbf{y}^r \mathbf{y}^s$ and at least one of the $\xi_i$s 
must be zero. If $\xi_z = 0$, it is easy to show that one of the other barycentric coordinates is also zero and 
equation \ref{eq:discre2} simplifies to 
\begin{equation}
\label{eq:discre4}
T = \min \left( T^r + \frac{\delta}{v_r}, T^s + \frac{\delta}{v_s} \right)\;,
\end{equation}
which is a causal discretization of the DSR singularity in equation \ref{eq:singular}. On the other hand, if 
$\xi_z \not= 0$ and $\xi_r \not= 0$ but $\xi_s = 0$, i.e. the slowness vector at $s$ is vertical, then
\begin{equation}
\label{eq:kkt3}
T = (\xi_z T^z + \xi_r T^r) + \frac{\delta}{v_r} \sqrt{\xi_r^2 + \mu^2 \xi_z^2} 
+ \frac{\Delta}{v_s} \xi_z\;.
\end{equation}
A similar Kuhn-Tucker-type argument shows that equation \ref{eq:kkt3} is also causal: if $\xi_z,\xi_r > 0$, then 
$T > \max(T^z,T^r)$. In this case, $T$ can be computed by solving 
\begin{equation}
\label{eq:discre5}
\frac{T-T^z}{\Delta} = 
\sqrt{\frac{1}{v_r^2} - \left( \frac{T-T^r}{\delta} \right)^2} + \frac{1}{v_s}\;.
\end{equation}
Equation \ref{eq:discre5} is equivalent to setting $\partial T / \partial s = 0$ in equation \ref{eq:raypath}. 
Analogously, when $\xi_z \not= 0$ and $\xi_s \not= 0$ but $\xi_r = 0$, we have $\partial T / \partial r = 0$ and
\begin{equation}
\label{eq:discre6}
\frac{T-T^z}{\Delta} = 
\frac{1}{v_r} + \sqrt{\frac{1}{v_s^2} - \left( \frac{T-T^s}{\delta} \right)^2}\;,
\end{equation}
with the causal solution satisfying $T > \max(T^z,T^s)$. Equations \ref{eq:discre5} and \ref{eq:discre6} provide 
a two-sided update at $T$. Finally, if $\xi_z \not= 0$ but $\xi_s = 0$ and $\xi_r = 0$, i.e. 
$\partial T / \partial s = 0$ and $\partial T / \partial r = 0$, we obtain the simplest one-sided update:
\begin{equation}
\label{eq:discre7}
\frac{T-T^z}{\Delta} = 
\frac{1}{v_r} + \frac{1}{v_s}\;.
\end{equation}

We note that the one-sided update \ref{eq:discre7} could be considered a special case of two-sided updates: if 
$T= T^r$ (or $T^s$), then \ref{eq:discre7} becomes equivalent to \ref{eq:discre5} (or, respectively, 
\ref{eq:discre6}). Similarly, the two-sided updates can be viewed as special versions of the three-sided one: 
e.g., if $T = T^r = \max(T^z,T^r,T^s)$, then \ref{eq:discre3} becomes equivalent to \ref{eq:discre6}. This means 
that the causal criteria for formulas \ref{eq:discre3}, \ref{eq:discre5} and \ref{eq:discre6} can be relaxed (the 
inequalities do not have to be strict). This relaxation is used to streamline the update strategy in the 
Numerical Implementation section.

In Figure~\ref{fig:update1} and the corresponding semi-Lagrangian discretization \ref{eq:discre1}, the ray-path 
is linearly approximated up to its intersection with the simplex $\mathbf{y}^z \mathbf{y}^r \mathbf{y}^s$ at a 
priori unknown depth $z + \xi_z \Delta$. An alternative \textit{explicit} semi-Lagrangian discretization can be 
obtained in the spirit of Figure~\ref{eq:raypath} by tracing the ray up to the pre-specified depth $z + \Delta$. 
In Figure~\ref{fig:update2}, we consider the DSR characteristic being straddled by 
$\mathbf{y}_*^z \mathbf{y}_*^r \mathbf{y}_*^s$, where 
$\mathbf{y}_*^z = (z+\Delta,r,s)$, $\mathbf{y}_*^r = (z+\Delta,r-\delta,s)$ and 
$\mathbf{y}_*^s = (z+\Delta,r,s+\delta)$. Denoting 
$\mathbf{\tilde{y}}_* = (z+\Delta,\tilde{r}_*,\tilde{s}_*)$ for the intersection point between DSR characteristic 
and simplex $\mathbf{y}_*^z \mathbf{y}_*^r \mathbf{y}_*^s$, we obtain the following discretization:
\begin{equation}
\label{eq:discre8}
T = \smash{\displaystyle\min_{\mathbf{\tilde{y}_*} \in \mathbf{y}_*^z \mathbf{y}_*^r \mathbf{y}_*^s}} 
\left\{ T (\mathbf{\tilde{y}}_*) + \frac{\sqrt{\Delta^2 + (r-\tilde{r}_*)^2}}{v (z,r)} 
+ \frac{\sqrt{\Delta^2 + (s-\tilde{s}_*)^2}}{v (z,s)} \right\}\;.
\end{equation}
One could perform the same analysis of \ref{eq:discre2} through \ref{eq:discre3} to equation \ref{eq:discre8}. 
For the sake of brevity, we omit the derivation and show the resulting explicit discretization scheme: 
\begin{equation}
\label{eq:discre9}
\frac{T-T_*^z}{\Delta} = 
\sqrt{\frac{1}{v_r^2} - \left( \frac{T_*^z-T_*^r}{\delta} \right)^2} + 
\sqrt{\frac{1}{v_s^2} - \left( \frac{T_*^z-T_*^s}{\delta} \right)^2}\;,
\end{equation}
where $T_*^i = T (\mathbf{y}_*^i)$ for $i = z,r,s$. More generally, to account for various possible directions 
of the diving wave, we can set $T_*^r = \min \left(T (z+\Delta,r-\delta,s), T (z+\Delta,r+\delta,s)\right)$ and 
$T_*^s = \min \left(T (z+\Delta,r,s-\delta), T (z+\Delta,r,s+\delta)\right)$.
\plot{update2}{width=0.6\textwidth}{An explicit discretization 
scheme. Compare with Figure~\ref{fig:update1}. The arrow 
again depicts a DSR characteristic with its root confined in the 
simplex $T_*^z T_*^r T_*^s$.}

Compared with equation \ref{eq:discre3}, equation \ref{eq:discre9} does not 
require solving a polynomial equation. Moreover, $T$ depends only on 
values in lower z-slices, which means that the system of equations can be solved in a single sweep in the $-z$ 
direction. Unfortunately, despite this efficiency on a fixed grid, the explicit discretization has a major 
disadvantage stemming from the requirement that the characteristic should be straddled by $y^z_* y^r_* y^s_*$. 
This imposes an upper bound on $\mu$ based on the slope of the diving wave. Moreover, since every diving ray is 
horizontal at its lowest point, the convergence is possible only if $\mu \to 0$ under mesh refinement. In 
practice, this means that the results are meaningful only if $\Delta$ is significantly smaller than $\delta$. We 
note that restrictive stability conditions also arise for time-dependent Hamilton-Jacobi equations of optimal 
control, where sufficiently strong inhomogenieties can make nonlinear/implicit schemes preferable to the usual 
linear/explicit approach \cite[]{vladimirsky}.

The above analysis also applies to the first branch of the DSR eikonal equation in Figure~\ref{fig:root}. 
However, in the discretized $(z,r,s)$ domain, the slowness vectors at $s$ and $r$ are always 
aligned in the $z$ direction, either upward or downward. For this reason, there is no DSR characteristic that 
accounts for the second and third scenarios. We will refer to the first and last branches in 
Figure~\ref{fig:root} as \textit{causal branches} of DSR eikonal equation, and the left-over two as 
\textit{non-causal branches}.
\plot{search}{width=0.6\textwidth}{When slowness vectors at $s$ 
and $r$ are pointing in the opposite directions, the ray-path must 
intersect with line $s-r$ at certain point $q$.}

Note that when the slowness vectors at $s$ and $r$ are pointing in opposite directions, there must be at least 
one intersection of the ray-path with the $z$ depth level in-between. As shown in Figure~\ref{fig:search}, ray 
segments between these intersections fall into the category of causal branches. Thus a search process for the 
intersections is sufficient in recovering the non-causal branches during forward modeling. Moreover, because we 
are interested in first-breaks only, the minimum traveltime requirement allows us to search for only one 
intersection, such as $q$ denoted in Figure~\ref{fig:search}: 
\begin{equation}
\label{eq:search}
T (z,r,s) = \smash{\displaystyle\min_{q \in (s,r)}}
\left\{ T (z,q,s) + T (z,r,q) \right\}\;.
\end{equation}
Other possible intersections in intervals $(s,q)$ and $(q,r)$ have already been recovered when computing 
$T (z,q,s)$ and $T (z,r,q)$, as long as we enable the intersection searching from the beginning of forward 
modeling. The traveltime of non-causal branches from equation \ref{eq:search} is then compared with that from 
causal branches, and the smaller one should be kept. 

Unfortunately, this search routine induces considerable computational cost. Moreover, we note 
that, under a dominant diving waves assumption, the first DSR branch, despite being 
causal, becomes useless if \ref{eq:search} is turned-off.

\appendix
\section{Appendix B: Frech\'{e}t derivative of DSR tomography}

To derive the Frech\'{e}t derivative, we start from equations \ref{eq:raypath2} and 
\ref{eq:chain}. Applying $\partial / \partial w_s$ to both sides of equation \ref{eq:raypath2} results in 
\begin{eqnarray}
\label{eq:dsrdds}
D_z \frac{\partial t^{dsr}}{\partial w_s} &=& 
- \frac{1}{2 \sqrt{w_s - D_s t^{dsr} \cdot D_s t^{dsr}}} \nonumber \\
&+& \left(\frac{D_s t^{dsr} \cdot D_s}{\sqrt{w_s - D_s t^{dsr} \cdot D_s t^{dsr}}} 
+ \frac{D_r t^{dsr} \cdot D_r}{\sqrt{w_r - D_r t^{dsr} \cdot D_r t^{dsr}}}\right) 
\frac{\partial t^{dsr}}{\partial w_s}\;.
\end{eqnarray}
Analogously
\begin{eqnarray}
\label{eq:dsrddr}
D_z \frac{\partial t^{dsr}}{\partial w_r} &=& 
- \frac{1}{2 \sqrt{w_r - D_r t^{dsr} \cdot D_r t^{dsr}}} \nonumber \\
&+& \left(\frac{D_s t^{dsr} \cdot D_s}{\sqrt{w_s - D_s t^{dsr} \cdot D_s t^{dsr}}} 
+ \frac{D_r t^{dsr} \cdot D_r}{\sqrt{w_r - D_r t^{dsr} \cdot D_r t^{dsr}}}\right) 
\frac{\partial t^{dsr}}{\partial w_r}\;.
\end{eqnarray}
Inserting equations \ref{eq:dsrdds} and \ref{eq:dsrddr} into \ref{eq:chain} and regrouping the terms, we prove 
equation \ref{eq:dsrderiv}
\begin{equation}
\label{eq:dsrderivapp}
J^{dsr} = B^{-1} (C_s + C_r)\;,
\end{equation}
where 
\begin{equation}
\label{eq:dsrderiv1}
B = D_z 
- \left( \frac{D_s t^{dsr} \cdot D_s}{\sqrt{w_s - D_s t^{dsr} \cdot D_s t^{dsr}}} \right) 
- \left( \frac{D_r t^{dsr} \cdot D_r}{\sqrt{w_r - D_r t^{dsr} \cdot D_r t^{dsr}}} \right)\;,
\end{equation}
and
\begin{equation}
\label{eq:dsrderiv2}
C_s = 
- \frac{1}{2 \sqrt{w_s - D_s t^{dsr} \cdot D_s t^{dsr}}}\, 
\frac{\partial w_s}{\partial w}\;;
\end{equation}
\begin{equation}
\label{eq:dsrderiv3}
C_r = 
- \frac{1}{2 \sqrt{w_r - D_r t^{dsr} \cdot D_r t^{dsr}}}\, 
\frac{\partial w_r}{\partial w}\;.
\end{equation}

At the singularity of DSR eikonal equation, the operators $B$, $C_s$ and $C_r$ take simpler forms 
and can be derived directly from equation \ref{eq:singular}.

\appendix
\section{Appendix C: Adjoint-state tomography with upwind finite-differences}

Following Appendix A, we let $T_i^{j,k}$ in the DSR case be the traveltime at vertex $(z_i,r_j,s_k)$ and 
approximate $D_z$ in equation \ref{eq:difference} by a one-sided finite-difference
\begin{equation}
\label{eq:upwind1}
D_z^{\pm} T_i^{j,k} = 
\pm \frac{T_{i \pm 1}^{j,k} - T_i^{j,k}}{\Delta}\;,
\end{equation}
where the $\pm$ sign corresponds to the two neighbors of $T_i^{j,k}$ in $z$ direction. An upwind scheme 
\cite[]{franklin} picks the sign by
\begin{equation}
\label{eq:upwind2}
D_z T_i^{j,k} = 
\max \left( D_z^- T_i^{j,k}, -D_z^+ T_i^{j,k}, 0 \right)\;.
\end{equation}
The above strategy can be applied to $D_r$ and $D_s$ straightforwardly. For the shot-indexed eikonal equation 
\ref{eq:eikonal2}, we approximate $D_x$ with the same upwind method while $T$ in this case is indexed for $z$ 
and $x$.

For a Cartesian ordering of the discretized $T$, i.e. vector $\mathbf{t}$, the discretized operators 
$D_m T \cdot D_m$ with $m = z,x,r,s$ are matrices. Thanks to upwind finite-differences, these matrices are 
sparse and contain only two non-zero entries per row. For instance, suppose $T_i^{j,k}$ has 
its upwind neighbor in $z$ at $T_{i-1}^{j,k}$, then 
\begin{equation}
\label{eq:upwind3}
D_z T \cdot D_z = \left[
\begin{array}{cccccc}
\ddots & & & & & \\
& \ddots & & & & \\
& & \ddots & & & \\
& -\kappa_z & & \kappa_z & & \\
& & & & & \ddots
\end{array} \right]\;,
\end{equation}
where
\begin{equation}
\label{eq:kappa}
\kappa_z \equiv  \frac{D_z T_i^{j,k}}{\Delta} = \frac{T_i^{j,k} - T_{i-1}^{j,k}}{\Delta^2}\;.
\end{equation}
Definitions of $\kappa_r$, $\kappa_s$ and $\kappa_x$ follow their upwind approximations, respectively. In 
equation \ref{eq:upwind3}, $\pm \kappa_z$ are located in the same row as that of $T_i^{j,k}$ in $\mathbf{t}$. 
While $\kappa_z$ sits on the diagonal, $- \kappa_z$ has a column index equals to the row of $T_{i-1}^{j,k}$ in 
$\mathbf{t}$. At $T = 0$, there is no upwind neighbor and the corresponding row contains 
all zeros.
 
We can sort entries of $\mathbf{t}$ by their values in an increasing order, which 
equivalently performs column-wise permutations to $D_m T \cdot D_m$. The results are lower triangular matrices. 
In fact, during FMM forward modeling, such an \textit{upwind ordering} is maintained and updated by the priority 
queue and thus can be conveniently imported for usage here.

Note that the summation and subtraction of two (or more) $D_m T \cdot D_m$ matrices are still lower 
triangular. These matrices are also invertible, except for a singularity at $T = 0$ where we may set the entries 
to be zero. Naturally, the inverted matrices are also lower triangular. One example is the linearized eikonal 
equation that gives rise to equation \ref{eq:stdderiv1}. Following notation \ref{eq:kappa} and assuming the 
upwind neighbors of $T_i^j$ are $T_{i-1}^j$ and $T_i^{j-1}$, the linearized equation \ref{eq:eikonal2} reads
\begin{equation}
\label{eq:lineiko}
2 \kappa_z (\delta T_i^j - \delta T_{i-1}^j) + 
2 \kappa_x (\delta T_i^j - \delta T_i^{j-1}) = \delta w_i^j\;.
\end{equation}
After regrouping the terms, we get
\begin{equation}
\label{eq:lineiko2}
\delta T_i^j = 
\frac{2 \kappa_z \delta T_{i-1}^j + 2 \kappa_x \delta T_i^{j-1} + \delta w_i^j}{2\,(\kappa_z + \kappa_x)}\;.
\end{equation}
Equation \ref{eq:lineiko2} means the inverse of the operator $D_z T \cdot D_z + D_x T \cdot D_x$ does not 
need to be computed by an explicit matrix inversion. Instead, we can perform its application to a 
vector by a single sweep based on causal upwind ordering. The same conclusion can be drawn for operator 
\ref{eq:dsrderiv1}.

Lastly, the adjoint-state calculations implied by equations \ref{eq:linearstd} and 
\ref{eq:lineardsr} multiply the transpose of these inverse matrices with the data residual. The matrix 
transposition leads to upper triangular matrices. Accordingly, we solve the linear system with anti-causal 
\textit{downwind ordering} that follows a decrease in values of $\mathbf{t}$.

\bibliographystyle{seg}
\bibliography{dsrtomof}
