\published{Geophysics, 76, no. 6, P23-P34, (2011)}

\title{Time-frequency analysis of seismic data using local attributes}

\renewcommand{\thefootnote}{\fnsymbol{footnote}}

%\ms{GEO-2011}

\address{
\footnotemark[1] State Key Laboratory of Petroleum Resources and Prospecting\\
China University of Petroleum\\
Beijing, China \\
\footnotemark[2] Bureau of Economic Geology,\\
John A. and Katherine G. Jackson School of Geosciences \\
The University of Texas at Austin \\
University Station, Box X \\
Austin, TX, USA, 78713-8924 \\
}

\author{Guochang Liu\footnotemark[1], Sergey Fomel\footnotemark[2], Xiaohong Chen\footnotemark[1]}

\footer{GEO-2011}
\lefthead{Liu etc.}
\righthead{Time-frequency analysis}
\maketitle

\begin{abstract}

Time-frequency analysis is an important technology in 
seismic data processing and interpretation. 
To localize frequency content in time, we have developed a novel method
for computing a time-frequency map for nonstationary signals
using an iterative inversion framework. We calculated
time-varying Fourier coefficients by solving a least-squares
problem that uses regularized nonstationary regression. We
defined the time-frequency map as the norm of time-varying
coefficients. Time-varying average frequency of the seismic
data can also be estimated from the time-frequency map calculated
by our method. We tested the method on benchmark
synthetic signals and compared it with the well-known Stransform.
Two field data examples showed applications
of the proposed method for delineation of sand channels
and for detection of low-frequency anomalies.

\end{abstract}

\section{Introduction}

Time-frequency decomposition maps a 1D signal into a 2D signal
of time and frequency, and describes how the spectral content of the
signal changes with time. Time-frequency analysis has been used
extensively in seismic data processing and interpretation, including
attenuation measurement  \cite[]{Reine2009}, direct hydrocarbon
detection \cite[]{Castagna2003}, and stratigraphic mapping
\cite[]{Partyka1999}. The widely used short-time Fourier transform
(STFT) method produces a time-frequency spectrum by taking the
Fourier transform of data windows \cite[]{Cohen1995}, which leads to a
tradeoff between temporal and spectral resolution.

Over the past two decades, wavelet-based methods have been
applied to time-frequency analysis of seismic data.  \cite{Chakraborty1995} 
compare wavelet-based with Fourier-based methods
for performing time-frequency analysis on seismic data and
showed that the wavelet-based method improves spectral resolution.
The continuous-wavelet transform (CWT) provides a time-scale
map, known as a scalogram \cite[]{Rioul1991}, rather than
a time-frequency spectrum. Because a scale represents a frequency
band, \cite{Hlawatsch1992} choose their scale
to be inversely proportional to the center frequency of the wavelet,
which allowed them to transform their scalogram into a time-frequency
map. \cite{Sinha2005} provide a novel method of obtaining
such a time-frequency map by taking a Fourier transform of
the inverse CWT.

Wigner-Ville distribution (WVD) \cite[]{Wigner1932} represents
time-frequency components by using the reverse of the signal as
an analysis window function. WVD varies resolution in the
time-frequency plane by providing good temporal resolution at high
frequencies and good frequency resolution at a low-frequency
\cite[]{Cohen1989}. Applications of WVD are hindered by cross-term
interference, which can be suppressed by using appropriate kernel
functions. A smoothed pseudo-WVD with independent time and
frequency functions as kernels achieves a relatively good resolution
in both time and frequency \cite[]{Li2008}. \cite{Wu2009} employed 
smoothed pseudo-WVD with a Gaussian kernel
function to reduce cross-term interference.

The S-transform is an invertible time-frequency analysis technique
that combines elements of wavelet transforms and short-time
Fourier transforms. The S-transform with an arbitrary and varying
shape was applied to seismic data analysis by \cite{Pinnegar2003}. 
Matching pursuit \cite[]{Mallat1993},
which was also applied in seismic signal analysis \cite[]{Chakraborty1995, Castagna2003, Liu2004, 
Liu2005, Wang2007}, decomposes a seismic trace into
a series of wavelets that belong to a comprehensive dictionary of
functions. These wavelets are selected so as to best match signal
structures. The spectrum of the signal is then the time-shifted
sum of each of the wavelets.

Local attributes \cite[]{Fomel2007a} can adaptively measure timevarying
seismic signal characteristics in the neighborhood of each
data point. Local attributes have been successfully applied to
seismic image registration \cite[]{Fomel2009Jin}, phase detection
\cite[]{van2009, Fomel2010van}, and stacking \cite[]{Liu2009fomel, Liu2011fomel}. A natural extension of local
attributes, regularized nonstationary regression, decomposes input
data into a number of nonstationary components \cite[]{Fomel2009, Liu2010, Liu2011chen}.

In this paper, we propose a new method of time-frequency analysis
in which we use time-varying Fourier coefficients to define a
time-frequency map. As in regularized nonstationary regression,
shaping regularization \cite[]{Fomel2007b} constrains continuity and
smoothness of the coefficients. Given the close connection between
Fourier transforms and the least-squares norm, the least-squares
approach to time-frequency analysis is not new, having been used
previously, for example, by \cite{Youn1985}. What is novel
about our approach is the use of regularization for explicitly
controlling the time resolution of time-frequency representations.

The paper is organized as follows.We first describe the proposed
method for time-frequency analysis. Then we show how to compute
the time-varying average frequency from the time-frequency map.
We use benchmark synthetic examples to test the performance of
the proposed method. Finally, we apply the proposed method to
channel detection and low-frequency anomaly detection in field
seismic data.

\section{Time-frequency analysis using local attributes}

The Fourier transform has found various applications in signal
analysis. The classic Fourier transform indicates the presence of different
frequencies within the analysis window, but does not show
where in that window the particular frequency components reside.
Localized frequency information can be obtained by computing the
Fourier transform with a temporally shifted window. Such an
approach to time-frequency analysis is known as the STFT \cite[]{Allen1977}. 
The window function is commonly parameterized by window
size, overlap, and taper. Once the window function has been
chosen for the STFT, temporal and spectral resolutions are fixed for
the entire time-frequency map.
 
The S-transform \cite[]{Stockwell1996} is similar to the STFT,
but with a Gaussian-shaped window whose width scales inversely
with frequency. The expression of the S-transform is
      \begin{equation}
          S(\tau,f)=\int_{-\infty}^{\infty}S(t)\left\{\frac{|f|}{\sqrt{2\pi}}\exp\left[ \frac{-f^2(\tau-t)^2}{2}\right ] \exp(-2\pi ift)\right\} dt,
        \label{eq:eq1}
      \end{equation} 
where $s(t)$ is a signal and $\tau$ is a parameter which controls the position
of the Gaussian window. The S-transform is conceptually a
hybrid of the STFT and wavelet analysis, containing elements of
both but having its own properties. Like STFT, the S-transform uses
a window to localize the complex Fourier sinusoid, but, unlike the
STFT and analogously to the wavelet transform, the width of the
window scales with frequency.

Consider a signal $f(x)$ on $[0,L]$. The Fourier series, assuming a
periodic extension of the boundary conditions, can be expressed as
      \begin{equation}
          f(t)\approx a_{0}+\sum_{k=1}^{\infty}\left[ a_{k} \cos \left( \frac{2k\pi t}{L} \right)+b_{k} \sin \left( \frac{2k\pi t}{L}\right)\right] ,
        \label{eq:eq2}
      \end{equation} 
where $a_{k}$ and $b_{K}$ are the series coefficients. The relationship between
$k$ and frequency $f$ is $k=Lf$. In the case of the discrete Fourier transform, 
frequency is finite. Letting $ \mathbf{\Psi}_{k}(t)$ represent the Fourier basis
      \begin{equation}
          \mathbf{\Psi}_{k}(t) = \left[\begin{array}{ll} \mathbf{\Psi}_{1k}(t) \\ \mathbf{\Psi}_{2k}(t) \end{array} \right] = \left[\begin{array}{ll} \cos \left( \frac{2k \pi t}{L} \right) \\ \sin \left( \frac{2k \pi t}{L}\right) \end{array} \right],
        \label{eq:eq3}
      \end{equation}
and $\mathbf{C}_k$ represent the series coefficients
      \begin{equation}
          \mathbf{C}_{k} = \left[a_{k} \quad b_{k}\right],
        \label{eq:eq4}
      \end{equation}
where $b_{0}=0$, equation~\ref{eq:eq1} can be written as
      \begin{equation}
          f(t)=\sum_{k=0}^{\infty}\mathbf{C}_{k}\mathbf{\Psi}_{k}(t).
        \label{eq:eq5}
      \end{equation}
If frequency is finite, the range of $k$ becomes $[0,N]$,
$N = k_{max} = Lf_{max}$. In linear notation, $\mathbf{C}_k$ can be obtained by solving
the least-squares problem
      \begin{equation}
          \min_{\mathbf{C}_{k}}\left| \left| f(t)-\sum_{k=0}^{N} \mathbf{C}_k \mathbf{\Psi}_{k}(t)\right| \right|_{2}^{2},
        \label{eq:eq6}
      \end{equation}
where $\left| \left| \right| \right|$denotes the squared $L-2$ norm of a function. By allowing
coefficients $\mathbf{C}_k$ to change with time $t$, we can define the timevarying
coefficients $\mathbf{C}_{k}(t)$ via the following least-squares problem
      \begin{equation}
          \min_{\mathbf{C}_{k}}\sum_{k=0}^{N}\left| \left| f(t)-\mathbf{C}_{k}\mathbf{\Psi}_{k}(t)\right|\right| _{2}^{2}.
        \label{eq:eq7}
      \end{equation}
The Fourier coefficient $\mathbf{C}_{k}(t)$ in equation ~\ref{eq:eq7} is a function of time $t$
and frequency $f = k∕L$ and $\mathbf{C}_{k}(t)=\left[a_{k} \quad b_{k}\right]$. The numerical
support of frequency f can be between zero and the Nyquist frequency
\cite[]{Cohen1995}, and the interval of frequency can be
$\Delta f = 1∕L $. In practical applications, the range of frequencies can
also be assigned by the user. In matrix notation, equation ~\ref{eq:eq7} can
be written as
      \begin{equation}
          \left[f(t) \quad f(t) ... \quad f(t)\right]^{T} \approx \left[D\left\{\mathbf{\Psi}_{1}(t) \quad ... \quad \mathbf{\Psi}_{N}(t)\right\}\right]\left[\mathbf{C}_{1}(t)\quad...\quad\mathbf{C}_{N}(t)\right]^{T},
        \label{eq:eq8}
      \end{equation}
where $D\{...\}$ denotes a diagonal matrix which is composed from
the elements of $\mathbf{Psi}_{k}(t)$.

The problem of minimization in equation ~\ref{eq:eq7} is mathematically
ill-posed because it is severely underconstrained: There are more
unknown variables than constraints. To solve this ill-posed problem,
we force the coefficients $\mathbf{C}_{k}(t)$ to have a desired behavior, such as
smoothness. With the addition of a regularization term, equation ~\ref{eq:eq7}
becomes
      \begin{equation}
          \min_{\mathbf{C}_{k}(t)}\sum_{k=0}^{N}\left|\left|f(t)-\mathbf{C}_{k}(t)\mathbf{\Psi}_{k}(t)\right|\right|_{2}^{2}+R\left[\mathbf{C}_{k}(t)\right],
        \label{eq:eq9}
      \end{equation}
where $R$ denotes the regularization operator. If we use classic
Tikhonov regularization \cite[]{Tikhonov1963}, equation ~\ref{eq:eq9} can be
written as
      \begin{equation}
          \min_{\mathbf{C}_{k}(t)}\sum_{k=0}^{N}\left|\left|f(t)-\mathbf{C}_{k}(t)\mathbf{\Psi}_{k}(t)\right|\right|_{2}^{2}+\varepsilon^{2} \sum_{k=0}^{N} \left|\left|\mathbf{D} \left[\mathbf{C}_{k}(t)\right]\right|\right|_{2}^{2},
        \label{eq:eq10}
      \end{equation}
where $\mathbf{D}$ is the Tikhonov regularization operator (roughening operator)
and $\varepsilon $ is a scalar regularization parameter.

Shaping regularization \cite[]{Fomel2007b} provides a particularly
convenient method of enforcing smoothness in iterative optimization
schemes. In the appendix, we review the method of shaping
regularization in detail. \cite{Fomel2009} used shaping regularization
to constrain the problem of nonstationary regression. In this paper,
we use shaping regularization, analogous to the problem of nonstationary
regression, to constrain estimated coefficients. We choose
our shaping operator to be Gaussian smoothing with an adjustable
radius. In shaping regularization, the radius of the Gaussian smoothing
operator controls the smoothness of coefficients $\mathbf{C}_{k}(t)$.

Once we obtain time-varying Fourier coefficients $\mathbf{C}_{k}(t)=\left[a_{k}(t) \quad b_{k}(t)\right]$, the time-frequency map is defined as
      \begin{equation}
          F(t,f=k/L)=\sqrt{a_{k}^{2}(t)+b_{k}^{2}(t)}.
        \label{eq:eq11}
      \end{equation}
It is also possible to do an invertible time-frequency transform, as
shown by Liu and \cite {Liu2010}.

\inputdir{syn}

Consider a simple signal (Figure~\ref{fig:s-0}), which includes three monophonic
components at 10, 20, and 30 Hz and two broad-band spikes
at 2 and 2.3 s. Time-frequency maps generated by the S-transform
and our method are shown, respectively, in Figure ~\ref{fig:st-0} and ~\ref{fig:proj-0}.
Frequency components appear to be represented well from low
to high frequencies in the proposed method. In comparison with
the S-transform, the proposed method provides superior resolution
for monophonic waves. At the lower frequencies, the S-transform
is as good as the proposed method on the spectral resolution for
monophonic waves. Note that the S-transform represents spikes better
at high frequencies. However, because the width of analysis window
is large at low frequencies, the S-transform provides poor time
resolution at low frequencies for spikes. When we use the shaping
operator as a regularization term, there is an edge effect from
smoothing, which appears near 0 s and 4 s (Figure ~\ref{fig:proj-0}). Figure ~\ref{fig:proj-0-30}
displays the time-frequency map using a different parameter
(30-point smoothing radius) to demonstrate adjustable time-frequency
representation of the proposed method.

\multiplot{4}{s-0,st-0,proj-0,proj-0-30}{width=0.5\columnwidth}{(a) Synthetic signal with three constant frequency 
components and two spikes. (b) Time-frequency map of the S-transform. 
(c) Time-frequency map of the proposed method with smoothing radius of 15 points. 
(d) Time-frequency map of the proposed method with smoothing radius of 30 points.
}
  
To show the effect of varying frequencies, we provide a composite
chirp signal (Figure ~\ref{fig:s-1}), which includes two parabolic frequencies,
each having a constant amplitude. Figure ~\ref{fig:st-1} and ~\ref{fig:proj-1} shows the
results of the S-transform and the proposed method with 10-point
smoothing radius, respectively. The two frequency components are
detected with higher time and frequency resolution by our method.
The S-transform has high spectral resolution near low frequencies,
but loses resolution at high frequencies. Resolution of the time-frequency
map deteriorates in both methods when the curvature
of the time-frequency curve becomes large (as indicated by
the arrow).

\multiplot{3}{s-1,st-1,proj-1}{width=0.5\columnwidth}{(a) Synthetic chirp signal with two parabolic frequency changes. 
(b) Time-frequency map of the S-transform. (c) Time-frequency
map of the proposed method.
}

In the proposed method, the smoothing radius used in shaping
regularization is a parameter which controls the smoothness of
the model. It is different from the method of the STFT and the
S-transform, in which division into local windows is applied to
the data to localize frequency content in time. In contrast to the
wavelet-based methods \cite[]{Chakraborty1995, Sinha2005, Wang2007}, our method, as a straightforward extension
of the classic Fourier analysis, is different because of the choice
of basis functions. The wavelet-based methods need to decide the
wavelets, which can be regarded as patterns of seismic data. The
closer the selected wavelets are to the true pattern of the input data,
the better the result of the time-frequency analysis. In this paper, we
use more general Fourier basis functions to compute the time-frequency
map.

\section{ESTIMATION OF TIME-VARYING AVERAGE FREQUENCY}

Seismic instantaneous frequency is the derivative of the instantaneous
phase
      \begin{equation}
          f_{i}(t)=\frac{1}{2\pi}\frac{d\phi (t)}{dt},
        \label{eq:eq12}
      \end{equation}
where $\phi (t)$ is the instantaneous phase. Instantaneous frequency can
be estimated directly using a discrete form of equation ~\ref{eq:eq12}. This
estimate is highly susceptible to noise. \cite{Fomel2007a} modified
the definition of instantaneous frequency to that of a local frequency
by recognizing it as a form of regularized inversion, and by using
regularization to constrain continuity and smoothness of the output.

Average frequency can be estimated from the time-frequency
map \cite[]{Claasen1980,Cohen1989, Hlawatsch1992, Steeghs2001, Sinha2009}. 
Average frequency at a given time is
      \begin{equation}
          f_{a}(t)=\frac{\int fF^{2}(f,t)df}{\int F^{2}(f,t)df},
        \label{eq:eq13}
      \end{equation}

where $F(f,t)$ is the time-frequency map. Average frequency measured
by equation ~\ref{eq:eq13} is the first moment along the frequency axis of
a time-frequency power spectrum. \cite{Saha1987} and \cite{Brian1993}
analyzed the relationship between instantaneous frequency
and the time-frequency map in detail. Extraction of the attributes
from the time-frequency map of the seismic trace leads to considerable
improvement of the signal-to-noise ratio of the attributes
\cite[]{Steeghs2001}. We therefore propose applying
equation 13 to our time-frequency map to compute the time-varying
average frequency.

\multiplot{2}{ref-6,s-6}{width=0.5\columnwidth}{(a) Random reflectivity series. (b) Synthetic seismic trace.
}

\multiplot{3}{spec,st-6,proj-6}{width=0.5\columnwidth}{(a) Theoretical time-frequency map, which is scaled by 
a maximum in the frequency axis. White and black lines indicate dominant
frequency and average frequency of Ricker wavelet, respectively. (b) Time-frequency map of the S-transform. 
(c) Time-frequency map of the proposed method.
}

\plot{lcf-6}{width=0.5\columnwidth}{Time-varying average-frequency estimation (blue solid
curve: estimated by our method; pink dashed curve: estimated
by S-transform; black dot-dashed curve: theoretical curve, which
is denoted by black line in Figure ~\ref{fig:spec}).
}


We used a synthetic nonstationary seismic trace to illustrate our
approach to estimating time-varying average frequency. Figure ~\ref{fig:s-6}
shows a synthetic seismic trace generated by nonstationary convolution
\cite[]{Margrave1998} of a random reflectivity series (Figure ~\ref{fig:ref-6})
using a Ricker wavelet, the dominant frequency of which is a function
of time, $f_{d}=25t^{2}+15$. Figure ~\ref{fig:spec} shows the scaled spectrum
of the Ricker wavelet. Both the dominant frequency (white line in
Figure ~\ref{fig:spec} and the bandwidth increase with time. We computed the
average frequency (black line in Figure ~\ref{fig:spec} using equation ~\ref{eq:eq13} from
the scaled spectrum of Ricker wavelets. We note that average
frequency is larger than the dominant frequency at high frequencies
for Ricker wavelets.

We generated the time-frequency map of the synthetic nonstationary
seismic trace using the S-transform (Figure ~\ref{fig:st-6}) and
the proposed method with a 10-point smoothing radius (Figure ~\ref{fig:proj-6}).
We observe that the time-frequency map by the proposed method
has a bandwidth more similar to that of the time-frequency map of
the Ricker wavelet (Figure ~\ref{fig:spec}), especially at the high frequencies.
We estimated time-varying average frequency curves from the
time-frequency map by the proposed method (blue solid curve in
Figure ~\ref{fig:lcf-6}) and S-transform (pink dashed curve in Figure ~\ref{fig:lcf-6}), respectively.
Compared with the theoretical curve (black dashed curve in
Figure ~\ref{fig:lcf-6}), which was computed using equation ~\ref{eq:eq13} on the scaled
spectrum of Ricker wavelets (Figure ~\ref{fig:spec}), the time-varying average
frequency estimated by the proposed method is closer to the
theoretical one.

\section{Examples}

We demonstrate the effectiveness of the proposed time-frequency
analysis on a benchmark synthetic data set and two field data sets.
We first use a synthetic seismic trace to illustrate how the proposed
method obtains a time-frequency map, and then we test our method
on two field data sets with applications to detecting channels and
low-frequency anomalies.

 \subsection{Synthetic data}
 
\multiplot{3}{s-2,st-2,proj-2}{width=0.5\columnwidth}{(a) Synthetic seismic trace. (b) Time-frequency map of the S-transform. (c) Time-frequency map of the proposed method.
}

A synthetic seismic trace (Figure ~\ref{fig:s-2}) was obtained by adding
Ricker wavelets with different frequencies and time shifts. The first
event is an isolated wavelet with a frequency of 30 Hz, and the
second event consists of a 15-Hz wavelet and a 60-Hz wavelet
overlapping in time. The last event is a superposition of two 50-Hz
wavelets with different arrival times. Figure ~\ref{fig:st-2} and ~\ref{fig:proj-2} show the
time-frequency maps by the S-transform and our method with
15-point smoothing radius, respectively. From the time-frequency
map of the first event at 0.2 s, we found that time duration is long
at low-frequency in the S-transform. The proposed method produces
a more temporally limited ellipsoid spectrum for the first
event. Note that the proposed method has higher spatial resolution
at 0.6 s and temporal resolution at 1 s. This simple test shows that
our method can be effective in representing

\subsection{3D Gulf of Mexico data for channel detection}

\inputdir{chev}

Detecting channel structures is a common application of spectral
decomposition \cite[]{Partyka1999}. Different frequency slices show
different stratigraphic features. Figure~\ref{fig:chev} shows field
data from the Gulf of Mexico \cite[]{Lomask2006}. To generate stratal
slices \cite[]{Zeng1998}, we flattened this data set using predictive
painting \cite[]{Fomel2010}. The flattened data set is shown in
Figure~\ref{fig:flats}. The flattening procedure can remove structural
distortions and allows the interpreter to see geologic features as
they were emplaced \cite[]{Lomask2006}. Predictive painting does a
careful job of restoring true geological frequencies, as is evident
from vertical sections. Figure~\ref{fig:spec} shows the average
spectrum of the data set before flattening (dashed blue curve) and
after flattening (solid red curve).  We calculated the spectral
decomposition of flattened data using the proposed method with a
five-point smoothing
radius. Figure~\ref{fig:slice,slice10,slice20,slice30,slice40,slice50}
shows horizon slices from six different volumes at the same level in
reference time. The horizontal blue lines on Figure~\ref{fig:flats}b
and~\ref{fig:flats}c identify where the time slice is. Comparing
seismic amplitudes
(Figure~\ref{fig:slice,slice10,slice20,slice30,slice40,slice50}a) with
spectral decomposition at different frequencies
(Figures~\ref{fig:slice,slice10,slice20,slice30,slice40,slice50}b-f),
several channels in 30 Hz slice are easily visible. The high-frequency
spectral-decomposition maps, such as the 30-, 40-, and 50-Hz slices,
can highlight detailed geologic features.

\plot{chev}{height=0.85\textheight}{Seismic image from Gulf of
  Mexico. (a) Time slice. (b) Inline section. (c) Crossline
  section. Blue lines in (a) identify the location of inline and
  crossline sections. The horizontal blue lines in (b) and (c)
  identify where the time slice is.}

\plot{flats}{height=0.85\textheight}{Seismic image from
  Figure~\ref{fig:chev} after flattening by predictive painting. (a)
  Time slice. (b) Inline section. (c) Crossline section. The blue
  lines on (a) identify the location of inline and crossline
  sections. The horizontal blue lines on (b) and (c) identify where
  the time slice is.}

\plot{spec}{width=\textwidth}{Average data spectrum before flattening (dashed blue
curve) and after flattening (solid red curve).}

\multiplot{6}{slice,slice10,slice20,slice30,slice40,slice50}{width=0.4\textwidth}{Comparison
  of horizon slices, from four different volumes at the same level:
  (a) conventional amplitude; and spectraldecomposition at (b) 10 Hz,
  (c) 20 Hz, (d) 30 Hz, (e) 40 Hz, and (f) 50 Hz. The slice of 30 Hz
  displays most clearly visible channel features.}

 \subsection{2D data example for low-frequency anomaly detection}
 
\inputdir{lowf}

Low-frequency anomalies are often attributed to abnormally high
attenuation in gas-filled reservoirs and can be used as a hydrocarbon
indicator \cite[]{Castagna2003}. A number of studies have investigated
possible mechanisms of low-frequency anomalies associated
with hydrocarbon reservoirs, but no adequate explanation is
accepted absolutely \cite[]{Ebrom2004, Kazemeini2009}.

Figure ~\ref{fig:old-1} shows a poststack field data set with a bandwidth of
about 10–150 Hz (Figure ~\ref{fig:spectra}).We used our proposed method with a
ten-point smoothing radius to generate a time-frequency spectral
map. Figure ~\ref{fig:tf-l,tf-s} shows the time-frequency map of one trace from
the field data set. Note that our method has a higher temporal resolution
than that of the S-transform, especially at 0.6-0.8 s.We can
observe a general decay of frequency with time caused by seismic
attenuation from the time-frequency map of one trace (Figure 13).
The low-frequency anomaly is shown at 1.2-1.3 s (arrows).
Figures ~\ref{fig:s1,s2,s3} and ~\ref{fig:l1,l2,l3} show single-frequency sections at 15, 31, and
70 Hz from two methods, the S-transform and the proposed method.
We further found that the proposed method provides higher temporal
and spatial resolution, especially in low-frequency sections.
Figures ~\ref{fig:s1} and ~\ref{fig:l1} both show high-amplitude, low-frequency
anomalies (ellipse). However, these anomalies gradually disappear
in the high-frequency section (Figures ~\ref{fig:s2} and ~\ref{fig:s3} and ~\ref{fig:l2}
and ~\ref{fig:s3}). Note that the spatial resolution of low-frequency anomalies
in the proposed method (Figure ~\ref{fig:l1}) is higher than that in the
S-transform (Figure ~\ref{fig:s1}). We also computed the time-varying
average frequency section for this example using equation ~\ref{eq:eq13}
(Figure ~\ref{fig:lcf-l}). We find that the average frequency is about 60 Hz
in shallow layers, but 25 Hz in deep layers. The average frequency
of low-frequency anomalies is very low, just about 15 Hz. This
example shows that comparison of single low-frequency sections
from the proposed method is able to detect low-frequency anomalies
that might be caused by hydrocarbons, as demonstrated in other
studies \cite[]{Castagna2003, Korneev2004, Zhenhua2008}. What we demonstrate by this example is that the proposed
method can be used to detect low-frequency anomalies in seismic
sections. Well control is generally needed to interpret this section
more accurately.

\plot{old-1}{width=0.5\columnwidth}{A field marine seismic data set.
}

\plot{spectra}{width=0.5\columnwidth}{The averaged Fourier spectra of the field data.
}

\multiplot{2}{tf-l,tf-s}{width=0.25\columnwidth}{Time-frequency map of one trace (denoted by black line
in Figure ~\ref{fig:spectra}). (a) S-transform; (b) the proposed method.
}

\multiplot{3}{s1,s2,s3}{width=0.5\columnwidth}{Comparison of common-frequency slices from
S-transform; (a) 10 Hz, (b) 31 Hz, and (c) 70 Hz. The lowfrequency
abnormality is indicated by an ellipse.
}

\multiplot{3}{l1,l2,l3}{width=0.5\columnwidth}{Comparison of common-frequency slices from the proposed
method; (a) 10 Hz, (b) 31 Hz, and (c) 70 Hz. The lowfrequency
abnormity is indicated by an ellipse.
}

\plot{lcf-l}{width=0.5\columnwidth}{Estimated time-varying average frequency of the field
data from time-frequency map.
}


\section{Conclusion}

We have presented a novel numerical method for computing
time-frequency representation using least-squares inversion with
shaping regularization. This time-frequency-analysis technology
can be applied to nonstationary signal analysis. The method is a
straightforward extension of the classic Fourier analysis. The
parameter used in shaping regularization, the radius of the Gaussian
smoothing operator, controls the smoothness of time-varying
Fourier coefficients. The smooth time-varying average frequency
attribute can also be estimated from the proposed time-frequency
analysis technology. We have demonstrated applications of the
proposed time-frequency analysis for detecting channels and lowfrequency
anomalies in seismic images.
Finally, we realize that many different algorithms are capable of
computing time-frequency representations. We have provided some
comparisons of our method with one of them (the S-transform) but
cannot make the comparison exhaustive and do not claim that our
method will necessarily perform better in all practical situations.
Both approaches to time-frequency analysis have advantages and
disadvantages. What we see as the main advantages of our method
are its conceptual simplicity, computational efficiency, and explicit
controls on time and frequency resolution.

\section{Acknowledgments}

We would like to thank Thomas J. Browaeys, Yang Liu, and
Mirko van der Baan for inspiring discussions, and the China
Scholarship Council (grant 2007U44003) for partial support of
the first author’s research. This work is financially supported by
the National Basic Research Program of China (973 program, grant
2007CB209606) and by the National High Technology Research
and Development Program of China (863 program, grant
2006AA09A102-09).We also thank associate editor Kurt J. Marfurt
and three anonymous reviewers for their constructive comments,
which improved the quality of the paper.

\appendix
\section{SHAPING REGULARIZATION FOR INVERSE PROBLEMS}

In this appendix, we review the theory of shaping regularization
in inversion problems. \cite{Fomel2007b} introduces shaping regularization,
a general method of imposing regularization constraints. A
shaping operator provides an explicit mapping of the model to the
space of acceptable models.

Consider a system of linear equation given as $\mathbf{Ax} = \mathbf{b}$, where $\mathbf{A}$ is
a forward-modeling operator, $\mathbf{x}$ is the model, and $\mathbf{b}$ is the data. By
equation ~\ref{eq:eq8}, we can find that the proposed time-frequency decomposition
can be written as the form $\mathbf{Ax} = \mathbf{b}$. The standard regularized
least-squares approach to solving this equation seeks to
minimize $\left| \left| \mathbf{Ax} = \mathbf{b} \right| \right| _{2}^{2}+\varepsilon ^{2} \left| \left| \mathbf{Dx} \right| \right| _{2}^{2}$, where $\mathbf{D}$ is the Tikhonov regularization
operator \cite[]{Tikhonov1963} and $\varepsilon$ is scaling.

The formal solution, denoted by $\hat{\mathbf{x}}$, is given by
      \begin{equation}
          \hat{x}=\left( \mathbf{A}^{T}\mathbf{A}+\varepsilon ^{2}\mathbf{D}^{T}\mathbf{D}\right)^{-1}\mathbf{A}^{T}\mathbf{b},
        \label{eq:eqa1}
      \end{equation}
where $\mathbf{A}^{T}$ denotes the adjoint operator. \cite{Fomel2007b} defined a
relation between a shaping operator S and a regularization operator
$\mathbf{D}$ as
      \begin{equation}
          \mathbf{S}=\left(\mathbf{I}+\varepsilon ^{2}\mathbf{D}^{T}\mathbf{D}\right)^{-1}.
        \label{eq:eqa2}
      \end{equation}
Substituting equation  ~\ref{eq:eqa2} into equation  ~\ref{eq:eqa1} yields a formal
solution of the estimation problem regularization by shaping
      \begin{equation}
          \mathbf{\hat{x}}=\left(\mathbf{A}^{T}\mathbf{A}+\mathbf{S}^{-1} -\mathbf{I}\right)^{-1}\mathbf{A}^{T}\mathbf{b}=\left[\mathbf{I}+\mathbf{S}\left(\mathbf{A}^{T}\mathbf{A}-\mathbf{I}\right)\right]^{-1}\mathbf{S}\mathbf{A}^{T}\mathbf{b}.
        \label{eq:eqa3}
      \end{equation}

Introducing scaling of $\mathbf{A}$ by $1∕λ$ in equation ~\ref{eq:eqa3} , we obtain
      \begin{equation}
          \hat{\mathbf{x}}=\left[\lambda ^{2}\mathbf{I}+\mathbf{S}\left(\mathbf{A}^{T}\mathbf{A}-\lambda ^{2}\mathbf{I}\right)\right]^{-1}\mathbf{S}\mathbf{A}^{T}\mathbf{b}.
        \label{eq:eqa4}
      \end{equation}
The conjugate-gradient method can be used to compute the inversion
in equation ~\ref{eq:eqa4} iteratively. As shown by \cite{Fomel2009}, the
iterative convergence for inversion in equation ~\ref{eq:eqa4} can be dramatically
faster that the one in equation ~\ref{eq:eqa1}.

The main advantage of shaping regularization is the relative ease
of controlling the selection of $\lambda$ and $S\mathbf{S}$ in comparison with $\varepsilon$ and $\mathbf{D}$
cite[]{Fomel2009}. In this paper, we choose $\lambda$ to be the median value of
$\mathbf{\Psi_{k}(t)}$ and the shaping operator to be a Gaussian smoothing operator.
The Gaussian smoothing operator is a convolution operator with
the Gaussian function that is used to smooth images and remove
detail and noise. In this sense it is similar to the mean filter, but
it uses a different kernel to represent the shape of a Gaussian
(bell-shaped) hump. The Gaussian function in 1D has the form,
      \begin{equation}
          G(x)=\frac{1}{\sqrt{2 \pi \sigma}} \exp \left( -\frac{x^{2}}{2\sigma^{2}}\right).
        \label{eq:eqa5}
      \end{equation}
One can implement Gaussian smoothing by Gaussian filtering in
either the frequency domain or time domain. \cite{Fomel2007b} shows
that repeated application of triangle smoothing can also be used to
implement Gaussian smoothing efficiently, in which case the only
additional parameter is the radius of the triangle smoothing
operator. In this paper, we use the repeated triangle smoothing operator
to implement the Gaussian smoothing operator.

The computational cost of generating a time-frequency representation
with our method is $O\left(N_{t}N_{f}N_{iter}/N_{p}\right)$, where $N_{t}$ is the number
of time samples, $N_{f}$ is the number of frequencies, $N_{iter}$ is the number
of conjugate-gradient iterations, $N_{p}$ is the number of processors
used to process different frequencies in parallel, and $N_{iter}$ decreases
with the increase of smoothing and is typically around 10.


\bibliographystyle{seg}
\bibliography{paper}

