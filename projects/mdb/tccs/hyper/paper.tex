\published{Geophysics, 75, no. 2, U9-U18, (2010)}
\title{Generalized nonhyperbolic moveout approximation}

\renewcommand{\thefootnote}{\fnsymbol{footnote}}

\lefthead{Fomel \& Stovas}
\righthead{Generalized moveout approximation}
\ms{GEO-2009-0309}

\author{Sergey Fomel\footnotemark[1] and Alexey Stovas\footnotemark[2]}

\address{
\footnotemark[1]Bureau of Economic Geology, \\
John A. and Katherine G. Jackson School of Geosciences \\
The University of Texas at Austin \\
University Station, Box X \\
Austin, TX 78713-8972 \\
USA \\
sergey.fomel@beg.utexas.edu \\
\footnotemark[2]Department of Petroleum Engineering and
Applied Geophysics \\
Norwegian University of Science and Technology (NTNU) \\
S.P. Andersenvei 15A \\
7491 Trondheim \\
Norway \\
alexey.stovas@ntnu.no
}

\maketitle

\begin{abstract}
  Reflection moveout approximations are commonly used for velocity
  analysis, stacking, and time migration.  We introduce a novel
  functional form for approximating the moveout of reflection
  traveltimes at large offsets. While the classic hyperbolic
  approximation uses only two parameters (the zero-offset time and the
  moveout velocity), our form involves five parameters, which can be
  determined, in a known medium, from zero-offset computations and
  from tracing one non-zero-offset ray. We call it a generalized
  approximation because it reduces to some known three-parameter forms
  (the shifted hyperbola of Malovichko, de Baziliere, and Castle; the
  Pad\'{e} approximation of Alkhalifah and Tsvankin; and others) with
  a particular choice of coefficients. By testing the accuracy of the
  proposed approximation with analytical and numerical examples, we
  show that it can bring several-orders-of-magnitude improvement in
  accuracy at large offsets compared to known analytical
  approximations, which makes it as good as exact for many practical
  purposes.
\end{abstract}

\section{INTRODUCTION}

Reflection moveout approximations are commonly used for velocity
analysis, stacking, and time migration \cite[]{yilmaz}.  The
reflection traveltime as a function of the source-receiver offset has
a well-known hyperbolic form in the case of plane reflectors in
homogeneous isotropic (or elliptically anisotropic) overburden. A
hyperbolic behavior of the PP moveout is always valid around the zero
offset thanks to the source-receiver reciprocity and the first-order
Taylor series expansion. However, any deviations from this simple
model may cause nonhyperbolic behavior at large offsets
\cite[]{fandg}.

Considerable research effort has been devoted to developing
nonhyperbolic moveout approximations in both isotropic and anisotropic
media. The work on isotropic approximations goes back to
\cite{bolshykh}, \cite{GEO34-06-08590881}, \cite{malovichko},
\cite{GEO53-02-01430157}, \cite{GEO59-06-09830999}, and
others. \cite{fowler} provides a comprehensive review of many
different approximations developed for non-hyperbolic moveout in
anisotropic (VTI -- vertically transversally isotropic) media. A
particularly simple ``velocity acceleration'' model for nonhyperbolic
moveout is suggested by
\cite{taner1,taner2}. \cite{causse} approximates nonhyperbolic moveout
by expanding it into a sum of basis functions. \cite{douma} and
\cite{douma2} build an accurate moveout approximation by using
rational interpolation between several rays.

In this paper, we propose a general functional form for nonhyperbolic
approximations that can be applied to different kinds of seismic
media. The proposed form includes five coefficients as opposed to two
coefficients in the classic hyperbolic approximation. In certain
cases, the number of coefficients can be reduced. In the case of a
homogeneous VTI medium and the ``acoustic approximation'' of
\cite{GEO63-02-06230631}, our approximation becomes identical to the
one proposed previously by \cite{GPR52-03-02470259}. In the general
case, determining the optimal coefficients requires tracing of only
one non-zero-offset ray.

Using analytical ray-tracing solutions and numerical experiments, we
compare the accuracy of our approximation with the accuracy of other
known approximations and discover an improvement in accuracy of
several orders of magnitude. Potential applications of the new
approximation include velocity analysis and time-domain imaging.

\section{NONHYPERBOLIC MOVEOUT APPROXIMATION}

Let $t(x)$ represent the reflection traveltime as a function of the
source-receiver offset $x$. We propose the following general form of
the moveout approximation:
\begin{equation}
\label{eq:mo}
t^2(x) \approx (1-\xi)\,(t_0^2+a\,x^2) + \xi\,\sqrt{t_0^4 + 2\,b\,t_0^2\,x^2 + c\,x^4}\;.
\end{equation}
The five parameters $a$, $b$, $c$, $\xi$, and $t_0$
describe the moveout behavior. By simple algebraic manipulations, one
can also rewrite equation~\ref{eq:mo} as
\begin{equation}
\label{eq:mo2}
t^2(x) \approx t_0^2+\frac{x^2}{v^2} + \frac{A\,x^4}
{\displaystyle v^4\,\left(t_0^2+B\,\frac{x^2}{v^2} + \sqrt{t_0^4 + 2\,B\,t_0^2\,\frac{x^2}{v^2} + C\,\frac{x^4}{v^4}}\right)}\;,
\end{equation}
where the new set of parameters $A$, $B$, $C$, $v$, and $t_0$ is
related to the previous set by the equalities
\begin{eqnarray}
\label{eq;alpha}
a & = & \frac{A\,B+B^2-C}{v^2\,\left(A+B^2-C\right)}\;; \\
\label{eq:beta}
b & = & \frac{B}{v^2}\;; \\
\label{eq:gamma}
c & = & \frac{C}{v^4}\;; \\
\label{eq:xi}
\xi & = & \frac{A}{C-B^2}\;.
\end{eqnarray}
The inverse transform is given by
\begin{eqnarray}
\label{eq:v}
v^2 & = & \frac{1}{a\,(1-\xi) + b\,\xi}\;; \\
\label{eq:a}
A & = & \frac{\xi\,\left(c - b^2\right)}
{\left[a\,(1-\xi) + b\,\xi\right]^2}\;; \\
\label{eq:b}
B & = & \frac{b}{a\,(1-\xi) + b\,\xi}\;; \\
\label{eq:c}
C & = & \frac{c}{\left[a\,(1-\xi) + b\,\xi\right]^2}\;.
\end{eqnarray}
The existence of the nonhyperbolic part in the traveltime
approximation~\ref{eq:mo} and~\ref{eq:mo2} is controlled by parameter
$A$. When $A$ is zero (which implies that $\xi=0$ or $c=b^2$),
approximation~\ref{eq:mo} is hyperbolic. When both $B$ and $C$ are
very large, approximation~\ref{eq:mo2} also reduces to the hyperbolic
form.

\subsection{Connection with other approximations}

Equations~\ref{eq:mo}-\ref{eq:mo2} reduce to some well-known
approximations with special choices of parameters.
\begin{itemize}

\item If $A=0$, the proposed approximation reduces to
the classic hyperbolic form
\begin{equation}
\label{eq:hyper}
t^2(x) \approx t_0^2 + \frac{x^2}{v^2}\;,
\end{equation}
which is a two-parameter approximation.

\item The choice of parameters $A=(1-s)/2$; $B=s/2$; $C=0$ reduces the proposed
approximation to the shifted hyperbola \cite[]{malovichko,GEO53-02-01430157,GEO59-06-09830999}, which is the following three-parameter
approximation:
\begin{equation}
\label{eq:shifted}
t(x) \approx t_0\,\left(1-\frac{1}{s}\right) + 
\frac{1}{s}\,\sqrt{t_0^2+s\,\frac{x^2}{v^2}}\;.
\end{equation}

\item The choice of parameters $A = - 4\,\eta$; $B = 1 + 2\,\eta$;
  $C=(1 + 2\,\eta)^2$ reduces approximation~\ref{eq:mo2} to the form
  proposed by \cite{GEO60-05-15501566} for VTI media, which is the following
  three-parameter approximation:
\begin{equation}
  \label{eq:tsvankin}
  t^2(x) \approx t_0^2 + \frac{x^2}{v^2} - 
  \frac{2\,\eta\,x^4}
  {\displaystyle v^4\,\left[t_0^2 + (1+2\,\eta)\,\frac{x^2}{v^2}\right]}\;.
\end{equation}

\item The choice of parameters $A = - 2\,\gamma\,t_0^2\,v^2$; $B =
  -A/2$; $C=A^2/4$ reduces approximation~\ref{eq:mo2} to the following
  three-parameter approximation suggested by \cite{blias2} and
  reminiscent of the ``velocity acceleration'' equation proposed by
  \cite{taner1,taner2}:
  \begin{equation}
    \label{eq:taner}
    t^2(x) \approx t_0^2 + \frac{x^2}{v^2\,(1+\gamma\,x^2)}\;.
  \end{equation}
  
\item The choice of parameters $A=(1-s)/2$; $B=1$; $C=2-s$ reduces the proposed approximation to the following 
  three-parameter approximation suggested by \cite{blias}:
  \begin{equation}
    \label{eq:blias}
    t(x) \approx 
    \frac{1}{2}\,\sqrt{t_0^2+\left(1-\sqrt{s-1}\right)\,\frac{x^2}{v^2}} +
    \frac{1}{2}\,\sqrt{t_0^2+\left(1+\sqrt{s-1}\right)\,\frac{x^2}{v^2}} \;.
  \end{equation}
  
\item The choice of parameters $B=0$; $C=2\,A$ reduces the proposed approximation to the following 
  three-parameter approximation also suggested by \cite{blias}:
  \begin{equation}
    \label{eq:blias2}
    t^2(x) \approx \frac{t_0^2}{2} + \frac{x^2}{v^2} + \frac{1}{2}\,\sqrt{t_0^4 + \frac{2\,A\,x^4}{v^4}}\;.
  \end{equation}
  
\item The choice of parameters $A = 2\,\tan^2{\theta}$,
  $B=1-\tan^2{\theta}$, $C=1/\cos^4{\theta}$ reduces the proposed
  approximation to the double-square-root expression 
  \begin{eqnarray}
    \nonumber
    t(x) & \approx & \frac{1}{2}\,
    \sqrt{t_0^2 + \frac{x\,(x+t_0\,v\,\sin{2\theta})}{v^2\,\cos^2{\theta}}} +
    \frac{1}{2}\,
    \sqrt{t_0^2 + \frac{x\,(x-t_0\,v\,\sin{2\theta})}{v^2\,\cos^2{\theta}}} \\
    & = & \frac{\sqrt{z^2 + (y+x/2)^2}}{V} + \frac{\sqrt{z^2 + (y-x/2)^2}}{V}\;,
    \label{eq:diffr}
  \end{eqnarray}
  where $V = v\,\cos{\theta}$, $z=(t_0\,V/2)\,\cos{\theta}$, and
  $y=(t_0\,V/2)\,\sin{\theta}$. Equation~\ref{eq:diffr} describes
  moveout precisely for the case of a diffraction point in a constant
  velocity medium.
\end{itemize}

Thus, the proposed approximation encompasses some other known forms but
introduces more degrees of freedom for optimal fitting.

\subsection{General method for parameter selection}
\subsubsection{Zero-offset ray}

The Taylor expansion of approximation~\ref{eq:mo2} around the zero offset
\begin{equation}
  \label{eq:taylor2}
  t^2(x) = t_0^2 + \frac{x^2}{v^2} + \frac{A}{2}\,\frac{x^4}{v^4\,t_0^2} + O(x^6)
\end{equation}
provides a convenient method for evaluating coefficients $t_0$, $v$,
and $A$ by matching expansion~\ref{eq:taylor2} to the corresponding
expansion of the exact traveltime. This is the method used previously
for deriving approximations~\ref{eq:hyper} and~\ref{eq:shifted}. 

In the special case of an isotropic $V(z)$ medium,
the coefficients are readily available and reduce to statistical
averages of the velocity distribution \cite[]{bolshykh}
\begin{eqnarray}
  \label{eq:m-1}
  t_0 & = & 2\,m_{-1}\;, \\
  \label{eq:m1}
  v^2 & = & \frac{m_1}{m_{-1}}\;, \\
  \label{eq:m3}
  A & = & \frac{1}{2}\left(1-\frac{m_3\,m_{-1}}{m_1^2}\right)\;,
\end{eqnarray}
where
\[
m_k = \int\limits_{0}^{z} V^k(\zeta)\,d \zeta
\]
Equations~\ref{eq:m-1}-\ref{eq:m3} are easily extensible to the
vertical transverse isotropy (VTI) case
\cite[]{lyakh,blias0,GEO62-02-06620675,ursin}.

\subsubsection{Nonzero-offset ray}

To determine uniquely the remaining coefficients $B$ and $C$, we
propose to use just one additional ray reflected at a nonzero offset.
Suppose that a reflection ray with the ray parameter $P$ arrives at
the offset $X$ and traveltime $T$. Substituting
approximation~\ref{eq:mo2} into equations $t(X)=T$ and $dt/dX=P$ and
solving for $B$ and $C$ produces the explicit analytical solution
\begin{eqnarray}
  \label{eq:bp}
  B & = & \frac{t_0^2\,(X - P\,T\,v^2)}{X\,(t_0^2-T^2+P\,T\,X)} -
  \frac{A\,X^2}{X^2 + v^2\,(t_0^2-T^2)}\;, \\
  \label{eq:cp}
  C & = & \frac{t_0^4\,(X - P\,T\,v^2)^2}{X^2\,(t_0^2-T^2+P\,T\,X)^2} +
  \frac{2\,A\,v^2\,t_0^2}{X^2 + v^2\,(t_0^2-T^2)}\;.
\end{eqnarray}

\subsubsection{Horizontal ray}

If the reference ray happens to be horizontal, both $X$ and $T$ are
infinite, and equations~\ref{eq:bp}-\ref{eq:cp} are not directly
applicable. However, one can use the same principle and match two
terms for the behavior of the traveltime at infinitely large
offsets. If the traveltime behaves as
\begin{eqnarray}
  \label{eq:infty}
  t^2(x) \approx T_{\infty}^2 + P_{\infty}^2\,x^2
\end{eqnarray}
for $x$ approaching infinity, then, matching the corresponding
behavior of approximation~\ref{eq:mo2}, we find that
\begin{eqnarray}
  \label{eq:bi}
  B & = & \frac{t_0^2\,(1 - v^2\,P_{\infty}^2)}{t_0^2-T_{\infty}^2} -
  \frac{A}{1 - v^2\,P_{\infty}^2}\;, \\
  \label{eq:ci}
  C & = & \frac{t_0^4\,(1 - v^2\,P_{\infty}^2)^2}{(t_0^2-T_{\infty}^2)^2}\;.
\end{eqnarray}

\section{ACCURACY TESTS}
To illustrate the applicability of the proposed approximation, we try
several analytical and numerical models. Using these models, we test
the proposed approximation against the hyperbolic
approximation~\ref{eq:hyper}, the shifted hyperbola
approximation~\ref{eq:shifted}, and the Alkhalifah-Tsvankin
approximation~\ref{eq:tsvankin}.

\subsection{Analytical examples}
\inputdir{Math}

\subsubsection{Linear velocity and linear sloth}

We start with two analytical isotropic three-parameter models: linear
velocity model (described in Appendix A) and linear sloth model
(described in Appendix B). In both models, it is possible to compute
the exact moveout analytically and thus to compare directly the
accuracy of different approximations with the exact moveout. We show
this comparison in Figures~\ref{fig:linvel} and~\ref{fig:linsloth},
where the relative absolute approximation error is plotted for
different approximations against a large range of the offset-to-depth
ratio and the maximum-to-minimum velocity ratio. As evident from the
figures, three-parameter approximations (shifted-hyperbola and
Alkhalifah-Tsvankin) improve the accuracy of the two-parameter
hyperbolic approximation. However, the proposed five-parameter
generalized approximation brings a more significant improvement and
reduces the error by several orders of magnitude.

\plot{linvel}{width=\textwidth}{Relative absolute error of different traveltime
  approximations as a function of velocity contrast and offset/depth
  ratio for the case of a linear velocity model. (a) Hyperbolic
  approximation, (b) Shifted hyperbola approximation, (c)
  Alkhalifah-Tsvankin approximation, (d) Generalized nonhyperbolic
  approximation. The proposed generalized approximation reduces the
  maximum approximation error by several orders of magnitude.}

\plot{linsloth}{width=\textwidth}{Relative absolute error of different
  traveltime approximations as a function of velocity contrast and
  offset/depth ratio for the case of a linear sloth model. (a)
  Hyperbolic approximation, (b) Shifted hyperbola approximation, (c)
  Alkhalifah-Tsvankin approximation, (d) Generalized nonhyperbolic
  approximation. The proposed generalized approximation reduces the
  maximum approximation error by several orders of magnitude.}

\subsubsection{Curved reflector in a constant-velocity medium}
\inputdir{Math}

Our next analytical example is a curved reflector under a
constant-velocity overburden. The reflector curvature is one of the
possible causes of non-hyperbolic moveout \cite[]{fandg}. The Taylor
expansion around zero offset for the case of a curved reflector has
the form of equation~\ref{eq:taylor2} with the following set of
parameters \cite[]{fomel}
\begin{eqnarray}
\label{eq:ct}
t_0 & = & \frac{2\,L}{V}\;, \\
\label{eq:cv}
v & = & \frac{V}{\cos{\beta}}\;, \\
\label{eq:ca}
A & = & 2\,\tan^2{\beta}\,G\;, 
\end{eqnarray}
where $L$ is the length of the normal (zero-offset) ray, $V$ is the
true velocity, $\beta$ is the reflector dip angle at the normal
reflection point, $G=K\,L/(1+K\,L)$, and $K$ is the reflector
curvature at the normal reflection point. 

The two additional parameters $B$ and $C$ depend on the particular
shape of the reflector. In the case of a hyperbolic reflector,
analyzed in Appendix C, equation~\ref{eq:mo2} happens to be exact. In
this case, 
\begin{eqnarray}
\label{eq:ctinf}
T_{\infty}^2 & = & t_0^2\,\frac{G}{G+\tan^2{\beta}}\;, \\
\label{eq:cpinf}
P_{\infty} & = & \frac{1}{V} =  \frac{1}{v\,\cos{\beta}}\;, 
\end{eqnarray}
which, after substitution in equations~\ref{eq:bi}-\ref{eq:ci}, produce
\begin{eqnarray}
\label{eq:cb}
B & = & G - \tan^2{\beta}\;, \\
\label{eq:cc}
C & = & \left(G + \tan^2{\beta}\right)^2\;.
\end{eqnarray}
In the special case of a plane (zero curvature) reflector, $G=0$, and
the generalized approximation reduces to a hyperbolic form. In the
special case of a diffraction point (infinite curvature), $G=1$, and
the generalized approximation reduces to the double-square-root
equation~\ref{eq:diffr}. In both of those cases, as well as in the
case of a hyperbolic reflector, the generalized approximation is
simply exact.

Figure~\ref{fig:circle} shows a comparison between different
approximations for the case of a circular reflector, analyzed in
Appendix D.  As in the other examples, the proposed five-parameter
generalized approximation brings an improvement in accuracy in several
orders of magnitude in comparison with the three-parameter
approximations.

\plot{circle}{width=\textwidth}{Relative absolute error of different traveltime
  approximations for the case of a circular reflector as a function of
  the radius/depth ratio and the offset/depth ratio. The midpoint
  location with respect to the center of the circle is equal to the
  depth of the reflector. (a) Hyperbolic approximation, (b) Shifted
  hyperbola approximation, (c) Alkhalifah-Tsvankin approximation, (d)
  Generalized nonhyperbolic approximation. The proposed generalized
  approximation reduces the maximum approximation error by several
  orders of magnitude.}

\subsubsection{Homogeneous VTI layer}

Our next analytical example is a horizontal reflector in a homogeneous
VTI (vertically transverse isotropic) medium. As derived in Appendix
E, the approximation coefficients, under the assumption of the
acoustic approximation of \cite{GEO63-02-06230631}, take the form
\begin{eqnarray}
\label{eq:avti2}
A & = & -4\,\eta\;, \\
\label{eq:bvti2}
B & = & \frac{1 + 8\,\eta + 8\,\eta^2}{1 + 2\,\eta}\;, \\
\label{eq:cvti2}
C & = & \frac{1}{(1 + 2\,\eta)^2}\;.
\end{eqnarray}
Equation~\ref{eq:mo2} with coefficients given by
equations~\ref{eq:avti2}-\ref{eq:cvti2} is precisely equivalent to the
traveltime approximation suggested previously by
\cite{GPR52-03-02470259}. \cite{GPR52-03-02470259} shows comparisons
with alternative non-hyperbolic approximations, which demonstrate
superior accuracy of equation~\ref{eq:mo2} in case of strongly
anisotropic media.

\subsection{Numerical example}
\inputdir{amarm} 

For a numerical test, we create a one-dimensional velocity model by
extracting a depth column out of the anisotropic Marmousi model,
created by \cite{Alkhalifah.sep.95.tariq3} and shown in
Figure~\ref{fig:vz,v,eta}. We evaluate exact reflection traveltimes by
ray tracing (Figure~\ref{fig:vz1}). Next, we compare the exact time
for different reflection rays with values predicted by different
traveltime approximations. As shown in
Figure~\ref{fig:hynmo,shnmo,atnmo,fsnmo}, only the proposed
generalized approximation is able to predict the true traveltime
accurately over the full range of offsets. To define approximation
parameters, we used equations~\ref{eq:m-1}-\ref{eq:m3}
and~\ref{eq:bp}-\ref{eq:cp} and the ray with the largest offset as the
reference ray.

\multiplot{3}{vz,v,eta}{width=0.8\textwidth}{ Anisotropic Marmousi
  model. (a) Vertical velocity. (b) NMO velocity. (c) $\eta$
  parameter.}

\plot{vz1}{width=\textwidth}{One-dimensional model
  extracted from the left column of the anisotropic Marmousi model and
  corresponding reflection rays.}

\multiplot{4}{hynmo,shnmo,atnmo,fsnmo}{width=0.45\textwidth}{Exact
  moveout from ray tracing in the one-dimensional anisotropic Marmousi
  model (dots) and different approximations (solid lines). (a)
  Hyperbolic approximation, (b) Shifted hyperbola approximation, (c)
  Alkhalifah-Tsvankin approximation, (d) Generalized nonhyperbolic
  approximation.}

\section{Discussion}
\inputdir{Math}

Approximation is more of an art than a science. We don't have a
justification for suggesting equations~\ref{eq:mo} or~\ref{eq:mo2}
other than pointing out that they reduce to known forms with
particular choices of parameters.

The choice of a proper functional form is important for the
approximation accuracy. Suppose that we try to replace the
five-parameter approximation in equation~\ref{eq:mo2} with the
four-parameter equation
\begin{equation}
\label{eq:four}
t^2(x) \approx t_0^2+\frac{x^2}{v^2} + \frac{A\,x^4}
{\displaystyle v^4\,\left(2\,t_0^2+D\,\frac{x^2}{v^2}\right)}\;,
\end{equation}
where $D=B+\sqrt{C}$. Equation~\ref{eq:four} has the same behavior as
equation~\ref{eq:mo2} at small offsets and the same asymptote as $x$
approaches infinity. However, its accuracy is not nearly as
spectacular (Figure~\ref{fig:linsloth2}).

A proper selection of the reference ray for equations~\ref{eq:bp}
and~\ref{eq:cp} is also important for approximation accuracy. If this
ray is taken not at the largest possible offset, the accuracy will
deteriorate. As an extreme example, suppose that we try to define $B$
and $C$ by fitting subsequent terms of the Taylor
expansion~\ref{eq:taylor2} near the zero offset rather than the
behavior of the approximation at large
offsets. Figure~\ref{fig:linsloth1} shows the result for the case of a
linear sloth model: the approximation is more accurate than
alternatives (shown in Figure~\ref{fig:linsloth}) but not nearly as
accurate as the generalized approximation fitted at the critical offset.

Possible extensions of this work may include nonhyperbolic
approximations for diffraction traveltimes (for use in prestack time
migration) and reflection surfaces (for use in
common-reflection-surface methods) as well as approximations for
anisotropic phase and group velocities in ray tracing and wave
extrapolation.

\plot{linsloth2}{width=0.5\textwidth}{Relative absolute error of Pad\'{e} approximation 
in equation~\ref{eq:four} as a function of velocity contrast and
  offset/depth ratio for the case of a linear sloth model. Compare
  with Figure~\ref{fig:linsloth}.}
\plot{linsloth1}{width=0.5\textwidth}{Relative absolute error of 
the generalized approximation 
fitted to the zero offset as opposed to the critical offset. Compare
with Figure~\ref{fig:linsloth}.}

\section{CONCLUSIONS}

We propose a five-parameter nonhyperbolic moveout approximation that
generalizes the classic two-parameter hyperbolic approximation as well
as some known three-parameter approximations. We propose a method for
selecting the approximation parameters, which involves only two rays:
the normal-incident ray and one additional ray, preferably at a large
offset. The special case of the additional ray being horizontal can be
handled as well.

A comparison with the classic hyperbolic approximation, the shifted
hyperbola approximation and the Alkhalifah-Tsvankin approximation for
analytical and numerical isotropic and transversely isotropic models
shows that the proposed generalized nonhyperbolic approximation can
bring an improvement of several orders of magnitude in approximation
accuracy. Based on these experiments, we claim that, for many
practical purposes, the proposed approximation can be used in place of
the exact moveout.

\section{ACKNOWLEDGMENTS}

We would like to thank Tariq Alkhalifah, Emil Blias, Houb Douma, and
Bin Wang for many helpful suggestions that improved the paper.

This publication is authorized by the Director, Bureau 
of Economic Geology, The University of Texas at Austin.

\appendix
\section{Appendix A: LINEAR VELOCITY MODEL}

The linear velocity model is defined by 

\begin{equation}
\label{eq:linear}
V(z) = V_0\,(1+g\,z)\,
\end{equation}
where $g$ is the velocity gradient and $V_0$ is velocity at zero depth.

The reflection traveltime can be expressed in an analytical form as a
function of offset \cite[]{LSC00-00-02680268}
\begin{equation}
  t(x) = {\frac{2\,H}{V_0\,(r-1)}}\,
  {\arccosh{\left[1 + \frac{(r-1)^2}{2\,r}\,\left(1+\frac{x^2}{4\,H^2}\right)\right]}}\;,
  \label{eq:tlinear}
\end{equation}
where $H$ is the depth of the reflector, and $r=V(H)/V_0$ is
the ratio of velocity at the bottom and the top of the model. The
traveltime parameters are given by
\begin{eqnarray}
  \label{eq:t0linear}
  t_0 & = & \frac{2\,H}{V_0}\,\frac{\ln{r}}{r-1}\;, \\
  \label{eq:v2linear}
  v^2 & = & V_0^2\,\frac{r^2-1}{2\,\ln{r}}\;, \\
  \label{eq:alinear}
  A & = & \frac{1}{2}\,\left(1-\frac{r^2+1}{r^2-1}\,\ln{r}\right)\;.
\end{eqnarray}
This model has maximum (critical) offset and traveltime that are
defined by
\begin{eqnarray}
  \label{eq:xmlinear}
  X & = & 2\,H\,\sqrt{\frac{r+1}{r-1}}\;, \\
  \label{eq:tmlinear}
  T & = & \frac{2\,H}{V_0}\,\frac{\arccosh\,{r}}{r-1}\;.
\end{eqnarray}
Substituting equations~\ref{eq:xmlinear} and \ref{eq:tmlinear} into
equations~\ref{eq:bp}-\ref{eq:cp} and also using the expressions for
traveltime parameters~\ref{eq:t0linear}, \ref{eq:v2linear}, and
\ref{eq:alinear} results in complicated but analytical expressions for
additional parameters $B$ and $C$.

\appendix
\section{Appendix B: LINEAR SLOTH MODEL}

The linear sloth model is defined by 
\begin{equation}
  {\frac{1}{V^2(z)}} = {\frac{1}{V_0^2}}\,(1+G\,z)\;.
  \label{eq:sloth}
\end{equation}
where $G$ is the sloth gradient and $V_0$ is velocity at zero depth.

The equation for traveltime can be computed analytically, as follows
\cite[]{cerveny}:
\begin{equation}
  t^2(x) = t_0^2 + \frac{x^2}{v^2} - 
  \frac{x^4\,(r^2-1)^2\,(2\,Q+1)}{144\,Q^3\,H^2\,v^2\,(Q+1)^2}\;,
\end{equation}
where $H$ is the depth of the reflector, $r=V(H)/V_0$ is the
ratio of velocity at the bottom and the top of the model and
\[
Q = \sqrt{1-\frac{x^2\,(r^2-1)^2}{16\,r^2\,H^2}}\;.
\]
The main traveltime parameters are given by
\begin{eqnarray}
  \label{eq:t0sloth}
  t_0 & = & \frac{4\,H}{3\,V_0}\,\frac{1+r+r^2}{r\,(r+1)}\;, \\
  \label{eq:v2sloth}
  v^2 & = & V_0^2\,\frac{3\,r^2}{1+r+r^2}\;, \\
  \label{eq:asloth}
  A & = & -\frac{(r-1)^2}{6\,r}\;.
\end{eqnarray}
The maximum offset and traveltime are defined by
\begin{eqnarray}
  \label{eq:xmsloth}
  X & = & \frac{4\,H}{\sqrt{r^2-1}}\;, \\
  \label{eq:tmsloth}
  T & = & \frac{4\,H}{3\,V_0}\,\frac{r^2+2}{r\,\sqrt{r^2-1}}\;.
\end{eqnarray}
Substituting equations~\ref{eq:xmsloth} and \ref{eq:tmsloth} into
equations~\ref{eq:bp}-\ref{eq:cp} and using the expressions for
traveltime parameters~\ref{eq:t0sloth}, \ref{eq:v2sloth}, and
\ref{eq:asloth} results in the following analytical expressions for
additional parameters $B$ and $C$:
\begin{eqnarray}
\label{eq:bsloth}
B & = & - \frac{(r-1)^2\,(1+r+r^2)}{2\,r\,(r+2)\,(2\,r+1)}\;, \\
\label{eq:csloth}
C & = & - \frac{(r-1)^4\,(1+r+r^2)^2}{3\,r\,(r+2)\,(2\,r+1)^2}\;.
\end{eqnarray}

\appendix
\section{Appendix C: REFLECTION FROM A HYPERBOLIC REFLECTOR IN A HOMOGENEOUS VELOCITY MODEL}
\inputdir{XFig}

In this appendix, we derive an analytical expression for reflection
traveltime from a hyperbolic reflector in a homogeneous velocity
model (Figure~\ref{fig:hyper}). Similar derivations apply to an elliptic reflector and were
used previously in the theory of dip moveout, offset continuation, and
non-hyperbolic common-reflection surface
\cite[]{stovas,GEO68-02-07180732,crs}.

\plot{hyper}{width=0.75\textwidth}{Reflection from a hyperbolic reflector in a 
homogeneous velocity model (a scheme).}

Consider the source point $x_s$ and the receiver point $x_r$ at the
surface $z=0$ above a 2-D constant-velocity medium and a hyperbolic
reflector  defined by the equation 
\begin{equation}
  \label{eq:hypref}
  z(x) = \sqrt{h^2 +
    x^2\,\tan^2{\alpha}}\;.
\end{equation}
The reflection traveltime as a function of the reflection point
location $y$ is
\begin{equation}
  \label{eq:tofy}
  t = \frac{\sqrt{(x_s-y)^2 + z^2(y)} + \sqrt{(x_r-y)^2+z^2(y)}}{V}\;.
\end{equation}
According to Fermat's principle, the traveltime should be stationary
with respect to the reflection point~$y$:
\begin{equation}
\label{eq:fermat}
0 = \frac{\partial t}{\partial y} =
\frac{y-x_s + y\,\tan^2{\alpha}}{V\,\sqrt{(x_s-y)^2 + h^2 + y^2\,\tan^2{\alpha}}} +
\frac{y-x_r + y\,\tan^2{\alpha}}{V\,\sqrt{(x_r-y)^2 + h^2 + y^2\,\tan^2{\alpha}}}\;.
\end{equation}
Putting two terms in equation~\ref{eq:fermat} on different sides of
the equation, squaring them, and reducing their difference to a common
denominator, we arrive at the equation
\begin{eqnarray}
\nonumber
0 & = & 
\left[\frac{y}{\cos^2{\alpha}} - x_s\right]^2\,\left[(x_r-y)^2 + h^2 + y^2\,\tan^2{\alpha}\right] \\
& - & 
\left[\frac{y}{\cos^2{\alpha}} - x_r\right]^2\,\left[(x_s-y)^2 + h^2 + y^2\,\tan^2{\alpha}\right] 
\label{eq:fermat2}
\end{eqnarray}
which simplifies to the following quadratic equation with respect to $y$:
\begin{equation}
  \label{eq:y2}
  y^2\,(x_s+x_r)\,\tan^2{\alpha} - 2\,y\,\left(x_s\,x_r\,\sin^2{\alpha} - h^2\right) -
  h^2\,(x_s+x_r)\,\cos^2{\alpha} = 0\;.
\end{equation}
The discriminant is
\begin{equation}
  \label{eq:disc}
  D = \left(x_s\,x_r\,\sin^2{\alpha}-h^2\right)^2 + h^2\,(x_s+x_r)^2\,\sin^2{\alpha}
  = (h^2+x_s^2\,\sin^2{\alpha})\,(h^2+x_r^2\,\sin^2{\alpha})\;.
\end{equation}
Only one of the two branches of the solution
\begin{eqnarray}
  \nonumber
  y & = & 
  \frac{x_s\,x_r\,\sin^2{\alpha}-h^2 + \sqrt{(h^2+x_s^2\,\sin^2{\alpha})\,(h^2+x_r^2\,\sin^2{\alpha})}}{(x_s+x_r)\,\tan^2{\alpha}} \\
  & = &
  \frac{h^2\,(x_s+x_r)\,\cos^2{\alpha}}{h^2 - x_s\,x_r\,\sin^2{\alpha} +
    \sqrt{(h^2+x_s^2\,\sin^2{\alpha})\,(h^2+x_r^2\,\sin^2{\alpha})}}
  \label{eq:ys}
\end{eqnarray}
has physical meaning. Substituting equation~\ref{eq:ys} into
equation~\ref{eq:tofy}, we obtain, after a number of algebraic
simplifications,
\begin{equation}
  \label{eq:t}
  t = \frac{\sqrt{2 h^2 + x_s^2 + x_r^2 - 2\,x_s\,x_r\,\cos^2{\alpha} + 
      2\,\sqrt{(h^2+x_s^2\,\sin^2{\alpha})\,(h^2+x_r^2\,\sin^2{\alpha})}}}{V}\;.
\end{equation}
Making the variable change in equation~\ref{eq:t} from $x_s$ and $x_r$
to the midpoint and offset coordinates $m$ and $x$ according to
$x_s=m-x/2$, $x_r=m+x/2$, we notice that this equation is exactly
equivalent to equation~\ref{eq:mo} with the following definition of
parameters:
\begin{eqnarray}
\label{eq:th}
t_0 & = & \frac{2\,\sqrt{h^2 + m^2\,\sin^2{\alpha}}}{V}\;, \\
\label{eq:ah}
a & = & \frac{2-\sin^2{\alpha}}{V^2}\;, \\
\label{eq:bh}
b & = & \frac{\sin^2{\alpha}}{V^2}\,\frac{h^2 - m^2\,\sin^2{\alpha}}{h^2 + m^2\,\sin^2{\alpha}}\;, \\
\label{eq:ch}
c & = & \frac{\sin^4{\alpha}}{V^4}\;, \\
\label{eq:xh}
\xi & = & \frac{1}{2}\;.
\end{eqnarray}

\appendix
\section{Appendix D: REFLECTION FROM A CIRCULAR REFLECTOR IN A HOMOGENEOUS VELOCITY MODEL}
\inputdir{XFig}

In the case of a circular (cylindrical or spherical) reflector in a
homogeneous velocity model, there is no closed-form analytical
solution. However, the moveout can be described analytically by
parametric relationships \cite[]{mirror}.

\plot{crefl}{width=\textwidth}{Reflection from a circular reflector in
  a homogeneous velocity model (a scheme).}

Consider the geometry of the reflection shown in
Figure~\ref{fig:crefl}. According to the trigonometry of the
reflection triangles, the source and receiver positions can be expressed as
\begin{eqnarray}
\label{eq:xs}
x_s & = & R\,\sin{\alpha} + (H+R - R\,\cos{\alpha})\,\tan{(\alpha-\theta)}\;, \\
\label{eq:xr}
x_r & = & R\,\sin{\alpha} + (H+R - R\,\cos{\alpha})\,\tan{(\alpha+\theta)}\;,
\end{eqnarray}
where $R$ is the reflector radius, $H$ is the minimum reflector depth,
$\alpha$ is the reflector dip angle at the reflection point, and
$\theta$ is the reflection angle. Correspondingly, the midpoint and
offset coordinates can be expressed as
\begin{eqnarray}
\label{eq:m}
m & = & \frac{x_s+x_r}{2} = R\,\sin{\alpha} + (H+R - R\,\cos{\alpha})\,\frac{\cos{\alpha}\,\sin{\alpha}}{\cos^2{\theta} - \sin^2{\alpha}}\;, \\
\label{eq:x}
x & = & x_r - x_s = 2\,(H+R - R\,\cos{\alpha})\,\frac{\cos{\theta}\,\sin{\theta}}{\cos^2{\theta} - \sin^2{\alpha}}\;,
\end{eqnarray}
and the reflection traveltime can be expressed as
\begin{eqnarray}
\nonumber
t & = & \frac{H+R - R\,\cos{\alpha}}{V}\,\left[\frac{1}{\cos{(\alpha-\theta)}} + \frac{1}{\cos{(\alpha+\theta)}}\right] \\
& = & 2\,\frac{H+R - R\,\cos{\alpha}}{V}\,\frac{\cos{\alpha}\,\cos{\theta}}{\cos^2{\theta} - \sin^2{\alpha}}\;,
\label{eq:tcirc}
\end{eqnarray}
where $V$ is the medium velocity.
Expressing the reflection angle $\theta$ from equation~\ref{eq:m} and
substituting it into equations~\ref{eq:x} and~\ref{eq:tcirc}, we obtain a
pair of parametric equations
\begin{eqnarray}
\label{eq:xa}
x^2(\alpha) & = & 4\,\frac{[m\,\cos{\alpha} - (H+R)\,\sin{\alpha}]\,[m\,\sin{\alpha} + (H+R)\,\cos{\alpha} - R]}{\cos{\alpha}\,\sin{\alpha}}\;, \\
\label{eq:ta}
t^2(\alpha) & = & \frac{4}{V^2}\,\frac{(m - R\,\sin{\alpha})\,[m\,\sin{\alpha} + (H+R)\,\cos{\alpha} - R]}{\sin{\alpha}}\;,
\end{eqnarray}
which define the exact reflection moveout for the case of a circular reflector in a homogeneous medium.

The connection with parameters of equations~\ref{eq:ct}-\ref{eq:ca} is given by
\begin{eqnarray}
\label{eq:l}
L & = & \sqrt{m^2 + (H+R)^2} - R\;,   \\
\label{eq:cbeta}
\cos{\beta} & = & \frac{H+R}{\sqrt{m^2 + (H+R)^2}}\;, \\
\label{eq:g}
G & = & \frac{L}{L+R} = 1 - \frac{R}{\sqrt{m^2 + (H+R)^2}}\;.
\end{eqnarray}
The behavior of the moveout at infinitely large offsets is controlled by $P_{\infty} = 1/V$ and
\begin{equation}
\label{eq:ctinf3}
T_{\infty} = \frac{2\,H}{V} = t_0\,\frac{G+\cos{\beta} - 1}{G\,\cos{\beta}}\;. 
\end{equation}
After substitution in equations~\ref{eq:bi}-\ref{eq:ci}, we obtain
somewhat complicated but analytical expressions for parameters $B$ and
$C$.

\appendix
\section{Appendix E: HOMOGENEOUS VTI MODEL}

According to the acoustic approximation of
\cite{GEO63-02-06230631}, one can use the following parametric
equations to define the traveltime-offset relationship in a
homogeneous VTI model:
\begin{eqnarray}
\label{eq:xpvti}
x(p) & = & \frac{2\,H}{v_z}\,\frac{p\,v^2}{(1-2\,\eta\,p^2\,v^2)^2\,\sqrt{1-\frac{p^2\,v^2}{1-2\,\eta\,p^2\,v^2}}}\;, \\
\label{eq:tpvti}
t(p) & = & \frac{2\,H}{v_z}\,\frac{(1-2\,\eta\,p^2\,v^2)^2 + 2\,\eta\,p^4\,v^4}
{(1-2\,\eta\,p^2\,v^2)^2\,\sqrt{1-\frac{p^2\,v^2}{1-2\,\eta\,p^2\,v^2}}}\;,
\end{eqnarray}
where $p$ is the ray parameter, $H$ is the depth of the reflector,
$v_z$ is the vertical velocity, $v$ is the NMO velocity, and $\eta$ is
the dimensionless parameter introduced by \cite{GEO60-05-15501566}.

At small offsets, the homogeneous VTI traveltime behaves as
\begin{equation}
\label{eq:vtismall}
t^2(x) \approx t_0^2 + \frac{x^2}{v^2} - \frac{2\,\eta\,x^4}{t_0^2\,v^4}\;,
\end{equation} 
which allows us to define $A=-4\,\eta$ according to equation~\ref{eq:taylor2}.

At large offsets, the homogeneous VTI traveltime behaves as
\begin{equation}
\label{eq:vtilarge}
t^2(x) \approx t_0^2\,(1+2\,\eta) + \frac{x^2}{v^2\,(1+2\,\eta)}\;.
\end{equation} 
Comparing with equation~\ref{eq:infty}, we note that $T_{\infty} =
t_0\,\sqrt{1+2\,\eta}$ and $P_{\infty} =
1/(v\,\sqrt{1+2\,\eta})$. Substituting into
equations~\ref{eq:bi}-\ref{eq:ci}, we derive the coefficients $B$ and
$C$ to be
\begin{eqnarray}
\label{eq:bvti}
B & = & \frac{1 + 8\,\eta + 8\,\eta^2}{1 + 2\,\eta}\;, \\
\label{eq:cvti}
C & = & \frac{1}{(1 + 2\,\eta)^2}\;.
\end{eqnarray}

\nocite{Sword.sep.51.313}

\bibliographystyle{seg}

\bibliography{SEG,SEP2,hyper}
