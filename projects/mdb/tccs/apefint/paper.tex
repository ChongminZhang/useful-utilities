
\published{Geophysics, 76, V69-V77, (2011)}
\title{Seismic data interpolation beyond aliasing using regularized nonstationary autoregression}

\renewcommand{\thefootnote}{\fnsymbol{footnote}}

%\ms{GEO-2010-0231-final}

\address{
\footnotemark[1] College of Geo-exploration Science and Technology,\\
Jilin University \\
No.6 Xi minzhu street, \\
Changchun, China, 130026 \\
\footnotemark[2] Bureau of Economic Geology,\\
John A. and Katherine G. Jackson School of Geosciences \\
The University of Texas at Austin \\
University Station, Box X \\
Austin, TX, USA, 78713-8924}

\author{Yang Liu\footnotemark[1], Sergey Fomel\footnotemark[2]}

\footer{GEO-2010-0231-final}
\lefthead{Liu and Fomel}
\righthead{Regularized nonstationary autoregression}
\maketitle

\begin{abstract}
Seismic data are often inadequately or irregularly sampled along
spatial axes. Irregular sampling can produce
artifacts in seismic imaging results. We present a new approach to
interpolate aliased seismic data based on adaptive prediction-error
filtering (PEF) and regularized nonstationary autoregression. Instead
of cutting data into overlapping windows (patching), a popular method
for handling nonstationarity, we obtain smoothly nonstationary PEF
coefficients by solving a global regularized least-squares
problem. We employ shaping regularization to
control the smoothness of adaptive PEFs. Finding the interpolated
traces can be treated as another linear least-squares problem, which
solves for data values rather than filter coefficients. Compared with
existing methods, the advantages of the proposed method include an
intuitive selection of regularization parameters and fast
iteration convergence. Benchmark synthetic and
field data examples show that the proposed technique can successfully
reconstruct data with decimated or missing traces.

\end{abstract}

\section{Introduction}
The regular and fine sampling along the time
axis is common, whereas good spatial sampling is often more expensive
or prohibitive and therefore is the main bottleneck for seismic
resolution. Too large a spatial sampling interval may lead to
aliasing problems that adversely affect the resolution
of subsurface images. An alternative to expensive dense spatial
sampling is interpolation of seismic traces. One important approach to
trace interpolation is prediction interpolating methods
\cite[]{GEO56-06-07850794}, which use low-frequency non-aliased data
to extract antialiasing prediction-error filters (PEFs) and then
interpolates high frequencies beyond aliasing. \cite{Claerbout92}
extends Spitz's method using PEFs in the $t$-$x$
domain. \cite{GEO64-05-14611467} proposes a half-step PEF scheme that
makes the interpolation process more efficient. \cite{Huard96}
and \cite{GEO67-04-12321239} extend $f$-$x$ trace interpolation to higher
spatial dimensions. \cite{GEO68-01-03550369} introduces an algorithm
similar to $f$-$x$ prediction filtering, which has an elegant
representation in the $f$-$k$ domain. \cite{Curry06} uses
multi-dimensional nonstationary PEFs to interpolate diffracted
multiples. \cite{Naghizadeh09} propose an adaptive $f$-$x$
interpolation using exponentially weighted recursive least
squares. More recently, \cite{Naghizadeh10a} propose a prediction
approach similar to Gulunay's method but using the curvelet transform
instead of the Fourier transform. \cite{Abma05} compare the
performance of several different interpolation methods.

Correcting irregular spatial sampling is another application for
seismic data interpolation algorithms. A variety of interpolation
methods have been published in the recent years. One approach is to
estimate the PEF on multiple rescaled copies of the irregular
data \cite[]{Curry03}, where the data are rescaled with a number of
progressively-larger bin sizes. \cite{Curry04} further improves the
rescaling method by introducing multiple scales of the data where the
location of the grid cells are varied in addition to the size of the
cells. \cite{Curry08} use pseudo-primary data by crosscorrelating
multiples and primaries to estimate nonstationary PEF and then
interpolated missing near offsets. \cite{Naghizadeh10b} propose
autoregressive spectral estimates to reconstruct aliased data and data
with gaps.

Seismic data are nonstationary. The standard PEF is designed under the
assumption of stationary data and becomes less effective when this
assumption is violated \cite[]{Claerbout92}.  Cutting data into
overlapping windows (patching) is a common method to handle
nonstationarity \cite[]{Claerbout10}, although it occasionally fails
in the presence of variable 
dips. \cite{Crawley99} propose smoothly-varying
nonstationary PEFs with ``micropatches'' and radial smoothing, which
typically produces better results than the rectangular patching
approach. \cite{GEO67-06-19461960} develops a nonstationary plane-wave
destruction (PWD) filter as an alternative to $t$-$x$ PEF
\cite[]{Claerbout92} and applies the PWD operator to trace
interpolation. The PWD method depends on the assumption of a small
number of smoothly variable seismic dips. \cite{Curry03} uses
Laplacian and radial rougheners to ensure a nonstationary PEF that
varies smoothly in space, which specifies an appropriate
regularization operator.

In this paper, we use the two-step strategy, similar to that of
\cite{Claerbout92} and \cite{Crawley99}, but calculate the
adaptive PEF by using regularized nonstationary autoregression
\cite[]{Fomel09} to handle both nonstationarity and
aliasing. The key idea is the use of shaping regularization
\cite[]{Fomel07} to constrain the spatial smoothness of filter
coefficients. We provide an approach to nonstationary data
interpolation, which has an intuitive selection of
parameters and fast iteration
convergence. We test the new method by using several benchmark
synthetic examples. Results of applying the proposed method to a field
data example demonstrate that it can
be effective in trace interpolation problems, even in the presence of
multiple strongly variable slopes.

\section{Theory}

A common constraint for interpolating missing seismic traces is to
ensure that the interpolated data, after specified filtering, have
minimum energy \cite[]{Claerbout92}. Filtering is equivalent to
spectral multiplication. Therefore, specified filtering is a way of
prescribing a spectrum for the interpolated data. A sensible choice is
a spectrum of the recorded data, which can be captured by finding the
data's PEF \cite[]{GEO56-06-07850794,Crawley00}. The PEF, also known
as the autoregression filter, plays the role of the
`inverse-covariance matrix' in statistical estimation theory. A signal
is regressed on itself in the estimation of PEF. The PEF can be
implemented in either $t$-$x$ (time-space) or $f$-$x$
(frequency-space) domain.  Time-space PEFs are less likely to create
spurious events in the presence of noise than $f$-$x$ PEFs
\cite[]{Abma95,Crawley00}. When data interpolation is cast as an
inverse problem, a PEF can be used to find missing data. This involves
a two-step approach. In the first step, a PEF is
estimated by minimizing the output of convolution of known
data with an unknown PEF. In the second step, the
missing data is found by minimizing the convolution of the recently
calculated PEF with the unknown model, which is constrained where the
data are known \cite[]{Curry04}.

\subsection{Step 1: Adaptive PEF estimation}
 \subsubsection{Regular trace interpolation} 

An important property of PEFs is scale invariance, which allows
estimation of PEF coefficients $A_n$ (including the leading ``$-1$''
and prediction coefficients $B_n$) for incomplete aliased data
$S(t,x)$ that include known traces
$S_{known}(t,x_k)$ and unknown or zero traces
$S_{zero}(t,x_z)$. For trace decimation, zero 
traces interlace known traces. To avoid zeroes that influence filter
estimation, we interlace the filter coefficients with zeroes. For
example, consider a 2-D PEF with seven prediction coefficients:
\begin{equation}
   \begin{array}{ccccc}
      B_3    &B_4    &B_5     &B_6     &B_7     \\
      \cdot  &\cdot  &-1      &B_1     &B_2 \end{array}
\label{eqn:noilace}
\end{equation}
Here, the horizontal axis is time, the vertical axis is space, and
``$\cdot$'' denotes zero. Rescaling both time and spatial axes assumes
that the dips represented by the original filter in
equation~\ref{eqn:noilace} are the same as those represented by the
scaled filter
\cite[]{Claerbout92}:
\begin{equation}
   \begin{array}{ccccccccc} 
   B_3 &\cdot &B_4 &\cdot &B_5 &\cdot &B_6 &\cdot &B_7 \\
   \cdot &\cdot &\cdot &\cdot &\cdot &\cdot &\cdot &\cdot &\cdot \\
   \cdot &\cdot &\cdot &\cdot &-1 &\cdot &B_1 &\cdot &B_2
   \end{array}
\label{eqn:ilacefil}
\end{equation}
For nonstationary situations, we can also assume locally stationary
spectra of the data because trace decimation makes the space between
known traces small enough, thus making adaptive PEFs locally
scale-invariant. For estimating adaptive PEF coefficients,
nonstationary autoregression allows coefficients $B_n$ to change with
both $t$ and $x$. The new adaptive filter can look something like
\begin{equation}
   \begin{array}{ccccccccc} 
     B_3(t,x) &\cdot &B_4(t,x) &\cdot &B_5(t,x) 
                &\cdot &B_6(t,x) &\cdot &B_7(t,x)
   \\ \cdot &\cdot &\cdot &\cdot &\cdot &\cdot &\cdot &\cdot &\cdot \\
   \cdot &\cdot &\cdot &\cdot &-1 &\cdot &B_1(t,x) &\cdot &B_2(t,x)
   \end{array}
\label{eqn:ailacefil}
\end{equation}
In other words, prediction coefficients $B_n(t,x)$ are
obtained by solving the least-squares problem,
%{\setlength\arraycolsep{2pt}
\begin{eqnarray}
  \label{eq:pred2}
\widehat{B_n}(t,x) &=& \arg\min_{B_n}\|S(t,x)-\sum_{n=1}^{N} 
   B_n(t,x)S_n(t,x)\|_2^2 \nonumber \\
  & & + \epsilon^2\, \sum_{n=1}^{N} \|\mathbf{D}[B_n(t,x)]\|_2^2\;,
\end{eqnarray}
where $S_n(t,x)$=$S(t-m\,i\,\Delta\,t,x-m\,j\,\Delta\,x)$, which
represents the causal translation of $S(t,x)$, with time-shift index
$i$ and spatial-shift index $j$ scaled by decimation interval
$m$. Note that predefined constant $m$ uses the interlacing value as
an interval; i.e., the shift interval equals 2 in
equation~\ref{eqn:ailacefil}. Subscript $n$ is the general shift index
for both time and space, and the total number of $i$ and $j$ is
$N$. $\mathbf{D}$ is the regularization operator, and $\epsilon$ is a
scalar regularization parameter.  All coefficients $B_n(t,x)$ are
estimated simultaneously in a time/space variant manner. This approach
was described by \cite{Fomel09} as regularized nonstationary
autoregression (RNA). If $\mathbf{D}$ is a linear operator,
least-squares estimation reduces to linear inversion
\begin{equation}
  \label{eq:inv}
  \mathbf{b} = \mathbf{A}^{-1}\,\mathbf{d}\;,
\end{equation}
where 
\begin{equation}
  \label{eq:bvar}
  \mathbf{b} = \left[\begin{array}{cccc}B_1(t,x) & B_2(t,x) & \cdots & B_N(t,x)\end{array}\right]^T\;,
\end{equation}
\begin{equation}
  \label{eq:dvar}
\mathbf{d} = \left[\begin{array}{cccc}S_1(t,x)\,S(t,x) & S_2(t,x)\,S(t,x) & \cdots & S_N(t,x)\,S(t,x)\end{array}\right]^T\;,
\end{equation}
and the elements of matrix $\mathbf{A}$ are
\begin{equation}
  \label{eq:aij}
  A_{nk}(t,x) = S_n(t,x)\,S_k(t,x) + \epsilon^2\,\delta_{nk}\,\mathbf{D}^T\,\mathbf{D}\;.
\end{equation}

Shaping regularization \cite[]{Fomel07} incorporates a shaping
(smoothing) operator $\mathbf{G}$ instead of $\mathbf{D}$ and
provides better numerical properties than Tikhonov's
regularization \cite[]{Tikhonov63} in
equation~\ref{eq:pred2} \cite[]{Fomel09}.
Inversion using shaping regularization takes the form
\begin{equation}
  \label{eq:rinv}
  \mathbf{b} = \widehat{\mathbf{A}}^{-1}\,\widehat{\mathbf{d}}\;,
\end{equation}
where 
\begin{equation}
  \label{eq:dvar2}
\widehat{\mathbf{d}} = \left[\begin{array}{cccc}\mathbf{G}\left[S_1(t,x)\,S(t,x)\right] & 
\mathbf{G}\left[S_2(t,x)\,S(t,x)\right] & \cdots & 
\mathbf{G}\left[S_N(t,x)\,S(t,x)\right]\end{array}\right]^T\;,
\end{equation}
the elements of matrix $\widehat{\mathbf{A}}$ are
\begin{eqnarray}
  \label{eq:raij}
  \widehat{A}_{nk}(t,x) &=& \lambda^2\,\delta_{nk} + \mathbf{G}\left[S_n(t,x)\,S_k(t,x) - 
    \lambda^2\,\delta_{nk}\right]\;,
\end{eqnarray}
and $\lambda$ is a scaling coefficient. One advantage of the shaping
approach is the relative ease of controlling the selection of
$\lambda$ and $\mathbf{G}$ in comparison with $\epsilon$ and
$\mathbf{D}$. We define $\mathbf{G}$ as Gaussian smoothing with an
adjustable radius, which is designed by repeated application of
triangle smoothing \cite[]{Fomel07}, and choose $\lambda$ to be the
mean value of $S_n(t,x)$.

Coefficients $B_n(t,x_z)$ at zero traces $S_{zero}(t,x_z)$ get constrained
(effectively smoothly interpolated) by regularization. The required
parameters are the size and shape of the filter $B_n(t,x)$ and the
smoothing radius in $\mathbf{G}$.

\subsubsection{Missing data interpolation} 

Irregular gaps occur in the recorded data for many different reasons,
and prediction-error filters are known to be a powerful method for
interpolating missing data. Missing data interpolation is a particular
case of data regularization, where the input data are already given on
a regular grid, and one needs to reconstruct only the missing values
in empty bins \cite[]{Fomel01a}. One can use existing traces to
directly estimate adaptive PEF coefficients instead of scaling the
filter as in regular trace interpolation problem. However, finding the
adaptive PEF needs to avoid using any regression equations that
involve boundaries or missing data. This can be achieved by creating
selection mask operator $K(t,x)$, a diagonal matrix with ones at the
known data locations and zeros elsewhere, for both causal translations
and input data \cite[]{Claerbout10}.

Analogously to the stationary prediction-error filter
(\ref{eqn:noilace}), adaptive PEF coefficients $B_n(t,x)$ use the
unscaled format and appear as
\begin{equation} \begin{array}{ccccc}
B_3(t,x) &B_4(t,x) &B_5(t,x) &B_6(t,x) &B_7(t,x) \\ \cdot &\cdot &-1
&B_1(t,x) &B_2(t,x) \end{array} \label{eqn:npef} \end{equation} The
nonstationary coefficients $B_n(t,x)$ can be obtained by solving the
least-squares problem
\begin{eqnarray}
  \label{eq:pred3}
\widehat{B_n}(t,x) &=& \arg\min_{B_n}\|K(t,x)[S(t,x)-\sum_{n=1}^{N} 
   B_n(t,x)S_n(t,x)]\|_2^2 \nonumber \\
  & & + \epsilon^2\, \sum_{n=1}^{N} \|\mathbf{D}[B_n(t,x)]\|_2^2\;.
\end{eqnarray}
where $S_n(t,x)=S(t-i,x-j)$. By using shaping regularization, 
adaptive PEF coefficients are smoothly filled at missing trace
locations.

\subsection{Step 2: Data interpolation with adaptive PEF}

In the second step, a similar problem is solved, except that the
filter is known, and the missing traces are unknown. In the
decimated-trace interpolation problem, we squeeze (by throwing away
alternate zeroed rows and columns) the filter in
equation~\ref{eqn:ailacefil} to its original size
and then formulate the least-squares problem,
\begin{equation}
  \label{eq:pred4}
     \widehat{S}(t,x) = \arg\min_{S}\|S(t,x)-\sum_{n=1}^{N} 
       \widehat{B_n}(t,x)S_n(t,x)\|_2^2\;,
\end{equation}
subject to
\begin{equation}
  \label{eq:pred5}
      \widehat{S}(t,x_k) = S_{known}(t,x_k)\;,
\end{equation}
where $\widehat{S}(t,x)$ represents the interpolated output, and $i$
and $j$ use the original shift as the interval; i.e., the shift
interval equals 1.

We carry out the minimization in
equations~\ref{eq:pred2}, \ref{eq:pred3}, and \ref{eq:pred4} by the
conjugate gradient method \cite[]{Hestenes52}. The constraint
condition (equation~\ref{eq:pred5}) is used as the initial model and
constrains the output by using the known traces for each iteration in
the conjugate-gradient scheme. The computational cost is proportional to
$N_{iter}\times N_{f}\times N_{t}\times N_{x}$, where $N_{iter}$
is the number of iterations, $N_f$ is the filter size, and $N_t \times
N_x$ is the data size. In our tests, $N_f$ and $N_{iter}$ were
approximately equal to 100. Increasing the smoothing radius in shaping
regularization decreases $N_{iter}$ in the filter estimation step.

\section{Synthetic data tests}

\subsection{Aliasing decimated-trace interpolation test}

We start with a strongly aliased synthetic example from
\cite{Claerbout09}. The sparse spatial sampling makes the gather
severely aliased, especially at the far offset positions
(Figure~\ref{fig:jaliasp,dealias,jamiss}a). For comparison, we used
PWD \cite[]{GEO67-06-19461960} to interpolate the traces
(Figure~\ref{fig:jaliasp,dealias,jamiss}b). Interpolation with PWD
depends on dip estimation. In this example, the true dip is
non-negative everywhere and is easily distinguished from the aliased
one. Therefore, the PWD method recovers the interpolated traces
well. However, in the more general case, an additional interpretation
may be required to determine which of the dip components is
contaminated by aliasing. According to the theory described in the
previous section, the PEF-based methods use the lower (less aliased)
frequencies to estimate PEF coefficients, and then interpolate the
decimated traces (high-frequency information) by minimizing the
convolution of the scale-invariant PEF with the unknown model, which
is constrained where the data is known. We designed adaptive PEFs
using 10 (time) $\times$ 2 (space) coefficients for each sample and a
50-sample (time) $\times$ 2-sample (space) smoothing radius and then
applied them so as to interpolate the aliased trace. The nonstationary
autoregression algorithm effectively removes all spatial aliasing
artifacts (Figure~\ref{fig:jaliasp,dealias,jamiss}c). The proposed
method compares well with the PWD method. The CPU times, for single
2.66GHz CPU used in this example, are 20 seconds for adaptive PEF
estimation (step 1) and 2 seconds for data interpolation (step 2).

\inputdir{alias}
\multiplot{3}{jaliasp,dealias,jamiss}{width=0.45\columnwidth}{Aliased
               synthetic data (a), trace interpolation with plane-wave
               destruction (b), and trace interpolation with
               regularized nonstationary autoregression (c). Three
               additional traces were inserted between each of the
               neighboring input traces.}

\subsection{Abma decimated-trace interpolation tests}

A benchmark example created by Raymond Abma (personal communication)
shows a simple curved event
(Figure~\ref{fig:jcurve,jcscov,jcacov}a). The challenge in this
example is to account for both nonstationarity and
aliasing. Figure~\ref{fig:jcurve,jcscov,jcacov}b shows the
interpolated result using Claerbout's stationary $t$-$x$ PEF, which
was estimated and applied in one big window, with each PEF coefficient
$B_n$ constant at every data location. Note that the $t$-$x$ PEF
method can recover the aliasing trace only in the dominant slope
range. The trace-interpolating result using regularized nonstationary
autoregression is shown in Figure~\ref{fig:jcurve,jcscov,jcacov}c. The
adaptive PEF has 20 (time) $\times$ 3 (space) coefficients for each
sample and a 20-sample (time) $\times$ 3-sample (space) smoothing
radius. The proposed method eliminates all nonstationary aliasing and
improves the continuity of the curved event.
\inputdir{ray}
\multiplot{3}{jcurve,jcscov,jcacov}{width=0.75\columnwidth}{Curve
              model (a), trace interpolation with stationary PEF (b),
              and trace interpolation with adaptive PEF (c).}

\cite{Abma05} present a comparison of several algorithms used for
trace interpolation. We chose the most challenging benchmark Marmousi
example from Abma and Kabir to illustrate the performance of RNA
interpolation. Figure~\ref{fig:jmarm,jmacov}a shows a zero-offset
section of the Marmousi model, in which curved events violate the
assumptions common for most trace-interpolating
methods. Figure~\ref{fig:jmarm,jmacov}b shows that our method produces
reasonable results for both curved and weak events and does not
introduce any undesirable noise. The adaptive PEF parameters
correspond to 7 (time) $\times$ 5 (space) coefficients for each sample
and a 40-sample (time) $\times$ 30-sample (space) smoothing radius.


 \multiplot{2}{jmarm,jmacov}{width=0.75\columnwidth}{Marmousi model
                (a) and trace interpolation with regularized
                nonstationary autoregression (b).}

\subsection{Missing-trace interpolation test}

A missing trace test is shown in Figure~\ref{fig:zero,jamiss}a and
comes from decimated-trace interpolation result
(Figure~\ref{fig:jcurve,jcscov,jcacov}c) after removing 70\% of
randomly selected traces. The curved event makes it difficult to
recover the missing traces. The interpolated result is shown in
Figure~\ref{fig:zero,jamiss}b, which uses a regularized adaptive PEF
with 4 (time) $\times$ 2 (space) coefficients for each sample and a
50-sample (time) $\times$ 10-sample (space) smoothing radius. In the
interpolated result, it is visually difficult to distinguish the
missing trace locations, which is an evidence of successful
interpolation. The filter size along space direction needs to be
small in order to generate enough regression equations.

\inputdir{misscurv}
   \multiplot{2}{zero,jamiss}{width=0.75\columnwidth}{Curved model
                 (Figure~\ref{fig:jcurve,jcscov,jcacov}c) with 70\%
                 randomly selected traces removed (a) and trace
                 interpolation with regularized nonstationary
                 autoregression (b).}

\section{Field Data Examples}
We use a set of marine 2-D shot gathers from a deepwater Gulf of
Mexico survey \cite[]{Crawley99,GEO67-06-19461960} to further
test the proposed method. Figure~\ref{fig:sean,sean2} shows the data
before and after subsampling in the offset direction. The shot gather
has long-period multiples and complicated diffraction events caused by
a salt body. Amplitudes of the events are not uniformly
distributed. Subsampling by a factor of 2
(Figure~\ref{fig:sean,sean2}b) causes visible aliasing of the steeply
dipping events. We designed a nonstationary PEF, with 15 (time)
$\times$ 5 (space) coefficients for each sample and a 50-sample (time)
$\times$ 20-sample (space) smoothing radius to handle the variability
of events. Figure~\ref{fig:samiss,serr} shows the interpolation result
and the difference between interpolated traces and original traces
plotted at the same clip value. The proposed method succeeds in the
sense that it is hard to distinguish interpolated traces from the
interpolation result alone. A close-up comparison between the original
and interpolated traces (Figure~\ref{fig:win,win2}) shows some small
imperfections. Some energy of the steepest events is partly
missing. Coefficients of the adaptive PEF are illustrated in
Figure~\ref{fig:sfcoe,smcoe}, which displays the first coefficient
($B_1$) and the mean coefficient of $B_n$, respectively. The filter
coefficients vary in time and space according to the curved
events. The interpolated results are relatively insensitive to the
smoothing parameters.
\inputdir{sean}
   \multiplot{2}{sean,sean2}{width=0.47\columnwidth}{A 2-D marine shot
                gather. Original input (a) and input subsampled by a
                factor of 2 (b).}

   \multiplot{2}{samiss,serr}{width=0.47\columnwidth}{Shot gather
                 after trace interpolation (adaptive PEF with 15
                 $\times$ 5) (a) and difference between original
                 gather (Figure~\ref{fig:sean,sean2}a) and
                 interpolated result (Figure~\ref{fig:samiss,serr}a)
                 (b).}

   \multiplot{2}{win,win2}{width=0.47\columnwidth}{Close-up comparison
                 of original data (a) and interpolated result by
                 RNA (b).}

   \multiplot{2}{sfcoe,smcoe}{width=0.47\columnwidth}{Adaptive PEF
                 coefficients. First coefficient $B_1$ (a) and mean
                 coefficient of $B_n$ (b).}

For a missing-trace interpolation test
(Figure~\ref{fig:zero,ramiss}a), we removed 40\% of randomly selected
traces from the input data
(Figure~\ref{fig:sean,sean2}a). Furthermore, the first five traces
were also removed to simulate traces missing at near offset. The
adaptive PEF can only use a small number of coefficients in the
spatial direction because of a small number of fitting equations
(where the adaptive PEF lies entirely on known data). However, it also
limits the ability of the proposed method to interpolate dipping
events. We used a nonstationary PEF with 4 (time) $\times$ 3 (space)
coefficients for each sample and a 50-sample (time) $\times$ 10-sample
(space) smoothing radius to handle the missing trace recovery. The
result is shown in Figure~\ref{fig:zero,ramiss}b. By comparing the
results with the original input (Figure~\ref{fig:sean,sean2}a), the
missing traces are interpolated reasonably well except for weaker
amplitude of the steeply dipping events.
% which also appear in the
%difference between interpolated traces and original traces
%(Figure~\ref{fig:diff}).

\inputdir{missing}
   \multiplot{2}{zero,ramiss}{width=0.47\columnwidth}{Field data
                 with 40\% randomly missing traces (a), and
                 reconstructed data using RNA (b).}

%    \plot{diff}{width=0.47\columnwidth}{Difference between original
%                 gather (Figure~\ref{fig:sean,sean2}a) and
%                 interpolated result (Figure~\ref{fig:zero,ramiss}b).}

An extension of the method to 3-D is straightforward and follows a
two-step least-squares method with 3-D adaptive PEF
estimation. We use a set of shot gathers as the input data volume to
further test our method (Figure~\ref{fig:sean3,zero3}a). We removed
50\% of randomly selected traces and five near offset traces for all
shots (Figure~\ref{fig:sean3,zero3}b). For comparison, we used
PWD to recover the missing traces
(Figure~\ref{fig:miss3,diff3,amiss3,adiff3}a). The PWD method produces
a reasonable result after carefully estimating dip information, but
the interpolated error is slightly larger in the diffraction
locations. (Figure~\ref{fig:miss3,diff3,amiss3,adiff3}b). The additional
direction provided more information for interpolation but also
increased the number of zeros in the mask operator $K(t,x)$, which
constrains enough fitting equations in equation~\ref{eq:pred3}. To use
the available fitting equations for adaptive PEF estimation, we chose
a smaller number of coefficients in the spatial direction. The
proposed method is able to handle conflicting dips, although it does
not appear to improve the dipping-event recovery compared to the 2-D
case. This characteristic partly limits the application of RNA in 3-D
case. We used a 3-D nonstationary PEF with 4 (time) $\times$ 2
(space) $\times$ 2 (space) coefficients for each sample and a
50-sample (time) $\times$ 10-sample (space) $\times$ 10-sample (space)
smoothing radius was selected. Similar to the result in the 2-D
example, Figure~\ref{fig:miss3,diff3,amiss3,adiff3}c shows the
interpolation result, in which only steeply-dipping low-amplitude
diffraction events with are lost
(Figure~\ref{fig:miss3,diff3,amiss3,adiff3}d).

   \multiplot{2}{sean3,zero3}{width=0.42\columnwidth}{A
                3-D field data volume (a) and data with
                50\% randomly missing traces (b).}

   \multiplot{4}{miss3,diff3,amiss3,adiff3}{width=0.42\columnwidth}{
                Reconstructed data volume using 3-D plane-wave
                destruction (a), difference between original input
                (Figure~\ref{fig:sean3,zero3}a) and interpolated
                result (Figure~\ref{fig:miss3,diff3,amiss3,adiff3}a),
                reconstructed data volume using 3-D regularized
                nonstationary autogression (c), and difference between
                original input (Figure~\ref{fig:sean3,zero3}a) and
                interpolated result
                (Figure~\ref{fig:miss3,diff3,amiss3,adiff3}c) (d).}


\section{Conclusions}
We have introduced a new approach to adaptive prediction-error
filtering for seismic data interpolation. Our approach uses
regularized nonstationary autoregression to handle time-space
variation of nonstationary seismic data. We apply this method to
interpolating seismic traces beyond aliasing and to reconstructing
data with missing and decimated traces. Experiments with benchmark
synthetic examples and field data tests show that the proposed filters
can depict nonstationary signal variation and provide a useful
description of complex wavefields having multiple curved events. These
properties are useful for applications such as seismic data
interpolation and regularization. Other possible applications may
include seismic noise attenuation.

\section{Acknowledgments}
We are grateful to Raymond Abma for providing benchmark data sets. We
thank Eric Verschuur, Bill Curry, and two anonymous reviewers for
helpful suggestions, which improved the quality of the paper.

This publication was authorized by the Director, Bureau of Economic
Geology, The University of Texas at Austin.

\bibliographystyle{seg}
\bibliography{SEG,SEP2,seg2010}

