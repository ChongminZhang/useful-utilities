\published{Geophysics, 80, C89-C105, (2015)}

\title{On anelliptic approximations for qP velocities in TI and orthorhombic media}
\author{Yanadet Sripanich and Sergey Fomel}

\address{
Bureau of Economic Geology \\
John A. and Katherine G. Jackson School of Geosciences \\
The University of Texas at Austin \\
University Station, Box X \\
Austin, TX 78713-8924 
}
\maketitle

\righthead{qP velocities approximations}
\lefthead{Sripanich \& Fomel}
\footer{TCCS-9}

\begin{abstract}
Anelliptic approximations for phase and group velocities of qP waves in transversely isotropic (TI) media have been widely applied in various seismic data processing and imaging tasks. 
We revisit \old{several successful anelliptic} \new{previously proposed} approximations and suggest two improvements. 
\old{The first improvement involves an alteration of the anelliptic parameter term to achieve better fitting along the non-symmetry axis. 
The second improvement involves finding an empirical connection between anelliptic parameters along symmetry and non-symmetry axes based on laboratory measurements of anisotropy of rock samples of different types. }\new{The first improvement involves finding an empirical connection between anelliptic parameters along different fitting axes based on laboratory measurements of anisotropy of rock samples of different types. The relationship between anelliptic parameters observed is strongly linear suggesting a novel set of anisotropic parameters suitable for the study of qP-wave signatures. The second improvement involves suggesting a new functional form for the anelliptic parameter term to achieve better fitting along the horizontal axis.} These modifications lead to improved three-parameter and four-parameter approximations for phase and group velocities of qP waves in TI media. In a number of model comparisons, the new three-parameter approximations appear to be more accurate than \old{the} previous approximations \new{with the same number of parameters}.
These modifications also serve as \new{a} \old{foundations to}\new{foundation for} an extension to orthorhombic media where qP velocities involve nine independent \new{elastic} parameters. \new{However,} as shown by previous researchers, qP wave propagation in orthorhombic media can be adequately approximated using just six combinations of those nine parameters. 
We propose novel six-parameter approximations for phase and group velocities for qP wave\new{s} in orthorhombic media.
The proposed orthorhombic phase-velocity approximation provides a more accurate alternative to previously known approximations and can find applications in full-wave modeling, imaging, and inversion. 
The proposed group-velocity approximation is also highly accurate and can find applications in ray tracing and \old{seismic data}\new{velocity} analysis.
\end{abstract}

\section{Introduction}

Anellipticity is a well-known characteristic of elastic wave propagation in anisotropic media. 
The simplest, yet \new{practically} important case \new{of anellipticity,} occurs in transversely isotropic media \old{,
where qP waves often exhibit strong anellipticity} \cite[]{grechkabook,tsvankinbook,thomsenbook}.
In recent years, it has been recognized that transverse isotropy may not be sufficient to characterize the actual media encountered in \old{seismic experiments} \new{many regions of the world} and as a result, orthorhombic anisotropy has become a significant topic of interest \cite[e.g.][]{tsvankinortho,tsvankinbook,bakulin2,xu,vascon,grechkabook,fowlerortho,thomsenbook}. 
One important example of an orthorhombic medium is a sedimentary basin exhibiting parallel vertical cracks embedded in a background medium 
with vertical transverse isotropy \cite[]{helbig,tsvankinortho,tsvankinbook,grechkabook}. 
In such \old{a medium} \new{media}, \new{three-dimensional} anellipticity remains an important characteristic of elastic wave propagation.\old{Moreover,}\cite{tsvankinortho,tsvankinbook} \old{observed}\new{pointed out} that the elastic wave propagation in TI media resembles the \old{in-plane}elastic wave propagation \new{in the symmetry plane} of orthorhombic media. This observation enables an accurate description of orthorhombic anellipticity using only a limited number of parameters \old{and the concept}\new{by extending the approach} used to approximate anellipticity in TI media.

The exact expressions for qP phase and group velocities in TI media involve four independent parameters \cite[]{gass,berryman}. 
\cite{alkatsvankin} \new{and \cite{alkavti}} showed that three combinations of those four parameters are sufficient to describe qP wave propagation with high accuracy. Although the exact expression of phase velocity in terms of phase angle is known, the exact expression for group velocity 
in terms of group angle \old{can be}\new{appears} too complicated for practical use. Therefore, accurate approximations involving a small number of independent parameters are needed. \old{On the other hand,}In orthorhombic media, the exact expression for qP phase velocity can be derived as a solution of a cubic equation and involves nine parameters. However, only six combinations of those nine parameters are sufficient to accurately describe qP wave propagation \cite[]{tsvankinortho,tsvankinbook}. The exact expression of qP group velocity in orthorhombic media can be derived from phase velocity expressions, but this expression is again \old{complicated}\new{cumbersome} and \new{can only be} expressed in terms of \new{the} phase angle instead of group angle. 
Therefore, \old{it}\new{this expression} is not always convenient for practical applications such as ray tracing and moveout \old{approximations} \new{correction}, where the expression in terms of the group angle (seismic ray direction) is often preferred. 

Many approximations have been proposed previously for both phase and group velocities in TI media
\cite[e.g.][]{mdk,alkatsvankin,tsvankinvti,alkavti,alkavti2,alkavti3,schohoop,stopin,zhang,daley,fomel,ursin,fomelstovas,stovas2010,farra}. 
\cite{fowlervti} presented a comprehensive comparative review of many of these approximations.
Accuracy comparison of several group-velocity approximations \new{(}in terms of moveout approximations\new{)} was also presented by \new{\cite{aleixo} and} \cite{golikov}. Among these different approaches, \cite{fomel} proposed an extension of the Muir-Dellinger approach \cite[]{md,mdk} 
using the shifted-hyperbola functional form. The resultant three-parameter approximation for phase velocity is identical to
the acoustic approximation of \cite{alkavti,alkavti2} and the empirical approximation of \cite{stopin}. 
The corresponding three-parameter approximation for group velocity was new  at the time and proved to be 
exceptionally accurate in comparison with other known approximations. 
\old{This three-parameter approximation is also a particular case of the five-parameter generalized approximation (Fomel and Stovas, 2010; Stovas, 2010).} 

In the first part of this study, we revisit the anelliptic approximations by \cite{fomel} and further improve their accuracy \new{by using an empirical relationship between the vertical and horizontal anelliptic parameters extracted from many laboratory measurements of stiffness tensor coefficients}. We \new{also} modify the functional form of the approximations to improve their behavior at large angles.
\old{We further improve their accuracy by using the relationships between the anelliptic parameters extracted from many laboratory measurements of stiffness tensor coefficients.}

\old{Several} \new{Many} studies of elastic wave propagation and velocity approximations in orthorhombic media have been \old{conducted in the past}\new{reported in the literature}, and several alternative six-parameter approximations for qP phase velocity have been proposed \cite[]{tsvankinortho,alkaortho,grechkabook,song,hao}.
Several group-velocity approximations \new{for orthorhombic media} have been proposed in the form of moveout approximations \old{for orthorhombic media} \cite[]{xu,vascon}. \old{Because}\new{Using the fact that} the elastic wave propagation in each of the three symmetry planes of orthorhombic media is controlled by the same Christoffel equation as in the case of TI media \cite[]{tsvankinortho,tsvankinbook}, we develop novel approximations for orthorhombic qP velocities by starting from our approximations in TI media. We extend our anelliptic TI approximations to a 3D form suitable for approximation of phase and group velocities of qP waves in orthorhombic media. \old{Following the approach of Fomel(2004), we apply the shifted-hyperbola approach to extend the approach of Muir and Dellinger (1985). We then construct anelliptic six-parameter approximations for both phase and group velocities of qP waves.}
Using a set of test models, we check the accuracy of the proposed approximations and verify that they provide more accurate alternatives to the previously known approximations. In some of the models, the improvement in accuracy is dramatic \old{(an order of magnitude = 10 times)}\new{and reaches a factor of ten}. The proposed approximations can readily be used in seismic data processing and imaging applications. \new{We show examples of applying the proposed phase-velocity approximations for TI and orthorhombic media in wave extrapolation experiments.}


\new{\section{Choice of anisotropic parameters}}

\new{Previous researchers have shown that in TI and orthorhombic media, only three and six combinations of stiffness coefficients respectively are sufficient to describe qP-wave propagation. The most widely used parameters are Thomsen's parameters \cite[]{thomsen} in TI media and their extension to orthorhombic media \cite[]{tsvankinortho}. Another notable parameterization scheme involves the $\eta$ parameter \cite[]{alkatsvankin} and is commonly used when the so-called acoustic approximation is assumed \cite[]{alkavti,alkavti2,alkavti3,alkaortho}. In this study, following the work of \cite{md} and \cite{mdk}, we adopt the set of anisotropic parameters, referred to as \textit{Muir-Dellinger parameters}, which represent different combinations of elastic moduli.} 

\new{In our notation, the Muir-Dellinger parameters include $w_1$, $w_3$, $q_1$, and $q_3$, where $w_1$ denotes the horizontal velocity squared and $w_3$ denotes the vertical velocity squared, and $q_1$ and $q_3$ are anelliptic parameters derived from fitting the phase velocity curvatures along the horizontal axis and vertical axis. In terms of density-normalized stiffness tensor coefficients in Voigt notation, $w_1=c_{11}$, $w_3=c_{33}$, and
\begin{eqnarray}
\label{eq:qh}
q_1 & = & \frac{c_{55}(c_{11}-c_{55})+(c_{55}+c_{13})^2}{c_{33}(c_{11}-c_{55})}~, \\
\label{eq:qv}
q_3 & = & \frac{c_{55}(c_{33}-c_{55})+(c_{55}+c_{13})^2}{c_{11}(c_{33}-c_{55})}~.
\end{eqnarray}
Equations~\ref{eq:qh} and~\ref{eq:qv} were derived previously by \cite{md}. Thomsen's parameters and the parameter $\eta$ used by \cite{alkatsvankin} are related to $q_1$ and $q_3$ by the conversion shown in Table~\ref{tbl:ticonversion}. Table~\ref{tbl:tischeme} shows a comparison between parameterization schemes for anisotropic parameters in TI media. Uncertainty analysis for different choices of anisotropic parameters in TI media is discussed in Appendix A.}

\tabl{ticonversion}{Conversion table between anelliptic parameters, Thomsen's parameters \cite[]{thomsen} and the time-processing parameter $\eta$ \cite[]{alkatsvankin}. $R=V^2_{S0}/V^2_{P0}$ denotes the ratio between vertical velocity squared of qP-wave and qS-wave. }
{
\centering
      	   \begin{tabular}{c c}
      	   \hline Anelliptic & Thomsen \\
           \hline \hline \\ $ q_1 = 1+\frac{2(R-1)(\delta-\epsilon)}{R-(1+2\epsilon)}$ & $\delta =  \frac{(q_3-q_1)+R(q_1q_3-2q_3+1)}{2(q_1-q_3 +R (q_3-1))}$ \\ \\
            $ q_3 = \frac{1+2\delta}{1+2\epsilon} = \frac{1}{1+2\eta} $ & $\epsilon = \frac{(q_1-q_3)(R-1)}{2(q_1-q_3 +R (q_3-1))}$\\ \\
      \hline
    \end{tabular}
}

\tabl{tischeme}{Comparison of four-parameter parameterization schemes for qP-wave anisotropic parameters.}
{
\centering
     	     \begin{tabular}{c c c c}
     	     \hline Schemes & Parameters & Elliptical Ani. & Acoustic Approx. \\ 
     	     \hline \hline \\ \cite{thomsen} & $V_{P0}$, $V_{S0}$, $\epsilon$, and $\delta$ & $\epsilon = \delta$ & $V_{S0}=0$\\ \\
     	     \cite{alkavti} & $V_{P0}$, $V_{S0}$, $V_{nmo}$ and $\eta$  & $\eta=0$ & $V_{S0}=0$\\ \\
     	     \cite{md},  & \multirow{2}*{$w_1$, $w_3$, $q_1$ and $q_3$} & \multirow{2}*{$q_1 = q_3 = 1$} &   \multirow{2}*{$q_1 = q_3$} \\ 
             \cite{fomel}, and proposed &  &  & \\ \\
      \hline
    \end{tabular}
}

\new{To study correlations between anisotropic parameters, we compiled laboratory measurements of stiffness tensor elements or other equivalent measurements of anisotropy  \cite[]{jw,thomsen,helbig,vernik,wang}. Computing $q_1$ and $q_3$ according to equations~\ref{eq:qh} and~\ref{eq:qv}, we discover an empirical relationship between anelliptic parameters shown in Figure~\ref{fig:qshalelegnew,qsandlegnew,qcarbonateleg}. The results suggest that $q_1$ and $q_3$ exhibit a linear relationship ($q_1 = a q_3 + b$) that appears to depend solely on lithology, regardless of the geographical location of the samples. We can also observe that the proportionality constant $a$ is close to 1 for nearly isotropic rocks (carbonates and sandstones) but deviates from 1 for highly anisotropic rocks (shales). We assume that each relationship between $q_1$ and $q_3$ given in Figure~\ref{fig:qshalelegnew,qsandlegnew,qcarbonateleg} is valid for that type of media and therefore, the proportionality constant $a$ and the intercept $b$ are not new independent parameters.}

\new{In consideration of a linear relationship ($q_1 = a q_3 + b$) above, the ratio of veritical qP- and qSV-wave velocites squared $R=V^2_{S0}/V^2_{P0}$ can be expressed as,
\begin{equation}
R = \frac{(1+2\epsilon)[(1+2\delta)(a-1) + b(1+2\epsilon)]}{a(1+2\delta)+(1+2\epsilon)(b-1-2(\delta-\epsilon))}~.
\end{equation}
Therefore, if $a=1$ and $b=0$, $R=0$. This result agrees with that of \cite{fomel}, who previously showed that if $w_1 \ne w_3$ , $q_1 \ne 1$, and $q_3 \ne 1$, setting $q_1=q_3$ is equivalent to the assumption used in the acoustic approximation \cite[]{alkavti}. In other words,
\begin{equation}
\lim_{R \to 0} q_1 = q_3~.
\end{equation}
A similar condition is discussed by \cite{fowlervti}.}

\multiplot{3}{qshalelegnew,qsandlegnew,qcarbonateleg}{width= 0.45\textwidth}{Relationship between $q_1$ and $q_3$ in different lithology. 
Data are obtained from various publications on laboratory measurements. Dashed line indicates graph of $q_1=q_3$ in each case. }
%The parameters for Greenhorn shales are $c_{11}=14.47$ $ km^2/s^2$, $c_{33}=9.57$ $ km^2/s^2$, $c_{55}=2.28$ $ km^2/s^2$, 
%and $c_{13}=4.51$ $ km^2/s^2$.

\new{Because the in-plane qP-wave propagation in orthorhombic media behaves identically to the case of TI media \cite[]{tsvankinbook}, we can extend the Muir-Dellinger parameters from 2D to 3D appropriately for studies of orthorhombic media. The full set of parameters in 3D includes $w_1$, $w_2$, $w_3$, $q_{21}$, $q_{31}$, $q_{12}$, $q_{32}$, $q_{13}$, and $q_{23}$, where $w_i$ denotes the velocity squared in the $n_i$ direction and $q_{ij}$ denotes the anelliptic parameters derived from fitting the phase velocity curvatures along the $n_i$ axis in the symmetry plane defined by the $n_j$ axis (Figure~\ref{fig:fittingnotation}). For example, in plane [$n_1$,$n_3$], we consider $q_{12}$ and $q_{32}$ because we can find $q$ either by fitting along $n_1$ or $n_3$ axis with $n_2$ as the axis defining the symmetry plane. The expressions for the anelliptic parameters are as follows:
\begin{eqnarray}
q_{21} & = & \frac{c_{44}(c_{22}-c_{44})+(c_{44}+c_{23})^2}{c_{33}(c_{22}-c_{44})}~, \\
q_{31} & = & \frac{c_{44}(c_{33}-c_{44})+(c_{44}+c_{23})^2}{c_{22}(c_{33}-c_{44})}~, \\
%q_{12} & = & \frac{c_{55}(c_{11}-c_{55})+(c_{55}+c_{13})^2}{c_{33}(c_{11}-c_{55})}~, \\
%q_{32} & = & \frac{c_{55}(c_{33}-c_{55})+(c_{55}+c_{13})^2}{c_{11}(c_{33}-c_{55})}~, \\
q_{13} & = & \frac{c_{66}(c_{11}-c_{66})+(c_{66}+c_{12})^2}{c_{22}(c_{11}-c_{66})}~, \\
q_{23} & = & \frac{c_{66}(c_{22}-c_{66})+(c_{66}+c_{12})^2}{c_{11}(c_{22}-c_{66})}~.
\end{eqnarray}
}
\new{Expressions for $q_{12}$ and $q_{32}$ are equivalent to expressions for $q_1$ and $q_3$ in equations~\ref{eq:qh} and~\ref{eq:qv}. For subscript convention on Muir-Dellinger parameters and associating expressions in 3D, we reserve $i$, $j$, and $k$ specifically to indicate the fitting location in each symmetry plane. The summation convention for repeating indices is not assumed. Alternatively, we adopt the notation that, for each combination of $i$, $j$, and $k$, the first digit indicates the index of the fitting axis and the second digit indicates the index of the axis defining the symmetry plane.  Therefore, in our notation, $i$, $j$, and $k$ are integers between 1 and 3 and in each expression, they must be different from one another. This set of $q_{ij}$ parameters may also be considered as elements of a 3$\times$3 matrix with the diagonal elements of this matrix absent. Table~\ref{tbl:orthoscheme} shows a comparison between parameterization schemes for anisotropic parameters in orthorhombic media.}

\tabl{orthoscheme}{Comparison of parameterization schemes for qP-wave anisotropic parameters and the assumption for acoustic approximation \new{in orthorhombic media}.}
{
\centering
     	     \begin{tabular}{c c c c}
     	     \hline Schemes & Parameters & Acoustic Approx. \\ 
     	     \hline \hline \\ 
             \multirow{3}*{\cite{tsvankinortho}} & $V_{P0}$, $V_{S0}$, $\delta^{(3)}$, & \multirow{3}*{-}\\
                        & $\epsilon^{(1)}$, $\delta^{(1)}$,  $\gamma^{(1)}$, & \\
                        & $\epsilon^{(2)}$, $\delta^{(2)}$,  $\gamma^{(2)}$ & \\ \\
     	     \multirow{3}*{\cite{alkaortho}} & $V_v$, $V_{S1}$, $V_{S2}$,& $V_{S1}=0$\\
                        & $V_{S3}$, $V_{nmo_1}$,  $V_{nmo_2}$, & $V_{S2}=0$\\
                        & $\eta_1$, $\eta_2$,  $\delta$ & $V_{S3}=0$\\ \\
     	     \multirow{3}*{Proposed} & $w_1$, $w_2$, $w_3$, &  $q_{21} = q_{31}$,\\
                        & $q_{12}$, $q_{21}$, $q_{13}$, & $q_{12} = q_{32}$ \\
                        & $q_{31}$, $q_{23}$, $q_{32}$ &  $q_{13} = q_{23}$ \\ \\
      \hline
    \end{tabular}
}

\new{On the basis of the Muir-Dellinger parameters in TI and orthorhombic media introduced in this section, we propose novel phase- and group-velocity approximations for qP waves in the subsequent sections. We first suggest a symmetric extension of the velocity approximations in TI media by \cite{fomel} and also extend the expressions to the orthorhombic case. We then utilize the relationships from Figure~\ref{fig:qshalelegnew,qsandlegnew,qcarbonateleg} to reduce the number of independent parameters in the proposed approximations to three and six in TI and orthorhombic media respectively so that they require the similar number of parameters as the other previously suggested approximations.}


\section{Transversely isotropic media}
\subsection{Exact Expression}

The phase velocity of qP waves in TI media has the following well-known explicit expression \cite[]{gass,berryman}:
\begin{equation}
\label{eq:exactphase}
v^2_{phase} = \frac{1}{2}[(c_{11}+c_{55})n^2_1 + (c_{33}+c_{55})n^2_3] + \frac{1}{2}\sqrt{[(c_{11}-c_{55})n^2_1 - (c_{33}-c_{55})n^2_3]^2 + 4(c_{13}+c_{55})^2n^2_1n^2_3}~,\\
\end{equation}
where $c_{ij}$ are density-normalized stiffness tensor coefficients in Voigt notation, $n_1 = \sin\theta$, $n_3 = \cos\theta $, and $\theta$ is the phase angle 
(measured from the vertical axis). Group velocity can be determined from phase velocity using the general expression \cite[]{cerveny}
\new{\begin{equation}
\label{eq:generalgroup}
\mathbf{v}_{group} = v_{phase}\mathbf{n} + (\mathbf{I} - \mathbf{n}\mathbf{n}^T)\nabla_{\mathbf{n}}v_{phase}~,\\
\end{equation}
}
where $\mathbf{I}$ denotes the identity matrix, $\mathbf{n}=\{n_1,n_3\}$ is the phase direction vector, and \old{$\nabla_{\mathbf{n}}v_{phase}$} \new{$\nabla_{\mathbf{n}}v_{phase} = \{\frac{\partial v_{phase}}{n_1},\frac{\partial v_{phase}}{n_3}\}^T$} is 
the gradient of $v_{phase}$ with respect to $\mathbf{n}$. \new{Using Muir-Dellinger parameters, the exact phase velocity for qP waves (equation~\ref{eq:exactphase}) can be expressed as
\begin{equation}
\label{eq:exactq}
v^2_{phase} = \frac{1}{2}\left[w_1n^2_1 + w_3n^2_3 + w_{13}\right] + \frac{1}{2}\sqrt{f}~,\\
\end{equation} 
where
\begin{eqnarray}
\nonumber
f & = & \left[w_1n^2_1 + w_3n^2_3 - w_{13}\right]^2 + \frac{4(q_1-1)(q_3-1)(w_3-w_1)w_{13}n^2_1n^2_3}{(q_1-q_3)}~,\\
\nonumber
w_{13} & = & \frac{(q_1-q_3)w_1w_3}{(q_1-1)w_3-(q_3-1)w_1}~.
\end{eqnarray}} 

\subsection{Muir and Dellinger Approximations}

Similar to the derivations by \cite{fomel}, the Muir-Dellinger approximations \cite[]{md,mdk} serve as the starting point of our derivation. The Muir-Dellinger phase-velocity approximation is of the following form:
\begin{equation}
v^2_{phase}(n_1,n_3) \approx  e(n_1,n_3) + \frac{(q-1)w_1w_3n^2_{1}n^2_{3}}{e(n_1,n_3)}~, \\
\label{eq:mdphase}
\end{equation}
where $q$ is the anelliptic parameter ($q = 1$ in case of elliptical anisotropy), $w_1=c_{11}$ denotes the horizontal ($n_1$) velocity squared, $w_3=c_{33}$ denotes the vertical ($n_3$) velocity squared,
and $e(n_1,n_3)$ describes the elliptical part of the velocity and is defined by
\begin{equation}
e(n_1,n_3) = w_1n^2_1 + w_3n^2_3~.
\end{equation}

The group-velocity approximation takes a similar form, but with symmetric changes in the coefficients and variables,
\begin{equation}
\frac{1}{v^2_{group}(N_1,N_3)} \approx E(N_1,N_3) + \frac{(Q-1)W_1W_3N^2_{1}N^2_{3}}{E(N_1,N_3)}~, \\
\label{eq:mdgroup}
\end{equation}
where $N_1 = \sin\Theta$, $N_3 = \cos\Theta $, $\Theta$ is group angle (from vertical), $W_1=1/w_{1}$ denotes the horizontal \old{($n_1$)} slowness squared, $W_3=1/w_{3}$ denotes the vertical \old{($n_3$)} slowness squared,
$Q = 1/q$, and $E(N_1,N_3)$ describes the elliptical part of the slowness and is defined by
\begin{equation}
E(N_1,N_3) = W_1N^2_1 + W_3N^2_3~.
\end{equation}

\new{As suggested by \cite{md},} the $q$ parameter can be found by fitting the phase-velocity curvature around either the vertical axis ($\theta = 0$)  or the horizontal axis ($\theta = \pi/2$). The explicit expressions of $q$ fitting in those two cases are \old{as follows (Fomel,2004):}\new{given in equations~\ref{eq:qh} and~\ref{eq:qv}.} If we define $Q$ in equation~\ref{eq:mdgroup} by fitting the group velocity curvature around either $\Theta = 0$ or $\Theta = \pi/2$, we find that
\begin{equation}
 Q_i=1/q_i~.
\end{equation}
\new{Extending this idea, \cite{mdk} proposed four-parameter approximations for phase and group velocites using both $q_1$ and $q_3$.}

\subsection{Previous Approximations}

To obtain more accurate approximations, \cite{fomel} suggested applying the shifted-hyperbola functional form, 
which introduces \new{shift} parameters $s$ (for phase velocity) and $S$ (for group velocity) into the approximations \old{The approximations then become}\new{ using the following functional form:}
\begin{equation}
\label{eq:sphase}
v^2_{phase} \approx e(n_1,n_3)(1-s) + s\sqrt{e^2(n_1,n_3) + \frac{2(q-1)w_1w_3n^2_{1}n^2_{3}}{s}}~,
%v^2_{phase} & \approx & e(n_1,n_3)(1-s) + \\
%		& & s\sqrt{e^2(n_1,n_3) + \frac{2(q-1)c_{11}c_{33}n^2_{1}n^2_{3}}{s}}~,
\end{equation}
\begin{equation}
\label{eq:sgroup}
\frac{1}{v^2_{group}} \approx E(N_1,N_3)(1-S) + S\sqrt{E^2(N_1,N_3) + \frac{2(Q-1)W_1W_3N^2_{1}N^2_{3}}{S}}~.
%\frac{1}{v^2_{group}} & \approx & E(N_1,N_3)(1-S) + \\
%		& & S\sqrt{E^2(N_1,N_3) + \frac{2(Q-1)N^2_{1}N^2_{3}}{Sc_{11}c_{33}}}~.
\end{equation}

Parameters $s$ and $S$ can be found by fitting the fourth derivatives $d^4v_{phase}/d\theta^4$ and $d^4v_{group}/d\Theta^4$ to the exact phase and group velocities, respectively. The results derived by \cite{fomel} for the vertical fitting ($\theta=0$ or $\Theta = 0$) are
\begin{eqnarray}
\label{eq:fomels}
 s & = & \frac{(w_1-w_3)(q_3-1)(q_1-1)}{2[w_1(1-q_1-q_3(1-q_3)) - w_3((q_1-1)^2 + q_1(q_3-q_1))]}~, \\
\label{eq:fomelS}
 S & = & \frac{(W_3-W_1)(Q_3-1)(Q_1-1)}{2[W_1(Q_1-Q^3_3+Q^2_3-1) + W_3(Q_1(Q^2_3-Q_3-1)+1)]}~. 
% s & = & a/b~, \\
% \nonumber
% a & = & (c_{11}-c_{33})(q_3-1)(q_1-1)~, \\
% \nonumber
% b  & = & 2[c_{11}(1-q_1-q_3(1-q_3)) - c_{33}((q_1-1)^2 + q_1(q_3-q_1))]~,\\
%% \nonumber
%% 	  &   & c_{33}((q_1-1)^2 + q_1(q_3-q_1))]~,\\
% S & = & A/B, \\
% \nonumber
% A & = & (c_{11}-c_{33})(Q_3-1)(Q_1-1)~, \\
% \nonumber
% B  & = & 2[c_{33}(Q_1-Q^3_3+Q^2_3-1) +  c_{11}(Q_1(Q^2_3-Q_3-1)+1)]~.
%% \nonumber
%% 	  &   & c_{11}(Q_1(Q^2_3-Q_3-1)+1)]~.
\end{eqnarray}

The introduction of parameters $s$ and $S$ leads to an increase in the number of parameters from three to four.
To reduce this number back to three\old{again}, \cite{fomel} suggested setting $q_1 = q_3$, 
or equivalently, $Q_1 = Q_3$, which results in $s = 1/2$ and $S = 1/2(1+Q_3)$. Note that if we use equation~\ref{eq:qv} and set $q_1 = q_3$, this substitution will transform approximations~\ref{eq:sphase} and~\ref{eq:sgroup} to the following form:
\begin{equation}
v^2_{phase}(\theta) \approx \frac{1}{2}e(\theta) + \frac{1}{2}\sqrt{e^2(\theta)+4(q_3-1)w_1w_3\sin^2\theta \cos^2\theta}~, \\
\label{eq:fomelphase}
\end{equation}
%where
%\begin{equation}
%\nonumber
%w = e^2(\theta)+4(q_3-1)c_{11}c_{33}\sin^2\theta \cos^2\theta~,\\
%\end{equation}
and
\begin{equation}
\frac{1}{v^2_{group}(\Theta)} \approx \frac{1+2Q_3}{2(1+Q_3)}E(\Theta) + \frac{1}{2(1+Q_3)}\sqrt{E^2(\Theta) + 4(Q^2_3-1)W_1W_3\sin^2\Theta \cos^2\Theta}~,\\
\label{eq:fomelgroup}
\end{equation}
%where
%\begin{equation}
%\nonumber
%W = E^2(\Theta) + \frac{4(Q^2_3-1)\sin^2\Theta \cos^2\Theta}{c_{11}c_{33}}~.
%\end{equation}

The phase-velocity approximation in equation~\ref{eq:fomelphase} \old{happens to be} \new{is} equivalent to the acoustic approximation 
of \cite{alkavti,alkavti2} and the empirical approximation of \cite{stopin}, which were derived in a different way. \new{As discussed in the previous section, taking $q_1 = q_3$, or, equivalently, $Q_1 = Q_3$, is not necessarily the optimal choice, because the values of these two parameters may depend on the material properties of the media of interest. The actual empirical relationship between $q_1$ and $q_3$ can be extracted from laboratory or in situ measurements of the stiffness tensor elements in various media (Figure~\ref{fig:qshalelegnew,qsandlegnew,qcarbonateleg}). Furthermore, Equations~\ref{eq:fomelphase} and~\ref{eq:fomelgroup} are derived by fitting the derivatives up to fourth-order at either the vertical or horizontal axis, whereas fitting at the other axis is only first-order. This low-order fitting may lead to a loss of accuracy at larger angles ($\theta$ or $\Theta$).}

\old{Thomsen's parameters ($v_P$, $v_S$, $\epsilon$ and $\delta$) and the parameter $\eta$ used by Alkhalifah and Tsvankin (1995) are related to $q_1$ and $q_3$ by}

\old{Although approximations~\ref{eq:fomelphase} and~\ref{eq:fomelgroup} produce remarkably accurate results, two possibilities exist for improvement:}
	\old{1. Equations~\ref{eq:fomelphase} and~\ref{eq:fomelgroup} are derived by fitting the derivatives up to fourth-order at either the vertical or horizontal axis, whereas fitting at the other axis is only first-order. This low-order fitting may lead to a loss of accuracy 
	at larger angles ($\theta$ or $\Theta$).}
	\old{2.	Taking $q_1 = q_3$, or, equivalently, $Q_1 = Q_3$, is not necessarily the best choice, because the values of these two parameters are model-dependent. The actual empirical relationship between $q_1$ and $q_3$ can be extracted from laboratory or in situ measurements of the stiffness tensor elements in various TI media (Figure~\ref{fig:qshalelegnew,qsandlegnew,qcarbonateleg}).}

\subsection{Proposed Approximations}

To \old{address the first opportunity for improvement} \new{derive a more symmetric form}, we \new{return to the four-parameter expressions (equations~\ref{eq:sphase} and~\ref{eq:sgroup}) and} propose \old{a modification of equations~\ref{eq:sphase} and~\ref{eq:sgroup} }\new{to modify them} as follows:
\begin{equation}
\label{eq:mshphase}
v^2_{phase} \approx e(n_1,n_3)(1-\hat{s}) + \hat{s}\sqrt{e^2(n_1,n_3) + \frac{2(\hat{q}-1)w_1w_3n^2_{1}n^2_{3}}{\hat{s}}}~,\\
\end{equation}
and
\begin{equation}
\label{eq:mshgroup}
\frac{1}{v^2_{group}} \approx E(N_1,N_3)(1-\hat{S}) + \hat{S}\sqrt{E^2(N_1,N_3) + \frac{2(\hat{Q}-1)W_1W_3N^2_{1}N^2_{3}}{\hat{S}}}~,
\end{equation}
where
\begin{eqnarray}
\label{eq:qhatti}
\hat{q} &= ~~\frac{q_1 w_1 n^2_1 + q_3 w_3 n^2_3}{w_1 n^2_1 + w_3 n^2_3},~\hat{Q} &=~~ \frac{Q_1 W_1 N^2_1 + Q_3 W_3 N^2_3}{W_1 N^2_1 + W_3 N^2_3}~, \\
\label{eq:shatti}
\hat{s} &= ~~\frac{s_1 w_1 n^2_1 + s_3 w_3 n^2_3}{w_1 n^2_1 + w_3 n^2_3},~\hat{S} &=~~ \frac{S_1 W_1 N^2_1 + S_3 W_3 N^2_3}{W_1 N^2_1 + W_3 N^2_3}~.
\end{eqnarray}

\old{The expressions} \new{The modifications} in equation~\ref{eq:qhatti} are equivalent to the second anelliptic approximations by \cite{mdk}. \new{Again,} parameters $q_3$ and $q_1$ can be found by fitting the velocity profile curvatures at the vertical  ($\theta=0$) and horizontal ($\theta=\pi/2$) axis, respectively and are defined in equations~\ref{eq:qh} and~\ref{eq:qv}. Analogously, $s_3$ and $s_1$ can be found by fitting the fourth-order derivative ($d^4v_{phase}/d\theta^4$) at the same angle. A similar strategy applies to fitting \new{parameters} for the \old{group velocity} \new{group-velocity approximation}. \new{Note that expressions for $s_3$ and $S_3$ are different from equations~\ref{eq:fomels} and~\ref{eq:fomelS}.}

%The proposed approximations allow fitting of both phase- and group-velocity expressions along both axes leading to two different expressions for each fitting parameter ($q$, $Q$, $s$, and $S$) and both are needed in the approximations. This is different in the case of original Muir-Dellinger approximations (equations~\ref{eq:mdphase} and~\ref{eq:mdgroup}) where only one expression, for examples $q_1$ or $q_3$, is allowed.

Following \old{the fitting process as described} \new{this approach}, we derive the following expressions for $s_1$, $s_3$, $S_1$, and $S_3$:
\begin{eqnarray}
 \label{eq:s12}
 s_1 & = & a_{1}/b_{1}~, \\
 \nonumber
 a_{1} & = & (w_3-w_1)(q_1-1)^2(q_3-1)~, \\
 \nonumber
 b_{1}  & = & 2[w_3(q_1(q_1(q_1-2)+3)-2q_1q_3+q_3^2-1) - w_1(q_3(q_1(q_1-4)+q_3+1) +2q_1 -1 )]~,\\ \nonumber \\
  \label{eq:s32}
  s_3 & = & a_{3}/b_{3}~, \\
 \nonumber
 a_{3} & = & (w_1-w_3)(q_1-1)(q_3-1)^2~, \\
 \nonumber
 b_{3}  & = & 2[w_1(q_3(q_3(q_3-2)+3)-2q_1q_3+q_1^2-1) - w_3(q_1(q_3(q_3-4)+q_1+1) +2q_3 -1 )]~,\\ \nonumber
\end{eqnarray}
\begin{eqnarray}
 \label{eq:S12}
  S_1 & = & A_{1}/B_{1}~, \\
 \nonumber
 A_{1} & = & (W_1-W_3)(Q_1-1)^2(Q_3-1)~, \\
 \nonumber
 B_{1}  & = & 2[W_1(Q_3^2 +2Q_1 +Q_1Q_3(Q_1(Q_1-2)-1)-1) \\
 \nonumber
 	&   &- W_3(Q_3^2-2Q_1Q_3+Q_1(Q_1(Q_1-1)^2+2)-1)]~, \\ \nonumber \\
 \label{eq:S32}
 S_3 & = & A_{3}/B_{3}~, \\
 \nonumber
 A_{3} & = & (W_3-W_1)(Q_1-1)(Q_3-1)^2~, \\
 \nonumber
 B_{3}  & = & 2[W_3(Q_1^2 +2Q_3 +Q_1Q_3(Q_3(Q_3-2)-1)-1) \\
 \nonumber
 	&   & - W_1(Q_1^2-2Q_1Q_3+Q_3(Q_3(Q_3-1)^2+2)-1)]~.
\end{eqnarray}

\old{This modification introduces} \new{Note that equations~\ref{eq:mshphase} and~\ref{eq:mshgroup} introduce} three more parameters \old{making}\new{generating} six parameters in total, \old{including}\new{namely} 
$w_1$, $w_3$, $q_1$, $q_3$, $s_1$, and $s_3$ for equation~\ref{eq:mshphase} or $W_1$, $W_3$, $Q_1$, $Q_3$, $S_1$, and $S_3$ for equation~\ref{eq:mshgroup}. \old{By}\new{However,} expressing $s_i$ and $S_i$ in terms of $q_i$ and $Q_i$ in equations~\ref{eq:s12}-\ref{eq:S32}, we effectively reduce the dependency to four parameters. This reduction leads to four-parameter anelliptic approximations, which fit up to the fourth-order accuracy along both axes. \new{The exact phase- and group-velocity expressions also require the total of four independent parameters. However, the advantage of the proposed approximations lies in the existence of the group-velocity expression (equation~\ref{eq:mshgroup}) with analogous functional form as the phase-velocity expression (equation~\ref{eq:mshphase}).} To\old{address the second opportunity for improvement and to} reduce the number of parameters to three, \old{we compile laboratory measurements of stiffness tensor elements or other equivalent measurements 
of anisotropy (Jones and Wang, 1981; Thomsen, 1986; Schoenberg and Helbig, 1997; Vernik and Liu, 1997; Wang, 2002)
and compute $q_1$ and $q_3$ from equations~\ref{eq:qh} and~\ref{eq:qv} to try to find a more accurate empirical
relationship. The plots, shown in Figure~\ref{fig:qshalelegnew,qsandlegnew,qcarbonateleg},
 suggest that $q_1$ and $q_3$ do indeed exhibit a linear relationship ($q_1 = a q_3 + b$) that appears to depend only on lithology, regardless of the geographical location of the samples. The proportionality constant ($a$) is close to 1 for nearly isotropic rocks (carbonates and sandstones) but deviates noticeably for highly anisotropic rocks (shales). Introduction of this linear dependence leads to a reduction of the number of parameters from four to three.} \new{we utilize the linear relationships between $q_1$ and $q_3$ given in Figure~\ref{fig:qshalelegnew,qsandlegnew,qcarbonateleg}. The required $Q_1$ and $Q_3$ parameters for the group-velocity approximations can be found from the reciprocals of $q_1$ and $q_3$ for phase-velocity approximations, as mentioned above. Therefore, both phase- and group-velocity approximations derived on the basis of this approach require the same number of parameters. }

%A summary table for required independent parameters in various qP-wave phase-velocity approximations in TI media for both the ones mentioned in this paper and some notable ones proposed elsewhere is given in Table~\ref{tbl:tiparameter}. 

%\tabl{tiparameter}{Comparison of exact parameters, and required independent parameters and assumptions used for various qP-wave phase-velocity approximations.}
%{
%\centering
%     	     \begin{tabular}{c c c c}
%     	     \hline Schemes & $\#$ of Parameters & Parameters & Assumption \\ 
%     	     \hline \hline \\ Exact (Equation~\ref{eq:exactphase})& 4 & $c_{11}$, $c_{33}$, $c_{13}$ and $c_{55}$ & -\\ \\
%             \multirow{2}*{\cite{thomsen}} & \multirow{2}*{3} & \multirow{2}*{$V_{P0}$, $\epsilon$ and $\delta$} & Weak anisotropy\\
%              & & & ($\epsilon<1$ and $\delta<1$) \\ \\
%     	     \multirow{2}*{\cite{alkavti}} & \multirow{2}*{3} & \multirow{2}*{$V_{P0}$, $V_{nmo}$ and $\eta$}  & Acoustic approximation\\
%             & & & ($V_{S0}=0$) \\ \\
%             \cite{md} & 3 & $w_1$, $w_3$, and $q$  & -\\ \\
%     	     \cite{fomel} & 3 & $w_1$, $w_3$ and $q_3$ &  $q_1 = q_3$ \\ \\
%     	     Proposed (Equation~\ref{eq:mshphase}) & 3 & $w_1$, $w_3$ and $q_3$ &  $q_1 = a q_3 + b$ (Figure~\ref{fig:qshalelegnew,qsandlegnew,qcarbonateleg}) \\ \\
%      \hline
%    \end{tabular}
%}

\new{\subsubsection{Moveout approximation}}

\old{Note that} The group-velocity approximation in equation~\ref{eq:mshgroup} can be easily converted into the corresponding moveout equation using the relationship between
offset ($x$), vertical distance ($z$), and total reflection traveltime ($t$) given by
\begin{equation}
\label{eq:moveout}
t(x) = \frac{2\sqrt{(x/2)^2+z^2}}{V(\arctan(x/2z))}~,
\end{equation}
where $z = t_0 V(0)/2$ is the depth of the reflector, $t_0$ is the vertical two-way reflection traveltime, and V($\Theta$) is the approximated group velocity.
The moveout equation corresponding to equation~\ref{eq:mshgroup} is thus,
\begin{equation}
\label{eq:timoveout}
t^2(x) = H(x)(1-\hat{S}) + \hat{S}\sqrt{H^2(x)+\frac{2(\hat{Q}-1)t^2_0x^2}{\hat{S}Q_3V^2_{nmo}}}~,\\
\end{equation}
where
\begin{equation}
\nonumber
\hat{Q} = \frac{\frac{Q_1}{Q_3V^2_{nmo}}x^2 + Q_3t^2_0}{\frac{1}{Q_3V^2_{nmo}}x^2 + t^2_0},~~\hat{S} = \frac{\frac{S_1}{Q_3V^2_{nmo}}x^2 + S_3t^2_0}{\frac{1}{Q_3V^2_{nmo}}x^2 + t^2_0}~,
\end{equation}
$V_{nmo}$ denotes the NMO-velocity \cite[]{alkatsvankin} \new{and is given by
\begin{equation}
V^2_{nmo} = \frac{1}{W_1Q_3} = w_1 q_3 = V^2_{P0}\frac{1+2\epsilon}{1+2\eta} = V^2_{P0}(1+2\delta)~.
%w_1 & = & w_3 (1+2\epsilon) = V^2_{P0} (1+2\epsilon)~,
\end{equation}}
\old{and} $H(x)$ denotes the hyperbolic part of \new{the} reflection traveltime \new{squared and is} given \old{below:} \new{by}
\begin{equation}
H(x) = t^2_0 + \frac{x^2}{Q_3V^2_{nmo}}~.
\end{equation}
\old{By applying to equations~\ref{eq:S12} and~\ref{eq:S32} a linear relationship between $q_1$ and $q_3$, we reduce the number of moveout parameters from four to three.} \new{Assuming a particular media type and using a linear relationship between $q_1$ and $q_3$, we reduce the number of independent moveout parameters in the similar manner. However, note that $S_1$ (equations~\ref{eq:S12}) and $S_3$ (equation~\ref{eq:S32}) also depend on $W_1$ and $W_3$. Therefore, to effectively reduce the number of parameters in the moveout approximation (equation~\ref{eq:timoveout}) to three, we suggest, as an approximation, to adopt $Q_1=Q_3$ only for equations~\ref{eq:S12} and~\ref{eq:S32}, which lead to
\begin{equation}
\label{eq:strick}
S_1 = S_3 = \frac{1}{2(1+Q_3)}~.
\end{equation}
As a result, the moveout approximation depends on $t_0$, $V_{nmo}$, and $Q_3$.}

\new{For small offsets, the Taylor expansion of equation~\ref{eq:timoveout} is
\begin{equation}
t^2(x) \approx t^2_0 +\frac{x^2}{V^2_{nmo}} -\frac{1-2S_3(Q_1+1)+Q_3(4S_3+Q_3-2)}{2S_3Q_3^2t_0^2V^4_{nmo}} x^4 ~,
\end{equation}
which reduces to the expression given by \cite{fomel} by setting $Q_1=Q_3$. The asymptote of this expression for unbounded offset $x$ is given by
\begin{equation}
\frac{1}{Q_3 V^2_{nmo}} = \frac{1}{w_1}~,
\end{equation}
which is the horizontal velocity squared.}

\new{In the Muir-Dellinger notation, another nonhyperbolic moveout approximation, the generalized nonhyperbolic moveout approximation \cite[]{fomelstovas,stovas2010} can be expressed as
%\begin{eqnarray}
%\label{eq:gmat}
%t^2(x) &\approx&(1-\xi)(t_0^2 + ax^2) + \xi\sqrt{t_0^4+2bt_0^2x^2 + cx^4}~,\\
%\nonumber
%a & = & \frac{W_1 [(1-Q_3) (2 Q_3 (Q_3+2)+1) W_1+(Q_3 (2 Q_1 (Q_3+1)-3)-1) W_3]}{\left(Q_1-4 Q_3^2+3\right) W_1+(Q_1 (4 Q_3-1)-3) W_3}~,\\
%\nonumber
%b & = & \frac{W_1(Q_3-1)\left[\left(2 Q_3^2-1\right) W_1+W_3-2 Q_1 Q_3 W_3\right]}{(Q_1-1) (W_1-W_3)}~,\\
%\nonumber
%c & = & \frac{(Q_3-1)^2 W_1^2}{(Q_1-1)^2}~,\\
%\nonumber
%\xi & = & \frac{(Q_1-1) (W_1-W_3)}{4 \left(\left(Q_3^2-1\right) W_1+W_3-Q_1 Q_3 W_3\right)}~,
%\end{eqnarray}
%or alternatively, in the more popular form of moveout approximations' expression,
\begin{eqnarray}
\label{eq:gmaT}
t^2(x) &\approx&t_0^2 + \frac{x^2}{V^2_{nmo}} + \frac{Ax^4}{V^4_{nmo}\left(t_0^2 +B\frac{x^2}{V^2_{nmo}}+\sqrt{t_0^4+2Bt_0^2\frac{x^2}{V^2_{nmo}}+C\frac{x^4}{V^4_{nmo}}} \right)}~,\\
\nonumber
A & = & \frac{(Q_3-1)^2 (Q_1 W_3-Q_3 W_1)}{Q_3(Q_1-1) (W_1-W_3)}~,\\
\nonumber
B & = & \frac{(Q_3-1) \left[\left(2 Q_3^2-1\right) W_1+W_3-2 Q_1 Q_3 W_3\right]}{Q_3(Q_1-1) (W_1-W_3)}~,\\
\nonumber
C & = & \frac{(Q_3-1)^2}{(Q_1-1)^2 Q_3^2}~.
\end{eqnarray}
If the empirical assumption of $Q_1=Q_3$, or equivalently acoustic approximation is used, equation~\ref{eq:gmaT} reduces to the moveout approximation of \cite{fomel}.}

\subsection{Examples}

To investigate the accuracy of the proposed approximations, we make the relative error comparison \old{using}\new{with} both plots and tables \old{with multiple}\new{using several} anisotropy models based on values from laboratory rock samples. 
The plots in Figure~\ref{fig:vtiphaseplotlegnew,vtigroupplotleg1new,vtigroupplotleg2new} are generated\old{, for the sake of historical consistency,} using the stiffness tensor measurements of Greenhorn shales \cite[]{jw}, which have been applied for various approximation comparisons in the past \cite[e.g.][]{joethesis,fomel,stovas2010,farra}. 
Additionally, Tables~\ref{tbl:tirephase} and \ref{tbl:tiregroup} show the RMS relative error results of the new approximations, in comparison with results from some of the previously suggested approximations using the normalized stiffness tensor measurements given in Table~\ref{tbl:tisample}. \new{The RMS error computation is based on
\begin{equation}  
\mbox{RMS error} = \sqrt{\sum_{\psi=0}^{90} (v_{exact}(\psi)-v_{approx}(\psi))^2}~,
\end{equation}
where $\psi$ denotes phase or group angle as appropriate.} In all comparisons, we apply the relationships shown in Figure~\ref{fig:qshalelegnew,qsandlegnew,qcarbonateleg} to reduce the number of parameters from four to three.
For each model, the best-performing approximation is denoted in \new{red and} bold. The proposed approximations appear to be the most accurate in nearly all of the cases.

%\tabl{tisamplethomsen}{Thomsen parameters for samples in Table~\ref{tbl:tisample}. Velocities are given in $km$/$s$. }
%{
%\centering
%     	     \begin{tabular}{|l|c|c|c|c|}
%     	     \hline Shales sample & $ V_{P0}$ & $ V_{S0}$ & $ \epsilon $ & $ \delta $\\ 
%     	     \hline 1. Greenhorn  & 3.094 & 1.510 & 0.256 & -0.0505\\
%     	     \hline 2. Hard (brine) & 3.727 & 2.378 & 0.252 & 0.0347\\
%     	     \hline 3. North Sea (brine) & 2.291 & 1.341 & 0.195 & -0.0139\\
%     	     \hline 4. Dog Creek & 1.875 & 0.826 & 0.225 & 0.0998\\
%     	     \hline 5. Mesaverde & 3.749 & 2.621 & 0.128 & 0.0781\\
%     	     \hline 6. North Sea (dry) & 3.860 & 2.220 & 0.240 & 0.0199\\
%      \hline
%    \end{tabular}
%}

\tabl{tirephase}{RMS relative error (\%) from $0$-$90\,^{\circ}$ of phase-velocity approximations by \cite{thomsen}, \cite{alkavti} (similar to \cite{fomel}), and of the proposed three-parameter approximation for transversally-isotropic elastic models from Table~\ref{tbl:tisample}. Bold red highlight indicates the best-performing approximation. In all the cases, except sample 4, the proposed approximation appears to be the most accurate.}
{
\centering
     	     \begin{tabular}{|c|c|c|c|c|}
     	     \hline Sample & \cite{thomsen} & \cite{alkavti} & Proposed\\ 
     	     \hline 1 & 0.6789 & 0.1422 & \textcolor{red}{\textbf{0.0978}}\\
     	     \hline 2 & 0.6482 & 0.2254 & \textcolor{red}{\textbf{0.0503}}\\
     	     \hline 3 & 0.4564 & 0.1399 & \textcolor{red}{\textbf{0.0273}}\\
     	     \hline 4 & 0.2978 & \textcolor{red}{\textbf{0.0485}} & 0.0506\\
     	     \hline 5 & 0.1244 & 0.0541 & \textcolor{red}{\textbf{0.0201}}\\
     	     \hline 6 & 0.5710 & 0.1631 & \textcolor{red}{\textbf{0.0149}}\\
      \hline
    \end{tabular}
}

\tabl{tiregroup}{RMS relative error (\%) from $0$-$90\,^{\circ}$ of group-velocity approximations by \cite{alkatsvankin}, \cite{fomel}, \cite{farra} (second-order) and of the proposed three-parameter approximation for transversally-isotropic elastic models from Table~\ref{tbl:tisample}. Bold red highlight indicates the best-performing approximation. In all the cases, except samples 4 and 5, the proposed approximation appears to be the most accurate.}
{
\centering
     	     \begin{tabular}{|c|c|c|c|c|}
     	     \hline Sample & \cite{alkatsvankin} & \cite{fomel} & \cite{farra} & Proposed\\ 
     	     \hline 1 & 1.0149 & 0.1210 & 0.2530 & \textcolor{red}{\textbf{0.0801}}\\
     	     \hline 2 & 0.3306 & 0.2179 & 0.1351 &\textcolor{red}{\textbf{0.0564}}\\
     	     \hline 3 & 0.4602 & 0.1311 & 0.0977 & \textcolor{red}{\textbf{0.0194}}\\
     	     \hline 4 & 0.1369 & \textcolor{red}{\textbf{0.0467}}& 0.0983 & 0.0492\\
     	     \hline 5 & \textcolor{red}{\textbf{0.0188}} & 0.0540 & 0.0194 & 0.0202\\
     	     \hline 6 & 0.4258 & 0.1541 & 0.1412 & \textcolor{red}{\textbf{0.0084}}\\
      \hline
    \end{tabular}
}

\tabl{tisample}{\old{Elastic parameters from laboratory measurements in rock samples} \new{Normalized stiffness tensor coefficients (in $km^2$/$s^2$) from different TI samples}: 1 is from \cite{jw}, 2 and 3 are from \cite{wang}, 4 and 5 are from \cite{thomsen}, and 6 is from \cite{vernik}.}
{
\centering
     	     \begin{tabular}{|l|c|c|c|c|c|c|c|c|}
     	     \hline Shales sample & $ c_{11}$ & $ c_{33}$ & $ c_{13}$ & $ c_{55} $ & $ V_{P0}$ & $ V_{S0}$ & $ \epsilon $ & $ \delta $\\ 
     	     \hline 1. Greenhorn  & 14.47 & 9.57 & 4.51 & 2.28 & 3.094 & 1.510 & 0.256 & -0.0505\\
     	     \hline 2. Hard (brine) & 20.89 & 13.89 & 3.048 & 5.655 & 3.727 & 2.378 & 0.252 & 0.0347\\
     	     \hline 3. North Sea (brine) & 7.292 & 5.248 & 1.578 & 1.798 & 2.291 & 1.341 & 0.195 & -0.0139\\
     	     \hline 4. Dog Creek & 5.098 & 3.5163 & 2.4832 & 0.6823 & 1.875 & 0.826 & 0.225 & 0.0998\\
     	     \hline 5. Mesaverde & 17.653 & 14.055 & 1.3391 & 6.87 & 3.749 & 2.621 & 0.128 & 0.0781\\
     	     \hline 6. North Sea (dry) & 22.051 & 14.90 & 5.336 & 4.928 & 3.860 & 2.220 & 0.240 & 0.0199\\
      \hline
    \end{tabular}
}

\multiplot{3}{vtiphaseplotlegnew,vtigroupplotleg1new,vtigroupplotleg2new}{width= 0.55\textwidth}{Relative error plots using Greenhorn Shale measurements.
a) Phase velocity. b) Group velocity. c) Group velocity (finer scale).}

\section{Orthorhombic media}
\subsection{Exact Expressions}

\new{As the analog of equation~\ref{eq:exactphase}}\old{in orthorhombic media}, qP wave\new{s}\old{has} \new{have} the following well-known explicit exact expression for phase velocity \new{in orthorhombic media} \cite[]{helbig,tsvankinortho,tsvankinbook}:
\begin{equation}
\label{eq:orthoexactphase}
v^2_{phase} = 2\sqrt{\frac{-d}{3}}\cos(\frac{\nu}{3})-\frac{a}{3} ~,\\
\end{equation}
where
\begin{eqnarray}
\nonumber
\nu & = & \arccos \left(\frac{-q}{2\sqrt{(-d/3)^3}}\right)~,\\
\nonumber
q & = & 2\left(\frac{a}{3}\right)^3 - \frac{ab}{3}+c~,~~~d~~=~~-\frac{a^2}{3} + b~,\\
\nonumber
a & = & -(G_{11}+G_{22}+G_{33})~,\\
\nonumber
b & = & G_{11}G_{22}+G_{11}G_{33}+G_{22}G_{33}-G^2_{12}-G^2_{13}-G^2_{23}~,\\
\nonumber
c & = & G_{11}G^2_{23}+G_{22}G^2_{13}+G_{33}G^2_{12}-G_{11}G_{22}G_{33}-2G_{12}G_{13}G_{23}~,
%\nonumber
%&& -2G_{12}G_{13}G_{23}~,
\end{eqnarray}
and
\begin{eqnarray}
\nonumber
G_{11} & = & c_{11}n^2_1 + c_{66}n^2_2 + c_{55}n^2_3~,\\
\nonumber
G_{22} & = & c_{66}n^2_1 + c_{22}n^2_2 + c_{44}n^2_3~,\\
\nonumber
G_{33} & = & c_{55}n^2_1 + c_{44}n^2_2 + c_{33}n^2_3~,\\
\nonumber
G_{12} & = & (c_{12}+c_{66})n_1n_2~,\\
\nonumber
G_{13} & = & (c_{13}+c_{55})n_1n_3~.
\end{eqnarray}
Here, $c_{ij}$ are density-normalized stiffness tensor coefficients, $n_1 = \sin\theta \cos\phi$, $n_2 = \sin\theta \sin\phi$, $n_3 = \cos\theta$, $\theta$ is \new{zenith} phase angle (measured from $n_3$), and  $\phi$ is azimuthal phase angle (measured from $n_1$) \new{in the local orthorhombic frame of reference where the axes $n_1$, $n_2$, and $n_3$ are intersections of the corresponding planes of symmetry}. \old{Group velocity}\new{The corresponding group-velocity} expression can be determined from equation~\ref{eq:generalgroup} \new{extended to 3D}.


\subsection{Extended Muir-Dellinger Approximations}

\new{Considering} the Muir-Dellinger approximations for qP velocities in TI media \new{(equations~\ref{eq:mdphase} and~\ref{eq:mdgroup}) and the subscript convention introduce in the first section}\old{(Muir and Dellinger, 1985; Dellinger et al., 1993) can be naturally extended}\new{, we can naturally extend them} to orthorhombic media as follows:
\begin{equation}
\label{eq:emdphase}
v^2_{phase}(n_1,n_2,n_3) \approx e(n_1,n_2,n_3) + \sum\limits_{j=1}^3 \frac{(q_{ij}-1)w_iw_kn^2_{i}n^2_{k}}{e(n_1,n_2,n_3)}~,
\end{equation}
where $q_{ij}$ are the anelliptic parameters ($q_{ij} = 1$ in case of elliptical anisotropy),
$w_i=c_{ii}$ denotes \old{the}velocity squared along \new{the} $n_i$ axis, and $e(n_1,n_2,n_3)$ describes the ellipsoidal part of the velocity and is defined by
\begin{equation}
e(n_1,n_2,n_3) = w_1n^2_1 + w_2n^2_2 + w_3n^2_3~.
\end{equation}
\old{Note that, for subscript convention for the rest of this text, we use $l$ and $m$ as general-purpose indices and reserve $i$, $j$, and $k$ specifically to indicate the
fitting location in each symmetry plane. We adopt the notation that, for each combination of $i$, $j$, and $k$, the first digit indicates the index of the 
fitting axis and the second digit indicates the index of the symmetry axis. For example, in plane [$n_1$,$n_3$], we consider $q_{12}$ and $q_{32}$ because we can find $q$ 
either by fitting along $n_1$ or $n_3$ axis with $n_2$ as the symmetry axis. Therefore, in our notation, $i$, $j$, $k \in\{1,2,3\}$, and in each expression, they must be different from one another.
In other words, $i$, $j$, and $k$ are in cyclic order without variation in permutation. Therefore in [$n_1$,$n_3$], either $q_{12}$ or $q_{32}$ can be used, depending on the choice of fitting location, but the two cannot be used at the same time. 
This rule applies to the other two planes as well (Figure~\ref{fig:coordinate,fittingnotation}).} \new{This extension is based on the consideration of an ellipsoid in 3D as opposed to an ellipse in 2D and the additional two anelliptic terms involving $q_{ij}$ parameters from considering a total of three symmetry planes in orthorhombic media. It is also valid to consider $q_{kj}$ instead of $q_{ij}$ because in consistent with the original Muir-Dellinger approximation (equation~\ref{eq:mdphase}), only one anelliptic parameter is needed in each symmetry plane.} According to the Muir-Dellinger approach, we can also derive the group-velocity approximation, which takes a similar form, with symmetric changes in the coefficients and variables as shown below:
\begin{equation}
\label{eq:emdgroup}
\frac{1}{v^2_{group}(N_1,N_2,N_3)} \approx  E(N_1,N_2,N_3) + \sum\limits_{j=1}^3 \frac{(Q_{ij}-1)W_iW_kN^2_{i}N^2_{k}}{E(N_1,N_2,N_3)}~,
\end{equation}
where $N_1 = \sin\Theta \cos\Phi$, $N_2 = \sin\Theta \sin\Phi$, $N_3 = \cos\Theta$, $\Theta$ is \new{zenith} group angle (from vertical), 
$\Phi$ is azimuthal group angle (from $n_1$), $Q_{ij} = 1/q_{ij}$,
$W_i=1/w_{i}$ denotes \old{the}slowness squared along \new{the} $N_i$ axis,
and $E(N_1,N_2,N_3)$ describes the elliptical part of the slowness, defined by
\begin{equation}
E(N_1,N_2,N_3) = W_1N^2_1 + W_2N^2_2 + W_3N^2_3~.
\end{equation}

\old{These extended expressions} \new{This simple extension from 2D to 3D} stem\new{s} from the observation that elastic wave propagation in each symmetry plane of orthorhombic media is controlled by the same Christoffel equation as in the case of TI media \cite[]{tsvankinortho,tsvankinbook}. Therefore, if \new{any} \old{$n_l$ or $N_l$}\new{$n_i$ or $N_i$} is zero\old{for any $l=1,2,$ and $3$}, the extended expressions will simply reduce to the 2D Muir-Dellinger approximations for TI media in equations~\ref{eq:mdphase} and~\ref{eq:mdgroup}. Note that the expression in equation~\ref{eq:emdphase} is equivalent to the two leading terms of the phase-velocity pseudo-acoustic approximation derived by \cite{fowlerlapilli} and \cite{fowlerortho}.


\subsection{Proposed Approximations}

In the preceding section, we suggest\new{ed} \old{two improvements} \new{an improvement} to the anelliptic approximations for VTI media previously proposed by \cite{fomel}. \old{These two modifications led to more accurate three- and four-parameter approximations for qP velocities in TI media. Combining these modifications with}\new{Applying a similar modification to} the extended Muir and Dellinger approximations (equations~\ref{eq:emdphase} and~\ref{eq:emdgroup}), we can \old{extend}\new{write} the resultant approximations \old{to}\new{in} a new form suitable for approximation of velocities in orthorhombic media, as follows:
\begin{equation}
\label{eq:mshphaseortho}
v^2_{phase} \approx e(n_1,n_2,n_3)(1-\hat{s}) + \hat{s}\sqrt{e^2(n_1,n_2, n_3) + \frac{2\sum\limits_{j=1}^3 (\hat{q}_{j}-1)w_iw_kn^2_{i}n^2_{k}}{\hat{s}}}~,\\
\end{equation}
where
\begin{equation}
\hat{s} = \frac{\hat{s}_1w_1n^2_1 + \hat{s}_2w_2n^2_2 + \hat{s}_3w_3n^2_3}{w_1n^2_1 + w_2n^2_2 + w_3n^2_3}~,
\end{equation}
and
\begin{equation}
\label{eq:mshgrouportho}
\frac{1}{v^2_{group}} \approx E(N_1,N_2,N_3)(1-\hat{S}) + \hat{S}\sqrt{E^2(N_1,N_2,N_3) + \frac{2\sum\limits_{j=1}^3(\hat{Q}_{j}-1)W_iW_kN^2_{i}N^2_{k}}{\hat{S}}}~,\\
\end{equation}
where
\begin{equation}
\hat{S} = \frac{\hat{S}_1W_1N^2_1 + \hat{S}_2W_2N^2_2 + \hat{S}_3W_3N^2_3}{W_1N^2_1 + W_2N^2_2 + W_3N^2_3}~,
\end{equation}
\begin{eqnarray}
\label{eq:qi}
\hat{q}_j &= ~~\frac{q_{ij} w_in^2_i + q_{kj} w_kn^2_k}{w_in^2_i + w_kn^2_k},~\hat{Q}_j &=~~ \frac{Q_{ij} W_iN^2_i + Q_{kj} W_kN^2_k}{W_iN^2_i + W_kN^2_k}~, \\
\label{eq:Si}
\hat{s}_j &= ~~\frac{s_{jk} w_in^2_i + s_{ji} w_kn^2_k}{w_in^2_i + w_kn^2_k},~\hat{S}_j &=~~ \frac{S_{jk} W_iN^2_i + S_{ji} W_kN^2_k}{W_iN^2_i + W_kN^2_k}~.
\end{eqnarray}

Figure~\ref{fig:fittingnotation} shows the locations of the fitting indices in each symmetry plane. Note that the subscript rule explained earlier still applies, and the relationship $Q_{ij}=1/q_{ij}$ still holds. \new{In our notation,} $\hat{q}$ and $\hat{Q}$ represent weighted averages of anelliptic parameters ($q$ and $Q$) in a plane whereas $\hat{s}$ and $\hat{S}$ represent weighted averages at an axis.

\multiplot{2}{coordinate,fittingnotation}{width=0.45\textwidth}{a) Parametization rule for working coordinates. b) Locations of fitting indices $q_{ij}$ in each symmetry plane.}

Similar to the derivation in the TI case, $q_{ij}$ and $Q_{ij}$ can be found by fitting the curvatures to the exact velocities along \old{ $n_l$}\new{orthogonal directions} in each of the three planes. Likewise, $s_{ij}$ and $S_{ij}$ can be found by fitting the fourth-order derivative ($d^4v_{phase}/d\theta^4$ and $d^4v_{group}/d\Theta^4$) at the same positions. Note that the expressions of $q_{i2},Q_{i2},s_{i2}$, and $S_{i2}$ are similar to those in the TI case for \old{$q_l,Q_l,s_l$, and $S_l$ for $l=1,3$}\new{$q_i,Q_i,s_i$, and $S_i$ for $i=1,3$}. As a result, we need to specify parameter expressions only for the other two planes. \new{Fitting both phase- and group-velocity expressions along both axes in each  symmetry plane leads to six different expressions for each of the fitting parameters. This is different for the case of extended Muir-Dellinger approximations (equations~\ref{eq:emdphase} and~\ref{eq:emdgroup}) where only three expressions are allowed.}

Since the Christoffel equation in the three symmetry planes is similar to that in TI media, the functional forms of every parameter expression remain the same. Therefore, we can compute parameters needed by equations~\ref{eq:mshphaseortho} and \ref{eq:mshgrouportho} using the formulas derived in [$n_1$,$n_3$] plane for the TI case. Thus, we obtain:
\begin{eqnarray}
\label{eq:qij}
 q_{ij} & = & \frac{(c_{ik}+c_{pp})^2+c_{pp}(c_{ii}-c_{pp})}{c_{kk}(c_{ii}-c_{pp})}~, \\ \nonumber \\
  \label{eq:sij}
 s_{ij} & = & a_{ij}/b_{ij}~, \\
 \nonumber
 a_{ij} & = & (w_k-w_i)(q_{ij}-1)^2(q_{kj}-1)~, \\
 \nonumber
 b_{ij}  & = & 2[w_k(q_{ij}(q_{ij}(q_{ij}-2)+3)-2q_{ij}q_{kj}+q_{kj}^2-1) - w_i(q_{kj}(q_{ij}(q_{ij}-4)+q_{kj}+1) +2q_{ij} -1 )]~,\\ \nonumber \\
   \label{eq:Sij}
 S_{ij} & = & A_{ij}/B_{ij}~, \\
  \nonumber
 A_{ij} & = & (W_i-W_k)(Q_{ij}-1)^2(Q_{kj}-1)~, \\
 \nonumber
 B_{ij}  & = & 2[W_i(Q_{kj}^2 +2Q_{ij} +Q_{ij}Q_{kj}(Q_{ij}(Q_{ij}-2)-1)-1)  \\
 \nonumber
 	  &   & - W_k(Q_{kj}^2-2Q_{ij}Q_{kj}+Q_{ij}(Q_{ij}(Q_{ij}-1)^2+2)-1)]~.
\end{eqnarray}
where \old{$p=4,5,6$ when $j=1,2,3$ respectively} \new{$p=j+3$}. \new{Recall that the indices $ij \ne ji$ and therefore, there are six expressions corresponding to each formula from equations~\ref{eq:qij}-\ref{eq:Sij}.} Equations~\ref{eq:mshphaseortho} and \ref{eq:mshgrouportho} amount to the nine-parameter approximations for both velocities where the required parameters include $w_1$, $w_2$, $w_3$, $q_{21}$, $q_{31}$, $q_{12}$, $q_{32}$, $q_{13}$, and $q_{23}$ for equation~\ref{eq:mshphaseortho} or their reciprocals, $W_1$, $W_2$, $W_3$, $Q_{21}$, $Q_{31}$, $Q_{12}$, $Q_{32}$, $Q_{13}$, and $Q_{23}$ for equation~\ref{eq:mshgrouportho}. However, these nine parameters can be \new{easily} reduced to six using the linear relationships between $q_{ij}$ and $q_{kj}$ for different lithologies from the \old{previous section for}\new{previous discussion on} TI media (Figure~\ref{fig:qshalelegnew,qsandlegnew,qcarbonateleg}).
\old{Note that our anelliptic parameters $q_{ij}$ relate to parameters proposed by Tsvankin (1997, 2012) and Alkhalifah (2003) by
where $j=1$ and $2$ for planes [$n_2$,$n_3$] and [$n_1$,$n_3$] respectively.}\new{Similar conversion rules apply for anisotropic parameters in each symmetry plane and are summarized in Table~\ref{tbl:ticonversion}. Note that the required independent parameters for the group-velocity approximations derived based on Muir-Dellinger approach can be found in a one-to-one relationship simply from the reciprocals of the required parameters for phase-velocity approximations as mentioned before. Therefore, phase- and group-velocity approximations derived on the basis of this approach require exactly the same number of parameters.}

%The summary table for required parameters in various qP-wave phase-velocity approximations in orthorhombic media, including both the ones mentioned in this paper and some notable ones proposed elsewhere, can be found in Table~\ref{tbl:orthoparameter}. 

%\tabl{orthoparameter}{Comparison of exact parameters, and required independent parameters and assumptions used for various qP-wave phase-velocity approximations.}
%{
%\centering
%     	     \begin{tabular}{c c c c}
%     	     \hline Schemes & $\#$ of Parameters & Parameters & Assumption \\ 
%     	     \hline \hline 
%             \\ \multirow{3}*{Exact (Equation~\ref{eq:orthoexactphase})} & \multirow{3}*{9} & $c_{11}$, $c_{22}$, $c_{33}$, & \multirow{3}*{-}\\
%                       & & $c_{12}$, $c_{23}$, $c_{13}$, & \\
%                       & & $c_{44}$, $c_{55}$, $c_{66}$ & \\ \\
%             \\ \multirow{2}*{\cite{tsvankinortho}} & \multirow{2}*{6} & $V_{P0}$, $\epsilon^(1)$, $\epsilon^{(2)}$, & \multirow{2}*{Weak anisotropy}\\
%                       & & $\delta^{(1)}$, $\delta^{(2)}$, $\delta^{(3)}$ & \\ \\
%             \\ \multirow{2}*{\cite{alkaortho}} & \multirow{2}*{6} & $V_v$, $V_{nmo_1}$, $V_{nmo_2}$, & \multirow{2}*{Acoustic approximation}\\
%                       & & $\eta_1$, $\eta_2$, $\delta$ & \\ \\
%     	     \multirow{3}*{Proposed} & \multirow{3}*{6} & $w_1$, $w_2$, &  $q_{21} = a q_{31} + b$ (Figure~\ref{fig:qshalelegnew,qsandlegnew,qcarbonateleg})\\
%                        & & $w_3$, $q_{31}$, & $q_{12} = a q_{32} + b$ (Figure~\ref{fig:qshalelegnew,qsandlegnew,qcarbonateleg})\\
%                        & & $q_{13}$, $q_{32}$ & $q_{23} = q_{13}$\\ \\
%      \hline
%    \end{tabular}
%}

\new{\subsubsection{Moveout approximation}}

To convert the proposed group-velocity approximation (equation~\ref{eq:mshgrouportho}) to the corresponding moveout approximation, we \old{again use}\new{apply again} the general expression given in equation~\ref{eq:moveout}. Adopting the same notation rules, \old{we have} \new{the moveout approximation takes the form}:
\begin{eqnarray}
\label{eq:orthomoveout}
t^2 & = & H_{ortho}(x,y)(1-\hat{S}) + \hat{S}\sqrt{F}~,\\
\nonumber
F & = & H^2_{ortho}(x,y)+ \frac{2}{\hat{S}}\bigg(\frac{(\hat{Q}_1-1)t^2_0y^2}{Q_{31}V^2_{nmo_1}} + \frac{(\hat{Q}_2-1)t^2_0x^2}{Q_{32}V^2_{nmo_2}}+ \frac{(\hat{Q}_3-1)x^2y^2}{(Q_{31}V^2_{nmo_1})(Q_{32}V^2_{nmo_2})}\bigg)~,
\end{eqnarray}
where
\begin{equation}
\hat{S} = \frac{\frac{\hat{S}_1}{Q_{32}V^2_{nmo_2}}x^2 + \frac{\hat{S}_2}{Q_{31}V^2_{nmo_1}}y^2 + \hat{S}_3t^2_0}{\frac{1}{Q_{32}V^2_{nmo_2}}x^2 + \frac{1}{Q_{31}V^2_{nmo_1}}y^2 + t^2_0}~,\\
\end{equation}
\begin{eqnarray}
\hat{Q}_1 =\frac{\frac{Q_{21}}{Q_{31}V^2_{nmo_1}}y^2 + Q_{31}t^2_0}{\frac{1}{Q_{31}V^2_{nmo_1}}y^2 + t^2_0},~~
\hat{Q}_2 = \frac{\frac{Q_{12}}{Q_{32}V^2_{nmo_2}}x^2 + Q_{32}t^2_0}{\frac{1}{Q_{32}V^2_{nmo_2}}x^2 + t^2_0},~~
\hat{Q}_3 = \frac{\frac{Q_{13}}{Q_{32}V^2_{nmo_2}}x^2 + \frac{Q_{23}}{Q_{31}V^2_{nmo_1}}y^2}{\frac{1}{Q_{32}V^2_{nmo_2}}x^2 + \frac{1}{Q_{31}V^2_{nmo_1}}y^2}~,\\
\hat{S}_1 = \frac{\frac{S_{13}}{Q_{31}V^2_{nmo_1}}y^2 + S_{12}t^2_0}{\frac{1}{Q_{31}V^2_{nmo_1}}y^2 + t^2_0},~~
\hat{S}_2 = \frac{\frac{S_{23}}{Q_{32}V^2_{nmo_2}}x^2 + S_{21}t^2_0}{\frac{1}{Q_{32}V^2_{nmo_2}}x^2 + t^2_0},~~
\hat{S}_3 = \frac{\frac{S_{32}}{Q_{32}V^2_{nmo_2}}x^2 + \frac{S_{31}}{Q_{31}V^2_{nmo_1}}y^2}{\frac{1}{Q_{32}V^2_{nmo_2}}x^2 + \frac{1}{Q_{31}V^2_{nmo_1}}y^2}~,
\end{eqnarray}
$x$ denotes the offset in \old{$n_1$}\new{$N_1$} direction, $y$ denotes the offset in \old{$n_2$}\new{$N_2$} direction, $V_{nmo_2} = \sqrt{1/W_1Q_{32}}$ denotes the NMO-velocity in \old{$n_1$}\new{$N_1$} direction, $V_{nmo_1} = \sqrt{1/W_2Q_{31}}$ denotes the NMO-velocity in \old{$n_2$}\new{$N_2$} direction, 
and $H_{ortho}(x,y)$ denotes the hyperboloidal part of reflection traveltime \new{squared} given below,
\begin{equation}
H_{ortho}(x,y) = t^2_0 + \frac{x^2}{Q_{32}V^2_{nmo_2}} + \frac{y^2}{Q_{31}V^2_{nmo_1}}~.
\end{equation}
We apply the same strategy \old{as previously mentioned (Figure~\ref{fig:qshalelegnew,qsandlegnew,qcarbonateleg}) to reduce the number of parameters from nine to six}\new{to reduce the number of parameters with an approximation on $Q_{ij}$ for $S_{ij}$ as in equation~\ref{eq:strick}}.

\new{For small offset, the Taylor expansion of equation~\ref{eq:orthomoveout} is
\begin{eqnarray}
t^2(x) & \approx &t^2_0 +\frac{x^2}{V^2_{nmo_2}} + \frac{y^2}{V^2_{nmo_1}} - \\
\nonumber
    &&\frac{1-2S_{32}(Q_{12}+1)+Q_{32}(4S_{32}+Q_{32}-2)}{2S_{32}Q_{32}^2t_0^2V^4_{nmo_2}} x^4 -\\
\nonumber
    &&\frac{1-2S_{31}(Q_{21}+1)+Q_{31}(4S_{31}+Q_{31}-2)}{2S_{31}Q_{31}^2t_0^2V^4_{nmo_1}} y^4 +\\
\nonumber
    &&\frac{S_{31}(Q_{32}-1)^2-S_{32}(Q_{32}-1)(Q_{32}+2Q_{31}-3)-2S^2_{32}(Q_{32}+Q_{31}-Q_{13}-1)}{2S^2_{32}Q_{31}Q_{32}t_0^2V^2_{nmo_1}V^2_{nmo_2}} x^2y^2 + ...
\end{eqnarray}
The asymptote of this expression for unbounded offsets $x$ and $y$ is given by
\begin{equation}
\frac{1}{Q_{32} V^2_{nmo_2}} = \frac{1}{w_1}~~~\mbox{and}~~~\frac{1}{Q_{31} V^2_{nmo_1}} = \frac{1}{w_2}~,
\end{equation}
which denote the horizontal velocities squared along $N_1$ and $N_2$ directions respectively.}

\subsection{Examples}

To evaluate the accuracy of the proposed approximations, we produce relative error plots and tables, using several sets of normalized stiffness tensor measurements summarized in Table~\ref{tbl:sample}\new{, which can be converted to any parameterization scheme (Table~\ref{tbl:orthoscheme})}. The error plots in Figures~\ref{fig:phaseweak90leglow,phaseweakleglow}-\ref{fig:groupmshtrueq90leglow,groupmshtrueqleglow} are generated using the standard model \cite[]{helbig} and are presented as both 3D surfaces and sterographic projections with $\theta$ (or $\Theta$) changing radially and $\phi$ (or $\Phi$) changing azimuthally. The standard model assumes a shale background with \old{two sets}\new{a set} of parallel vertical cracks; therefore, we use the following relationship between anelliptic parameters in shales: $q_1 = 0.83734 q_3 +0.1581$\old{from the previous section} to reduce the number of parameters in the vertical [$n_1$,$n_3$] and [$n_2$,$n_3$] planes. \old{Because the anisotropy in the horizontal plane [$n_1$,$n_2$] corresponds to fractures,}\new{The anisotropy in the horizontal plane [$n_1$,$n_2$], on the other hand, corresponds to a different cause, which in this case is assumed to be vertical fractures. Because we \old{do not have}\new{do not know} a \new{proper} relationship between anelliptic parameters for such feature,} we resort to the previously used assumption of $q_{13}=q_{23}$. Tables~\ref{tbl:rephase} and \ref{tbl:regroup} show RMS relative error results of our approximations in comparison with results from some of the previously suggested approximations \new{, which are computed based on
\begin{equation}  
\mbox{RMS error} = \sqrt{\sum_{\psi_1=0}^{90} \sum_{\psi_2=0}^{90} (v_{exact}(\psi_1,\psi_2)-v_{approx}(\psi_1,\psi_2))^2}~,
\end{equation}
where $\psi_1$ and $\psi_2$ denote the zenith and azimuthal phase or group angles as appropriate.} The best-performing approximation is denoted in \new{red and} bold. In all examples, the proposed approximations appear to be significantly more accurate than \new{the} other known approximations.

\tabl{sample}{Normalized stiffness tensor coefficients (in $km^2$/$s^2$) from different orthorhombic samples: 1 is from \cite{helbig}, 2 and 3 are from \cite{tsvankinortho}, and 4 and 5 are from \cite{alkaortho}.}
{
\centering
     	     \begin{tabular}{|l|c|c|c|c|c|c|c|c|c|}
     	     \hline Sample & $ c_{11}$ & $ c_{22}$ & $ c_{33}$ & $ c_{44} $ & $ c_{55}$ & $ c_{66}$ & $ c_{12}$ & $ c_{23}$ & $ c_{13}$\\ 
     	     \hline 1. Standard model & 9 & 9.84 & 5.938 & 2 & 1.6 & 2.182 & 3.6 & 2.4 & 2.25 \\
     	     \hline 2. Tsvankin 1 & 11.7 & 13.5 & 9 & 1.728 & 1.44 & 2.246 & 8.824 & 5.981 & 5.159 \\
     	     \hline 3. Tsvankin 2 & 17.1 & 13.5 & 9 & 1.728 & 1.44 & 2.246 & 9.772 & 4.580 & 7.745 \\
     	     \hline 4. Alkhalifah 1 & 1.452 & 2.016 & 1 & 0.25 & 0.25 & 0.25 & 1.089 & 0.695 & 0.599 \\
     	     \hline 5. Alkhalifah 2 & 1.452 & 2.016 & 1 & 0.49 & 0.36 & 0.49 & 0.608 & 0.206 & 0.378 \\
      \hline
    \end{tabular}
}

%\tabl{sampletsvankin}{Tsvankin parameters for samples in Table~\ref{tbl:sample}. Velocities are given in $km$/$s$.}
%{
%\centering
%     	     \begin{tabular}{|l|c|c|c|c|c|c|c|c|c|}
%     	     \hline Sample & $ V_{P0}$ & $ V_{S0}$ & $ \epsilon_2$ & $ \delta_2$ & $ \gamma_2$ & $ \epsilon_1 $ & $ \delta_1$ & $\gamma_1 $ & $\delta_3$\\ 
%     	     \hline 1. Standard model & 2.437 & 1.265 & 0.2579 & -0.0775 & 0.0455 & 0.3286 & 0.0825 & 0.1819 & -0.1064 \\
%     	     \hline 2. Tsvankin 1 & 3.0 & 1.2 & 0.15 & -0.1 & 0.15 & 0.25 & 0.05 & 0.28 & 0.15 \\
%     	     \hline 3. Tsvankin 2 & 3.0 & 1.2 & 0.45 & 0.2 & 0.15 & 0.25 & -0.1 & 0.28 & -0.15 \\
%     	     \hline 4. Alkhalifah 1 & 1.0 & 0.5 & 0.226 & 0.1055 & 0.0 & 0.508 & 0.2204 & 0.0 & 0.1 \\
%     	     \hline 5. Alkhalifah 2 & 1.0 & 0.6 & 0.226 & 0.1055 & 0.0 & 0.508 & 0.22 & 0.1806 & 0.1 \\
%      \hline
%    \end{tabular}
%}

\tabl{rephase}{RMS relative error (\%) from $0$-$90\,^{\circ}$ (both $\theta$ and $\phi$) of orthorhombic phase-velocity approximations by \cite{tsvankinortho}, \cite{alkaortho}, and of proposed six-parameter approximation. Bold red highlight indicates the best-performing approximation. In all cases, the proposed approximation appears to be the most accurate.}
{
\centering
     	     \begin{tabular}{|c|c|c|c|c|}
     	     \hline Sample & \cite{tsvankinortho} & \cite{alkaortho} & Proposed\\ 
     	     \hline 1 & 0.5787 & 0.1742 & \textcolor{red}{\textbf{0.1029}}\\
     	     \hline 2 & 0.5918 & 0.0645 & \textcolor{red}{\textbf{0.0275}}\\
     	     \hline 3 & 0 .7104 & 0.0952 & \textcolor{red}{\textbf{0.0637}}\\
     	     \hline 4 & 0.8960 & 0.1382 & \textcolor{red}{\textbf{0.0293}}\\
     	     \hline 5 & 1.0736 & 0.3274 & \textcolor{red}{\textbf{0.2084}}\\
      \hline
    \end{tabular}
}

\tabl{regroup}{RMS relative error (\%) from $0$-$90\,^{\circ}$ (both $\Theta$ and $\Phi$) of orthorhombic group-velocity approximations by \cite{xu} and \cite{vascon}, and of proposed six-parameter approximation. Bold red highlight indicates the best-performing approximation. In all cases, the proposed approximation appears to be more accurate.}
{
\centering
     	     \begin{tabular}{|c|c|c|c|c|}
     	     \hline Sample & Xu-Vasconcelos & Proposed\\ 
     	     \hline 1 & 0.8985 & \textcolor{red}{\textbf{0.1446}}\\
     	     \hline 2 & 0.6066 & \textcolor{red}{\textbf{0.1354}}\\
     	     \hline 3 & 0.7966 & \textcolor{red}{\textbf{0.0311}}\\
     	     \hline 4 & 0.4907 & \textcolor{red}{\textbf{0.0387}}\\
     	     \hline 5 & 0.4588 & \textcolor{red}{\textbf{0.1729}}\\
      \hline
    \end{tabular}
}


\multiplot{2}{phaseweak90leglow,phaseweakleglow}{width= 0.45\textwidth}{Relative error of phase-velocity approximation by \cite{tsvankinortho}.
a) from azimuth 0 to $90\,^{\circ}$.
b) from azimuth 0 to $360\,^{\circ}$.}
\multiplot{2}{phaseacoustic90leglow,phaseacousticleglow}{width= 0.45\textwidth}{Relative error of phase-velocity approximation by \cite{alkaortho}.
a) from azimuth 0 to $90\,^{\circ}$.
b) from azimuth 0 to $360\,^{\circ}$.}
\multiplot{2}{phasemshappq90leglownew,phasemshappqleglownew}{width= 0.45\textwidth}{Relative error of proposed six-parameter phase-velocity approximation.
a) from azimuth 0 to $90\,^{\circ}$.
b) from azimuth 0 to $360\,^{\circ}$.}
\multiplot{2}{phasemshtrueq90leglow,phasemshtrueqleglow}{width= 0.45\textwidth}{Relative error of proposed nine-parameter phase-velocity approximation.
a) from azimuth 0 to $90\,^{\circ}$.
b) from azimuth 0 to $360\,^{\circ}$.}
\multiplot{2}{groupxu90leglow,groupxuleglow}{width= 0.45\textwidth}{Relative error of group-velocity approximation by \cite{xu} and \cite{vascon}.
a) from azimuth 0 to $90\,^{\circ}$.
b) from azimuth 0 to $360\,^{\circ}$.}
\multiplot{2}{groupmshappq90leglownew,groupmshappqleglownew}{width= 0.45\textwidth}{Relative error of the proposed six-parameter group-velocity approximation.
a) from azimuth 0 to $90\,^{\circ}$.
b) from azimuth 0 to $360\,^{\circ}$.}
\multiplot{2}{groupmshtrueq90leglow,groupmshtrueqleglow}{width= 0.45\textwidth}{Relative error of the proposed nine-parameter group-velocity approximation.
a) from azimuth 0 to $90\,^{\circ}$.
b) from azimuth 0 to $360\,^{\circ}$.}


\section{Application to wave extrapolation}

One possible \old{way to test the accuracy}\new{application} of the proposed phase-velocity approximations (equations~\ref{eq:mshphase} and~\ref{eq:mshphaseortho}) is\old{through} seismic wave extrapolation based on the anisotropic wave equation. 
\cite{fomellr} and \cite{junzhe} presented a lowrank approximation method to accomplish \old{such}\new{this} task. 
The proposed phase-velocity approximations (both in TI and orthorhombic media) are converted to their corresponding dispersion relations \old{relating} \new{involving} frequency\old{($\omega$)} and wavenumber\old{($k$)} and incorporated into the wave extrapolator formulated in the Fourier domain.  An example of wave extrapolation in the complex BP 2007 TTI model is shown in Figure~\ref{fig:snaptssum-0,snaptssum-1,errortssum-2,errortssum-1}.
The same portion of the model was investigated by \cite{fomellr} and \cite{junzhe}.
For simplicity, we take the shear-wave velocity ($V_{S0}=\sqrt{c_{55}}$) to be $V_{P0}/2$.
\old{This test shows}\new{The results are shown in Figure~\ref{fig:snaptssum-0,snaptssum-1,errortssum-2,errortssum-1} and demonstrate noticeably} smaller phase errors obtained from the proposed approximation\old{(Figure~\ref{fig:errortssum-1})}.

\inputdir{bptti}
%\multiplot{4}{vpend2,thetaend2,q1,q3}{width=0.45\textwidth}{Portion of BP-2007 anisotropic benchmark model.
%a) Velocity along the axis of symmetry.
%b) Tilt of the symmetry axis.
%c) Anellipticity parameter along the axis perpendicular to the axis of symmetry ($q_1$).
%d) Anellipticity parameter along the axis of symmetry ($q_3$).
%}

\multiplot{4}{snaptssum-0,snaptssum-1,errortssum-2,errortssum-1}{width=0.40\textwidth}{Multiple snapshots and errors of the wavefield extrapolation results for the BP TTI 2007 model.
a) Wavefield extrapolation using exact phase velocity.
b) Wavefield extrapolation using proposed phase-velocity approximation~(\ref{eq:mshphase}).
c) Absolute error in $n_1$-$n_3$ plane of the acoustic approximation.
d) Absolute error in $n_1$-$n_3$ plane of the proposed approximation \new{(six-parameter)}.
}

An example in a heterogeneous tilted orthorhombic medium is shown in Figures~\ref{fig:wave0,wave1} and~\ref{fig:error3,error2,error1,error4} 
using the parameters of the tilted orthorhombic model from \cite{song}. In this model, the anisotropic parameters in the notation of \cite{alkaortho} are specified in the range, $v_1 = 1500:3088$ m/s, $v_2 = 1500:3686$ m/s, $v_v = 1500:3474$ m/s, $\eta_1=0.3$, $\eta_2=0.1$, and $\gamma=1.03$. \new{The exact formulas for $v_1$, $v_2$, and $v_v$ are
\begin{eqnarray}
v_1 &=& 1500 + 40x^2+ 30(y-1.5)^2+30(z-1)^2~,\\
v_2 &=& 1500 + 60x^2+ 40(y-1.5)^2+40(z-1)^2~,\\
v_v &=& 1500 + 50x^2+ 35(y-1.5)^2+40(z-1)^2~,
\end{eqnarray}
where $x$, $y$, and $z$ are components in the model.}
\old{,which corresponds to}\new{These values of parameters correspond} to $q_{21}=0.857:0.879$, $q_{31}=0.833$, $q_{12}=0.670:0.727$, $q_{32}=0.625$, $q_{13}=0.993:1.414$, and $q_{23}=0.993:1.442$. The model, according to the right-hand rule is rotated $45\,^{\circ}$ counterclockwise around the $n_3$ axis and subsequently $45\,^{\circ}$ counterclockwise around the $n_1$ axis. We perform wave extrapolation using the exact dispersion relation and compare it with the results from the proposed approximation (equation~\ref{eq:mshphaseortho})\new{, as well as the weak-anisotropy approximation \cite[]{tsvankinortho}}, and the acoustic approximation \cite[]{alkaortho,song}. \old{This test shows} \new{The error plots shown in Figure~\ref{fig:error3,error2,error1,error4} demonstrate} noticeably smaller phase errors from the proposed approximation.

\inputdir{tiltorthocompare}
\multiplot{2}{wave0,wave1}{width=0.40\textwidth}{Wavefield extrapolation results in an example of the tilted orthorhombic model.
a) Wavefield extrapolation using the exact phase velocity.
b) Wavefield extrapolation using the proposed phase-velocity approximation (equation~\ref{eq:mshphaseortho}). The wavefields are virtually identical.
}
\multiplot{4}{error3,error2,error1,error4}{width=0.40\textwidth}{Errors in wavefield extrapolation results in an example of the tilted orthorhombic.
a) Absolute error of the weak-anisotropy phase-velocity approximation.
b) Absolute error of the acoustic phase-velocity approximation.
c) Absolute error of the proposed phase-velocity approximation \new{(six-parameter)}.
d) Absolute error of the proposed phase-velocity approximation (nine-parameter).
}

\new{\section{Discussion}}

\new{Our choice of Muir-Dellinger parametrization leads naturally to a four-parameter velocity approximation in TI media and a nine-parameter approximation in orthorhombic media. The approximations are improved by shifted-hyperboloid functional form. Although highly accurate, these approximations require the same number of parameters as the exact expressions. The benefits of their introduction may not be apparent in the case of phase velocity but are apparent in the consideration of group-velocity approximations because the exact expressions for group velocity in both types of media can be prohibitively complex and cannot be expressed easily in terms of group angle. As observed by previous researchers, the sufficient number of parameters to describe qP wave propagation in TI and orthorhombic media is smaller: three and six respectively. Therefore, we apply the novel relationships between anisotropic parameters summarized in Figure~\ref{fig:qshalelegnew,qsandlegnew,qcarbonateleg} to effectively reduce the number of parameters from the proposed four- and nine-parameter approximations to three- and six-parameter approximations respectively.}

\new{Apart from the functional form proposed in this paper, many other forms of phase-velocity and group-velocity approximations, especially in TI media, have been extensively investigated in the past \cite[e.g.][]{alkatsvankin,tsvankinvti,menschraso,psencikgajewski98,alkavti,alkavti2,alkavti3,farrahighorder,stopin,zhang,farrapsencikhighorder,daley,ursin,fomelstovas,stovas2010,aleixo,farra,hao}. While some of them are based on physical assumptions, others are derived purely from mathematical arguments. 
%The former, although provides a justifiable link to the actual physical phenomena obeserved, often leads to lower accurate results. On the other hand, the latter often leads to higher accurate results but lacks the robust physical relation and/or the compactness of expressions. 
Our proposed approximation is an alternative, which provides both accuracy and connection with the physical wave phenomena. They are based on the original Muir-Dellinger approximatons \cite[]{md,mdk}, which were derived on the basis of perturbation from ellipitical group-velocity surfaces.\old{This claim is further supported by the discovery of the strongly linear relationships between the proposed anelliptic parameters in various lithologies (Figure~\ref{fig:qshalelegnew,qsandlegnew,qcarbonateleg}).} The primary advantage of the Muir-Dellinger parameterization is the ease of conversion between the phase- and group-velocity approximations (e.g. equations~\ref{eq:mshphase} and~\ref{eq:mshgroup}), which provides practical convenience. Alternative highly accurate form for phase- and group-velocity approximations is the generalized moveout approximation \cite[]{fomelstovas}, which was recently applied to anisotropic velocity approximations by \cite{hao} and \cite{zonegma}.}

%In addition, the set of anisotropic parameters used in this study, referred to as Muir-Dellinger parameters in the text, represent a fundamentally different way of combining the elastic moduli in comparison to Thomsen parameters \cite[]{thomsen} and their extension to orthorhombic media \cite[]{tsvankinortho}. Even though it remains to be investigated, the fact that our parameterization leads to interestingly strong linear relationships between anisotropic parameters along different axessuggest that perhaps, they are more desirable for use in anisotropic signatures analysis.

\new{Our approximations are readily applicable to approximate phase and group velocities in the case of transversely isotropic and orthorhombic media whose symmetry axis is aligned with the coordinate axis, e.g., VTI, HTI, and VOR. In the case of TTI (tilted tranversly isotropic) and TOR (tilted orthorhombic), the coordinates simply need to be rotated via Bond transformation before applying the proposed approximations.} 

\section{Conclusions}

We have introduced \old{two modifications to the previously developed}\new{novel forms of} anelliptic approximations for qP velocities in TI \new{and orthorhombic} media. \new{The first modification is an empirical connection between $q_1$ and $q_3$ parameters, which depends on the dominant lithology.} The \old{first}\new{second} modification \old{leads to}\new{is} a new functional form of the phase- and group-velocity approximations, allowing up to fourth-order fitting along both symmetry and non-symmetry axes. \old{The second modification incorporates an empirical connection between $q_1$ and $q_3$ parameters, which depends on the dominant lithology.} As a \old{consequence}\new{result of these modifications}, we arrive at \old{extremely}\new{highly} accurate four-parameter approximations and new three-parameter approximations \new{for TI media} with \old{higher}\new{better} accuracy than previously suggested three-parameter approximations for both phase and group velocites.  
On the basis of the modified anelliptic approximations in TI media, we also propose anelliptic approximations for qP velocities in orthorhombic media, which can be implemented using either six or nine parameters. 
The proposed orthorhombic phase-velocity approximation \new{maintains the algebraic symmetry and} appears to be a more accurate alternative to previously proposed approximations. The group-velocity approximation has an analogous functional form and is \old{highly} \new{also very} accurate. The superior accuracy of the proposed phase-velocity approximations \old{in TI and orthorhombic media} is confirmed additionally using wave extrapolation experiments.

\section{Acknowledgments}
\new{We would like to thank Mirko van der Baan, Igor Ravve, Alexey Stovas, and an anonymous reviewer for constructive comments and helpful suggestions.} We also thank Joe Dellinger, Paul Fowler, \new{Qi Hao}, and Junzhe Sun for useful discussions.
We thank the sponsors of the Texas Consortium for Computational Seismology (TCCS) for financial support.


\appendix
\section{Appendix A: Uncertainty Analysis}
\inputdir{.}
\new{In this appendix, we study the variation of the phase-velocity expression in TI media with respect to different choices of parameters, particularly Thomsen's parameters and the proposed Muir-Dellinger parameters. To study resolution, we use the general formula,
\begin{equation}
\label{eq:sens}
R_{ij} = \int_0^{\pi/2} \frac{\partial V^2}{\partial m_i} \frac{\partial V^2}{\partial m_j} d\theta~,
\end{equation}
where $V$ is the exact phase-velocity expression (equation~\ref{eq:exactphase}), $m_i$ and $m_j$ are two of the four parameters present in the expression, and $\theta$ is the phase angle measured from vertical. The matrix $R_{ij}$, in both cases, is computed based on the stiffness tensor of Greenhorn shales given in Table~\ref{tbl:tisample}. The results are shown in Tables~\ref{tbl:thomsensens} and~\ref{tbl:qsens}. Note that the matrix is symmetric, so the values are shown only on one side of the diagonal.}

\tabl{thomsensens}{$R_{ij}$ of the exact phase-velocity expression with the two considered Thomsen parameters are denoted in each row and column.}
{
\centering
     	     \begin{tabular}{|c|c|c|c|c|}
     	     \hline Parameters & $ V_{P0}$ & $ V_{S0}$ & $ \epsilon$ & $ \delta$\\ 
     	     \hline $V_{P0}$   &  87.11  &  0.467 & 105.39 & 25.19 \\ 
     	     \hline $V_{S0}$   &  & 0.005 & 0.649 & 0.20 \\ 
     	     \hline $\epsilon$ &  &  & 181.74 & 22.54 \\ 
     	     \hline $\delta$   &  &  & & 13.16 \\ 
      \hline
    \end{tabular}
}

\tabl{qsens}{$R_{ij}$ of the exact phase-velocity expression with the two considered anelliptic parameters are denoted in each row and column.}
{
\centering
     	     \begin{tabular}{|c|c|c|c|c|}
     	     \hline Parameters & $ w_1$ & $ w_3$ & $ q_1$ & $q_3$\\ 
     	     \hline $w_1$   & 0.576 & 0.143 & 0.651 & 0.599 \\ 
     	     \hline $w_3$   &  & 0.538 & 0.257 &1.061 \\ 
     	     \hline $q_1$   &  &  & 1.325 & 1.279 \\ 
     	     \hline $q_3$   &  &  &  & 4.124 \\ 
      \hline
    \end{tabular}
}

%It is crucial to emphasize that the numbers shown in Tables~\ref{tbl:thomsensens} and~\ref{tbl:qsens} must be considered only in a relative sense and separately for each case. This means that, in a given case, the larger value corresponds to higher correlation and vice versa. On the basis of unit (considering $V_{P0}$ with $V_{S0}$ and $\epsilon$ with $\delta$),
\new{Table~\ref{tbl:thomsensens} shows a significantly larger correlation between the change in phase velocity with $V_{P0}$ in comparison with that of $V_{S0}$, which agrees with the general assumption of the independency of $V_{S0}$ in qP velocities approximations. Likewise, the effect from $\epsilon$ has a higher correlation with the change of phase velocity than $\delta$ because the exact qP phase-velocity formula (equation~\ref{eq:exactphase}) can be expressed in terms of Thomsen parameters with $\epsilon$ corresponding to the lower order of $\sin \theta$ than $\delta$. Moreover, $\epsilon$ and $\delta$ also have high correlation with $V_{P0}$, which is apparent from their definitions.}

\new{Table~\ref{tbl:qsens} shows relatively similar correlations from $w_1$ and $w_3$ to the change in exact phase velocity suggesting a more symmetric contribution from both parameters. The dimensionless anelliptic parameters $q_1$ and $q_3$ exhibit a strong correlation, which  is consistent with the relationships shown in Figure~\ref{fig:qshalelegnew,qsandlegnew,qcarbonateleg}.}

\new{By ignoring the effect of $V_{S0}$ in the case of Thomsen parameters or using the relationship between $q_1$ and $q_3$ (Figure~\ref{fig:qshalelegnew,qsandlegnew,qcarbonateleg}) to reduce the number of parameters to three, we can transform the matrix $R_{ij}$ from $4\times4$ to $3\times3$ ($\tilde{R}_{ij}$). Note that the matrix for Thomsen parameters is similar to Table~\ref{tbl:thomsensens} with the omittance of the row and column associated with $V_{S0}$. Table~\ref{tbl:qsens3} shows the three-parameter matrix for anelliptic parameters with similar behavior of relatively equal correlations from $w_1$ and $w_3$ as before. }

\tabl{qsens3}{$\tilde{R}_{ij}$ of the exact phase-velocity expression with the two considered anelliptic parameters are denoted in each row and column.}
{
\centering
     	     \begin{tabular}{|c|c|c|c|c|}
     	     \hline Parameters & $ w_1$ & $ w_3$  & $q_3$\\ 
     	     \hline $w_1$   & 0.578 & 0.144 & 1.166 \\ 
     	     \hline $w_3$   &  & 0.534 & 1.286 \\ 
     	     \hline $q_1$   &  &  & 7.411  \\ 
      \hline
    \end{tabular}
}

\new{To better visualize the variational effect from the change in the three parameters in both cases, we follow the approach of \cite{osypov}, compute the quadratic form of $\tilde{R}_{ij}$ and plot its contour at a given amount of change in the exact phase velocity expression,
\begin{equation}
\Delta V^2= \mathbf{x}^T \tilde{R}_{ij} \mathbf{x}~,
\end{equation}
where $\mathbf{x}$ denotes the vector of parameter variations: [$\Delta V_{P0}$, $ \Delta \epsilon$, $\Delta \delta$]$^T$ or [$\Delta w_1$, $\Delta w_3$, $\Delta q_3$]$^T$ and $\tilde{R}_{ij}$ is computed at the known values of the anisotropic parameters of the model (Greenhorn shales). The resultant plots are shown in Figure~\ref{fig:thomsenmatrix,zonematrix}. For Thomsen's parameters, Figure~\ref{fig:thomsenmatrix} shows a strongly oblate ellipsoid with high degree of deviation (stretch) from a sphere for all three parameters. On the contrary, Figure~\ref{fig:zonematrix} shows oblate ellipsoid with smaller deviation suggesting that the Muir-Dellinger parameters may represent a more orthogonal parameterization scheme than Thomsen's parameters. This observation is important for the problem of estimating anisotropic parameters, which goes beyond the scope of this paper.}

%The dependencies between different parameters can be illustrated through the magnitude of deviation from a sphere the ellipsoid is. The larger the deviation, the higher the dependency. Note that the size of the ellipsoids is of no importance because one can always resize the ellipsoid by varying $\Delta V^2$ as desired.

\multiplot{2}{thomsenmatrix,zonematrix}{width=0.4\textwidth}{Ellipsoids obtained from the quadratic form of $\tilde{R}_{ij}$ in the case of a) Thomsen parameters  b) anelliptic parameters.}


%\newpage
\bibliographystyle{seg}
\bibliography{seg2014}




