\published{Geophysics, 76, T123-T129, (2011)}
\title{Fourier finite-difference wave propagation}

\lefthead{Song \& Fomel}
\righthead{FFD wave propagation}

\author{Xiaolei Song and Sergey Fomel}



\address{Bureau of Economic Geology \\
John A. and Katherine G. Jackson School of Geosciences \\
The University of Texas at Austin \\
University Station, Box X \\
Austin, TX 78713-8924}

%\ms{GEO-2010}

\maketitle

\begin{abstract}
  We introduce a novel technique for seismic wave extrapolation in
  time.  The technique involves cascading a Fourier Transform operator
  and a finite difference operator to form a chain operator: 
  Fourier Finite Differences (FFD).  We derive the FFD operator from a
  pseudo-analytical solution of the acoustic wave equation.  2-D
  synthetic examples demonstrate that the FFD operator can have high
  accuracy and stability in complex velocity media. 
  Applying the FFD method to the anisotropic case overcomes some
  disadvantages of other methods, such as the coupling of qP-waves and
  qSV-waves.  The FFD method can be applied to enhance accuracy and
  stability of seismic imaging by reverse-time migration.
\end{abstract}

\section{Introduction}

The wavefield extrapolation problem refers to advancement of a
wavefield through space or time.  Both extrapolation in depth and
extrapolation in time can be used in seismic modeling and seismic
migration.  Reverse time migration, or RTM
\cite[]{baysal,McMechan,whitmore,levin}, involves wave extrapolation
forward and backward in time. RTM is useful for accurate imaging in
complex areas and is drawing more and more attention as the most
powerful depth-imaging method \cite[]{yoon,sym,fletcher,fowler}.\\

Reverse-time migration can correctly handle complex velocity models
without dip limitations on the image.  However, it has large memory
requirements and needs a significant amount of computation.  The most
popular and straightforward way to implement reverse-time migration is
the method of explicit finite differences, which is only conditionally
stable because of the limit on time-step size.  Finite-difference
methods also suffer from numerical dispersion problems, which can be
overcome either by decreasing the time step or by high-order schemes
\cite[]{Wu,liu}.  Several alternative algorithms have been developed
for seismic wave extrapolation in variable velocity media.
\cite{yu} introduced an algorithm based on a high-order differential operator, 
which allows a large extrapolation time step by solving a coefficient optimization problem.
\cite{zhang} proposed one-step extrapolation method by introducing a square-root operator. This method can formulate the two-way wave equation as a first-order partial differential equation in time similar to the one-way wave equation.
\cite{etgen} modified the Fourier Transform of the Laplacian operator to compensate exactly for the error
resulting from the second-order time marching scheme used in conventional pseudo spectral methods \cite[]{pseudo}.
\cite{fowler} provided an accurate VTI P-wave modeling method with coupled second-order pseudo-acoustic equations.
\cite{pestana} presented an application of Rapid Expansion Method (REM) \cite[]{tal} for forward modeling with one-step time evolution algorithm and RTM with recursive time stepping algorithm.\\

In this paper, we present a new wave extrapolator derived from the pseudo-analytical approach of \cite{etgen}.
Our method combines FFT and finite differences. 
We call it the Fourier Finite Difference method 
because it is analogous to the concept introduced previously for one-way wave extrapolation by \cite{ffd}.\\ 

As a chain operator of Fast Fourier Transform and Finite Difference
operators, the proposed extrapolator can be as accurate as the
parameter interpolation approach employed by \cite{etgen} but at a
cost of only one Fast Fourier Transform (FFT) and inverse Fast Fourier
Transform (IFFT) operation.  The advantages of the FFD operator are
even more apparent in the anisotropic case: no need for several
interpolations for different parameters with the corresponding
computational burden of several FFTs and IFFTs.  In addition, the
operator can overcome the coupling of qP-waves and qSV-waves
\cite[]{zhang2}.  We demonstrate the method on synthetic examples and
propose to incorporate FFD into reverse-time migration in order to
enhance migration accuracy and stability.

\section{Theory}

The acoustic wave equation is widely used in 
forward seismic modeling and reverse-time migration \cite[]{bednar,etgen1}:
\begin{equation}
\label{eq:acoustic} 
\frac{\partial^2p}{\partial t^2} = v(\mathbf{x})^2\,\nabla^2p\;,
\end{equation}
where $p(\mathbf{x},t)$ is the seismic pressure wavefield, 
and $v(\mathbf{x})$ is the propagation velocity.
Assuming a constant velocity $v$, after Fourier transform in space,
 we can get the following explicit expression:
\begin{equation}
\label{eq:ode} 
\frac{d^2\hat{p}}{dt^2} = - v^2|\mathbf{k}|^2\hat{p}\;,
\end{equation}
where
\begin{equation}
\label{eq:p} 
\hat{p}(\mathbf{k},t)=\int^{+\infty}_{-\infty}{p(\mathbf{x},t)e^{i\mathbf{k}\cdot\mathbf{x}}d\mathbf{x}}\;.
\end{equation}
\\
Equation~\ref{eq:ode} has the following solution:
\begin{equation}
\label{eq:fourier} 
\hat{p}(\mathbf{k},t+\Delta t) = e^{\pm i|\mathbf{k}|v\Delta t}\hat{p}(\mathbf{k},t)\;.
\end{equation}
A second-order time-marching scheme and the inverse Fourier transform lead to 
the well-known expression \cite[]{Etgen.sep.60.131,yu}\hyphenation{Sou-ba-ras}:
\begin{eqnarray}
\label{eq:exact} 
\lefteqn{p(\mathbf{x},t+\Delta t)+p(\mathbf{x},t-\Delta t)-2p(\mathbf{x},t)  = }\nonumber \\
& & 2\int^{+\infty}_{-\infty}{\hat{p}(\mathbf{k},t)(\cos(|\mathbf{k}|v\Delta t)-1)e^{-i\mathbf{k}\cdot\mathbf{x}}d\mathbf{k}}\;.
\end{eqnarray} 


Equation~\ref{eq:exact} provides an elegant and efficient solution 
in the case of a constant-velocity medium with the aid of FFT. 
In the case of a variable-velocity medium, 
equation~\ref{eq:exact} can provide an approximation by replacing $v$ with $v(\mathbf{x})$. 
However, FFT can no longer be applied directly for 
the inverse Fourier transform from the wavenumber domain back to the space domain.
To overcome this problem, \cite{etgen} propose a velocity interpolation method. 
They present an implementation for isotropic, VTI (vertical transversely isotropic) and TTI (tilted transversely isotropic) media.  
In the isotropic case, two FFTs can be sufficient. 
For anisotropic media, more than one velocity parameter must be used. 
Therefore, it is necessary to perform velocity interpolation 
by combining different parameters and 
computing the corresponding forward and inverse FFTs 
for each of the velocity parameters, thus increasing the computational burden.
Other FFT-based solutions include the optimized separable approximation or OSA \cite[]{song,morton,zhang,du} 
and the lowrank approximation \cite[]{ying}.

We propose an alternative approach.
First, we adopt the following form of the right-hand side of equation~\ref{eq:exact} in the variable velocity case: 
\begin{eqnarray}
\label{eq:numerator}
\lefteqn{2\left[\cos(v(\mathbf{x})|\mathbf{k}|\Delta t)-1\right] = }\nonumber \\
& & 2\left[\cos(v_0|\mathbf{k}|\Delta t)-1\right]\left[\frac{\cos(v(\mathbf{x})|\mathbf{k}|\Delta t)-1}{\cos(v_0|\mathbf{k}|\Delta t)-1}\right]\;,
\end{eqnarray} 
where $v_0$ is the reference velocity, such as the RMS (root-mean-square) velocity of the medium.
After that, we apply the following approximation:
\begin{equation}
\label{eq:approximate}
\frac{\cos(v(\mathbf{x})|\mathbf{k}|\Delta t)-1}{\cos(v_0|\mathbf{k}|\Delta t)-1} \approx a + 2\sum^3_{n=1}{b_n\cos(k_n\Delta x_n)}\;,
\end{equation} 
where coefficients $a$ and $b_n$ are defined using the Taylor expansion around $k=0$,
\begin{equation}
\label{eq:coefficient}
 {\begin{array}{*{20}c}
\displaystyle a=\frac{v^2(\mathbf{x})}{v_0^2}\left[1+\frac{(\Delta t)^2(v_0^2-v^2(\mathbf{x}))(\Delta x_1^2\Delta x_2^2+\Delta x_2^2\Delta x_3^2+\Delta x_3^2\Delta x_1^2)}{6\Delta x_1^2\Delta x_2^2\Delta x_3^2}\right] \,\\ 
\displaystyle b_n=\frac{(\Delta t)^2v^2(\mathbf{x})(v^2(\mathbf{x})-v_0^2)}{12(\Delta x_n^2)v_0^2} \, \\ 
 \end{array} }  \;, 
\end{equation}
and $\Delta x_n$ is the sampling in the n-th direction. 
We only need to calculate these coefficients once. After completing the calculation, they can be used at each time step during the wave extrapolation process.


Equation~\ref{eq:numerator} consists of two terms: the first term is independent of $\mathbf{x}$ and  only depends on $\mathbf{k}$. 
For this part, we use inverse FFT to return to the space domain from the wavenumber domain.
For the remaining part, however, we can avoid phase shift in the wavenumber domain by implementing space shifts 
through finite differences (approximation~\ref{eq:approximate}) with coefficients provided by equation~\ref{eq:coefficient}.
This approach is analogous to the FFD method proposed by \cite{ffd} for one-way extrapolation in depth.

\inputdir{cos}
Figure~\ref{fig:cos-side1} (a) shows approximations for $[\cos(v(\mathbf{x})|\mathbf{k}|\Delta t)-1]$  
by the 4th-order FD method (dash line) and pseudo-spectral method (dotted line). Figure~\ref{fig:cos-side1} (b) shows approximations by the FFD method (2nd-order: dash line, 4th-order: dotted line). 
The solid lines stand for the exact values for function $[\cos(v(\mathbf{x})|\mathbf{k}|\Delta t)-1]$ with true velocity: $v=4.0 km/s$ (bottom solid line) and $v_0=2.0 km/s$ (top solid line), which indicates a significant velocity contrast ($100\%$ difference).
In this situation, all the approximations deviate from the exact solution as the wavenumber $|k|$ becomes large.
However, the 4th-order FFD method approximates the exact solution with the most accuracy, as shown in the error plot (Figure~\ref{fig:diff1}).
In order to enhance the stability, one can suppress the wavefield at high wavenumbers for both pseudo-spectral and the FFD method. 

\plot{cos-side1}{width=0.9\textwidth}{Different approximations for $\cos(v(\mathbf{x})|\mathbf{k}|\Delta t)-1$. 
Solid lines: exact solution ($\cos(v(\mathbf{x})|\mathbf{k}|\Delta t)-1$) for $v=4.0$ km/s and $v_0=2.0$ km/s.
(a) Dash line: the 4th-order FD. Dotted line: pseudo-spectral method. (b) Dash line: the 2nd-order FFD method. Dotted line: the 4th-order FFD with $v_0$ as reference velocity.  $\Delta t=0.001$ s. 
$\Delta x=0.005\,km$.}

\plot{diff1}{width=0.9\textwidth}{Errors for different approximations for $\cos(v(\mathbf{x})|\mathbf{k}|\Delta t)-1$. 
Solid lines: exact solution ($\cos(v(\mathbf{x})|\mathbf{k}|\Delta t)-1$) for $v=4.0$ km/s and $v_0=2.0$ km/s.
(a) Dash line: the 4th-order FD. Dotted line: pseudo-spectral method. (b) Dash line: the 2nd-order FFD method. Dotted line: the 4th-order FFD with $v_0$ as reference velocity.  $\Delta t=0.001$ s. 
$\Delta x=0.005\,km$.}

Assuming that $p(\mathbf{x},t-\Delta t)$ and $p(\mathbf{x},t)$ are already known, the time-marching algorithm can be specified as follows:

\begin{enumerate}
\item Transform $p(\mathbf{x},t)$ to $\hat{p}(\mathbf{k},t)$ by 3-D FFT;
\item Multiply $\hat{p}(\mathbf{k},t)$ by $2\left[\cos(v_0|\mathbf{k}|\Delta t)-1\right]$ to get $\hat{q}(\mathbf{k},t)$;
\item Transform $\hat{q}(\mathbf{k},t)$ to $q(\mathbf{x},t)$ by inverse FFT;
\item Apply finite differences to $q(\mathbf{x},t)$ with coefficients in equation~\ref{eq:coefficient} to get $q(\mathbf{x},t+\Delta t)$. 
Namely,
\begin{eqnarray}
\label{eq:fd}
\lefteqn{q^{i,j,k}(t+\Delta t) = a^{i,j,k}q^{i,j,k}(t)}\nonumber \\
& & +b_1^{i,j,k}(q^{i-1,j,k}(t)+q^{i+1,j,k}(t)) \nonumber \\
& & +b_2^{i,j,k}(q^{i,j-1,k}(t)+q^{i,j+1,k}(t))\nonumber \\
& & +b_3^{i,j,k}(q^{i,j,k-1}(t)+q^{i,j,k+1}(t))\;,
\end{eqnarray} 
where $i$ is the grid index of $x_i$ direction;
\item $p(\mathbf{x},t+\Delta t) \leftarrow q(\mathbf{x},t+\Delta t) + 2p(\mathbf{x},t) - p(\mathbf{x}, t-\Delta t)$.
\end{enumerate}
Here $q(\mathbf{x},t)$ and $\hat{q}(\mathbf{k},t)$ are temporary functions.\\

The FFD approach is not limited to the isotropic case. 
In the case of transversally isotropic (TTI) media, the term $v(\mathbf{x})\,|\mathbf{k}|$ on the left-hand side of equation~\ref{eq:approximate},
can be replaced with the acoustic approximation \cite[]{alkhalifah1,alkhalifah2,anelliptic},

%\begin{eqnarray}
%\label{eq:ttiexact} 
%\lefteqn{f(\mathbf{v},\mathbf{\hat{k}},\eta)  = } \\
%&\sqrt{\frac{1}{2}(v_x^2\,\hat{k}_x^2+v_z^2\,\hat{k}_z^2)+\frac{1}{2}\sqrt{(v_x^2\,\hat{k}_x^2+v_z^2\,\hat{k}_z^2)^2-\frac{8\eta}{1+2\eta}v_x^2v_z^2\,\hat{k}_x^2\,\hat{k_z^2}}}\;,\nonumber 
%\end{eqnarray} 

\begin{equation}
\label{eq:ttiexact} 
f(\mathbf{v},\mathbf{\hat{k}},\eta)=\sqrt{\frac{1}{2}(v_1^2\,\hat{k}_1^2+v_2^2\,\hat{k}_2^2)+\frac{1}{2}\sqrt{(v_1^2\,\hat{k}_1^2+v_2^2\,\hat{k}_2^2)^2-\frac{8\eta}{1+2\eta}v_1^2v_2^2\,\hat{k}_1^2\,\hat{k_2^2}}}\;, 
\end{equation} 

where $v_1$ is the P-wave phase velocity in the symmetry plane, 
$v_2$ is the P-wave phase velocity in the direction normal to the symmetry plane, 
$\eta$ is the anisotropic elastic parameter \cite[]{alkhalifah} related to Thomsen's elastic parameters $\epsilon$ and $\delta$ \cite[]{thomsen} by
$$\frac{1+2\delta}{1+2\epsilon}=\frac{1}{1+2\eta}\,;$$
and $\hat{k}_1$ and $\hat{k}_2$ stand for the wavenumbers evaluated in a rotated coordinate system aligned with the symmetry axis:

\begin{equation}
\label{eq:wavenumber}
\begin{array}{*{20}c}
\hat{k}_1=k_1\cos{\phi}+k_2\sin{\phi}\;\\ 
\hat{k}_2=-k_1\sin{\phi}\cos{\theta}+k_2\cos{\phi}\cos{\theta}+k_3\sin{\theta}\;\\ 
\hat{k}_3=k_1\sin{\phi}\sin{\theta}-k_2\cos{\phi}\sin{\theta}+k_3\cos{\theta}\;,\\ 
 \end{array}
\end{equation}

where $\theta$ is the tilt angle measured with respect to vertical and
$\phi$ is the angle between the projection of the symmetry axis in the
horizontal plane and the original X-coordinate. The symmetry axis has
the direction of
$\left\{-\sin\theta\sin\phi,-\sin\theta\cos\phi,\cos\theta\right\}$.

Using these definitions, 
we develop a finite-difference approximation analogous to equation~\ref{eq:approximate} for FFD in TTI media. The details of the derivation are given in the appendix.
For the 2D TTI case, the corresponding FFD algorithm is as following:
\begin{enumerate}
\item Transform $p(\mathbf{x},t)$ to $\hat{p}(\mathbf{k},t)$ by FFT;
\item Multiply $\hat{p}(\mathbf{k},t)$ by $\displaystyle\frac{2[\cos(f(\mathbf{v_0},\mathbf{\hat{k}},\eta_0)\Delta t)-1]}{|\mathbf{\hat{k}}|^2}$ to get $\hat{q}(\mathbf{k},t)$;
\item Transform $\hat{q}(\mathbf{k},t)$ to $q(\mathbf{x},t)$ by inverse FFT;
\item Apply finite differences to $q(\mathbf{x},t)$ with coefficients in Table~\ref{tbl:equations} to get $q(\mathbf{x},t+\Delta t)$. Namely,
\begin{eqnarray}
\label{eq:fd}
\lefteqn{q^{i,j,t+\Delta t} = a^{i,j}q^{i,j,t}} \\
& & +b_1^{i,j}(q^{i-1,j,t}+q^{i+1,j,t}) \nonumber \\
& & +b_2^{i,j}(q^{i,j-1,t}+q^{i,j+1,t})\nonumber \\
& & +d_1^{i,j}(q^{i-2,j,t}+q^{i+2,j,t}) \nonumber \\
& & +d_2^{i,j}(q^{i,j-2,t}+q^{i,j+2,t})\nonumber \\
& & +c^{i,j}(q^{i-1,j-1,t}+q^{i-1,j+1,t}+q^{i+1,j-1,t}+q^{i+1,j+1,t})\nonumber\;.
\end{eqnarray} 
where $i$ is the grid index of $x_i$ direction;
\item $p(\mathbf{x},t+\Delta t) \leftarrow q(\mathbf{x},t+\Delta t) + 2p(\mathbf{x},t) - p(\mathbf{x}, t-\Delta t)$.
\end{enumerate}
Here, $q(\mathbf{x},t)$ and $\hat{q}(\mathbf{k},t)$ are temporary functions.



\section{Numerical Examples}

%\subsection{Isotropic Cases}

\inputdir{ffd}
Our first example is a comparison of four methods: the 4th-order
finite differences, pseudo-spectral method, velocity interpolation,
and the FFD method in a velocity model with smooth variation,
formulated as
$$v(x,z)\,=\,550+1.5\times10^{-4}(x-800)^2+10^{-4}(z-500)^2;$$
$$0\le x\le2560,\;0\le z\le2560.$$
The velocity is between 550 m/s and 1439 m/s.
A Ricker-wavelet source with maximum frequency 70 Hz is located at the center of the model. For all the numerical simulations based on this model, we use the same grid size: $\Delta x\,=\,5$ m and $\Delta t\,=\,2$ ms.

Figure~\ref{fig:wavfd} shows an obvious numerical dispersion from the
snapshot of the acoustic wavefield computed by the 4th-order finite
difference method.  Figure~\ref{fig:wavsp} shows a slight dispersion
from the snapshot computed by pseudo-spectral method \cite[]{reshef}.
Figure~\ref{fig:wavpspi} shows the corresponding snapshot of the
velocity-interpolation method \cite[]{etgen,crawley}, calculated using
two reference velocities.  It is practically free of dispersion thanks
to spectral compensation.  Figure~\ref{fig:wavffd} shows a snapshot of
the proposed FFD method.  It is almost exactly the same as
Figure~\ref{fig:wavpspi}; however, only one reference velocity is used
instead of two.  As comparison between Figure~\ref{fig:wavffd} and
Figure~\ref{fig:wavpspi} implies, the FFD method has practically the
same accuracy as the velocity interpolation method while having only
one reference velocity and therefore replacing the cost of one
additional FFT with the cost of a low-order finite-difference
operator.

\inputdir{ffd}

\multiplot{4}{wavfd,wavsp,wavpspi,wavffd}{width=0.4\textwidth}{Acoustic wavefield snapshot by: (a) 4th-order Finite Difference method; (b) pseudo-spectral method; (c) velocity interpolation method with 2 reference velocities; (d) FFD method with RMS velocity.} 

\inputdir{bpmodel}

Our next example is a snapshot of the acoustic wavefield calculated by FFD in the BP model \cite[]{bp}. We use a Ricker-wavelet at a point source. The maximum frequency is 50 Hz. The horizontal grid size $\Delta x$ is 37.5 m, the vertical grid size $\Delta z$ is 12.5 m and the time step is 1 ms. 
Figure~\ref{fig:vel} shows a part of the model with a salt body.
Figure~\ref{fig:wavsnap} shows a wavefield snapshot confirming that the FFD method can work in a complex-velocity medium as well.\\

\inputdir{bpmodel}

\plot{vel}{width=0.9\textwidth}{Portion of BP 2004 synthetic velocity model.}
\plot{wavsnap}{width=0.9\textwidth}{Wavefield snapshot in the BP Model shown in Figure~\ref{fig:vel}.}

Next we apply our FFD algorithm to RTM with a simple exponential decaying boundary condition \cite[]{cerjan}. The dominant frequency is 27 Hz. The space grid is 25 m and the time step is 1.5 ms. Figure~\ref{fig:img} shows the output image. The inner and outer flanks of the salt body are clearly imaged.\\ 

\inputdir{.}
\plot{img}{width=0.9\textwidth}{RTM image of BP salt model.}

\inputdir{anisotropic}
The cost advantage of FFD is even more appealing in anisotropic (TTI) media, 
which require multiple velocity parameters and increase the cost of velocity interpolation.
Figure~\ref{fig:TTI-snapshot} shows the impulse response of a 4th-order FFD operator 
in a TTI model with the tilt of $45\,^{\circ}$ and a smooth velocity variation ($v_x$: 800-1225.41 m/s, $v_z$: 700-883.6 m/s). The space grid size is 5 m and the time step is 1 ms. 
Note no coupling of qP-waves and qSV-waves \cite[]{zhang2} in the figure, 
thanks to the Fourier construction of the operator.\\

For this paper, we implemented a 2nd-order 5-point FD scheme for 2D isotropic case. 
One can observe little dispersion in isotropic examples. For TTI media, we use a 13-point scheme which minimizes the error in the symmetry plane and in the direction normal to the symmetry plane. One can still see some dispersion in the corresponding snapshot; which indicates that a higher order FD scheme might be required to further supress the dispersion.\\ 

\inputdir{anisotropic}
\plot{TTI-snapshot}{width=0.9\textwidth}{Wavefield snapshot in a TTI medium with tilt of 45 degrees. 
$v_x(x,z)=800+10^{-4}(x-1000)^2+10^{-4}(z-1200)^2$; $v_z(x,z)=700+10^{-4}(z-1200)^2$; $\eta=0.3$; $\theta=45\,^{\circ}$.}

\inputdir{bptti}

Our last example is qP-wave simulation in the BP 2D TTI model. 
Figure~\ref{fig:vp2}-\ref{fig:theta2} shows parameters for part of the model.
The maximum frequency is 50 Hz. The space grid size is 12.5 m and the time step is 1 ms.
The snapshot of the acoustic wavefield in Figure~\ref{fig:ttisnapshot} demonstrates the stability of our approach in a complicated anisotropic model.
Some small dispersion is present in the TTI examples, pointing to a possible need to extend the FD part of the FFD scheme from second order to higher orders.

\inputdir{bptti}
%\multiplot{4}{vp2,epsilon2,delta2,theta2}{width=0.2\textwidth}{Partial region of the 2D BP TTI model. a: $v_z$. b: $\epsilon$. c: $\delta$. d:$\theta$.}
\multiplot{4}{vp2,vx2,yita2,theta2}{width=0.45\textwidth}{Partial region of the 2D BP TTI model. a: $v_z$. b: $v_x$. c: $\eta$. d:$\theta$.}
\plot{ttisnapshot}{width=0.9\textwidth}{Scalar wavefield snapshot in the 2D BP TTI model, shown in Figure~\ref{fig:vp2,vx2,yita2,theta2}.}


%\plot{ttisnapshot}{width=0.45\textwidth}{width=\columnwidth}{Scalar wavefield snapshot in the 2D BP TTI model, shown in Figure~\ref{fig:vp2,vx2,yita2,theta2}.}
\section{Conclusions}
Accurate and efficient numerical wave propagation in variable velocity media is crucial for seismic modeling and seismic migration,
particularly for reverse-time migration.
The FFD technique proposed in this paper promises higher accuracy than that of the conventional, explicit finite-difference method, 
at the cost of only one forward and inverse Fast Fourier Transform.
Results in synthetic isotropic and anisotropic models illustrate FFD's stability in complicated velocity models. 
The method can be used in reverse-time migration to enhance its accuracy and stability.

\section{Acknowledgement}

We thank BP for releasing benchmark synthetic models, Bjorn Enquist,
Paul Fowler and Lexing Ying for useful discussions, and John Etgen,
Stig Hestholm, Erik Saenger and two anonymous reviewers for helpful
reviews.

This publication is authorized by the Director, Bureau of Economic
Geology, The University of Texas at Austin.


\append{FFD for TTI media}
To develop a 25-point finite-difference scheme analogous to equation~\ref{eq:approximate} for FFD in 3D TTI media, we first apply the following approximation:
\begin{eqnarray}
\label{eq:ttiapprox} 
\lefteqn{\frac{\cos(f(\mathbf{v},\mathbf{\hat{k}},\eta)\Delta t)-1}{\cos(f(\mathbf{v_0},\mathbf{\hat{k_0}},\eta_0)\Delta t)-1}|\mathbf{\hat{k}}|^2 \approx  } \\
\displaystyle & a + 2 \sum^{3}_{n=1}{(b_n\cos(k_n\Delta x_n)+d_n\cos(2k_n\Delta x_n))} \nonumber \\
\displaystyle & + 2 \sum^{3}_{n=1}{c_n [\cos(k_i\Delta x_i+k_j\Delta x_j) + \cos(k_i\Delta x_i-k_j\Delta x_j)]} \nonumber \;,\nonumber 
\end{eqnarray} 
where $i,j=1,2,3;\,i\neq j;\,i,j\neq n$.

In approximation~\ref{eq:ttiapprox},
$f(\mathbf{v},\mathbf{\hat{k}},\eta)$ is a function as in
expression~\ref{eq:ttiexact} and $a$, $b_n$, $c_n$ and $d_n$ are
coefficients determined from the Taylor expansion around $k=0$.

Notice that we multiply the left-hand side with
$|\mathbf{\hat{k}}|^2$, so one needs to multiply
$\hat{p}(\mathbf{\hat{k},t})$ with
$\displaystyle\frac{2[\cos(f(\mathbf{v_0},\mathbf{k},\eta_0)\Delta
t)-1]}{|\mathbf{\hat{k}}|^2}$.

The coefficients for Equation~\ref{eq:ttiapprox} are derived in
Table~\ref{tbl:equations} and Table~\ref{tbl:wh}.  $w_{n0}$, $h_{n0}$,
$p_{n0}$ and $q_{n0}$ have similar expressions as above in
Table~\ref{tbl:wh} with $\mathbf{v}$, $\eta$ and $\theta$ substited by
the corresponding reference values: RMS velocity $\mathbf{v}_0$,
average anisotropic parameter $\eta_0$ and average tilt angles
$\theta_0$ and $\phi_0$.
\tabl{equations}{Coefficients for equation~\ref{eq:ttiapprox}.}
{
    \begin{center}
     \begin{tabular}{|c|c|}
      \hline $a$ &
     $\displaystyle a=-2b_1-2b_2-2b_3-4c_1-4c_2-4c_3-2d_1-2d_2-2d_3$ \\ 
      \hline $b$ &
      $\displaystyle b_1=-2c_2-2c_3-4d_1-\frac{w_1+h_1}{\Delta x_1^2(w_{10}+h_{10})}$\\ 
      \hline $c$ &
      $\displaystyle b_2=-2c_1-2c_3-4d_2-\frac{w_2+h_2}{\Delta x_2^2(w_{20}+h_{20})}$\\ 
      \hline $d$ &
      $\displaystyle b_3=-2c_1-2c_2-4d_3-\frac{w_3+h_3}{\Delta x_3^2(w_{30}+h_{30})}$\\ 
      \hline $e$ &
      $\displaystyle d_1=\frac{(w_1+h_1)(2x_1^2+\Delta t^2(w_{10}+h_{10}-w_1-h_1))}{24\Delta x_1^4(w_{10}+h_{10})}$\\
      \hline $f$ &
      $\displaystyle d_2=\frac{(w_2+h_2)(2x_2^2+\Delta t^2(w_{20}+h_{20}-w_2-h_2))}{24\Delta x_2^4(w_{20}+h_{20})}$\\
      \hline $g$ &
      $\displaystyle d_3=\frac{(w_3+h_3)(2x_3^2+\Delta t^2(w_{30}+h_{30}-w_3-h_3))}{24\Delta x_3^4(w_{30}+h_{30})}$\\
      \hline $h$ &
 $\begin{array}{c}  \displaystyle c_1=\frac{1}{12\Delta x_2^2\Delta x_3^2}\left[\frac{\Delta x_2^2(w_2+h_2)}{w_{20}+h_{20}}+\frac{\Delta x_3^2(w_3+h_3)}{w_{30}+h_{30}}\right]-d_2\frac{\Delta x_2^2}{\Delta x_3^2}-d_3\frac{\Delta x_3^2}{\Delta x_2^2} \\  
\displaystyle\; +\frac{\Delta t^2(p_1+q_1)(p_10+q_10-p1-q1)}{12\Delta x_2^2\Delta x_3^2(p_10+q_10)}\end{array}$ \\
      \hline $i$ &
 $\begin{array}{c}  \displaystyle c_2=\frac{1}{12\Delta x_1^2\Delta x_3^2}\left[\frac{\Delta x_1^2(w_1+h_1)}{w_{10}+h_{10}}+\frac{\Delta x_3^2(w_3+h_3)}{w_{30}+h_{30}}\right]-d_1\frac{\Delta x_1^2}{\Delta x_3^2}-d_3\frac{\Delta x_3^2}{\Delta x_1^2} \\  
\displaystyle\; +\frac{\Delta t^2(p_2+q_2)(p_20+q_20-p2-q2)}{12\Delta x_1^2\Delta x_3^2(p_20+q_20)}\end{array}$ \\
      \hline $j$ &
 $\begin{array}{c}  \displaystyle c_3=\frac{1}{12\Delta x_1^2\Delta x_2^2}\left[\frac{\Delta x_1^2(w_1+h_1)}{w_{10}+h_{10}}+\frac{\Delta x_2^2(w_2+h_2)}{w_{20}+h_{20}}\right]-d_1\frac{\Delta x_1^2}{\Delta x_2^2}-d_2\frac{\Delta x_2^2}{\Delta x_1^2} \\  
\displaystyle\; +\frac{\Delta t^2(p_3+q_3)(p_30+q_30-p3-q3)}{12\Delta x_1^2\Delta x_2^2(p_30+q_30)}\end{array}$ \\
      \hline 
     \end{tabular}   
    \end{center}   
}

\tabl{wh}{Coefficients for Table~\ref{tbl:equations}.}
{
    \begin{center}
     \begin{tabular}{|c|c|}
      \hline $a$ &
$w_1=v_1^2\cos^2\phi+\sin^2\phi(v_1^2\cos^2\theta+v_2^2\sin^2\theta)$ \\ 
      \hline $b$ &
$w_2=(v_1^2+v_2^2)\cos^2\phi\cos^2\theta+v_1^2\sin^2\theta$ \\ 
      \hline $c$ &
$w_3=v_1^2\sin^2\theta+v_2^2\cos^2\theta$ \\ 
      \hline $d$ &
$\displaystyle h_1=\sqrt{w1^2-\frac{8\eta v_1^2v_2^2\sin^2\phi(\cos^2\phi+\sin^2\phi\cos^2\theta)\sin^2\theta}{1+2\eta}}$ \\ 
      \hline $e$ &
$\displaystyle h_2=\sqrt{w2^2-\frac{8\eta v_1^2v_2^2\cos^2\phi\cos^2\theta(\cos^2\phi\cos^2\theta+\sin^2\phi)}{1+2\eta}}$ \\ 
      \hline $f$ &
$\displaystyle h_3=\sqrt{w3^2-\frac{8\eta v_1^2v_2^2\cos^2\theta\sin^2\theta}{1+2\eta}}$ \\       \hline $g$ &
$p_1=w_2+w_3+v_1^2\cos\phi\sin2\theta-2v_2^2\cos\phi\cos^2\theta$ \\ 
      \hline $h$ &
$\displaystyle q_1=\sqrt{p_1^2-\frac{32\eta v_1^2v_2^2\cos^2\theta\sin^4\frac{\phi}{2}(\cos^2\phi\cos^2\theta+\sin^2\theta+\sin^2\phi+\cos\phi\sin2\theta)}{1+2\eta}}$ \\ 
      \hline $i$ &
$p_2=w_1+w_3+(v_2^2-v_1^2)\sin\phi\sin2\theta$ \\ 
      \hline $j$ &
$\displaystyle q_2=\sqrt{p_2^2-\frac{8\eta v_1^2v_2^2(\cos\theta+\sin\phi\sin\theta)^2(\cos^2\phi+(\cos\theta\sin\phi-\sin\theta)^2)}{1+2\eta}}$ \\ 
      \hline $k$ &
$p_3=w_1+w_2+v_1^2\sin^2\theta\sin2\phi+\frac{1}{2}v_2^2\sin2\phi\sin2\theta$ \\ 
      \hline $l$ &
$\displaystyle q _3=\sqrt{p_3^2-\frac{4\eta v_1^2v_2^2\cos^2(\phi+\theta)(\sin2\phi\cos2\theta-3-\cos2\theta-\sin2\phi)}{1+2\eta}}$ \\ 
      \hline 
     \end{tabular}   
    \end{center}   
}

\bibliographystyle{seg}

\bibliography{ffd,SEP2}
