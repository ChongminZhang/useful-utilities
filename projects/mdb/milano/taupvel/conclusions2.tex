\section{Discussion}

In the conventional NMO processing, one needs to scan over a range of
possible velocities and pick the \new{appropriate} velocity trend from
semblance maxima. Therefore, the cost of velocity scanning is roughly
proportional to the number of scanned velocities $N_{V}$ times the
input data size.  Anisotropic velocity analysis is performed by
simultaneously scanning two (or more) parameters. Consequently, the
number of trial velocities/parameters squares, which increases the
computational time dramatically. In oriented processing, the effective
anisotropy parameters turn into data attributes according to equations
\ref{eqn:vNmap}--\ref{eqn:etamap}. These parameters are directly
mapped from the slope field $R$ to the correct zero-slope/offset
traveltime $\tau_0$. The cost of local slope estimation with
plane-wave destruction method is proportional to the data size times
the number of estimation iterations $N_{I}$ times the 2D filter size
$N_{F}$. Typically $N_{I}=10$ and $N_{F}=6,$ which roughly correspond
to scanning $N_{V}=60$ velocities. However, unlike semblance analysis,
this cost does not increase if we are estimating one, two or more
parameters. The cost of the semblance scan becomes even more
prohibitive when processing wide-azimuth data. The computational
advantages of our approach are \old{really} encouraging especially
\old{towards} \new{with respect to} multi-azimuth processing and
orthorhombic velocity analysis, where time processing is controlled by
at least five parameters \citep{ilyabook2006}.

Automation, in addition to speed, is \old{the other real} \new{another
clear} advantage of \new{the} slope-based processing.  Slope
estimation provided by plane-wave destruction represents an automated
approach to velocity analysis. It may require a limited
user-interaction in choosing input parameters. A user-supplied initial
guess for the slope field can \old{help} \new{accelerate} the
nonlinear optimization \old{in convergence}, thereby providing a more
reliable estimate for the slopes. The smoothness of the output slopes
is controlled by shaping regularization; the length of a 2-D
triangular smoothing filter controls smoothness along $p$ and $\tau $
direction in the \taup transformed CMP data. If the \new{input}
seismic data are not regularly and properly sampled in space, as often
happens in wide-azimuth acquisition, the \taup transform may add to
the data coherent-noise artifacts. This can affect the final result of
PWD slope estimation. Thus, if the seismic \taup data are \old{very}
noisy, increasing the length of smoothing filters can help in
achieving a more stable solution, despite some loss in resolution. In
contrast, for high SNR\ data, less smoothing yields better-resolved
slope fields.

All the equations we have developed in this paper hold for S-wave data
as long as we use two parameters S-wave phase-velocity
approximation \citep{stovas:2506}. The combination of the results from
P-wave and S-wave processing may enable a retrieval of all the elastic
parameters needed to build an initial VTI anisotropic model suitable
for depth processing.

The application of the proposed method is also limited by the
underlying assumption of vertical variation of the velocity model with
the horizontal symmetry plane. In principle, the method can handle
limited lateral variation of the velocity. \old{Thus} \new{Therefore},
it can be used for dense anisotropic moveout analysis at the early
stages of processing.

\section{Conclusions}

Local slopes of seismic events carry complete information about the
structure of the subsurface. We have developed a velocity-independent
\taup imaging approach to perform moveout correction in 2D layered
VTI\ media. We process Radon-transformed data because \taup is the
natural domain for anisotropic parameter estimation in
vertically-variable media.
%In \taup domain, moveout modeling and inversion are simpler and more
%accurate than traditional formulation based on traveltime series
%expansion or group velocity approximation.  
Effective VTI parameters turn into data attributes through the use of
slopes and are directly mappable to the zero-slope
traveltime. Interval parameters turn into data attributes as well. We
have developed the analytical theory for the slope-based Dix inversion
in \taup, as well as two alternative sets of equations that can be
regarded as an extension of Claerbout's method for
\textit{straightedge determination of interval velocity}. Both sets of
equations exploit the intrinsic layer stripping power of the \taup
domain to estimate interval parameters directly without involving
effective parameters.

The equations we have introduced to retrieve both effective and
interval parameters in VTI media require directly or indirectly an
estimation of the local data curvature.
%Unlike slopes, we don't have yet a direct method for estimating the curvature field $Q$. Practically we compute the curvature field $Q$
%by numerical differentiation of the slope estimates $R$. This
%procedure can lead to a noisy estimate of curvature field that can affect
%the NMO correction and parameters estimation, especially if we are
%processing real data set. 
On the other hand, Fowler's equations do not require an explicit use
of the curvature. Therefore, we propose bypassing the curvature
estimation by exploiting a curvature-independent estimation of the
zero-slope time $\tau_{0}$ field that, together with the slopes,
provides the input to Fowler's method. The zero-slope time can be
found efficiently by employing the predictive painting algorithm.  A
reference trace at the zero-slope time $\tau _{0}$ is spread along the
local data slope to predict the $\tau_0$ field along reflection curves
in the $\tau$-$p$ CMP gather. This estimation appears robust and
efficient enough to enable automated, slope-based, dense estimation of
interval parameters.

\begin{comment}
This processing can be straightforwardly extended to pure S-waves data
using the dispersion relations introduced by \cite{stovas:2506}.
Moreover, we believe that our \taup slope-based processing can be
applied to 3D data as an alternative to \cite{burnett:WB129} and
\cite{wang:WB117} approaches.  We have demonstrated the practical
applicability of our method for interval VTI velocity estimations, on
both synthetic and field-data examples.
\end{comment}


