\author{Joseph Fourier}
%%%%%%%%%%%%%%%%%%%%%%%
\title{Homework 3}

\begin{abstract}
  This homework has four parts, one theoretical and three computational. 
  \begin{enumerate}
  \item Theoretical questions related to Fourier transform and forward interpolation.
  \item Data compression using 2-D Fourier transform.
  \item Data interpolation after coordinate transformation.
  \item Analyzing your own data.
  \end{enumerate}
\end{abstract}

\section{Prerequisites}

Completing the computational part of this homework assignment requires
\begin{itemize}
\item \texttt{Madagascar} software environment available from \\
\url{http://www.ahay.org/}
\item \LaTeX\ environment with \texttt{SEG}\TeX\ available from \\ 
\url{http://www.ahay.org/wiki/SEGTeX}
\end{itemize}
To do the assignment on your personal computer, you need to install
the required environments. Please ask for help if you don't know where
to start.

The homework code is available from the \texttt{Madagascar} repository
by running
\begin{verbatim}
svn co http://svn.code.sf.net/p/rsf/code/trunk/book/geo391/hw3
\end{verbatim}

\section{Theoretical part}

You can either write your answers to theoretical questions on paper or
edit them in the file \texttt{hw3/paper.tex}. Please show all the
mathematical derivations that you perform.

\begin{enumerate}

\item Show that, using the helix transform and imposing helical boundary conditions, it is possible to compute a 2-D digital Fourier transform using 1-D FFT program. Assuming that the input data is of size $N \times N$, would this approach have any computational advantages?

\item The Taylor series expansion of the inverse sine function around zero is
\begin{equation}
  \label{eq:arcsin}
  \arcsin{x} = x + \frac{1}{2}\,\frac{x^3}{3} + 
  \frac{1 \cdot 3}{2 \cdot 4}\,\frac{x^5}{5} + 
  \frac{1 \cdot 3 \cdot 5}{2 \cdot 4 \cdot 6}\,\frac{x^7}{7} + 
  \cdots
\end{equation}
\begin{enumerate}
\item Show how one can use expansion~(\ref{eq:arcsin}) to design a
  digital filter that approximates the derivative
  operator. 

  \textbf{Hint:} Use the identity $1/Z-Z = 2\,i\,\sin(\omega\,\Delta t)$.
\item In particular, find a seven-point derivative filter of the form
\begin{equation}
  \label{eq:d6}
  D(Z) = d_{-3}/Z^{3} + d_{-2}/Z^{2} + d_{-1}/Z + d_0 + 
  d_1\,Z + d_2\,Z^2 + d_3\,Z^3\;.
\end{equation}
\end{enumerate}

\item The parabolic B-spline $\beta_2(x)$ is a function defined as
 \begin{equation}
   \label{eq:b3} 
   \beta_2(x) = \int\limits_{-\infty}^{\infty} \beta_1(t)\,\beta_0(x-t)\,d t\;,
\end{equation}
where
\begin{equation}
   \label{eq:b1}
   \beta_0(x) = \left\{\begin{array}{lcl} 1 & \quad\mbox{for}\quad & |x| \le 1/2 \\
       0 &\quad \mbox{for}\quad& |x| > 1/2\end{array}\right.
 \end{equation}
and
\begin{equation}
  \label{eq:b2} 
   \beta_1(x) = \int\limits_{-\infty}^{\infty} \beta_0(t)\,\beta_0(x-t)\,d t\;
   = \left\{\begin{array}{lcl} 1-|x| &\quad \mbox{for}\quad& |x| \le 1 \\
       0 & \quad \mbox{for}\quad&  |x| > 1\end{array}\right.
\end{equation}

\begin{enumerate}
\item Find an explicit expression for $\beta_2(x)$.
\item Show that decomposing a continuous data function $d(x)$ into the convolution basis 
  with parabolic B-spines
  \begin{equation}
    \label{eq:basis} 
    d(x) = \sum\limits_k c_k\,\beta_2(x-k)
  \end{equation}
  leads to an interpolation filter of the form
  \begin{equation}
    \label{eq:bz}
    Z^{\sigma} \approx B_2(Z) = \frac{a_0(\sigma)\,Z^{-1} + a_1(\sigma) + a_2(\sigma)\,Z}{b_0\,Z^{-1} + b_1 + b_2\,Z}\;.
  \end{equation}
  Define $a_0(\sigma)$, $a_1(\sigma)$, $a_2(\sigma)$, $b_0$, $b_1$, and $b_2$.
\end{enumerate}
\end{enumerate}

\section{Fourier compression}
\inputdir{fourier}

In this exercise, we will use a depth slice selected from a 3-D
seismic volume and shown in Figure~\ref{fig:data} \cite[]{hall}. Notice
a channel structure.

\sideplot{data}{width=\textwidth}{Seismic depth slice with a channel structure.}

\sideplot{fft}{width=\textwidth}{Absolute value of the Fourier transform of 
the seismic slice from Figure~\ref{fig:data}. The circle inside shows
a window selected for compression.}

The goal of your assignment is to find a compressed representation of
the data in the Fourier transform domain. Figure~\ref{fig:fft} shows
the Fourier transform of the data from Figure~\ref{fig:data}. We can
see that most of the energy gets concentrated near the center (zero
frequency).

There are two alternative ways to compress data in the Fourier domain:
\begin{itemize}
\item One approach is to
select a range of frequencies that contain the most important
information. An advantage of this approach is the ability to subsample
the original data by transforming back from a windowed range of frequencies.
The results from this method are shown in Figure~\ref{fig:sig,cut}.
\item Another approach is to zero all Fourier coefficients below a certain threshold value, regardless of which frequencies they represent.  
 The results from this method are shown in Figure~\ref{fig:thr,noi}. 
Figure~\ref{fig:hist} shows the selected threshold plotted against the histogram of Fourier coefficients.
\end{itemize}

\multiplot{2}{sig,cut}{width=0.45\textwidth}{Data separated into signal (a) and noise (b) by applying Fourier compression with windowing.}
\multiplot{2}{thr,noi}{width=0.45\textwidth}{Data separated into signal (a) and noise (b) by applying Fourier compression with thresholding.}

\sideplot{hist}{width=0.8\textwidth}{Normalized histogram of Fourier coefficients (by absolute value). The vertical line shows the selected threshold.}

\begin{enumerate}
\item Change directory to \texttt{hw3/fourier}.
\item Run 
\begin{verbatim}
scons view
\end{verbatim}
to reproduce the figures on your screen.
\item Modify the \texttt{SConstruct} file to decrease the size of the window so that the noise level increases in Figure~\ref{fig:cut}. How do you measure the noise level? Find a level that you find negligibly small.
\item Modify the \texttt{SConstruct} file to increase the threshold value so that the compression achieves the same quality as in the previous case. The noise level in Figure~\ref{fig:noi} should match that in Figure~\ref{fig:cut}.
\item Compare the number of nonzero Fourier coefficients in both cases. Which method achieves a better compression?
\item \textbf{EXTRA CREDIT} for finding a way for a better compression of the data in the Fourier domain. Your data reconstruction should have 
the same noise level, yet the number of non-zero coefficients in the Fourier domain should be smaller.
\end{enumerate}

\lstset{language=python,numbers=left,numberstyle=\tiny,showstringspaces=false}
\lstinputlisting[frame=single]{fourier/SConstruct}

\section{Interpolation after coordinate transformation}
\inputdir{rotate}

In this exercise, we will use a slice out of a 3-D CT-scan of a
carbonate rock sample, shown in
Figure~\ref{fig:circle}\footnote{Courtesy of Jim Jennings
  (currently at Shell.)}. Notice microfracture channels.

\multiplot{2}{circle,rotate}{width=0.45\textwidth}{Slice of a CT-scan
  of a carbonate rock sample. (a) Original. (b) After clockwise rotation
  by $90^{\circ}$.}

The goal of the exercise is to apply a coordinate transformation to
the original data. A particular transformation that we will study is
coordinate rotation. Figure~\ref{fig:rotate} shows the original slice
rotated by 90 degrees. A 90-degree rotation in this case amounts to
simple transpose. However, rotation by a different angle requires
interpolation from the original grid to the modified grid.

The task of coordinate rotation is accomplished by the C program
\texttt{rotate.c}. Two different methods are implemented: 
nearest-neighbor interpolation and bilinear interpolation.

To test the accuracy of different methods, we can rotate the original
data in small increments and then compare the result of rotating to
$360^{\circ}$ with the original data. Figure~\ref{fig:nearest,linear}
compares the error of the nearest-neighbor and bilinear interpolations
after rotating the original slice in increments of $20^{\circ}$. The
accuracy is comparatively low for small discontinuous features like
microfracture channels.

To improve the accuracy further, we need to employ a longer
filter. One popular choice is \emph{cubic convolution} interpolation,
invented by Robert Keys (a geophysicist, currently at ConocoPhillips).
The cubic convolution filter can be expressed as the
filter \cite[]{keys}
\begin{eqnarray}
\nonumber
Z^{\sigma} \approx C(Z) & = & -\frac{\sigma\,(1-\sigma)^2}{2}\,Z^{-1} + 
\frac{(1-\sigma)\,(2 + 2\,\sigma - 3 \sigma^2)}{2} + \\
&  & \frac{\sigma\,(1 + 4\,\sigma - 3\,\sigma^2)}{2}\,Z - \frac{(1-\sigma)\,\sigma^2}{2}\,Z^2\;.
\label{eq:cubic}
\end{eqnarray}
and is designed to approximate the ideal sinc-function interpolator.

\multiplot{2}{nearest,linear}{width=0.45\textwidth}{Error of different
  interpolation methods computed after full circle rotation in
  increments of 20 degrees. (a) Nearest-neighbor interpolation. (b)
  Bi-linear interpolation.}

\lstset{language=c,numbers=left,numberstyle=\tiny,showstringspaces=false}
\lstinputlisting[frame=single,title=rotate/rotate.c]{rotate/rotate.c}

\lstset{language=python,numbers=left,numberstyle=\tiny,showstringspaces=false}
\lstinputlisting[frame=single,title=rotate/SConstruct]{rotate/SConstruct}

Your task:
\begin{enumerate}
\item Change directory to \texttt{hw3/rotate}
\item Run 
\begin{verbatim}
scons view
\end{verbatim}
to reproduce the figures on your screen.
\item Additionally, you can run
\begin{verbatim}
scons nearest.vpl
\end{verbatim}
and
\begin{verbatim}
scons linear.vpl
\end{verbatim}
to see movies of incremental slice rotation with different methods.
\item Modify the \texttt{rotate.c} program and the \texttt{SConstruct} file to implement 
the cubic convolution interpolation and to compare 
its results with the two other methods.
\item \textbf{EXTRA CREDIT} for implementing an interpolation algorithm, which 
is more accurate than cubic convolution.
\end{enumerate}

\section{Your own data}

Your final task is to apply one of the data analysis techniques of the
previous sections (Fourier compression or coordinate transformation) to your own data:
\begin{enumerate}
\item Select a dataset suitable for compression or coordinate transformation. 
\item Apply one of the algorithms of the previous two sections and choose appropriate parameters.
\item Include the results in your homework.
\end{enumerate}

%\newpage

\section{Completing the assignment}

\begin{enumerate}
\item Change directory to \texttt{hw3}.
\item Edit the file \texttt{paper.tex} in your favorite editor and change the
  first line to have your name instead of Fourier's.
\item Run
\begin{verbatim}
sftour scons lock
\end{verbatim}
to update all figures.
\item Run
\begin{verbatim}
sftour scons -c
\end{verbatim}
to remove intermediate files.
\item Run
\begin{verbatim}
scons pdf
\end{verbatim}
to create the final document.
\item Submit your result (file \texttt{paper.pdf}) on paper or by
e-mail.
\end{enumerate}

\bibliographystyle{seg}
\bibliography{hw3}
